\documentclass[12pt]{article}
\usepackage{amsmath, amssymb}

\newtheorem{acknowledgement}{Acknowledgement}
\newtheorem{algorithm}{Algorithm}
\newtheorem{axiom}{Axiom}
\newtheorem{case}{Case}
\newtheorem{claim}{Claim}
\newtheorem{conclusion}{Conclusion}
\newtheorem{condition}{Condition}
\newtheorem{conjecture}{Conjecture}
\newtheorem{corollary}{Corollary}
\newtheorem{criterion}{Criterion}
\newtheorem{definition}{Definition}
\newtheorem{example}{Example}
\newtheorem{exercise}{Exercise}
\newtheorem{lemma}{Lemma}
\newtheorem{notation}{Notation}
\newtheorem{problem}{Problem}
\newtheorem{proposition}{Proposition}
\newtheorem{remark}{Remark}
\newtheorem{solution}{Solution}
\newtheorem{summary}{Summary}
\newtheorem{theorem}{Theorem}


\newcommand{\fancy}[1]{\mathcal{#1}}
\newcommand{\C}[1]{\fancy{C}_{#1}}


\newcommand{\IN}{\mathbb{N}}
\newcommand{\IZ}{\mathbb{Z}}
\newcommand{\IR}{\mathbb{R}}
\newcommand{\G}{\fancy{G}}
\newcommand{\CC}{\fancy{C}}
\newcommand{\D}{\fancy{D}}
\newcommand{\T}{\fancy{T}}
\newcommand{\B}{\fancy{B}}
\renewcommand{\L}{\fancy{L}}
\newcommand{\HH}{\fancy{H}}

\newcommand{\inj}{\hookrightarrow}
\newcommand{\surj}{\twoheadrightarrow}

\newcommand{\set}[1]{\left\{ #1 \right\}}
\newcommand{\setb}[3]{\left\{ #1 \in #2 : #3 \right\}}
\newcommand{\setbs}[2]{\left\{ #1 : #2 \right\}}
\newcommand{\card}[1]{\left|#1\right|}
\newcommand{\size}[1]{\left\Vert#1\right\Vert}
\newcommand{\ceil}[1]{\left\lceil#1\right\rceil}
\newcommand{\floor}[1]{\left\lfloor#1\right\rfloor}
\newcommand{\func}[3]{#1\colon #2 \rightarrow #3}
\newcommand{\funcinj}[3]{#1\colon #2 \inj #3}
\newcommand{\funcsurj}[3]{#1\colon #2 \surj #3}
\newcommand{\irange}[1]{\left[#1\right]}
\newcommand{\join}[2]{#1 \mbox{\hspace{2 pt}$\ast$\hspace{2 pt}} #2}
\newcommand{\djunion}[2]{#1 \mbox{\hspace{2 pt}$+$\hspace{2 pt}} #2}
\newcommand{\parens}[1]{\left( #1 \right)}
\newcommand{\brackets}[1]{\left[ #1 \right]}
\newcommand{\DefinedAs}{\mathrel{\mathop:}=}

\newcommand{\mic}{\operatorname{mic}}
\newcommand{\AT}{\operatorname{AT}}
\newcommand{\col}{\operatorname{col}}
\newcommand{\ch}{\operatorname{ch}}
\newcommand{\type}{\operatorname{type}}
\newcommand{\nonsep}{\bar{S}}
\newcommand{\type}{\operatorname{type}}
\def\adj{\leftrightarrow}
\def\nonadj{\not\!\leftrightarrow}
\newcommand{\gcd}{\operatorname{gcd}}

\newcommand\restr[2]{{% we make the whole thing an ordinary symbol
  \left.\kern-\nulldelimiterspace % automatically resize the bar with \right
  #1 % the function
  \vphantom{\big|} % pretend it's a little taller at normal size
  \right|_{#2} % this is the delimiter
  }}

\def\D{\fancy{D}}
\def\C{\fancy{C}}
\def\A{\fancy{A}}
\def\s{\sigma}

\newcommand{\claim}[2]{{\bf Claim #1.}~{\it #2}~~}
\newcommand{\case}[2]{{\bf Case #1.}~{\it #2}~~}
\newcommand\numberthis{\addtocounter{equation}{1}\tag{\theequation}}

\def\gcd{\bigtriangledown}
\def\lcm{\bigtriangleup}
\def\no{\natural}


\usepackage{tikz}
\usetikzlibrary{calc}

\pgfdeclarelayer{background}
\pgfsetlayers{background,main}
\newcommand{\Bond}[6]%
% start, end, thickness, incolor, outcolor, iterations
{ \begin{pgfonlayer}{background}
        \colorlet{InColor}{#4}
        \colorlet{OutColor}{#5}
        \foreach \I in {#6,...,1}
        {   \pgfmathsetlengthmacro{\r}{#3/#6*\I}
            \pgfmathsetmacro{\C}{sqrt(1-\r*\r/#3/#3)*100}
            \draw[InColor!\C!OutColor, line width=\r] (#1.center) -- (#2.center);
        }
    \end{pgfonlayer}
}

\newcommand{\BlackBond}[2]%
% start, end
{   \Bond{#1}{#2}{0.7071mm}{black!25}{black!25!black}{10}
}

\title{committees with scribes}
\author{landon rabern\thanks{landon.rabern@gmail.com}}
\begin{document}
\maketitle

\section*{the setup}
You have a group of people $P$ in the red room and those people are on some committees, possibly some people on none or many committees; that is, you can describe your
committees as a collection $X$ of subsets of $P$.  There are never committees without members. For each \emph{committee} $C \in X$, there is a group of \emph{scribes} $\s(C)$ for $C$,  where $\s(C) \subseteq C$.
Think of the scribes of $C$ as the people that are allowed to speak for the committee $C$.
Taken together, the tuple $(P,X,\sigma)$ is an \emph{oriented hypergraph}.  Given an oriented hypergraph $(P,X,\sigma)$, you are interested in finding a
group of people $K \subseteq P$ so that each $x \in P \setminus K$ is on at least one committee with a scribe in $K$.  
If you take such a group of people $K$ out of the red room into the green room,
then person in the green room is spoken for on some committee by someone in the red room. 
You also want to leave a representative of each committee (not necessarily one that can speak for the group) in the red room
at all times; that is, we want to make sure that $C \not \subseteq K$ for all committees $C$.  Such a group $K$ is a \emph{kernel} for the oriented hypergraph $(P,X,\sigma)$.  
More succinctly, $K \subseteq P$ is a kernel for $(P,X,\sigma)$ just in case $C \not \subseteq K$ for all committees $C \in X$ and for each person $x\in P\setminus K$, there is a committee $C \ni x$ such that $K \cap \sigma(C) \ne \emptyset$.  Deciding
whether or not a given oriented hypergraph has a kernel appears to be a hard problem (many known NP-hard problems are special cases of this).  But things are not hard for
all oriented hypergraphs and you can build some tools by looking for useful structure.  For constructing kernels the notion of an \emph{induced oriented subhypergraph} will be indispensable.
Let $\HH \DefinedAs (P,X,\sigma)$ be an oriented hypergraph.  If $A \subseteq P$, then the \emph{oriented subhypergraph of $\HH$ induced on $A$}, written $\HH[A]$ is the tuple 
$(A, X\brackets{A}, \sigma)$ where $X\brackets{A}$ is the set of $C \in X$ where $C \subseteq A$ and we view $\sigma$ as restricted to $X\brackets{A}$.  
An oriented hypergraph $\HH$ is \emph{kernel-perfect} if every induced oriented subhypergraph of $\HH$ has a kernel.  When $\HH \DefinedAs (P,X,\sigma)$, set $\card{\HH} \DefinedAs \card{P}$
and $\size{\HH} \DefinedAs \card{X}$.

\begin{lemma}
Let $\HH \DefinedAs (P,X,\sigma)$ be an oriented hypergraph.  If there is $K \subseteq P$ containing no committee of $\HH$ 
such that $\sigma(C) > 1$ for all $C \in X\brackets{P \setminus K}$, then $\HH$ is kernel-perfect.
\end{lemma}
\begin{proof}
Suppose for the moment that the lemma is false.  Go grab a counterexample $\HH \DefinedAs (P,X,\sigma)$ with $\card{\HH}$ as small as possible.
Then $\HH$ contains no kernel by minimality of $\card{\HH}$; in particular $\card{\HH} > 0$. Now grab $K \subseteq P$ containing no committee of $\HH$ such that $\sigma(C) > 1$ for all $C \in X\brackets{P \setminus K}$.  Since $K$ contains no
committee of $\HH$ and it isn't a kernel, we can grab $x \in P\setminus K$ so that $\sigma(C) \cap K = \emptyset$ for every $C \in X$ with $x \in C$.  

\end{proof}

\end{document}