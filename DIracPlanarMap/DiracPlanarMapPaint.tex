\documentclass[12pt]{article}
\usepackage{amsmath, amsthm, amssymb}
\usepackage{hyperref}
\usepackage[margin=1cm]{caption}
\usepackage{verbatim}
\usepackage[top=1.0in, bottom=1.0in, left=1.0in, right=1.0in]{geometry}
\usepackage{graphicx}

\pagestyle{plain}

\usepackage{tkz-graph}
\usetikzlibrary{arrows}
\usetikzlibrary{shapes}
\usepackage[position=bottom]{subfig}

\usepackage{longtable}
\usepackage{array}

\usepackage{sectsty}
\allsectionsfont{\sffamily}

\setcounter{secnumdepth}{5}
\setcounter{tocdepth}{5}

\makeatletter
\newtheorem*{rep@theorem}{\rep@title}
\newcommand{\newreptheorem}[2]{
\newenvironment{rep#1}[1]{
 \def\rep@title{#2 \ref{##1}}
 \begin{rep@theorem}}
 {\end{rep@theorem}}}
\makeatother

\theoremstyle{plain}
\newtheorem{thm}{Theorem}[section]
\newreptheorem{thm}{Theorem}
\newtheorem{prop}[thm]{Proposition}
\newreptheorem{prop}{Proposition}
\newtheorem{lem}[thm]{Lemma}
\newreptheorem{lem}{Lemma}
\newtheorem{conjecture}[thm]{Conjecture}
\newreptheorem{conjecture}{Conjecture}
\newtheorem{cor}[thm]{Corollary}
\newreptheorem{cor}{Corollary}
\newtheorem{prob}[thm]{Problem}
\newtheorem{observation}{Observation}
\newtheorem{obs}[observation]{Observation}
\newtheorem*{mainconj}{Main Conjecture}
\newtheorem*{mainthm}{Main Theorem}
\newtheorem{problem}{Problem}
\newtheorem{clm}{Claim}

\theoremstyle{definition}
\newtheorem{defn}{Definition}
\theoremstyle{remark}
\newtheorem*{remark}{Remark}
\newtheorem{example}{Example}
\newtheorem*{question}{Question}


\newcommand{\fancy}[1]{\mathcal{#1}}
\newcommand{\C}[1]{\fancy{C}_{#1}}
\newcommand{\IN}{\mathbb{N}}
\newcommand{\IR}{\mathbb{R}}
\newcommand{\G}{\fancy{G}}
\newcommand{\CC}{\fancy{C}}
\newcommand{\D}{\fancy{D}}

\newcommand{\inj}{\hookrightarrow}
\newcommand{\surj}{\twoheadrightarrow}

\newcommand{\set}[1]{\left\{ #1 \right\}}
\newcommand{\setb}[3]{\left\{ #1 \in #2 \mid #3 \right\}}
\newcommand{\setbs}[2]{\left\{ #1 \mid #2 \right\}}
\newcommand{\card}[1]{\left|#1\right|}
\newcommand{\size}[1]{\left\Vert#1\right\Vert}
\newcommand{\ceil}[1]{\left\lceil#1\right\rceil}
\newcommand{\floor}[1]{\left\lfloor#1\right\rfloor}
\newcommand{\func}[3]{#1\colon #2 \rightarrow #3}
\newcommand{\funcinj}[3]{#1\colon #2 \inj #3}
\newcommand{\funcsurj}[3]{#1\colon #2 \surj #3}
\newcommand{\irange}[1]{\left[#1\right]}
\newcommand{\join}[2]{#1 \mbox{\hspace{2 pt}$\ast$\hspace{2 pt}} #2}
\newcommand{\djunion}[2]{#1 \mbox{\hspace{2 pt}$+$\hspace{2 pt}} #2}
\newcommand{\parens}[1]{\left( #1 \right)}
\newcommand{\brackets}[1]{\left[ #1 \right]}
\newcommand{\nint}[1]{\widetilde{N}\left(#1\right)}
\newcommand{\DefinedAs}{\mathrel{\mathop:}=}
\newcommand{\pot}{\operatorname{pot}}
\newcommand{\mic}{\operatorname{mic}}
\def\adj{\leftrightarrow}
\def\nonadj{\not\!\leftrightarrow}

\def\D{\fancy{D}}
\def\C{\fancy{C}}
\def\Q{\fancy{Q}}
\def\Z{\fancy{Z}}
\def\H{\fancy{H}}
\def\L{\fancy{L}}

% any changes to \claim should be mirrored in \claimnonum and \subclaim
\newcommand{\claim}[2]{{\bf Claim #1.}~{\it #2}~~}
\newcommand{\claimnonum}[1]{{\bf Claim.}~{\it #1}~~}
\newcommand{\subclaim}[2]{{\bf Subclaim #1.}~{\it #2}~~}

\newcommand\numberthis{\addtocounter{equation}{1}\tag{\theequation}}

%
%  If the proof ends with a displayed equation, use \aftermath just
%  before \end{proof} to put the halmos in the ``right'' place.  This
%  may not work near page boundaries. 
%
\def\aftermath{\par\vspace{-\belowdisplayskip}\vspace{-\parskip}\vspace{-\baselineskip}}

\def\fr{\frac}
\def\adj{\leftrightarrow}
\def\ch{\textrm{ch}}

\renewcommand{\restriction}{\mathord{\upharpoonright}}
\begin{document}
\title{Dirac planar map for paint and AT notes}
\author{}
\maketitle

\section{Basics}
We consider graphs embedded on surfaces without boundary.  These come in two flavors: the \emph{orientable surfaces} surfaces $\Sigma_g$ which is the sphere with $g$ handles; and the \emph{non-orientable surfaces}  $\Pi_h$ which is the sphere with $h$ cross-caps.  The \emph{Euler genus} $\varepsilon$ of $\Sigma_g$ is $2g$ and the Euler genus of $\Pi_h$ is $h$.  The \emph{Euler genus} of a graph $G$ is the smallest Euler genus of a surface on which $G$ can be embedded.  By Euler's formula, if a graph $G$ is embedded on a surface of Euler genus $\varepsilon$, then $n-m+f \ge 2 - \varepsilon$ where $n = \card{G}$, $m = \size{G}$ and $f$ is the number of faces of $G$.  It follows that if $G$ is embedded on a surface of Euler genus $\varepsilon$, then $\size{G} \le 3\card{G} - 6 + 3\varepsilon$. The \emph{Heawood} number is given by
\[H(\varepsilon) = \floor{\frac{7 + \sqrt{24\varepsilon + 1}}{2}}.\]

When $\varepsilon \ge 1$, the bound on $\size{G}$ above implies that if $G$ is embedded on a surface of Euler genus $\varepsilon$, then $G$ has a vertex of degree at most $H(\varepsilon) - 1$.  In particular, the graphs embedded on a surface of Euler genus $\varepsilon \ge 1$ are $H(\varepsilon)$-AT and hence $H(\varepsilon)$-paintable.  The goal is to show that the only obstruction to $G$ being  $(H(\varepsilon) - 1)$-AT is $G$ containing $K_{H(\varepsilon)}$.

\begin{conjecture}\label{MainC}
	Let $G$ be a graph embedded on a surface of Euler genus $\varepsilon \ge 1$.  If $K_{H(\varepsilon)} \not \subseteq G$, then $G$ is $(H(\varepsilon) - 1)$-AT.
\end{conjecture}

\section{From Gallai's Bound}
Gallai proved that the low vertex subgraph of a critical graph is a Gallai forest.  This follows immediately from the later classification of $d_0$-choosable graphs.  The same classification holds for paint and AT, so we get that the low vertex
subgraph of an AT-critical graph is a Gallai forest.

\section{With Kernel Magic}

We can get almost all the way there for paint using the follow result from \cite{OreVizing}.
\begin{defn} The \emph{maximum independent cover number }of a graph $G$
	is the maximum $\mic(G)$ of $\sum_{v\in I}d_{G}(v)$ over all independent sets $I$
	of $G$. A set $I$ that witnesses this maximum is said to be optimal. 
\end{defn}

In \cite{OreVizing} it was shown that $\mic(G) \ge |G| - 1$ for all $G$ and $\mic(G) \ge |G|$ if $G$ is a connected graph that is not a Gallai tree.

\begin{defn} A graph $G$ is \emph{P-reducible} to $H$ if $H$ is a nonempty induced
	subgraph of $G$ which is $f_{H}$-paintable where $f_{H}(v)\DefinedAs\delta(G)+d_{H}(v)-d_{G}(v)$
	for all $v\in V(H)$. If $G$ is not P-reducible to any nonempty induced subgraph,
	then it is \emph{P-irreducible}. \end{defn}

\begin{thm}\label{ConsantListMicStrength} Every P-irreducible graph $G$ satisfies
	$\mic(G)\leq2\size{G}-(\delta(G)-1)\card{G}-1$. 
\end{thm}

A minimal counterexample $G$ to Conjecture \ref{MainC} for paint is clearly P-irreducible.  So, if $G$ has Euler genus $\varepsilon$, then we have
\[\mic(G) \le 2\size{G}-(\delta(G)-1)\card{G}-1 \le 2(3\card{G} - 6 + 3\varepsilon) -(\delta(G)-1)\card{G}-1.\]
Since $\delta(G) \ge H(\varepsilon) - 1$, this becomes 
\[\mic(G) \le (8 - H(\varepsilon))\card{G} + 6\varepsilon - 13.\]
Since $G$ is $2$-connected and not complete, it is not a Gallai tree, thus we have $\mic(G) \ge |G|$ which gives
\[0 \le (7 - H(\varepsilon))\card{G} + 6\varepsilon - 13.\]
Since $H(2) = 7$, we conclude that $\varepsilon \ne 2$.

\bibliographystyle{amsplain}
\bibliography{GraphColoring1}

\end{document}
