\documentclass[12pt]{article}
\usepackage{amsmath, amsthm, amssymb}
\usepackage{hyperref}
\usepackage[margin=1cm]{caption}
\usepackage{verbatim}
\usepackage[top=1.0in, bottom=1.0in, left=1.0in, right=1.0in]{geometry}
\usepackage{graphicx}

\pagestyle{plain}

\usepackage{tkz-graph}
\usetikzlibrary{arrows}
\usetikzlibrary{shapes}
\usepackage[position=bottom]{subfig}

\usepackage{longtable}
\usepackage{array}

\usepackage{sectsty}
\allsectionsfont{\sffamily}

\setcounter{secnumdepth}{5}
\setcounter{tocdepth}{5}

\makeatletter
\newtheorem*{rep@theorem}{\rep@title}
\newcommand{\newreptheorem}[2]{
\newenvironment{rep#1}[1]{
 \def\rep@title{#2 \ref{##1}}
 \begin{rep@theorem}}
 {\end{rep@theorem}}}
\makeatother

\theoremstyle{plain}
\newtheorem{thm}{Theorem}[section]
\newreptheorem{thm}{Theorem}
\newtheorem{prop}[thm]{Proposition}
\newreptheorem{prop}{Proposition}
\newtheorem{lem}[thm]{Lemma}
\newreptheorem{lem}{Lemma}
\newtheorem{conjecture}[thm]{Conjecture}
\newreptheorem{conjecture}{Conjecture}
\newtheorem{cor}[thm]{Corollary}
\newreptheorem{cor}{Corollary}
\newtheorem{prob}[thm]{Problem}
\newtheorem{observation}{Observation}
\newtheorem{obs}[observation]{Observation}
\newtheorem*{mainconj}{Main Conjecture}
\newtheorem*{mainthm}{Main Theorem}
\newtheorem{problem}{Problem}
\newtheorem{clm}{Claim}
\newtheorem*{VAL}{Vizing's Adjacency Lemma (VAL)}

\theoremstyle{definition}
\newtheorem{defn}{Definition}
\theoremstyle{remark}
\newtheorem*{remark}{Remark}
\newtheorem{example}{Example}
\newtheorem*{question}{Question}


\newcommand{\fancy}[1]{\mathcal{#1}}
\newcommand{\C}[1]{\fancy{C}_{#1}}
\newcommand{\IN}{\mathbb{N}}
\newcommand{\IR}{\mathbb{R}}
\newcommand{\G}{\fancy{G}}
\newcommand{\CC}{\fancy{C}}
\newcommand{\D}{\fancy{D}}

\newcommand{\inj}{\hookrightarrow}
\newcommand{\surj}{\twoheadrightarrow}

\newcommand{\set}[1]{\left\{ #1 \right\}}
\newcommand{\setb}[3]{\left\{ #1 \in #2 \mid #3 \right\}}
\newcommand{\setbs}[2]{\left\{ #1 \mid #2 \right\}}
\newcommand{\card}[1]{\left|#1\right|}
\newcommand{\size}[1]{\left\Vert#1\right\Vert}
\newcommand{\ceil}[1]{\left\lceil#1\right\rceil}
\newcommand{\floor}[1]{\left\lfloor#1\right\rfloor}
\newcommand{\func}[3]{#1\colon #2 \rightarrow #3}
\newcommand{\funcinj}[3]{#1\colon #2 \inj #3}
\newcommand{\funcsurj}[3]{#1\colon #2 \surj #3}
\newcommand{\irange}[1]{\left[#1\right]}
\newcommand{\join}[2]{#1 \mbox{\hspace{2 pt}$\ast$\hspace{2 pt}} #2}
\newcommand{\djunion}[2]{#1 \mbox{\hspace{2 pt}$+$\hspace{2 pt}} #2}
\newcommand{\parens}[1]{\left( #1 \right)}
\newcommand{\brackets}[1]{\left[ #1 \right]}
\newcommand{\nint}[1]{\widetilde{N}\left(#1\right)}
\newcommand{\DefinedAs}{\mathrel{\mathop:}=}
\newcommand{\pot}{\operatorname{pot}}

\def\adj{\leftrightarrow}
\def\nonadj{\not\!\leftrightarrow}

\def\D{\fancy{D}}
\def\C{\fancy{C}}
\def\Q{\fancy{Q}}
\def\Z{\fancy{Z}}
\def\H{\fancy{H}}
\def\T{\fancy{T}}
\def\X{\fancy{X}}

% any changes to \claim should be mirrored in \claimnonum and \subclaim
\newcommand{\claim}[2]{{\bf Claim #1.}~{\it #2}~~}
\newcommand{\claimnonum}[1]{{\bf Claim.}~{\it #1}~~}
\newcommand{\subclaim}[2]{{\bf Subclaim #1.}~{\it #2}~~}

\newcommand\numberthis{\addtocounter{equation}{1}\tag{\theequation}}

%
%  If the proof ends with a displayed equation, use \aftermath just
%  before \end{proof} to put the halmos in the ``right'' place.  This
%  may not work near page boundaries. 
%
\def\aftermath{\par\vspace{-\belowdisplayskip}\vspace{-\parskip}\vspace{-\baselineskip}}

\def\fr{\frac}
\def\adj{\leftrightarrow}
\def\ch{\textrm{ch}}


\begin{document}
	
\section{Definitions}
A \emph{2-partition} of a set $S$ is a partition of $S$ into sets of size two and at most one set of size one.

\section{Fixable graphs}
For different colors $a,b \in P$, let $S_{L,a,b}$ be all the vertices of $G$ that have exactly one of $a$
or $b$ in their list; more precisely, $S_{L,a,b} = \setb{v}{V(G)}{\,\card{\set{a,b} \cap L(v)} = 1}$.   
 If $\X$ is a 2-partition of $S_{L,a,b}$ and $J \subseteq \X$, let $L_{J}$ be the list assignment formed
from $L$ by swapping $a$ and $b$ in $L(v)$ for every $v \in \bigcup J$.  If $J = \set{X}$, we also write $L_X$ for $L_{J}$.

\begin{defn}
$G$ is \emph{$(L, P)$-fixable} if either
\begin{enumerate}
\item[(1)] $G$ has an $L$-edge-coloring; or
\item[(2)] there are different colors $a,b \in P$ such that for every 2-partition
$\X$ of $S_{L,a,b}$ there exists $J\subseteq \X$ so that $G$ is $(L_J, P)$-fixable.
\end{enumerate}
\end{defn}

The meaning of (1) is clear.  Intuitively, (2) says the following.  There is
some pair of colors, $a$ and $b$, such that regardless of how the vertices of
$S_{L,a,b}$ are paired via Kempe chains for colors $a$ and $b$ (or not paired
with any vertex of $S_{L,a,b}$), we can swap the colors on some subset $J$ of
the Kempe chains so that the resulting partial edge-coloring is fixable.

We write $L$-fixable as shorthand for $(L, \pot(L))$-fixable. When $G$ is $(L,
P)$-fixable, the choices of $a,b$, and $J$ in each application of (2) determine
a tree where all leaves have lists satisfying (1).  The \emph{height} of $(L,
P)$ is the minimum possible height of such a tree.  We write $h_G(L, P)$ for
this height and let $h_G(L, P) = \infty$ when $G$ is not $(L,P)$-fixable. 

\begin{lem}\label{FixableCompletesColoring}
If a multigraph $M$ has a partial $k$-edge-coloring $\pi$ such that $M_\pi$ is $(L_\pi, \irange{k})$-fixable, then $M$ is $k$-edge-colorable.
\end{lem}

\subsection{A necessary condition}
Since the edges incident to a vertex $v$ must all get different colors, 
if $G$ is $(L, P)$-fixable, then $|L(v)| \ge d_G(v)$ for all $v \in V(G)$.

By considering the maximum size of matchings in each color, we get a more
interesting necessary condition.
For each $C \subseteq \pot(L)$ and $H \subseteq G$, let $H_{L, C}$ be the
subgraph of $H$ induced by the vertices $v$ with $L(v) \cap C \ne \emptyset$. 
When $L$ is clear from context, we write $H_C$ for $H_{L,C}$. If $C =
\set{\alpha}$, we write $H_\alpha$ for $H_C$.  For $H \subseteq G$, let

\[\psi_L(H) = \sum_{\alpha \in \pot(L)} \floor{\frac{\card{H_{L, \alpha}}}{2}}.\]
Each term in the sum gives an upper bound on the size of a matching in color
$\alpha$. So $\psi_L(H)$ is an upper bound on the number of edges in a
partial $L$-edge-coloring of $H$.  The pair $(H, L)$ is \emph{abundant} if
$\psi_L(H) \ge \size{H}$ and $(G,L)$ is \emph{superabundant} if for every
$H \subseteq G$, the pair $(H, L)$ is abundant.  

\begin{lem}
\label{SuperabundanceIsNecessary} 
If $G$ is $(L, P)$-fixable, then $(G, L)$ is superabundant.  
\end{lem}


\begin{defn}
$G$ is \emph{$(L, P)$-subfixable} if either
\begin{enumerate}
\item[(1)] $G$ is $(L, P)$-fixable; or
\item[(2)] there is $xy \in E(G)$ and $\tau \in L(x) \cap L(y)$ such that
$G-xy$ is $L'$-subfixable, where $L'$ is formed from $L$ by removing $\tau$ from
$L(x)$ and $L(y)$.
\end{enumerate}
\end{defn}

Superabundance is a necessary condition for subfixability because coloring an
edge cannot make a non-abundant subgraph abundant.  The conjectures in the rest
of this paper may be easier to prove with subfixable in place of fixable.  That would
really be just as good since it would give the exact same results for edge coloring.

This may be useful.  For a multigraph $H$, let $\nu(H)$ be the number of edges in a maximum matching
of $H$.  For a list assignment $L$ on $H$, let 
$$\eta_L(H) = \sum_{\alpha \in \pot(L)} \nu(H_\alpha).$$  
Note that always $\psi_L(H) \ge \eta_L(H)$.

\begin{lem}[Marcotte and Seymour]\label{MultiTreeHall}
	Let $T$ be a multitree and $L$ a list assignment on $V(T)$.  If $\eta_L(H) \ge
	\size{H}$ for all $H \subseteq T$, then $T$ has an $L$-edge-coloring.
\end{lem}

\section{Swappable pairs}
Suppose $(G,L)$ is superabundant.  We say that $a,b \in \pot(L)$ are \emph{swappable} if $(G,L_X)$ is superabundant for every $X \subseteq S_{L,a,b}$ with $\card{X} \le 2$.

\begin{lem}\label{SwappableCondition}
	Suppose $(G, L)$ is superabundant.  Then $a, b \in \pot(L)$ are swappable if for every $H \subseteq G$, at least one of the following holds:
	\begin{enumerate}
		\item $\psi_L(H) > \size{H}$; or,
		\item $\card{H_{L, a}}$ is odd; or,
		\item $\card{H_{L, b}}$ is odd.
	\end{enumerate}
\end{lem}
%
\begin{proof}
	Suppose not and choose $X \subseteq S_{L,a,b}$ with $\card{X} \le 2$ such that $(G,L_X)$ is not superabundant.  
	Then we have $H \subseteq G$ such that$(H,L_X)$ is not abundant.  Note that $\card{H_{L,a}}$ and
	$\card{H_{L_X,a}}$ differ by at most 2, so their contributions to
	$\psi_L(H)$ and $\psi_{L_X}(H)$ differ by at most 1; the same is true for
	$\card{H_{L,a}}$ and $\card{H_{L_X,b}}$.  
	If $\psi_L(H)>\size{H}$, then $\psi_{L_X}(H) \ge \psi_L(H)-1\ge \size{H}$, a contradiction.
	So (2) or (3) holds.	The only way that we can have
	$\psi_{L_X}(H)<\psi_L(H)$ is if $\floor{\frac{\card{H_{L_X, a}}}{2}} +
	\floor{\frac{\card{H_{L_X, b}}}{2}} < \floor{\frac{\card{H_{L, a}}}{2}}
	+ \floor{\frac{\card{H_{L, b}}}{2}}$.  
	Since $\card{H_{L, b}} + \card{H_{L, a}} = \card{H_{L_X, b}} +
	\card{H_{L_X, a}}$,   this requires that both $\card{H_{L, b}}$ and
	$\card{H_{L, a}}$ are even; since (2) or (3) holds, this is impossible.
\end{proof}

\section{Stars with one edge subdivided}
\begin{conjecture}
	\label{StarWithOneEdgeSubdivided}
	Let $G$ be a star with one edge subdivided, where $r$ is the center of the
	star, $t$ the vertex at distance two from $r$, and $s$ the intervening vertex.  
	If $L$ is superabundant and $|L(v)| \ge d_G(v)$ for all 
	$v \in V(G)$, then $G$ is $L$-fixable if at least one of the following holds:
	\begin{enumerate}
		\item[(a)] $|L(r)| > d_G(r)$; or
		\item[(b)] $|L(s)| > d_G(s)$; or
		\item[(c)] $\psi_L(G) > \size{G}$.
	\end{enumerate}
\end{conjecture}

\begin{proof} [Proof of Conjecture \ref{StarWithOneEdgeSubdivided}(a)]
	Suppose not and choose a counterexample $G$ with list assignment $L$ so as to minimize $\card{G}$ and subject to that, to maximize $\eta_L(G-t)$.
	
	Create a bipartite graph $B$ with parts $X = \setb{uw}{E(G - t)}{L(u) \cap L(w) \ne \emptyset}$ and $Y = \setb{\alpha}{\pot(L)}{\nu((G - t)_\alpha) = 1}$, where $uw \in X$ is adjacent to $\alpha \in Y$ if and only if $\alpha \in L(u) \cap L(w)$.  Put $F = L(r) - \bigcup_{v \in N(r)} L(v)$.  Since $G$ is not $L$-fixable, for each pair of colors $a,b \in \pot(L)$ there is a 2-partition $\X_{a,b}$ of $S_{L,a,b}$ such that $G$ is not $L_J$-fixable for every $J \subseteq \X_{a,b}$.
	
	\claim{1}{For any $\beta \in Y$ that is swappable with some $\gamma \in F$, we have $\card{G_\beta - r - t} \le 2$.}
	
	Suppose $\card{G_\beta - r - t} \ge 3$.  Pick $v \in V(G_\beta - r - t)$.  Let $X \in \X_{\beta,\gamma}$ with $v \in X$.  Note that $L_X(r) = L(r)$ since $\gamma, \beta \in L(r)$.  Since $\card{X \cap V(G_\beta - r - t)} \le 2$,  we have $\eta_{L_X}(G - t) > \eta_L(G - t)$, which contradicts the maximality of $\eta_L(G - t)$.
	
	\claim{2}{For any $\beta \not \in Y$ that is swappable with some $\gamma \in F$, we have $\card{G_\beta - r - t} \le 1$.}
	
	Suppose $\card{G_\beta - r - t} \ge 2$.  Let $X \in \X_{\beta,\gamma}$ with $r \in X$.  Since $\card{X \cap V(G_\beta - r - t)} \le 1$, we again contradict the maximality of $\eta_L(G - t)$.

	\claim{3}{If every $\beta \in \pot(L) - F$ is swappable with some $\gamma \in F$, then $\psi_L(G - t) \le |Y|$.}
	
	Suppose every $\beta \in \pot(L) - F$ is swappable with some $\gamma \in F$.  Then, by Claim 1, the colors in $Y$ each contribute at most one to $\psi_L(G - t)$.  By Claim 2, the colors not in $Y$ contribute nothing to $\psi_L(G - t)$.  Hence $\psi_L(G - t) \le \card{Y}$.

	\claim{4}{We have $\eta_L(G - t) \ge \size{G - t}$.}
	
	Suppose $\eta_L(G - t) < \size{G - t}$. Then we have $\card{F} \ge 2$. 
	
	\subclaim{4a}{There is $\beta \in \pot(L) - F$ that is not swappable with any $\gamma \in F$.}
		
	Otherwise, by Claim 3 we would have $\psi_L(G - t) \le |Y| = \eta_L(G - t) < \size{G - t}$, contradicting superabundance of $(G,L)$.
	
	\subclaim{4b}{There is $\gamma \in F - L(t)$. In particular, $V(G_\gamma) = \set{r}$.}
	
	Suppose $F \subseteq L(t)$.  Then $\card{L(r) \cap L(t)} \ge |F| \ge 2$.  Fix $\gamma \in F$. Let $H \subseteq G$.  If $\card{H_{L, \gamma}}$ is even, then $r, t \in V(H)$ and hence $\psi_L(H) \ge \size{H-t} + \card{L(r) \cap L(t)} \ge \size{H} + 1$.  Therefore $\gamma$ is swappable with $\beta$ by Lemma \ref{SwappableCondition}, contradicting Subclaim 4a.

	\subclaim{4c}{$\gamma$ is swappable with every color in $\pot(L) - F - \beta$.  Moreover, $L(s) \cap L(t) = \set{\beta}$.}
	
	The only subgraph with edges where $\card{H_{L, \gamma}}$ is even is $G[s, t]$, so if $\gamma$ is not swappable with some color $c \in \pot(L) - F$, then it must be $H = G[s, t]$ that fails all conditions of Lemma \ref{SwappableCondition}.  Hence we have $L(s) \cap L(t) = \set{c}$.  So, there is only one such $c$ and by Subclaim 4a, it is $\beta$.
	
	\subclaim{4d}{If $\beta \in Y$, then $\card{G_\beta - r - t} = 3$.}

	If $\beta \in Y$ and $\card{G_\beta - r - t} \le 2$, then the argument in Claim 3 and Subclaim 4a gives $\psi_L(G - t) < \size{G - t}$, a contradiction.
		
	So, suppose $\beta \in Y$ and $\card{G_\beta - r - t} \ge 4$.  Pick $v_1, v_2, v_3 \in V(G_\beta - r - t - s)$.  Then there is $i \in \irange{3}$ and $X \in \X_{\beta,\gamma}$ with $v_i \in X$ such that $X \cap \set{s,t} = \emptyset$. Note that $L_X(r) = L(r)$, $L_X(s) = L(s)$ and $L_X(t) = L(t)$.  Since the only subgraph with edges where $\card{H_{L, \gamma}}$ is even is $G[s, t]$, the argument in Lemma \ref{SwappableCondition} shows that $(G,L_X)$ is superabundant.  Now $\set{\beta, \gamma} \subseteq L(v_1) \cup L(v_2) \cup L(v_3)$, so $\eta_{L_X}(G - t) > \eta_L(G - t)$, which contradicts the maximality of $\eta_L(G - t)$.

	\subclaim{4e}{If $\beta \not \in Y$, then $\card{G_\beta - r - t} = 2$.}
	
	If $\beta \not \in Y$ and $\card{G_\beta - r - t} \le 1$, then the argument in Claim 3 and Subclaim 4a gives $\psi_L(G - t) < \size{G - t}$, a contradiction.
		
	So, suppose $\beta \not \in Y$ and $\card{G_\beta - r - t} \ge 3$. Pick $v_1, v_2 \in V(G_\beta - r - t - s)$ and let $v_3 = r$.  Then there is $i \in \irange{3}$ and $X \in \X_{\beta,\gamma}$ with $v_i \in X$ such that $X \cap \set{s,t} = \emptyset$.  Since the only subgraph with edges where $\card{H_{L, \gamma}}$ is even is $G[s, t]$, the argument in Lemma \ref{SwappableCondition} shows that $(G,L_X)$ is superabundant.  Now $\set{\beta, \gamma} \cap L(r) \subseteq L(v_1) \cup L(v_2)$, so $\eta_{L_X}(G - t) > \eta_L(G - t)$, which contradicts the maximality of $\eta_L(G - t)$.
	
	\subclaim{4f}{There is $\delta \in L(t) - L(s)$ such that $\card{G_\delta - t}$ is odd.  Moreover, $\delta$ and $\gamma$ are swappable.}
	
	By Subclaim 4c and Claim 1, the colors in $Y - \beta$ contribute at most $|Y - \beta|$ to $\psi_L(G - t)$.  By Subclaim 4d and Subclaim 4e, the total contribution of $Y$ and $\beta$ to $\psi_L(G - t)$ is at most $\card{Y} + 1$.  Since nothing else contributes by Claim 2, we have $\psi_L(G - t) \le \eta_L(G - t) + 1 \le \size{G} - 1$.  Since $\psi_L(G) \ge \size{G}$, there must be $\delta \in L(t) - L(s)$ such that $\card{G_\delta - t}$ is odd.  The final statement follows by Subclaim 4c since $\delta \ne \beta$ because $\beta \in L(s)$.
	
	\subclaim{4g}{We may assume $\delta \in F \cap L(t)$.}
	
	Suppose  $\delta \not \in F \cap L(t)$.  First, suppose $\delta \in Y$. Then, by Claim 1 and since $\card{G_\delta - t}$ is odd, we have $\delta \in L(u) \cap L(w)$ for exactly two $u,w \in N(r) - s$.   Let $X \in \X_{\delta,\gamma}$ with $t \in X$.  We have $L_X(r) = L(r)$ and thus if $|X| = 2$, then $\eta_{L_X}(G - t) > \eta_L(G - t)$, a contradiction.  So, $|X| = 1$ and $L_X$ differs from $L$ only on $t$ where $L_X$ has $\gamma$ instead of $\delta$. If $L_X(t) \subseteq F$, then we get a contradiction by the arguments in Subclaim 4a and 4b.  So, we may choose $\alpha \in F - L_X(t)$. Now we can use $\alpha$ in place of $\gamma$ and $\gamma$ in place of $\delta$, which proves the subclaim in this case.
	
	Otherwise, by Claim 2, we must have $\delta \in L(u)$ for exactly one $u$ in $N(r) - s$.  Let $X \in \X_{\delta,\gamma}$ with $t \in X$. As before, we conclude $|X| = 1$ and again we can choose $\alpha \in F - L_X(t)$ and use $\alpha$ in place of $\gamma$ and $\gamma$ in place of $\delta$.
	
	\subclaim{4h}{Claim 4 is true.}
	
	We know that $\beta$ and $\delta$ are not swappable by Subclaim 4a.  Nevertheless, we are now going to swap $\beta$ and $\delta$, taking care to preserve superabundance.  Let $X \in \X_{\beta,\delta}$ with $s \in X$.   Suppose $(G, L_X)$ is not superabundant and choose an induced subgraph $H$ of $G$ such that $(H, L_X)$ is not abundant.  Then $\card{H_\delta}$ is even by Lemma \ref{SwappableCondition} and hence $r,s,t \in V(H)$.  Hmmm... i see ways to break superabundance, we need to do something else.

	
	First, suppose $\beta \in Y$.  Then, by Subclaim 4d, we have $\card{G_\beta - r - t} = 3$.  No matter what Breaker does, $\eta_L(G-t)$ has increased.  Similarly, if $\beta \not \in Y$, Subclaim 3b gives $\card{G_\beta - r - t} \ge 2$ and $\eta_L(G-t)$ has increased no matter Breaker's response.

	\claim{5}{If $\card{C} \ge \card{N_B(C)}$ for $C \subseteq Y$, then $C \cap L(t) \ne \emptyset$.}
	
	Suppose not. Let $B'$ be the subgraph of $B$ induced on $C \cup N_B(C)$. Then we may apply Lemma \ref{SpannerSpecial} to get a nonempty matching $M$ of $B'$ whose vertex set is $S \cup N_B(S)$ for some $S \subseteq C$.  For each $\set{uw, \alpha} \in M$, color $uw$ with $\alpha$.  Then since we used colors $S$, no edge in $X - N_B(S)$ has a color in $S$, so the graph $G' = G - V(N_B(S) - r)$ with lists $L'(v) = L(v) - S$ satisfies the hypotheses of the lemma.  Since $\card{G'} < \card{G}$, Fixer has a winning strategy on $G'$ with lists $L'$.  But this strategy wins on $G$ with $L$ as well since $N_B(S)$ is colored with $S$, a contradiction. 
	
	\noindent\textbf{Claim 5.  }\textit{We have $\card{C} \le \card{N_B(C)}$ for $C \subseteq Y$. In particular, $\eta_L(G-t) = \size{G-t}$ and $F \ne \emptyset$.}
	
	Suppose not and choose $C \subseteq Y$ such that $\card{C} > \card{N_B(C)}$ so as to minimize $\card{C}$.  For all $\tau \in C$, by minimality of $\card{C}$, we have $N_B(C - \tau) = N_B(C)$.  Since $\card{N_B(C')} \ge \card{C}$ for every $C' \subseteq C - \tau$, Hall's theorem gives a nonempty matching $M_\tau$ whose vertex set is $(C - \tau) \cup N_B(C-\tau) = (C - \tau) \cup N_B(C)$.  So, for every $\tau \in C$, we can color $N_B(C - \tau)$ using $C - \tau$ as in Claim 4; the key point is that each of these colorings colors the same edge set.
	
	Put $R = C \cap L(t)$.  By Claim 4, $R \ne \emptyset$.  For $\tau \in R$, we have $\card{C - \tau} \ge \card{N_B(C - \tau)}$, so Claim 4 gives $\card{R} \ge 2$. 
	
	First, suppose $rs \in N_B(C)$. Pick $\tau \in R \cap L(s)$ if possible; otherwise pick $\tau \in R$ arbitrarily. For each $\set{uw, \alpha} \in M_\tau$, color $uw$ with $\alpha$.  Put $G' = G - V(N_B(C) - r)$ and $L'(v) = L(v) - (C - \tau)$ for $v \in V(G')$.  We claim that $(G', L')$ is superabundant.  Suppose otherwise that we have $H \subseteq G'$ such that $(H, L')$ is not abundant.  Since $rs$ got colored, the only subgraph to worry about is $H = G[s,t]$.  If $\tau \in L(s)$, then $\tau \in L'(s) \cap L'(t)$, so we are good.  Otherwise, $R \cap L(s) = \emptyset$, so there must be some color $\delta \in L(s) \cap L(t) - C$; in particular, $\delta \in L'(s) \cap L'(t)$.  Since $(G', L')$ satisfies the hypotheses of the lemma and $\card{G'} < \card{G}$, Fixer has a winning strategy by minimality of $\card{G}$.  But this strategy wins on $G$ with $L$ as well since $N_B(C - \tau)$ is colored with $C - \tau$, a contradiction. 
	
	Hence, we may assume that $rs \not \in N_B(C)$. So, $R \cap L(s) = \emptyset$. Pick $\tau \in R$. For each $\set{uw, \alpha} \in M_\tau$, color $uw$ with $\alpha$.  Put $G' = G - V(N_B(C) - r)$ and $L'(v) = L(v) - (C - \tau)$ for $v \in V(G')$.  We claim that $(G', L')$ is superabundant.  Suppose otherwise that we have $H \subseteq G'$ such that $(H, L')$ is not abundant. Since $\tau \not \in L(s$, we must have $r, t \in V(H)$.  Now $V(H_\tau - t) = \set{r}$ since $N_B(\tau) \subseteq N_B(C)$.  So, when we add $t$ back in, $\tau$ contributes one to $\psi_{L'}(H)$.  But $(H - t, L')$ is abundant, so $(H, L')$ is abundant, a contradiction.  Since $(G', L')$ satisfies the hypotheses of the lemma and $\card{G'} < \card{G}$, Fixer has a winning strategy by minimality of $\card{G}$. But this strategy wins on $G$ with $L$ as well since $N_B(C - \tau)$ is colored with $C - \tau$, a contradiction. 
	
	\noindent\textbf{Claim 6.  }\textit{$G-t$ has an $L$-edge-coloring $\pi$.}
	
	By Claim 5 and Hall's theorem, $B$ has a perfect matching which gives an $L$-edge-coloring of $G-t$.
	
	\noindent\textbf{Claim 7.  }\textit{There is a color $\beta$ such that $L(s) \cap L(t) = \set{\beta}$.  Moreover, $\beta \in L(r)$.}
	
	Otherwise, we $L$-edge-color $G-t$ by Claim 6 and then use one of the two colors in $L(s) \cap L(t)$ to color $st$, a contradiction.
	
	\noindent\textbf{Claim 8.  }\textit{We have $F \cap L(t) = \emptyset$.}
	
	Suppose otherwise that there is $\gamma \in F \cap L(t)$.  Color the edges of $G - s - t$ via $\pi$ and let $L'$ be the resulting list assignment on $rst$.  Then $\beta \in L'(r) \cap L'(s) \cap L'(t)$ and $\gamma \in 'L(r) \cap L'(t)$.  Hence $(rst, L')$ is superabundant and hence Fixer has a winning strategy on $rst$ with $L'$ by Theorem \ref{HallGame}.  But then Lemma \ref{CanColorAndPlayOnRest} shows that Fixer has a winning strategy on $G$ with $L$, a contradiction.
	
	\noindent\textbf{Claim 9.  }\textit{We have $L(r) \cap L(s) = \set{\beta}$.}
	
	Suppose not and pick $\tau \in L(r) \cap L(s) - \beta$. Pick $\gamma \in F$. 
	
	Suppose $\tau$ appears in more than three lists.  Then Fixer swaps $\gamma$ for $\tau$ in $L(s)$.  If Breaker does nothing, then Fixer colors $G - s - t$ from $\pi$, colors $rs$ with $\gamma$ and $st$ with $\beta$ to win.  Hence Breaker must swap $\gamma$ for $\tau$ at some $v \in N(r) - s$.  But $\tau \in L(w)$ for some $w \in N(r) - s - v$, so $\eta_L(G-t)$ has increased, a contradiction. 
	
	Suppose $\beta$ appears in more than three lists. Then Fixer swaps $\tau$ for $\beta$ in $L(t)$.  If Breaker does nothing or swaps $\tau$ for $\beta$ somewhere, then Fixer can finish the coloring.  Hence Breaker must swap $\beta$ for $\tau$ at some $v \in N(r) - s$.  But now applying the previous paragraph with the roles of $\beta$ and $\tau$ reversed gives a contradiction.
	
	Therefore, $\beta$ and $\tau$ each contribute only one to $\psi_L(G)$. Since $\eta_L(G-t) = \size{G-t}$ by Claim 5, applying Claim 1 and Claim 2 shows that there must be $\delta \in L(t) - L(s)$ such that $\card{G_\delta - t}$ is odd.  But now by the same argument as in Subclaim 3f, Fixer can achieve a position contradicting Claim 8.
	
	\noindent\textbf{Claim 10.  }\textit{Fixer wins.}
	
	Pick $\gamma \in F$.  Since $L(r) \cap L(s) = \set{\beta}$ by Claim 9, there must be $\alpha \in L(r) \cap L(t)$ since $rst$ is abundant and $L(s) \cap L(t) = \set{\beta}$ by Claim 7.
	
	Pick $\tau \in L(s) - \beta$.  Suppose $\tau$ appears in more than one list.  Then Fixer should swap $\gamma$ for $\tau$ in $L(s)$.  Then $\eta_L(G - t)$ has increased unless Breaker swaps $\tau$ for $\gamma$ in $L(r)$.  But since $\tau$ is in more than one list, it must be in $L(v)$ for some $v \in N(r) - s$ and $\eta_L(G - t)$ has increased as well, a contradiction.
	
	Suppose $\beta$ appears in more than three lists.  Then Fixer should swap $\tau$ for $\beta$ in $L(t)$.  If Breaker does nothing or swaps $\tau$ for $\beta$ in any list other than $L(r)$, then Fixer can complete the coloring.  So Fixer must swap $\tau$ for $\beta$ in $L(r)$.  But $\beta \in L(v)$ for some $v \in N(r) - s$, so applying the previous paragraph with the roles of $\beta$ and $\tau$ reversed gives a contradiction.
	
	Suppose $\alpha$ appears in more than three lists.  Then Fixer should swap $\gamma$ for $\alpha$ in $L(t)$. Now Breaker must pass lest he increase $\eta_L(G - t)$.  But this is a position contradicting Claim 8.
	
	Therefore, $\beta$ and $\alpha$ each contribute only one to $\psi_L(G)$. Since $\eta_L(G-t) = \size{G-t}$ by Claim 5, applying Claim 1 and Claim 2 shows that there must be $\delta \in L(t) - L(s)$ such that $\card{G_\delta - t}$ is odd.  But now by the same argument as in Subclaim 3f, Fixer can achieve a position contradicting Claim 8.
\end{proof}

\bibliographystyle{amsplain}
\bibliography{GraphColoring}

\end{document}
