\documentclass{article}
\usepackage{amssymb}
\def\labelenumi{(\arabic{enumi})}
\def\chil{{\chi_\ell}}
\def\chiol{\chi_{\rm{OL}}}
\newcommand{\ch}{\textrm{ch}}
\newcommand{\IN}{\mathbb{N}}
\newcommand{\IR}{\mathbb{R}}
\begin{document}
Please thank the referees for reading our paper carefully, and suggesting a number
of improvements.  We have implemented nearly all of their suggestions.
Below are our detailed responses.  (We first address the more detailed reported.)

\setcounter{section}{-1}
\section{Abstract}
\begin{enumerate}
\item Changed the sentence to read ``The problem of determining
the minimum number of edges in a $k$-critical graph with $n$ vertices has been
widely studied\ldots''
\item Changed the sentence to read ``In this paper, we improve the best
known lower bound on the number of edges in a $k$-list-critical graph.''
\item Added the following sentence: `` In fact,
our result on $k$-list-critical graphs is derived from a lower bound on the
number of edges in a graph that is critical with respect to its Alon--Tarsi
number.''
\end{enumerate}
\section{Introduction}
\begin{enumerate}
\item Changed capitalization as requested.
\item Added citations to Brooks, Dirac, and Gallai.
\item Added missing parentheses.
\item Changed ``Alon-Tarsi'' to ``Alon--Tarsi''.
\item Changed the sentence to ``In contrast,
Krivelevich's proof [14] does not easily translate to list
coloring, since it uses a lemma of Stiebitz [19], which says that
in a color-critical graph, the subgraph induced by vertices of degree at least
$k$ has no more components than the subgraph induced by vertices of degree
$k-1$ (but no analogous lemma is known for list coloring).''
\item Changed the sentence to read ``The discharging argument is more
intuitive and we believe that it may be easier to modify in the future for use
with new reducibility lemmas.''
\item Changed ``\ldots$f$-assignment such that $f(v)=k$\ldots'' to
``\ldots$f$-assignment where $f(v)=k$\ldots'' 
\item We kept this notation as $d_0$-assignment to remain with consistent in
previous papers.  (We avoid $d$-assignment to avoid confusion with
$k$-assignment, where $f(v)=k$ for all $v\in V(G)$.)
\item We added the definition of $d_G(v)$ and $d(v)$ at the start of Section~2.
\item Changed ``called Gallai Trees'' to ``known as \emph{Gallai Trees}''.
\item Now, just after Lemma 1.2, we write: ``
From the definitions and Lemma 1.2, it is easy to check that always
$\chi(G)\le \chil(G)\le \chiol(G)\le AT(G)\le \Delta(G)+1$.''
\item
\item The first time we mention the discharging method, we added a citation of
the paper by Cranston and West.
\end{enumerate}
\section{Gallai's bound via discharging}
\begin{enumerate}
\item We have kept the (repeated) definition of Gallai Tree.  Our motivation is
to gather all of the definitions in one place for easy reference.  (If readers
read papers from start to end at one sitting this would be less necessary.  But
that is not the case.)
\item Changed ``\ldots precisely the $d_0$-choosable\ldots'' to ``\ldots
precisely the non-$d_0$-choosable\ldots''
\item Changed ``\ldots set of all Gallai Trees of degree\ldots'' to
``\ldots set of all Gallai Trees of maximum degree\ldots'' 
\item Added the following to the proof of Theorem 2.1:
(Suppose there exists $v\in V(G)$ with $d(v)\le k-2$. Since $G$ is
$k$-AT-critical, $G-v$ has an orientation $D'$ with $d^+_{D'}(w)<k$ for all
$w\in V(G)-v$ and $EE(D')\ne EO(D')$.  Orienting all edges incident to $v$ away
from $v$ gives an orientation $D$ of $G$ also with $d^+_D(w)<k$ for all $w\in
V(G)$ and $EE(D)\ne EO(D)$.  So, we assume that $\delta(G)\ge k-1$.)
\item Added to the proof of Theorem 2.1 the sentence ``Note that the sum of
hte charge assigned to the vertices is preserved by these operations and equals
the sum of the degrees. So\ldots''
\item Changed $ch^*(v)$ to $\ch^*(v)$.
\item Removed ``instead'' from the start of the sentence.
\item Changed previous wording to ``The total charge received by vertices of
$T$ from the $k^+$-vertices is precisely''
\item Defined $d_T(v)$ at start of Section~2.
\item Added the following to the proof of Theorem 2.1:
Recall from the introduction that a connected graph is not
$d_0$-choosable precisely when it is a Gallai Tree.  So $T$ must be a Gallai
Tree; otherwise, we can $(k-1)$-color $G-T$ (since $G$ is $k$-critical), and
extend the coloring to $T$, since $T$ is $d_0$-choosable.
\item Added citation to Gallai paper for Lemma 2.2.
\end{enumerate}
\section{A refined bound on $\|T\|$}
\begin{enumerate}
\item Changed (2) to (ii) in last sentence of first paragraph on page 6.
\item Changed 1., 2., 3., and 4. in statement of Lemma 3.1 to (1), (2), (3), and
(4).
\item The first line of Lemma 3.1 already states: ``Let $p:\IN\to\IR$,
$f:\IN\to\IR$.  For all $k\ge 5\ldots$''
\item Changed $t\in\{2,k-2\}$ to $t\in\{2,\ldots,k-2\}$.
\item Changed start of second paragraph in proof of Lemma 3.1 to read as
follows.  ``Suppose the lemma is false and choose a counterexample $T$ minimizing $|T|$. 
If $T$ is $K_t$ for some $t \in \{2,\ldots,k-2\}$, then $t(t-1) > (k-3 + p(k))t +
f(k)$.  After substituting $p(k)\ge \frac{-f(k)}{k-2}$ from (1), this
implies that $-t(k-2)(k-2-t)>(k-2-t)f(k)$.  When $t=k-2$, both sides are 0,
which is a contradiction.  Otherwise, dividing by $k-2-t$ gives that
$-t(k-2)>f(k)$, which contradicts (3).''
	
\item Revised paragraph in proof of Lemma 3.1 beginning ``To handle\ldots'' to
the following.
``Suppose next that $B$ is $K_{k-2}$ and notice that, since $T$ has maximum degree
at most $k-1$ and $T\ne K_{k-2}$, this implies that $d_T(x_B)$ is either $k-2$
or $k-1$.  In the former case, we set $D=B$ and obtain.
	\[(k-2)(k-3) +2 > (k-3 + p(k))(k-2),\]
	contradicting (4).''

\item Changed 1., 2.,\ldots in statement of Lemma 3.1 to (1), (2), \ldots
\end{enumerate}
\section{Discharging}
\begin{enumerate}
\item Changed the sentence in the first paragraph of the discharging section to
the following.  ``It is helpful to view our proof here as a refinement and
strengthening of the proof of Gallai's bound in Section~\ref{sec:gallai},
Theorem~\ref{thm:Gallai}.''  (The point of using the term ``Gallai's bound'' is
that it is likely easier for the reader to know what is meant, without looking
back to the earlier section of the paper.)
\item The term $(k-1)$-neighbor is defined at the start of Section~2.
\item On page 11 (in the paragraph starting ``First, note\ldots''), changed $Q$
to be italicized.
\end{enumerate}
\section{Reducible Configurations}
\section{References}
\begin{enumerate}
\item Added the missing accent to Kr\'{a}l'.
\end{enumerate}

\section*{Second Referee}
\begin{itemize}
\item p1: Dirac proved that every $k$-critical graph other than $K_k$ [...]

Changed as requested.
\item p2.14:  I do not think the term Ore degree is established. Maybe use "degree sums"?

We added the following footnote: The \emph{Ore degree} of an edge $xy$ is the sum
$d(x)+d(y)$.  The Ore degree of a graph is the maximum Ore degree of its edges.
\item p2.15: use "the" Alon-Tarsi number through out.

Changed as requested.

\item p3: Why do you mention the bounds for $k$-critcal graphs, $k\le10$ from [8]?
They are worse than previous bounds, and somewhat confusing in the table.

The other obvious option would be to use a `---'.  However, our interpretation
of a `---' (if we were unaware of the history of this problem) would be that [8]
had not proved any bounds for these values of $d(G)$, which is incorrect.  Thus,
we include them.

\item p5: Include a short argument that $T'$ is in $T_k$ in both cases, it tripped me up for a minute.
\item p6: Just write "The proof mirrors that of [...]". No need to say that this sentence is the proof outline.

Changed as requested.
\item p7.13: "require" is maybe a bit too strong here. You don't give an argument (and you shouldn't) that (3), (4) and (5) are absolutely necessary.

Changed ``require" to ``rely on''
\item p7.18: delete "need to"

Changed as requested.
\item p7.23: each $\to$ every

Changed as requested.
\item p9.1: see $\to$ study ?

Changed as requested.
\item p9.7: How is this "instead"? 

Here we are first choosing the value of $p(k)$ (which then constrains the value
of $h(k)$, instead of first choosing the value of $h(k)$ (which then constrains
the value of $p(k)$).

Alternately, the first paragraph assigns one pair of values to $p(k)$ and
$h(k)$.  Instead, the second paragraph assigns them a different pair of values.

\item p9: can you make the formulas in Cor3.3 and Lem3.4 look even more similar by multiplying out some factors?

\item p10.22: Should it be $k$-list-critical here?

Changed to $k$-AT-irreducible.

\item p10.27: the proof $\to$ our proof ?

Changed as requested.
\item p11.5: This sounds like your believe is the unfortunate part...

Changed to the following.
``However, we believe this is false. Fortunately, something similar is true.''

\item p15: I am sure you have discussed this, but in my opinion you should switch sections 4 and 5. You frequently refer to section 5 in section 4, and it would be comforting to know at that point that all results in section 5 were established before elsewhere.

We do not prove anything in Section~5.  We merely record (for reference) three
lemmas from [9].  We choose to put that section last, because we want to
emphasize the discharging, which we see as a primary contribution of the paper.
\end{itemize}
\end{document}
