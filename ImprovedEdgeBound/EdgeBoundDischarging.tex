\documentclass[12pt]{article}
\usepackage{amsmath, amsthm, amssymb}
\usepackage{hyperref}
\usepackage{verbatim}
\usepackage[top=1.0in, bottom=1.0in, left=1.0in, right=1.0in]{geometry}
\usepackage{color}
\pagestyle{plain}

\usepackage{sectsty}
\allsectionsfont{\sffamily}

\setcounter{secnumdepth}{5}
\setcounter{tocdepth}{5}

\makeatletter
\newtheorem*{rep@theorem}{\rep@title}
\newcommand{\newreptheorem}[2]{
\newenvironment{rep#1}[1]{
 \def\rep@title{#2 \ref{##1}}
 \begin{rep@theorem}}
 {\end{rep@theorem}}}
\makeatother

\theoremstyle{plain}
\newtheorem{thm}{Theorem}[section]
\newreptheorem{thm}{Theorem}
\newtheorem{prop}[thm]{Proposition}
\newreptheorem{prop}{Proposition}
\newtheorem{lem}[thm]{Lemma}
\newreptheorem{lem}{Lemma}
\newtheorem{conjecture}[thm]{Conjecture}
\newreptheorem{conjecture}{Conjecture}
\newtheorem{cor}[thm]{Corollary}
\newreptheorem{cor}{Corollary}
\newtheorem{prob}[thm]{Problem}

\newtheorem*{KernelLemma}{Kernel Lemma}
\newtheorem*{BK}{Borodin-Kostochka Conjecture}
\newtheorem*{BK2}{Borodin-Kostochka Conjecture (restated)}
\newtheorem*{Reed}{Reed's Conjecture}
\newtheorem*{ClassificationOfd0}{Classification of $d_0$-choosable graphs}


\theoremstyle{definition}
\newtheorem{defn}{Definition}
\theoremstyle{remark}
\newtheorem*{remark}{Remark}
\newtheorem*{problem}{Problem}
\newtheorem{example}{Example}
\newtheorem*{question}{Question}
\newtheorem*{observation}{Observation}

\newcommand{\fancy}[1]{\mathcal{#1}}
\newcommand{\C}[1]{\fancy{C}_{#1}}


\newcommand{\IN}{\mathbb{N}}
\newcommand{\IR}{\mathbb{R}}
\newcommand{\G}{\fancy{G}}
\newcommand{\CC}{\fancy{C}}
\newcommand{\D}{\fancy{D}}
\newcommand{\T}{\fancy{T}}
\newcommand{\B}{\fancy{B}}
\renewcommand{\L}{\fancy{L}}
\newcommand{\HH}{\fancy{H}}

\newcommand{\inj}{\hookrightarrow}
\newcommand{\surj}{\twoheadrightarrow}

\newcommand{\set}[1]{\left\{ #1 \right\}}
\newcommand{\setb}[3]{\left\{ #1 \in #2 : #3 \right\}}
\newcommand{\setbs}[2]{\left\{ #1 : #2 \right\}}
\newcommand{\card}[1]{\left|#1\right|}
\newcommand{\size}[1]{\left\Vert#1\right\Vert}
\newcommand{\ceil}[1]{\left\lceil#1\right\rceil}
\newcommand{\floor}[1]{\left\lfloor#1\right\rfloor}
\newcommand{\func}[3]{#1\colon #2 \rightarrow #3}
\newcommand{\funcinj}[3]{#1\colon #2 \inj #3}
\newcommand{\funcsurj}[3]{#1\colon #2 \surj #3}
\newcommand{\irange}[1]{\left[#1\right]}
\newcommand{\join}[2]{#1 \mbox{\hspace{2 pt}$\ast$\hspace{2 pt}} #2}
\newcommand{\djunion}[2]{#1 \mbox{\hspace{2 pt}$+$\hspace{2 pt}} #2}
\newcommand{\parens}[1]{\left( #1 \right)}
\newcommand{\brackets}[1]{\left[ #1 \right]}
\newcommand{\DefinedAs}{\mathrel{\mathop:}=}

\newcommand{\mic}{\operatorname{mic}}
\newcommand{\AT}{\operatorname{AT}}
\newcommand{\col}{\operatorname{col}}
\newcommand{\ch}{\operatorname{ch}}
\newcommand{\type}{\operatorname{type}}
\newcommand{\nonsep}{\bar{S}}

\def\adj{\leftrightarrow}
\def\nonadj{\not\!\leftrightarrow}

\newcommand\restr[2]{{% we make the whole thing an ordinary symbol
  \left.\kern-\nulldelimiterspace % automatically resize the bar with \right
  #1 % the function
  \vphantom{\big|} % pretend it's a little taller at normal size
  \right|_{#2} % this is the delimiter
  }}

\def\D{\fancy{D}}
\def\C{\fancy{C}}
\def\A{\fancy{A}}

\newcommand{\case}[2]{{\bf Case #1.}~{\it #2}~~}

\title{Edge lower bounds via discharging notes}

\begin{document}
\maketitle

\section{Introduction}
For a graph $G$, let $d(G)$ be the average degree of $G$. Let $\T_k$ be the Gallai trees with maximum degree at most $k-1$, excepting $K_k$. 

\section{Gallai's bound via discharging}

\begin{thm}[Gallai]
	For $k \ge 4$ and $G \ne K_k$ a $k$-AT-critical graph, we have
	\[d(G) < k-1 + \frac{k-3}{k^2-3}.\]
\end{thm}
\begin{proof}
	Start with initial charge function $\ch(v) = d_G(v)$.  Have each $k^+$-vertex give charge $\frac{k-1}{k^2-3}$ to each of its $(k-1)$-neighbors.  Then let the vertices in each component of the low vertex subgraph share their total charge equally.  Let $\ch^*(v)$ be the resulting charge function.  We finish the proof by showing that $\ch^*(v) \ge k-1 + \frac{k-3}{k^2-3}$ for all $v \in V(G)$.
	
	If $v$ is a $k^+$-vertex, then $ch^*(v) \ge d_G(v) - \frac{k-1}{k^2-3}d_G(v) = \parens{1- \frac{k-1}{k^2-3}}d_G(v) \ge \parens{1- \frac{k-1}{k^2-3}}k = k-1 + \frac{k-3}{k^2-3}$ as desired.

	Let $T$ be a component of the low vertex subgraph.  Then the vertices in $T$ receive total charge
	\[\frac{k-1}{k^2-3}\sum_{v \in V(T)} k-1 - d_G(v) = \frac{k-1}{k^2-3}\parens{(k-1)|T| - 2\size{T}}.\]
	So, after distributing this charge out equally, each vertex in $T$ receives charge
	\[\frac{1}{|T|}\frac{k-1}{k^2-3}((k-1)|T| - 2\size{T}) = \frac{k-1}{k^2-3}\parens{(k-1) - d(T)}.\]
	By Lemma \ref{BasicGallaiTreeBound}, this is at least
	\[\frac{k-1}{k^2-3}\parens{(k-1) - \parens{k-2 + \frac{2}{k-1}}} = \frac{k-1}{k^2-3}\parens{\frac{k-3}{k-1}} = \frac{k-3}{k^2-3}.\]
	Hence each low vertex ends with charge at least $k-1 + \frac{k-3}{k^2-3}$ as desired.
\end{proof}


\begin{lem}[Gallai]\label{BasicGallaiTreeBound}
	For $k \ge 4$ and $T \in \T_k$, we have $d(T) < k-2 + \frac{2}{k-1}$.
\end{lem}
\begin{proof}
	Suppose not and choose a counterexample $T$ minimizing $|T|$.  Then $T$ has at least two blocks.  Let $B$ be an endblock of $T$.  If $B$ is $K_t$ for $2 \le t \le k-2$, then remove the non-separating vertices of $B$ from $T$ to get $T'$.  By minimality of $|T|$, we have 
	\[2\size{T} - t(t-1) = 2\size{T'} < \parens{k-2 + \frac{2}{k-1}}|T'| = \parens{k-2 + \frac{2}{k-1}}|T| - \parens{k-2 + \frac{2}{k-1}}(t-1).\]
	Hence we have the contradiction
	\[2\size{T} < \parens{k-2 + \frac{2}{k-1}}|T| + (t+2 -k - \frac{2}{k-1})(t-1) \le \parens{k-2 + \frac{2}{k-1}}|T|.\]
	
	The case when $B$ is an odd cycle is the same as the above, a longer cycle just makes things better.  Finally, if $B = K_{k-1}$, remove all vertices of $B$ from $T$ to get $T'$. By minimality of $|T|$, we have 
	\begin{align*}
	  2\size{T} - (k-1)(k-2) - 2 &= 2\size{T'}\\
	  &< \parens{k-2 + \frac{2}{k-1}}|T'|\\
	  &= \parens{k-2 + \frac{2}{k-1}}|T| - \parens{k-2 + \frac{2}{k-1}}(k-1).
	\end{align*}

	Hence $2\size{T} < \parens{k-2 + \frac{2}{k-1}}|T|$, a contradiction.
\end{proof}

\section{An initial improved bound}
Lemma \ref{BasicGallaiTreeBound} is best possible as can be seen by the family of graphs with blocks on a path alternating $K_{k-1}$ and $K_2$.  But we have reducible configurations (see the last section for the precise statements) that place restrictions on $K_{k-1}$ blocks. To state these restrictions, we need the following auxiliary bipartite graph. 

For a $k$-AT-critical graph $G$, let $\L(G)$ be the subgraph of $G$ induced on the $(k-1)$-vertices and $\HH(G)$ the subgraph of $G$ induced on the $k$-vertices.   For $T \in \T_k$, let $W^k(T)$ be the set of vertices of $T$ that are contained in some $K_{k-1}$ in $T$.  Let $\B_k(G)$ be the bipartite graph with one part $V(\HH(G))$ and the other part the components of $\L(G)$.  Put an edge between $y \in V(\HH(G))$ and a component $T$ of $\L(G)$ if and only if $N(y) \cap W^k(T) \ne \emptyset$.  Then Lemma \ref{MultipleHighConfigurationEuler} says that $\B_k(G)$ is $2$-degenerate.

We can use this fact to refine our discharging argument.  Let $\epsilon$ and $\gamma$ be parameters that we will determine where $\epsilon \le \gamma < 2\epsilon$. Start with initial charge function $\ch(v) = d_G(v)$.  
\begin{enumerate}
	\item Each $k^+$-vertex gives charge $\epsilon$ to each of its $(k-1)$-neighbors not in a $K_{k-1}$,
	\item Each $(k+1)^+$-vertex give charge $\gamma$ to each of its $(k-1)$-neighbors in a $K_{k-1}$,
	\item Let $Q = \B_k(G)$.  Repeat the following steps until $Q$ is empty.
	  \begin{enumerate}
	  	\item Remove all components $T$ of $\L(G)$ in $Q$ that have degree at most two in $Q$.
	  	\item Pick $v \in V(\HH(G)) \cap V(Q)$.  Send charge $\gamma$ from $v$ to each $x \in N_G(v) \cap W^k(T)$ for each component $T$ of $\L(G)$ where $vT \in E(Q)$.
	  \end{enumerate}
	\item The vertices in each component of the low vertex subgraph share their total charge equally.
\end{enumerate}
Let $\ch^*(v)$ be the resulting charge function.

\section{Reducible Configurations}
\begin{defn}
	A graph $G$ is \emph{AT-reducible} to $H$ if $H$ is a nonempty induced subgraph of $G$ which is $f_H$-AT where $f_H(v) \DefinedAs \delta(G) + d_H(v) - d_G(v)$ for all $v \in V(H)$.  
	If $G$ is not AT-reducible to any nonempty induced subgraph, then it is \emph{AT-irreducible}.
\end{defn}

This first lemma tells us how a single high vertex can interact with the low vertex subgraph.  This is the version Hal and i used, it (and more) follows from the classification in ``mostlow''.

\begin{lem}\label{ConfigurationTypeOneEuler}
Let $k \ge 5$ and let $G$ be a graph with $x \in V(G)$ such that:
\begin{enumerate}
\item $K_k \not \subseteq G$; and
\item $G-x$ has $t$ components $H_1, H_2, \ldots, H_t$, and all are in $\T_k$; and
\item $d_G(v) \leq k - 1$ for all $v \in V(G-x)$; and
\item $\card{N(x) \cap W^k(H_i)} \ge 1$ for $i \in \irange{t}$; and
\item $d_G(x) \ge t+2$.
\end{enumerate}

\noindent Then $G$ is $f$-AT where $f(x) = d_G(x) - 1$ and $f(v) = d_G(v)$ for all $v \in V(G - x)$.
\end{lem}

To deal with more than one high vertex we need the following auxiliary bipartite graph.  For a graph $G$, $\set{X, Y}$ a partition of $V(G)$ and $k \ge 4$, let $\B_k(X, Y)$ be the bipartite graph with one part $Y$ and the other part the components of $G[X]$.  Put an edge between $y \in Y$ and a component $T$ of $G[X]$ if and only if $N(y) \cap W^k(T) \ne \emptyset$.   The next lemma tells us that we have a reducible configuration if this bipartite graph has minimum degree at least three.  

\begin{lem}
	\label{MultipleHighConfigurationEuler} Let $k\ge7$ and let $G$ be a graph with
	$Y\subseteq V(G)$ such that: 
	\begin{enumerate}
		\item $K_{k}\not\subseteq G$; and 
		\item the components of $G-Y$ are in $\T_{k}$; and 
		\item $d_{G}(v)\leq k-1$ for all $v\in V(G-Y)$; and 
		\item with $\B\DefinedAs\B_{k}(V(G-Y),Y)$ we have $\delta(\B)\ge3$. 
	\end{enumerate}
	\noindent Then $G$ has an induced subgraph $G'$ that is $f$-AT where $f(y)=d_{G'}(y)-1$
	for $y\in Y$ and $f(v)=d_{G'}(v)$ for all $v\in V(G'-Y)$.\end{lem}

We also have the following version with asymmetric degree condition on $\B$.  The point here is that this works for $k \ge 5$.  As we'll see in the next section, the consequence is that we trade a bit in our size bound for the proof to go through with $k \in \set{5,6}$.

\begin{lem}
	\label{MultipleHighConfigurationEulerLopsided} Let $k \ge 5$ and let $G$ be a graph with
	$Y\subseteq V(G)$ such that: 
	\begin{enumerate}
		\item $K_{k}\not\subseteq G$; and 
		\item the components of $G-Y$ are in $\T_{k}$; and 
		\item $d_{G}(v)\leq k-1$ for all $v\in V(G-Y)$; and 
		\item with $\B \DefinedAs \B_k(V(G-Y), Y)$ we have $d_{\B}(y) \ge 4$ for all $y \in Y$ and $d_{\B}(T) \ge 2$ for all components $T$ of $G-Y$.
	\end{enumerate}
	\noindent Then $G$ has an induced subgraph $G'$ that is $f$-AT where $f(y)=d_{G'}(y)-1$
	for $y\in Y$ and $f(v)=d_{G'}(v)$ for all $v\in V(G'-Y)$.\end{lem}



\bibliographystyle{amsplain}
\bibliography{GraphColoring1}
\end{document}

 
