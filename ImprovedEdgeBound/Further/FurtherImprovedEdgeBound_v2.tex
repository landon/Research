\documentclass[12pt]{article}
\usepackage{amsmath, amsthm, amssymb}
\usepackage{hyperref}
\usepackage{verbatim}
\usepackage[top=1.0in, bottom=1.0in, left=1.0in, right=1.0in]{geometry}
\usepackage{color}
\pagestyle{plain}

\makeatletter
\newtheorem*{rep@theorem}{\rep@title}
\newcommand{\newreptheorem}[2]{
\newenvironment{rep#1}[1]{
 \def\rep@title{#2 \ref{##1}}
 \begin{rep@theorem}}
 {\end{rep@theorem}}}
\makeatother

\theoremstyle{plain}
\newtheorem{thm}{Theorem}[section]
\newtheorem*{KernelMagic}{Kernel Magic}
\newtheorem*{MainTheorem}{Main Theorem}
\newreptheorem{thm}{Theorem}
\newtheorem{prop}[thm]{Proposition}
\newreptheorem{prop}{Proposition}
\newtheorem{lem}[thm]{Lemma}
\newreptheorem{lem}{Lemma}
\newtheorem{conjecture}[thm]{Conjecture}
\newreptheorem{conjecture}{Conjecture}
\newtheorem{cor}[thm]{Corollary}
\newreptheorem{cor}{Corollary}
\newtheorem{prob}[thm]{Problem}

\newtheorem*{KernelLemma}{Kernel Lemma}

\theoremstyle{definition}
\newtheorem{defn}{Definition}
\newtheorem*{TheDefinition}{Definition}
\theoremstyle{remark}
\newtheorem*{remark}{Remark}
\newtheorem*{problem}{Problem}
\newtheorem{example}{Example}
\newtheorem*{question}{Question}
\newtheorem*{observation}{Observation}

\newcommand{\fancy}[1]{\mathcal{#1}}
\newcommand{\C}[1]{\fancy{C}_{#1}}


\newcommand{\IN}{\mathbb{N}}
\newcommand{\IR}{\mathbb{R}}
\newcommand{\G}{\fancy{G}}
\newcommand{\CC}{\fancy{C}}
\newcommand{\D}{\fancy{D}}
\newcommand{\T}{\fancy{T}}
\newcommand{\B}{\fancy{B}}
\renewcommand{\L}{\fancy{L}}
\newcommand{\HH}{\fancy{H}}

\newcommand{\inj}{\hookrightarrow}
\newcommand{\surj}{\twoheadrightarrow}

\newcommand{\set}[1]{\left\{ #1 \right\}}
\newcommand{\setb}[3]{\left\{ #1 \in #2 : #3 \right\}}
\newcommand{\setbs}[2]{\left\{ #1 : #2 \right\}}
\newcommand{\card}[1]{\left|#1\right|}
\newcommand{\size}[1]{\left\Vert#1\right\Vert}
\newcommand{\ceil}[1]{\left\lceil#1\right\rceil}
\newcommand{\floor}[1]{\left\lfloor#1\right\rfloor}
\newcommand{\func}[3]{#1\colon #2 \rightarrow #3}
\newcommand{\funcinj}[3]{#1\colon #2 \inj #3}
\newcommand{\funcsurj}[3]{#1\colon #2 \surj #3}
\newcommand{\irange}[1]{\left[#1\right]}
\newcommand{\join}[2]{#1 \mbox{\hspace{2 pt}$\ast$\hspace{2 pt}} #2}
\newcommand{\djunion}[2]{#1 \mbox{\hspace{2 pt}$+$\hspace{2 pt}} #2}
\newcommand{\parens}[1]{\left( #1 \right)}
\newcommand{\brackets}[1]{\left[ #1 \right]}
\newcommand{\DefinedAs}{\mathrel{\mathop:}=}

\newcommand{\mic}{\operatorname{mic}}
\newcommand{\AT}{\operatorname{AT}}
\newcommand{\col}{\operatorname{col}}
\newcommand{\ch}{\operatorname{ch}}
\newcommand{\type}{\operatorname{type}}
\newcommand{\nonsep}{\bar{S}}

\def\adj{\leftrightarrow}
\def\nonadj{\not\!\leftrightarrow}

\newcommand\restr[2]{{% we make the whole thing an ordinary symbol
  \left.\kern-\nulldelimiterspace % automatically resize the bar with \right
  #1 % the function
  \vphantom{\big|} % pretend it's a little taller at normal size
  \right|_{#2} % this is the delimiter
  }}

\def\D{\fancy{D}}
\def\C{\fancy{C}}
\def\A{\fancy{A}}

\newcommand{\case}[2]{{\bf Case #1.}~{\it #2}~~}
\newcommand\numberthis{\addtocounter{equation}{1}\tag{\theequation}}

\title{A better lower bound on average degree of $k$-list-critical graphs.}
\author{Landon Rabern}
\begin{document}
\maketitle

\section{Introduction}

\begin{MainTheorem}
For $k \ge 7$, every non-complete $k$-list-critical graph has average degree at least \[k-1 + \frac{(k-3)^2 (2 k-3)}{k^4-2 k^3-11 k^2+28 k-14}.\]
\end{MainTheorem}

\section{The Proof}
Let $q_k(G)$ be the number of non-cut vertices in $G$ that appear in copies of $K_{k-1}$.  Let $\beta_k(G)$ be the independence number of the subgraph of $G$ induced on the vertices of degree $k-1$.  
When $k$ is defined in context, we just write $q(G)$ and $\beta(G)$.

Sections 2 and 3 prove the following upper bounds on $q(\L)$ and $\beta(\L)$.
\begin{lem}\label{qLemmaList}
	Let $G$ be a non-complete $k$-list-critical graph where $k \ge 5$.  Let $\L$ be the subgraph of $G$ induced on $(k-1)$-vertices, $\HH^-$ the subgraph of $G$ induced on $k$-vertices, 
	$\HH$ the subgraph of $G$ induced on $k^+$-vertices, $\HH^+$ the subgraph of $G$ induced on $(k+1)^+$-vertices and $\D$ the components of $\L$ containing $K_{k-1}$.  Then
	\[q(\L) \le \card{\D} + 4\card{\HH^-} + \size{\HH^+, \L},\] and if $k \ge 7$, then
	\[q(\L) \le 2\card{\D} + 3\card{\HH^-} + \size{\HH^+, \L}.\]
\end{lem}

\begin{lem}\label{betaLemmaList}
	Let $G$ be a $k$-list-critical graph.  Let $\L$ be the subgraph of $G$ induced on $(k-1)$-vertices and	$\HH$ the subgraph of $G$ induced on $k^+$-vertices.  
	Then
	\[\beta(\L) \le \size{\HH} + \frac{2\size{G} - (k-2)\card{G} - \parens{k - \frac12}\card{\HH}}{k-1}.\]
\end{lem}

The final piece we need is a bound on the number of edges in Gallai trees, proved in Section 4.
\begin{lem}\label{GallaiTreeBound}
For all $k \ge 5$ and Gallai trees $T \ne K_k$ with $\Delta(T) \le k - 1$,
\[2\size{T} \le \parens{k-3 + \frac{3k-7}{k^2-4k+5}}\card{T} - \frac{2(k-1)(k-4)}{k^2-4k+5} + \frac{(k-1)(k-4)}{k^2-4k+5}q(T) + 2\beta(T)\]
\end{lem}
\bigskip
\begin{proof}[Proof of Main Theorem]
Let $\L$ be the subgraph of $G$ induced on $(k-1)$-vertices, $\HH^-$ the subgraph of $G$ induced on $k$-vertices, $\HH$ the subgraph of $G$ induced on $k^+$-vertices, $\HH^+$ the subgraph of $G$ induced on $(k+1)^+$-vertices and $\D$ the components of $\L$ containing $K_{k-1}$.
Plainly, the following bounds hold.
\begin{equation}\label{eq1}
2\size{G} \ge k\card{G} - \card{\L}
\end{equation}
\begin{equation}\label{eq2}
2\size{G} \ge (k+1)\card{G} - \card{\HH^-} - 2\card{\L}
\end{equation}
\begin{equation}\label{eq3}
2\size{G} \ge k\card{\HH^-} + (k-1)\card{\L} + \size{\HH^+, \L}
\end{equation}
\begin{equation}\label{eq4}
\size{\HH,\L} = (k-1)\card{\L} - 2\size{\L}
\end{equation}
Put
\begin{align*}
p(k) &\DefinedAs \frac{3k-7}{k^2-4k+5}\\
h(k) &\DefinedAs \frac{(k-1)(k-4)}{k^2-4k+5}\\
f(k) &\DefinedAs 2h(k)
\end{align*}
By Lemma \ref{GallaiTreeBound},
\begin{equation}\label{eq5}
2\size{\L} \le \parens{k-3 + p(k)}\card{\L} - f(k)c(\L) + h(k)q(\L) + 2\beta(\L)
\end{equation}
By Lemma \ref{qLemmaList},
\[q(\L) \le 2\card{\D} + 3\card{\HH^-} + \size{\HH^+, \L},\]
plugging this into \eqref{eq5} and using the fact that $c(\L) \ge \card{\D}$ gives
\begin{equation}\label{eq6}
2\size{\L} \le \parens{k-3 + p(k)}\card{\L} + 3h(k)\card{\HH^-} + h(k)\size{\HH^+, \L} + 2\beta(\L).
\end{equation}
Now using \eqref{eq1} and \eqref{eq6},
\begin{align*}
	2\size{G} &= 2\size{\HH} + 2\size{\HH, \L} + 2\size{\L}\\
	&= 2\size{\HH} + 2((k-1)\card{\L} - 2\size{\L}) + 2\size{\L}\\
	&= 2\size{\HH} + 2(k-1)\card{\L} - 2\size{\L}\\
	&\ge 2\size{\HH} + \parens{k+1 - p(k)}\card{\L} - 3h(k)\card{\HH^-} - h(k)\size{\HH^+, \L} - 2\beta(\L)\numberthis \label{eq7}\\
\end{align*}
Adding $h(k)$ times \eqref{eq3} to \eqref{eq7} gives
\begin{equation}\label{eq8}
\parens{1 + h(k)}(2\size{G}) \ge 2\size{\HH} + \parens{k+1 +(k-1)h(k)- p(k)}\card{\L} + (k- 3)h(k)\card{\HH^-}  - 2\beta(\L)
\end{equation}
Lemma \ref{betaLemmaList} gives
\[\beta(\L) \le \size{\HH} + \frac{2\size{G} - (k-2)\card{G} - \parens{k - \frac12}\card{\HH}}{k-1}.\]
Plugging this into \eqref{eq8} yields
\begin{equation}\label{eq9}
\begin{align*}
\parens{1 + h(k) + \frac{2}{k-1}}(2\size{G}) &\ge \parens{k+1 +(k-1)h(k)- p(k)}\card{\L}\\ 
&+ (k- 3)h(k)\card{\HH^-} +\frac{2(k-2)}{k-1}\card{G} + \frac{2k-1}{k-1}\card{\HH}.
\end{align*}
\end{equation}
Now using $\card{\HH} = \card{G} - \card{\L}$ gives
\begin{equation}\label{eq10}
\begin{align*}
\parens{1 + h(k) + \frac{2}{k-1}}(2\size{G}) &\ge \parens{k+1 +(k-1)h(k)- p(k) - \frac{2k-1}{k-1}}\card{\L}\\ 
&+ (k- 3)h(k)\card{\HH^-} +\frac{4k-5}{k-1}\card{G}.
\end{align*}
\end{equation}
Now using \eqref{eq2} to get a lower bound on $\card{\HH^-}$ gives
\begin{equation}\label{eq11}
\begin{align*}
\parens{1 + (k-2)h(k) + \frac{2}{k-1}}(2\size{G}) &\ge \parens{k+1 - (k-5)h(k)- p(k) - \frac{2k-1}{k-1}}\card{\L}\\ 
&+\parens{(k+1)(k-3)h(k) + \frac{4k-5}{k-1}}\card{G}.
\end{align*}
\end{equation}
Finally, using \eqref{eq1} to get a lower bound on $\card{\L}$ gives
\begin{equation}\label{eq12}
\frac{2\size{G}}{\card{G}} \ge \frac{k(k+1) + 3(k-1)h(k) - kp(k) + \frac{4k-5}{k-1} - \frac{k(2k-1)}{k-1}}{k+2 + 3h(k) - p(k) + \frac{2}{k-1}- \frac{2k-1}{k-1}}.
\end{equation}
Plugging in the values for $p(k)$ and $h(k)$ and simplifying proves the theorem.
\end{proof}

\section{Bounding $q(\L)$}
This section is devoted to extracting the reusable Lemma \ref{qLemma} from the proof of Kierstead and R. \cite{OreVizing}.

\begin{defn}
	A graph $G$ is \emph{AT-reducible} to $H$ if $H$ is a nonempty induced subgraph of $G$ which is $f_H$-AT where $f_H(v) \DefinedAs \delta(G) + d_H(v) - d_G(v)$ for all $v \in V(H)$.  
	If $G$ is not AT-reducible to any nonempty induced subgraph, then it is \emph{AT-irreducible}.
\end{defn}

\begin{lem}\label{qLemma}
	Let $G$ be a non-complete AT-irreducible graph with $\delta(G) = k-1$ where $k \ge 5$.  Let $\L$ be the subgraph of $G$ induced on $(k-1)$-vertices, $\HH^-$ the subgraph of $G$ induced on $k$-vertices, 
	$\HH$ the subgraph of $G$ induced on $k^+$-vertices, $\HH^+$ the subgraph of $G$ induced on $(k+1)^+$-vertices and $\D$ the components of $\L$ containing $K_{k-1}$.  Then
	\[q(\L) \le \card{\D} + 4\card{\HH^-} + \size{\HH^+, \L},\] and if $k \ge 7$, then
	\[q(\L) \le 2\card{\D} + 3\card{\HH^-} + \size{\HH^+, \L}.\]
\end{lem}

\begin{observation}
The hypotheses of Lemma \ref{qLemma} are satisfied by non-complete $k$-critical, $k$-list-critical, online $k$-list-critical and $k$-AT-critical graphs.
\end{observation}

The proof of Lemma \ref{qLemma} requires the following four lemmas from \cite{OreVizing}.

\begin{lem}\label{DegenerateEuler}
Let $G$ be a graph and $\func{f}{V(G)}{\IN}$.  If $\size{G} > \sum_{v \in V(G)} f(v)$, then $G$ has an induced subgraph $H$ such that $d_H(v) > f(v)$ for each $v \in V(H)$.
\end{lem}
\begin{proof}
Suppose not and choose a counterexample $G$ minimizing $\card{G}$. Then $\card{G} \ge 3$ and we have $x \in V(G)$ with $d_G(x) \leq f(x)$. But now $\size{G-x} > \sum_{v \in V(G-x)} f(v)$, contradicting minimality of $\card{G}$.
\end{proof}

Let $\T_k$ be the Gallai trees with maximum degree at most $k-1$, excepting $K_k$. For a graph $G$, let $W^k(G)$ be the set of vertices of $G$ that are contained in some $K_{k-1}$ in $G$.  

\begin{lem}\label{ConfigurationTypeOneEuler}
Let $k \ge 5$ and let $G$ be a graph with $x \in V(G)$ such that:
\begin{enumerate}
\item $K_k \not \subseteq G$; and
\item $G-x$ has $t$ components $H_1, H_2, \ldots, H_t$, and all are in $\T_k$; and
\item $d_G(v) \leq k - 1$ for all $v \in V(G-x)$; and
\item $\card{N(x) \cap W^k(H_i)} \ge 1$ for $i \in \irange{t}$; and
\item $d_G(x) \ge t+2$.
\end{enumerate}

\noindent Then $G$ is $f$-AT where $f(x) = d_G(x) - 1$ and $f(v) = d_G(v)$ for all $v \in V(G - x)$.
\end{lem}

For a graph $G$, $\set{X, Y}$ a partition of $V(G)$ and $k \ge 4$, let $\B_k(X, Y)$ be the bipartite graph with one part $Y$ and the other part the components of $G[X]$.  Put an edge between $y \in Y$ and a component $T$ of $G[X]$ iff $N(y) \cap W^k(T) \ne \emptyset$.   The next lemma tells us that we have a reducible configuration if this bipartite graph has minimum degree at least three.  

\begin{lem}
	\label{MultipleHighConfigurationEuler} Let $k\ge7$ and let $G$ be a graph with
	$Y\subseteq V(G)$ such that: 
	\begin{enumerate}
		\item $K_{k}\not\subseteq G$; and 
		\item the components of $G-Y$ are in $\T_{k}$; and 
		\item $d_{G}(v)\leq k-1$ for all $v\in V(G-Y)$; and 
		\item with $\B\DefinedAs\B_{k}(V(G-Y),Y)$ we have $\delta(\B)\ge3$. 
	\end{enumerate}
	\noindent Then $G$ has an induced subgraph $G'$ that is $f$-AT where $f(y)=d_{G'}(y)-1$
	for $y\in Y$ and $f(v)=d_{G'}(v)$ for all $v\in V(G'-Y)$.\end{lem}

We also have the following version with asymmetric degree condition on $\B$.  
The point here is that this works for $k \ge 5$.  
The consequence is that we trade a bit in our bound for the proof to go through with $k \in \set{5,6}$.

\begin{lem}
	\label{MultipleHighConfigurationEulerLopsided} Let $k \ge 5$ and let $G$ be a graph with
	$Y\subseteq V(G)$ such that: 
	\begin{enumerate}
		\item $K_{k}\not\subseteq G$; and 
		\item the components of $G-Y$ are in $\T_{k}$; and 
		\item $d_{G}(v)\leq k-1$ for all $v\in V(G-Y)$; and 
		\item with $\B \DefinedAs \B_k(V(G-Y), Y)$ we have $d_{\B}(y) \ge 4$ for all $y \in Y$ and $d_{\B}(T) \ge 2$ for all components $T$ of $G-Y$.
	\end{enumerate}
	\noindent Then $G$ has an induced subgraph $G'$ that is $f$-AT where $f(y)=d_{G'}(y)-1$
	for $y\in Y$ and $f(v)=d_{G'}(v)$ for all $v\in V(G'-Y)$.
\end{lem}

\begin{proof}[Proof of Lemma \ref{qLemma}]
Put $W \DefinedAs W^k(\L)$ and $L' \DefinedAs V(\L) \setminus W$. Define an auxiliary bipartite graph $F$ with parts $A$ and $B$ where:
\begin{enumerate}
\item  $B = V(\HH^-)$ and $A$ is the disjoint union of the following sets
$A_1, A_2$ and $A_3$,
\item $A_1 = \D$ and each $T \in \D$ is adjacent to all $y \in B$
where $N(y) \cap W^k(T) \ne \emptyset$,
\item For each $v \in L'$, let $A_2(v)$ be a set of $\card{N(v) \cap
B}$ vertices connected to $N(v) \cap B$ by a matching in $F$.  Let
$A_2$ be the disjoint union of the $A_2(v)$ for $v \in L'$,
\item For each $y \in B$, let $A_3(y)$ be a set of $d_{\HH}(y)$ vertices
which are all joined to $y$ in $F$.  Let $A_3$ be the disjoint union
of the $A_3(y)$ for $y \in B$.
\end{enumerate}

Define $\func{f}{V(F)}{\IN}$ by $f(v) = 1$ for all $v \in A_1 \cup A_2 \cup A_3$ and $f(v) = 3$ for all $v \in B$.  First, suppose $\size{F} > \sum_{v \in V(F)} f(v)$.  
Then by Lemma \ref{DegenerateEuler}, $F$ has an induced subgraph $Q$ such that $d_Q(v) > f(v)$ for each $v \in V(Q)$.  
In particular, $V(Q) \subseteq B \cup A_1$ and $d_Q(v) \ge 4$ for $v \in B \cap V(Q)$ and $d_Q(v) \ge 2$ for $v \in A_1 \cap V(Q)$.  
Put $Y \DefinedAs B \cap V(Q)$ and let $X$ be $\bigcup_{T \in V(Q) \cap A_1} V(T)$. 
Now $Z \DefinedAs G[X \cup Y]$ satisfies the hypotheses of Lemma \ref{MultipleHighConfigurationEulerLopsided}, so $Z$ has an induced subgraph $G'$ that is $f$-AT 
where $f(y) = d_{G'}(y) - 1$ for $y \in Y$ and $f(v) = d_{G'}(v)$ for $v \in X$.  Since $Y \subseteq B$ and $X \subseteq V(\L)$, we have $f(v) = k-1 + d_{G'}(v) - d_G(v)$ for all $v \in V(G')$.  
Hence, $G$ is AT-reducible to $G'$, a contradiction.
Therefore $\size{F} \le \sum_{v \in V(F)} f(v) = 3\card{B} + \card{\D} + \card{A_2} + \card{A_3}$. 
By Lemma \ref{ConfigurationTypeOneEuler}, for each $y \in B$ we have $d_F(y) \ge k-1$.  
Hence $\size{F} \ge (k-1)\card{B}$.  This gives $(k-4)\card{B} \le \card{\D} + \card{A_2} + \card{A_3}$.  
Now the first inequality in the lemma follows since $B = V(\HH^-)$, $\card{A_3} = \sum_{v \in V(\HH^-)} d_{\HH}(v)$ and
\begin{align*}
\card{A_2} &= -q(\L) + \size{\HH, \L} \\
&= -q(\L) + k\card{\HH^-}+ \size{\HH^+, \L} - \sum_{v \in V(\HH^-)} d_{\HH}(v).
\end{align*}

Suppose $k \ge 7$.  Define $\func{f}{V(F)}{\IN}$ by $f(v) = 1$ for all $v \in A_2 \cup A_3$ and $f(v) = 2$ for all $v \in B \cup A_1$.  First, suppose $\size{F} > \sum_{v \in V(F)} f(v)$.  
Then by Lemma \ref{DegenerateEuler}, $F$ has an induced subgraph $Q$ such that $d_Q(v) > f(v)$ for each $v \in V(Q)$.  
In particular, $V(Q) \subseteq B \cup A_1$ and $\delta(Q) \ge 3$.  Put $Y \DefinedAs B \cap V(Q)$ and let $X$ be $\bigcup_{T \in V(Q) \cap A_1} V(T)$. 
Now $Z \DefinedAs G[X \cup Y]$ satisfies the hypotheses of Lemma \ref{MultipleHighConfigurationEuler}, so $Z$ has an induced subgraph $G'$ that is $f$-AT 
where $f(y) = d_{G'}(y) - 1$ for $y \in Y$ and $f(v) = d_{G'}(v)$ for $v \in X$.  Since $Y \subseteq B$ and $X \subseteq V(\L)$, we have $f(v) = k-1 + d_{G'}(v) - d_G(v)$ for all $v \in V(G')$.  
Hence, $G$ is AT-reducible to $G'$, a contradiction.

Therefore $\size{F} \leq \sum_{v \in V(F)} f(v) = 2(\card{B} + \card{\D}) + \card{A_2} + \card{A_3}$. 
By Lemma \ref{ConfigurationTypeOneEuler}, for each $y \in B$ we have $d_F(y) \ge k-1$.  
Hence $\size{F} \ge (k-1)\card{B}$.  This gives $(k-3)\card{B} \leq 2\card{\D} + \card{A_2} + \card{A_3}$.  
Now the second inequality in the lemma follows as before.
\end{proof}

\section{Bounding $\beta(\L)$}
This section is devoted to extracting the reusable Lemma \ref{betaLemma} from the proof of R. \cite{4ListCritical}.

\begin{defn} A graph $G$ is \emph{OC-reducible} to $H$ if $H$ is a nonempty induced
subgraph of $G$ which is online $f_{H}$-choosable where $f_{H}(v)\DefinedAs\delta(G)+d_{H}(v)-d_{G}(v)$
for all $v\in V(H)$. If $G$ is not OC-reducible to any nonempty induced subgraph,
then it is \emph{OC-irreducible}. 
\end{defn}

\begin{lem}\label{betaLemma}
	Let $G$ be an OC-irreducible graph with $\delta(G) = k-1$.  Let $\L$ be the subgraph of $G$ induced on $(k-1)$-vertices and	$\HH$ the subgraph of $G$ induced on $k^+$-vertices.  
	If $\chi(\HH) \le k$, then
	\[\beta(\L) \le \size{\HH} + \frac{2\size{G} - (k-2)\card{G} - \parens{k - \frac12}\card{\HH}}{k-1}.\]
\end{lem}

\begin{observation}
The hypotheses of Lemma \ref{betaLemma} are satisfied by $k$-critical, $k$-list-critical and online $k$-list-critical graphs.
\end{observation}

The proof of Lemma \ref{betaLemma} requires the following lemma from Kierstead and R. \cite{KernelMagic} 
that generalizes a kernel technique of Kostochka and Yancey \cite{kostochkayancey2012ore}.

\begin{TheDefinition} The \emph{maximum independent cover number }of a graph $G$
	is the maximum $\mic(G)$ of $\size{I, V(G) \setminus I}$ over all independent sets $I$
	of $G$. 
\end{TheDefinition}

\begin{KernelMagic}\label{ConsantListMicStrength} 
	Every OC-irreducible graph $G$ with $\delta(G) = k-1$ satisfies
	\[2\size{G} \ge (k-2)\card{G} + \mic(G) + 1.\]
\end{KernelMagic}

\begin{proof}[Proof of Lemma \ref{betaLemma}]
Let $M$ be the maximum of $\size{I, V(G) \setminus I}$ over all independent sets $I$ of $G$ with $I \subseteq \HH$.   Since the vertices in $\L$ with $k-1$ neighbors in $\L$ have no neighbors in $\HH$,
	\begin{equation}
		\mic(G) \ge M + (k-1)\beta(\L).\label{eq0}
	\end{equation}
	Let $\C$ be the components of $G[\HH]$.  Then $\alpha(C) \ge \frac{\card{C}}{\chi(C)}$ for all $C \in \C$.  Whence
	\begin{equation}
	  M + (k-1)\size{\HH} \ge \sum_{C \in \C} k\frac{\card{C}}{\chi(C)} + (k-1)\size{C}.
	  \label{Mbound}
	\end{equation}
	For every $C \in \C$,
	\begin{equation}
	 k\frac{\card{C}}{\chi(C)} + (k-1)\size{C} \ge \parens{k - \frac12}\card{C}.
	 \label{KFC}
	\end{equation}
	To see this, first suppose $C \in \C$ is a tree. Then $\chi(C) \le 2$ and hence 
	$k\frac{\card{C}}{\chi(C)} + (k-1)\size{C} \ge k\frac{\card{C}}{2} + (k-1)(\card{C} - 1) \ge (k-\frac12)\card{C}$ unless $\card{C} = 1$.  But the bound is trivially satisfied when $\card{C} = 1$.
	Instead suppose $C$ is not a tree. Then $\size{C} \ge \card{C}$ and $\chi(C) \le k$. Hence $k\frac{\card{C}}{\chi(C)} + (k-1)\size{C} \ge k\frac{\card{C}}{k} + (k-1)\card{C} \ge (k - \frac12)\card{C}$, as desired.
	
	Combining \eqref{eq0}, \eqref{Mbound} and \eqref{KFC} gives
	\begin{equation}
	 \mic(G) \ge \parens{k-\frac12}\card{\HH} - (k-1)\size{\HH} + (k-1)\beta(\L).
	 \label{eq3}
	\end{equation}
	Applying Kernel Magic using \eqref{eq3} and solving for $\beta(\L)$ proves the lemma.
\end{proof}

\begin{problem}
The right side of equation (\ref{KFC}) in the above proof can be improved to $k\card{C}$ unless $C$ is a $K_2$ where both vertices have degree $k$ in $G$.  
Handle these $K_2$'s and improve Lemma \ref{betaLemma}.
\end{problem}

\section{Gallai tree bounds}
\begin{lem}\label{BoundFamilyWithoutKKMinusOne}
	Let $\func{p}{\IN}{\IR}$, $\func{f}{\IN}{\IR}$, $\func{z}{\IN}{\IR}$.
	For all $k \ge 6$ and $T \in \T_k$ with $K_{k-1} \not \subseteq T$, we have
	\[2\size{T} \le (k-3 + p(k))\card{T} + f(k) + z(k)\beta(T)\]
	whenever $p$, $f$ and $z$ satisfy all of the following conditions:
	\begin{enumerate}
		\item $p(k) \ge \frac{-f(k)}{k-2}$; and
		\item $p(k) \ge \frac{-f(k)}{5} + 5 - k$; and
		\item $0\ge f(k)\ge -k+2$; and
		\item $p(k) \ge \frac{3 - z(k)}{k-2}$.
	\end{enumerate}
\end{lem}
\begin{proof}
A general outline for the proof is that it mirrors that of
Lemma~\ref{BasicGallaiTreeBound}, and we add as hypotheses all of the conditions that
we need along the way.

Suppose the lemma is false and choose a counterexample $T$ minimizing $|T|$. 
If $T$ is $K_t$ for some $t \in \{2,k-2\}$, then $t(t-1) > (k-3 + p(k))t +
f(k)$.  After substituting $p(k)\ge \frac{-f(k)}{k-2}$ from (1), this
simplifies to $-t(k-2)>f(k)$, which contradicts (3).  If $T$ is $C_{2r+1}$ for
$r \ge 2$, then $2(2r+1) > (k-3 + p(k))(2r+1) + f(k)$ and hence
$(5-k-p(k))(2r+1)>f(k)$.  Since $f(k)\le 0$, this contradicts (2).  (Note that
we only use conditions (1), (2), and (3) when $T$ has a single block;
these are the base cases when the proof is phrased using induction.)

Let $D$ be an %endblock of $T$ and $x_B$ the cutvertex of $T$ contained in $B$. 
induced subgraph such that $T\setminus D$ is connected.  (We will choose $D$ to
be a connected subgraph contained in at most three blocks of $T$.)
Let $T' = T \setminus D$. %\parens{V(B) \setminus \set{x_B}}$. 
By the minimality of $|T|$, we have
	\[2\size{T'} \le (k-3 + p(k))\card{T'} + f(k) + z(k)\beta(T').\]
	Since $T$ is a counterexample, subtracting this inequality from the inequality for
$2||T||$ gives
	\begin{equation}
	2\size{T} - 2\size{T'} > (k-3 + p(k))|D| + z(k)(\beta(T) - \beta(T')). \tag{*}\label{eqn:*}
	\end{equation}
	
Suppose $T$ has an endblock $B$ that is $K_t$ for some $t \in \{3,\ldots,
k-3\}$; let $x_B$ be a cut vertex of $B$ and let $D=B-x_B$.
Now \eqref{eqn:*} gives $2\size{T}-2\size{T'} =
\card{B}(\card{B}-1)>(k-3+p(k))(|B|-1)$, which 
is a contradiction, since $|B|\le k-3$ and $p(k)>0$.
Suppose instead that $T$ has an endblock $B$ that is an odd cycle.  Again, let
$D=B-x_B$.  Now we get $2|B|>(k-3+p(k))(|B|-1)$.  This simplifies to $|B|<1+\frac2{k-5+p(k)}$, which is a contradiction, 
since the denominator is always at least 1 (using (4) when $k=5$).
%and $2\size{B} = 2\card{B}$ if $\card{B} > k-3$.  
%Since $p(k) \ge \frac{3}{k-2}$ by (4), this contradicts \ref{eqn:*}.
Finally suppose that $T$ has an endblock $B$ that is $K_2$. Now \eqref{eqn:*} gives
$2 > k-3 + p(k)$, which is again a contradiction, since $k \ge 5$ and $p(k) > 0$.
	
To handle the case when $B$ is $K_{k-2}$ we need to remove $x_B$ from $T$ as
well, so we simply let $D=B$.  
Since $B=K_{k-2}$, we have either $d_T(x_B) = k - 2$ or $d_T(x_B) =
k-1$. When $d_T(x_B) = k - 2$, we have
	\[(k-2)(k-3) +2 > (k-3 + p(k))(k-2),\]
	contradicting (4).
	
The only remaining case is when $B$ is $K_{k-2}$ and $d_T(x_B) =
k - 1$.  Each case above applied when $B$ was any endblock of $T$, so we may
assume that every endblock of $T$ is a copy of $K_{k-2}$ that shares a vertex
with an odd cycle.  Choose an endblock $B$ that is the end of a longest path in
the block-tree of $T$.  Let $C$ be the odd cycle sharing a vertex $x_B$ with
$B$.  Consider a neighbor $y$ of $x_B$ on $C$ that either (i) lies only in $C$
or (ii) lies also in an endblock $A$ that is a copy of $K_{k-2}$ (such a
neighbor exists because $B$ is at the end of a longest path in the block-tree).
In (i), let $D=B\cup\{y\}+yx_B$; in (ii), let $D=B\cup A+yx_B$.

In (i), equation \eqref{eqn:*} gives

\[(k-2)(k-3)+2(3) > (k-3+p(k))(k-1) + z(k).\]
%
This simplifies to $6>k-3+(k-1)p(k) + z(k)$.  Since $p(k) \ge 0$ by (1) and (3), this implies $z(k) < 9-k$. Also, by (4) we get $6>k+\frac{3-z(k)}{k-2} > k + \frac{k-6}{k-2}$, which yields a contradiction since $k \ge 6$.

In (ii), equation \eqref{eqn:*} gives
	\[2(k-2)(k-3) + 2(3) > 2(k-3 + p(k))(k-2) + z(k),\]
	which simplifies to
	\[3 - z(k) > (k-2)p(k),\]
	contradicting (4).
\end{proof}

\begin{lem}\label{BoundFamilyWithKKMinusOne}
	Let $\func{p}{\IN}{\IR}$, $\func{f}{\IN}{\IR}$, $\func{h}{\IN}{\IR}$, $\func{z}{\IN}{\IR}$. 
	For all $k \ge 6$ and $T \in \T_k$ with $K_{k-1} \subseteq T$, we have
	\[2\size{T} \le (k-3 + p(k))\card{T} + f(k) + h(k)q(T) + z(k)\beta(T)\]
	whenever $p$, $f$, $h$ and $z$ satisfy all of the following conditions:
	\begin{enumerate}
		\item $f(k) \ge (k-1)(1- p(k) - h(k))$; and	
	    \item $p(k) \ge \frac{3 - z(k)}{k-2}$; and
		\item $p(k) \ge h(k) + 5 - k$; and
		\item $p(k) \ge \frac{2+h(k)}{k-2}$; and
		\item $(k-1)p(k) + (k-3)h(k) + z(k)\ge k+1$.
	\end{enumerate}
\end{lem}

The proof is similar to that of Lemma~\ref{BoundFamilyWithoutKKMinusOne}.  The
main difference is that now our only base case is $T=K_{k-1}$.  For this
reason, we replace hypotheses (1), (2), and (3) of
Lemma~\ref{BoundFamilyWithoutKKMinusOne}, which we used only for the base cases
of that proof, with our new hypothesis (1), which we use for the current base
case.  When some endblock $B$ is an odd cycle or $K_t$, with $t\in\{3, \ldots,
k-3\}$, the induction step is identical to that in
Lemma~\ref{BoundFamilyWithoutKKMinusOne}, since deleting $D$ does not change
$q(T)$.
It is easy to check that, as needed, $K_{k-1}\subseteq T\setminus D$.  Thus, we
need only to consider the induction step when $T$ has an endblock $B$ that is
$K_2$, $K_{k-2}$, or $K_{k-1}$.  As we will see, these three cases require
hypotheses (3), (4), and (5), respectively.
    
Let $T$ be a counterexample minimizing $|T|$.  Let $D$ be an induced subgraph
such that $T\setminus D$ is connected, and let $T'=T\setminus D$. The same
argument as in Lemma~\ref{BoundFamilyWithoutKKMinusOne} now gives
    \begin{equation}
		2||T||-2||T'|| > (k-3 + p(k))\card{D} + h(k)\parens{q(T) - q(T')} + z(k)\parens{\beta(T) - \beta(T')}.\tag{**}\label{eqn:**}
	\end{equation}
If $B$ is $K_2$, then $q(T') \le q(T) + 1$ and \eqref{eqn:**} gives $2 > k-3 +
p(k) - h(k)$, contradicting (3).
So every endblock of $B$ is $K_{k-2}$ or $K_{k-1}$. To handle these cases, we
will need to remove $x_B$ from $T$ as well.  Suppose some endblock $B$ is
$K_{k-1}$ and $K_{k-1} \subseteq T\setminus B$.  Let $D=B$.  Now
$q(T') \le q(T)-(k-2)+1$.  So \eqref{eqn:**} gives
	
\[ (k-1)(k-2)+2 > (k-3+p(k))(k-1)+h(k)(k-3) + z(k).\]
This simplifies to $k+1 > (k-1)p(k)+(k-3)h(k) + z(k)$, which contradicts (5).  Thus,
at most one endblock of $T$ is $K_{k-1}$.
Since the cases above apply when $B$ is any endblock, each other endblock must
be $K_{k-2}$.  Let $B$ be such an endblock, and $x_B$ its cut vertex.	So
$d_T(x_{B}) = k - 2$ or $d_T(x_{B}) = k-1$.  In the former case, $q(T') \le
q(T) + 1$, and in the latter, $q(T) = q(T')$.
If $d_T(x_{B}) = k - 2$, then \eqref{eqn:**} gives
\[(k-2)(k-3) +2 > (k-3 + p(k))(k-2) - h(k),\]
which simplifies to $\frac{2+h(k)}{k-2} > p(k)$, and contradicts (4).
	
Hence, all but at most one endblock of $T$ is a copy of $K_{k-2}$ with a
cut vertex that is also in an odd cycle.  Let $B$ be such an endblock 
at the end of a longest path in the block-tree of $T$, and let $C$ be the odd
cycle sharing a vertex $x_B$ with $B$.  Consider a neighbor $y$ 
of $x_B$ on $C$ that either (i) lies only in block $C$ or (ii) lies also in an
endblock $A$ that is a copy of $K_{k-2}$ (such a neighbor exists 
because $B$ is at the end of a longest path in the block-tree).  In (i), let
$D=B\cup\{y\}+yx_B$; in (ii), let $D=B\cup A+yx_B$.  Let $T'=T\setminus
V(D)$.	In each case, we have $q(T')=q(T)$, so the analysis is identical to that
in the proof of Lemma~\ref{BoundFamilyWithoutKKMinusOne}.
\end{proof}

\begin{cor}\label{SingleParameterBoundFamilyWithKKMinusOne}
	Let $\func{p}{\IN}{\irange{0,1}}$.
	For all $k \ge 6$ and $T \in \T_k$ with $K_{k-1} \subseteq T$, we have
	\[2\size{T} \le \parens{k-3 + p(k)}\card{T} + f(k) + h(k)q(T) + z(k)\beta(T),\]
	where
	\begin{align*}
	f(k) &= (k-1)\parens{3 - (k-1)p(k)},\\
	h(k) &= (k-2)p(k) - 2,\\
	z(k) &= 3k-5 - (k^2-4k+5)p(k).\\
	\end{align*}
\end{lem}

\bibliographystyle{amsplain}
\bibliography{GraphColoring1}
\end{document}

 
