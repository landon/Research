\documentclass[12pt]{article}
\usepackage{amsmath, amsthm, amssymb}
\usepackage{tkz-graph}
\usepackage{marginnote}
\usepackage{verbatim}
\usepackage[top=1.0in, bottom=1.0in, left=1.0in, right=1.0in]{geometry}
\usepackage{color}
\pagestyle{plain}

\usepackage[backref=page]{hyperref}

\usepackage{sectsty}
\allsectionsfont{\sffamily}

\setcounter{secnumdepth}{5}
\setcounter{tocdepth}{5}

\makeatletter
\newtheorem*{rep@theorem}{\rep@title}
\newcommand{\newreptheorem}[2]{
\newenvironment{rep#1}[1]{
 \def\rep@title{#2 \ref{##1}}
 \begin{rep@theorem}}
 {\end{rep@theorem}}}
\makeatother

\theoremstyle{plain}
\newtheorem{thm}{Theorem}
\newreptheorem{thm}{Theorem}
\newtheorem{prop}[thm]{Proposition}
\newreptheorem{prop}{Proposition}
\newtheorem{lem}[thm]{Lemma}
\newreptheorem{lem}{Lemma}
\newtheorem{conjecture}[thm]{Conjecture}
\newreptheorem{conjecture}{Conjecture}
\newtheorem{cor}[thm]{Corollary}
\newreptheorem{cor}{Corollary}
\newtheorem{prob}[thm]{Problem}

\newtheorem*{KernelMagic}{Kernel Magic}
\newtheorem*{MainTheorem}{Main Theorem}
\newtheorem*{BK}{Borodin-Kostochka Conjecture}
\newtheorem*{BK2}{Borodin-Kostochka Conjecture (restated)}
\newtheorem*{Reed}{Reed's Conjecture}
\newtheorem*{ClassificationOfd0}{Classification of $d_0$-choosable graphs}
\newtheorem*{ReedDecomp}{Reed's Decomposition}
\newtheorem*{FajtlowiczDecomp}{Fajtlowicz's Decomposition}


\theoremstyle{definition}
\newtheorem{defn}{Definition}
\theoremstyle{remark}
\newtheorem*{remark}{Remark}
\newtheorem*{problem}{Problem}
\newtheorem{example}{Example}
\newtheorem*{question}{Question}
\newtheorem*{observation}{Observation}

\newcommand{\fancy}[1]{\mathcal{#1}}
\newcommand{\C}[1]{\fancy{C}_{#1}}


\newcommand{\IN}{\mathbb{N}}
\newcommand{\IR}{\mathbb{R}}
\newcommand{\G}{\fancy{G}}
\newcommand{\CC}{\fancy{C}}
\newcommand{\D}{\fancy{D}}
\newcommand{\T}{\fancy{T}}
\newcommand{\B}{\fancy{B}}
\renewcommand{\L}{\fancy{L}}
\newcommand{\HH}{\fancy{H}}

\newcommand{\inj}{\hookrightarrow}
\newcommand{\surj}{\twoheadrightarrow}

\newcommand{\set}[1]{\left\{ #1 \right\}}
\newcommand{\setb}[3]{\left\{ #1 \in #2 : #3 \right\}}
\newcommand{\setbs}[2]{\left\{ #1 : #2 \right\}}
\newcommand{\card}[1]{\left|#1\right|}
\newcommand{\size}[1]{\left\Vert#1\right\Vert}
\newcommand{\ceil}[1]{\left\lceil#1\right\rceil}
\newcommand{\floor}[1]{\left\lfloor#1\right\rfloor}
\newcommand{\func}[3]{#1\colon #2 \rightarrow #3}
\newcommand{\funcinj}[3]{#1\colon #2 \inj #3}
\newcommand{\funcsurj}[3]{#1\colon #2 \surj #3}
\newcommand{\irange}[1]{\left[#1\right]}
\newcommand{\join}[2]{#1 \mbox{\hspace{2 pt}$\vee$\hspace{2 pt}} #2}
\newcommand{\djunion}[2]{#1 \mbox{\hspace{2 pt}$+$\hspace{2 pt}} #2}
\newcommand{\parens}[1]{\left( #1 \right)}
\newcommand{\brackets}[1]{\left[ #1 \right]}
\newcommand{\DefinedAs}{\mathrel{\mathop:}=}

\newcommand{\mic}{\operatorname{mic}}
\newcommand{\AT}{\operatorname{AT}}
\newcommand{\col}{\operatorname{col}}
\newcommand{\ch}{\operatorname{ch}}
\newcommand{\type}{\operatorname{type}}
\newcommand{\nonsep}{\bar{S}}

\def\adj{\leftrightarrow}
\def\nonadj{\not\!\leftrightarrow}

\newcommand\restr[2]{{% we make the whole thing an ordinary symbol
  \left.\kern-\nulldelimiterspace % automatically resize the bar with \right
  #1 % the function
  \vphantom{\big|} % pretend it's a little taller at normal size
  \right|_{#2} % this is the delimiter
  }}

\def\D{\fancy{D}}
\def\C{\fancy{C}}
\def\A{\fancy{A}}
\def\chil{{\chi_\ell}}
\def\chiol{\chi_{\rm{OL}}}

\newcommand{\case}[2]{{\bf Case #1.}~{\it #2}~~}
\newcommand{\aside}[1]{\marginnote{\scriptsize{#1}}[0cm]}
\newcommand{\aaside}[2]{\marginnote{\scriptsize{#1}}[#2]}

\newcommand\numberthis{\addtocounter{equation}{1}\tag{\theequation}}

\title{notes on the Borodin-Kostochka conjecture}
\author{}

\begin{document}
\maketitle

\section{Introduction}
\begin{conjecture}[Borodin and Kostochka \cite{borodin1977upper}]
Every graph $G$ with $\Delta(G) \ge 9$ satisfies $\chi(G) \le \max\set{\omega(G), \Delta(G) - 1}$.
\end{conjecture}

\section{Techniques}
Catlin/Mozhan shuffling, independent transversals and strong coloing, $d_1$-choosables, mules, fractional coloring, probabilistic method, kernels, Kostochka's method

claw-free, doubly-critical edges, squares, vertex-transitive

\section{Excluded induced subgraphs by $d_1$-choosability}\label{d1choosable}
A graph $G$ is \emph{$d_r$-choosable} if $G$ can be $L$-colored from every list assingment $L$ with $\card{L(v)} \ge d_G(v) - r$ for all $v \in V(G)$.
Every graph is $d_{-1}$-choosable.  
The $d_0$-choosable graphs were classified by Borodin \cite{borodin1977criterion} and independently by Erd\H{o}s, Rubin, and Taylor \cite{erdos1979choosability} as those
graphs whose every block is either complete or an odd cycle (a connected such graph is a \emph{Gallai tree}).  Classifying the $d_r$-choosable graphs for any $r \ge 1$ appears
to be a hard problem.  However, we can get useful sufficient conditions for a graph to be $d_1$-choosable.  For example, all of the graphs here are $d_1$-choosable (the vertex color indicates
components of the complement): \url{https://landon.github.io/graphdata/borodinkostochka/offline/index.html}

Cranston and Rabern \cite{mules} classified all $d_1$-choosable graphs of the form $\join{A}{B}$.

\section{Decompositions}
\subsection{Reed's decomposition}
In \cite{reed1999strengthening}, Reed proved the Borodin-Kostochka conjecture for graphs $G$ with $\Delta(G) \ge 10^{14}$.  A piece of that proof was a decomposition of $G$
into dense chunks and one sparse chunk that also works for smaller $\Delta(G)$.  The following tight form of this decomposition is given in \cite{denseneighborhoods}.
Let $\CC_t(G)$ be the maximal cliques in $G$ having at least $t$ vertices.

\begin{ReedDecomp}
Suppose $G$ is a graph with $\Delta(G) \ge 8$ that contains no $K_{\Delta(G)}$ and has no $d_1$-choosable induced sugraph. If
$\frac{\Delta(G) + 5}{2} \le t \le \Delta(G) - 1$, then $\bigcup \CC_t(G)$ can be
partitioned into sets $D_1, \ldots, D_r$ such that for each $i \in \irange{r}$
at least one of the following holds:
\begin{enumerate}
  \item $D_i = C_i \in \CC_t(G)$,
  \item $D_i = C_i \cup \set{x_i}$ where $C_i \in \CC_t(G)$ and $\card{N(x_i) \cap C_i} \geq t-1$.
\end{enumerate}
\end{ReedDecomp}

\subsection{Fajtlowicz's decomposition}
In \cite{fajtlowicz1984independence}, Fajtlowicz proved that every graph has $\alpha(G) \ge \frac{2\card{G}}{\omega(G) + \Delta(G) + 1}$.  The proof of this result
gives a decomposition which we state in the special case needed for the Borodin-Kostochka conjecture.

\begin{FajtlowiczDecomp}
Suppose $G$ is a vertex-critical graph with $\chi(G) = \Delta(G)$.  Then $V(G)$ can be partitioned into sets $M, T,$ and $K$ such that
\begin{enumerate}
\item $M$ contains a maximum independent set $I$ of $G$; and
\item each $v \in T$ has $d_G(v) = \Delta(G)$, two neighbors in $I$ and zero neighbors in $M\setminus I$; and
\item $K$ can be covered by $\alpha(G)$ (or fewer) cliques; and
\item each $v \in K$ has exactly one neighbor in $I$ and at most one neighbor in $M\setminus I$ (none if $d_G(v) < \Delta(G))$; and
\item the vertices in $M \setminus I$ can be ordered $v_1, \ldots, v_r$ such that for $i \in \irange{r}$, either $v_i$ has at least three neighbors in $I \cup \set{v_1, \ldots, v_{i-1}}$
or $d_G(v_i) < \Delta(G)$ and $v_i$ has at least two neighbors in $I \cup \set{v_1, \ldots, v_{i-1}}$.
\end{enumerate}
\end{FajtlowiczDecomp}
\begin{proof}
Let $I$ be a maximum independent set in $G$.  Construct a maximal length sequence $I = M_0 \subsetneq M_1 \subsetneq \cdots \subsetneq M_r$ such that for $j>0$, 
\begin{itemize}
\item every $v \in M_j$ with $d_G(v) = \Delta(G)$ either has at least three neighbors in $M_{j-1}$ or at least two neighbors in $M_{j-1} \setminus I$; and
\item every $v \in M_j$ with $d_G(v) = \Delta(G) - 1$ either has at least two neighbors in $M_{j-1}$ or at least one neighbor in $M_{j-1} \setminus I$.
\end{itemize}
Now let $M = M_r$, let $T$ be the vertices in $V(G) \setminus M$ with exactly two neighbors in $I$ and let $K$ be the vertices in $V(G) \setminus M$ with exactly one neighbor in $I$.
The decomposition has the properties 1,2,4 and 5 since the sequence $M_0 \subsetneq M_1 \subsetneq \cdots \subsetneq M_r$ was chosen to be maximal length.  Property 3 follows since
for each $v \in I$, the set of $x \in K$ adjacent to $v$ must be a clique for otherwise we could get an independent set larger than $I$.
\end{proof}

\section{Results from strong coloring}\label{sparseneighborhoods}
In \cite{denseneighborhoods}, using ideas from strong coloring \cite{haxell2004strong, aharoni2007independent}, Rabern showed that any counterexample to the 
Borodin-Kostochka conjecture must have some sparse neighborhood and large independence number.

\begin{thm}\label{TwoThirdsCliqueCor}
Every graph $G$ with $\chi(G) \geq \Delta(G) \geq 9$ such that every
vertex is in a clique on $\frac23\Delta(G) + 2$ vertices contains $K_{\Delta(G)}$.
\end{thm}

\begin{thm}\label{BKdense}
Every graph $G$ with $\omega(G) < \Delta(G)$ such that $d(G[N(v)]) \geq \frac23\Delta(G) + 4$ for each $v \in V(G)$ is $(\Delta(G)-1)$-colorable.
\end{thm}

\begin{thm}
Every graph $G$ satisfies $\chi(G) \le \max\set{\omega(G), \Delta(G) - 1, 4\alpha(G)}$.
\end{thm}

In the next subsection, we improve Theorem \ref{TwoThirdsCliqueCor} and Theorem \ref{BKdense} slightly.

\subsection{An improvement}
The proof of Theorem \ref{TwoThirdsCliqueCor} in \cite{denseneighborhoods} uses techniques developed for strong coloring.  
Here we show how to use the best known bounds for strong coloring directly and hence get a small improvement.

For a positive integer $r$, a graph $G$ with $\card{G} = rk$ is called \emph{strongly $r$-colorable} if for every partition 
of $V(G)$ into parts of size $r$ there is a proper coloring of $G$ that uses all $r$ colors on each part.  
If $\card{G}$ is not a multiple of $r$, then $G$ is strongly $r$-colorable iff the graph formed by adding $r\ceil{\frac{|G|}{r}} - |G|$ isolated vertices to $G$ is strongly $r$-colorable.  
The \emph{strong chromatic number} $s\chi(G)$ is the smallest $r$ for which $G$ is strongly $r$-colorable.

Note that a strong $r$-coloring of $G$ with respect to a partition $V_1, \ldots, V_k$ of $V(G)$ with $\card{V_i} = r$ must partition $V(G)$ into $r$ 
independent transversals of $V_1, \ldots, V_k$. In \cite{szabo2006extremal}, Szab{\'o} and Tardos constructed partitioned graphs with part sizes $2\Delta - 1$ 
that have no independent transversal.  So we must have $s\chi(G) \geq 2\Delta(G)$.  That the upper bound $s\chi(G) \le 2\Delta(G)$ holds is the \emph{strong coloring conjecture}.

Haxell \cite{haxell2004strong} proved that $s\chi(G) \leq 3\Delta(G) - 1$.  

\begin{thm}\label{TwoThirdsCliqueCorImproved}
Every graph $G$ with $\chi(G) \ge \Delta(G) \ge 8$ such that every
vertex is in a clique on $\frac23\Delta(G) + 1$ vertices contains $K_{\Delta(G)}$.
\end{thm}
\begin{proof}
Suppose the theorem does not hold and let $G$ be a counterexample minimizing $\card{G}$.  
Apply Reed's decomposition with $t \DefinedAs \ceil{\frac23\Delta(G)} + 1$ to get sets $D_1, \ldots, D_r$ with $\bigcup_{i\in\irange{r}} D_i = V(G)$.  
Create $G'$ from $G$ by removing all the edges in $G[D_i]$ for each $i \in \irange{r}$.  
By Haxell's bound, $s\chi(G') \le 3\Delta(G') - 1 \le 3(\Delta(G) - (t-1)) - 1 \le \Delta(G)-1$. If $\card{D_i} \le \Delta(G)- 1$, this gives
a $\parens{\Delta(G)-1}$-coloring of $G$, a contradiction.  By symmetry, we may assume $\card{D_1} \ge \Delta(G)$.  Then it must be that $D_1 = C_1 \cup \set{x}$ where
$C_1$ is a $K_{\Delta(G)-1}$ and $x$ has at least $\ceil{\frac23\Delta(G)}$ neighbors in $C_1$.
\end{proof}

If the strong coloring conjecture holds, we get the following improvement.

\begin{conjecture}\label{TwoThirdsCliqueCorImprovedMore}
Every graph $G$ with $\chi(G) \ge \Delta(G) \ge 8$ such that every
vertex is in a clique on $\frac{\Delta(G) + 5}{2}$ vertices contains $K_{\Delta(G)}$.
\end{conjecture}

\section{Properties of minimum counterexamples}\label{mules}
In \cite{cranstonrabernapriori} Cranston and Rabern used the $d_1$-choosable graphs in Section \ref{d1choosable} to prove properties 
of a minimum counterexample to the Borodin-Kostochka conjecture.  For example, the following improves a lemma Reed used in his proof \cite{reed1999strengthening}.

\begin{lem}
Let $G$ be a minimum counterexample to the Borodin-Kostochka conjecture.  If $X$ is a $K_{\Delta(G) - 1}$ in $G$, then every $v \in V(G-X)$ has at most one neighbor in $X$.
\end{lem}

\begin{lem}
Let $G$ be a minimum counterexample to the Borodin-Kostochka conjecture. Let $A$ and $B$ be disjoint subgraphs of $G$ with $\card{A} + \card{B} = \Delta(G)$ 
such that $\card{A},\card{B} \ge 3$.  If $G$ contains all edges between $A$ and $B$,
then $A = K_1 + K_{\card{A} - 1}$ and $B = K_1 + K_{\card{B} - 1}$.
\end{lem}


\section{Results from kernel methods}
In \cite{KernelMagic}, Kierstead and Rabern proved a general lemma that allows the user to get list colorings for free from large independent sets.  
Specialized to the Borodin-Kostochka conjecture,
this becomes.

\begin{KernelMagic}
Suppose $G$ is a vertex-critical graph with $\chi(G) = \Delta(G)$.  For every induced subgraph $H$ of $G$ and independent set $I$ in $H$, we have
\[\sum_{v \in I} d_H(v) < \sum_{v \in V(H)} \Delta(G) + 2 - d_G(v).\]
\end{KernelMagic}

Applied with $H=G$, this gives:
\begin{cor}
If $G$ is a vertex-critical graph with $\chi(G) = \Delta(G)$, then $\alpha(G) < \frac{2\card{G}}{\Delta(G)-1}$.
\end{cor}

\section{Mozhan partitions}\label{shuffle}
Extending ideas of Mozhan \cite{mozhan1983}, Cranston and Rabern \cite{bigcliques} proved the following.
\begin{thm}
If $G$ is a vertex-critical graph with $\chi(G) = \Delta(G) \ge 13$, then $\omega(G) \ge \Delta(G) - 3$.
\end{thm}

\section{Vertex-transitive graphs}
In \cite{vertextransitive} Cranston and Rabern used Reed's decomposition and the ideas in Sections \ref{sparseneighborhoods} and \ref{shuffle} to prove 
the Borodin-Kostochka conjecture for vertex-transitive graphs with $\Delta(G) \ge 13$.  It would be interesting to improve this to $\Delta(G) \ge 9$.

\begin{thm}
Every vertex-transitive graph $G$ with $\Delta(G) \ge 13$ satisfies $\chi(G) \le \max\set{\omega(G), \Delta(G) - 1}$.
\end{thm}

\section{Claw-free graphs}
In \cite{cranstonrabernclaw}, Cranston and Rabern proved the Borodin-Kostochka conjecture for claw-free graphs using some of the $d_1$-choosable graphs in Section \ref{d1choosable} 
combined with the structure theorem for quasi-line graphs of Chudnovsky and Seymour \cite{chudnovsky2005structure}.
\bibliographystyle{plain}
\bibliography{GraphColoring1}
\end{document} 