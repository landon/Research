\documentclass[12pt]{amsart}
\usepackage{amsmath, amsthm, amssymb}
\usepackage[top=1.25in, bottom=1.25in, left=1.0in, right=1.0in]{geometry}

\allowdisplaybreaks
\pagestyle{headings}

\makeatletter
\newtheorem*{rep@theorem}{\rep@title}
\newcommand{\newreptheorem}[2]{
\newenvironment{rep#1}[1]{
 \def\rep@title{#2 \ref{##1}}
 \begin{rep@theorem}}
 {\end{rep@theorem}}}
\makeatother

\theoremstyle{plain}
\newtheorem{thm}{Theorem}[section]
\newreptheorem{thm}{Theorem}
\newtheorem{prop}[thm]{Proposition}
\newreptheorem{prop}{Proposition}
\newtheorem{lem}[thm]{Lemma}
\newreptheorem{lem}{Lemma}
\newtheorem{conjecture}[thm]{Conjecture}
\newreptheorem{conjecture}{Conjecture}
\newtheorem{cor}[thm]{Corollary}
\newreptheorem{cor}{Corollary}
\newtheorem{prob}[thm]{Problem}
\newtheorem{claim}{Claim}
\newtheorem*{unnumberedClaim}{Claim}
\newtheorem*{SmallPotLemma}{Small Pot Lemma}
\newtheorem*{BK}{Borodin-Kostochka Conjecture}
\newtheorem*{BK2}{Borodin-Kostochka Conjecture (restated)}
\newtheorem*{Reed}{Reed's Conjecture}
\newtheorem*{ClassificationOfd0}{Classification of $d_0$-choosable graphs}

\theoremstyle{definition}
\newtheorem{defn}{Definition}[section]
\newtheorem*{CliqueGraph}{Clique Graph}

\theoremstyle{plain}
\newtheorem*{remark}{Remark}
\newtheorem{example}{Example}
\newtheorem*{question}{Question}
\newtheorem*{observation}{Observation}

\newcommand{\fancy}[1]{\mathcal{#1}}
\newcommand{\C}[1]{\fancy{C}_{#1}}
\newcommand{\IN}{\mathbb{N}}
\newcommand{\IR}{\mathbb{R}}
\newcommand{\G}{\fancy{G}}
\newcommand{\CC}{\fancy{C}}
\newcommand{\D}{\fancy{D}}

\newcommand{\inj}{\hookrightarrow}
\newcommand{\surj}{\twoheadrightarrow}

\newcommand{\set}[1]{\left\{ #1 \right\}}
\newcommand{\setb}[3]{\left\{ #1 \in #2 \mid #3 \right\}}
\newcommand{\setbs}[2]{\left\{ #1 \mid #2 \right\}}
\newcommand{\card}[1]{\left|#1\right|}
\newcommand{\size}[1]{\left\Vert#1\right\Vert}
\newcommand{\ceil}[1]{\left\lceil#1\right\rceil}
\newcommand{\floor}[1]{\left\lfloor#1\right\rfloor}
\newcommand{\func}[3]{#1\colon #2 \rightarrow #3}
\newcommand{\funcinj}[3]{#1\colon #2 \inj #3}
\newcommand{\funcsurj}[3]{#1\colon #2 \surj #3}
\newcommand{\irange}[1]{\left[#1\right]}
\newcommand{\join}[2]{#1 \mbox{\hspace{2 pt}$\ast$\hspace{2 pt}} #2}
\newcommand{\djunion}[2]{#1 \mbox{\hspace{2 pt}$+$\hspace{2 pt}} #2}
\newcommand{\parens}[1]{\left( #1 \right)}
\newcommand{\brackets}[1]{\left[ #1 \right]}

\newcommand{\DefinedAs}{\mathrel{\mathop:}=}

\newcommand{\mov}[2]{#1^{#2}}
\newcommand{\wt}[1]{w\parens{#1}}
\renewcommand{\vec}[1]{\mathbf{#1}}
\def\adj{\leftrightarrow}
\def\nonadj{\not\!\leftrightarrow}
\newcommand{\im}{\operatorname{im}}
\newcommand{\ex}{\operatorname{E}}

\title{more notes 3}

\begin{document}
\maketitle

\section{Lemma 4.6}

We need the following result from \cite{mules}.

\begin{lem}\label{K6Possibilities}
If $\join{K_6}{B}$ is not $d_1$-choosable, then $\omega(B) \geq \card{B} - 1$.
\end{lem}

\begin{lem}\label{neighborhood}
If $B$ is a graph with $\delta(B) \geq \frac{\card{B}}{2} + 1$ such that
$\join{K_1}{B}$ is not $d_1$-choosable, then $\omega(B) \geq \card{B} - 1$.
\end{lem}
\begin{proof}
Suppose the lemma is false and let $L$ be a bad $d_1$-assignment on $B$ with $\card{Pot(L)}$ minimal. First note that if $B$ does not contain disjoint nonadjacent pairs $x_1, y_1$
and $x_2, y_2$, then $\omega(B) \geq \card{B} - 1$ by Lemma \ref{K6Possibilities}.

By the Small Pot Lemma, $\card{Pot(L)} \leq \card{B}$.  Since $\card{L(x_i)} +
\card{L(y_i)} \geq \card{B} + 2$, we have then $\card{L(x_i) \cap L(y_i)} \geq 2$ for each $i$.  Color $x_1$ and $y_1$ with $c_1 \in L(x_1) \cap L(y_1)$ and color $x_2$ and $y_2$ with $c_2 \in L(x_2) \cap L(y_2) - c_1$.  By the minimum degree condition on $B$, each component of $B - \set{x_1, y_1, x_2, y_2}$ has a vertex joined to $\set{x_1, y_1}$ or $\set{x_2, y_2}$.  Hence we can complete the coloring to all of $B$ and then to the $K_1$.
\end{proof}

Now let $G$ be our minimum counterexample.

\begin{lem}\label{partition}
If $\frac{\Delta(G) + 7}{2} \leq t \leq \Delta(G) - 1$, then $\bigcup \CC_t$ can be
partitioned into sets $D_1, \ldots, D_r$ such that for each $i \in \irange{r}$
at least one of the following holds:
\begin{itemize}
  \item $D_i \in \CC_t$,
  \item $D_i = C_i \cup \set{x_i}$ where $C_i \in \CC_t$ and $\card{N(x_i) \cap
  C_i} \geq t-1$.
\end{itemize}
Moreover, each $v \in V(G) - D_i$ has at most $t-2$ neighbors in $C_i$.
\end{lem}
\begin{proof}
Suppose $\card{C_i} \leq \card{C_j}$ and $C_i \cap C_j \neq \emptyset$. 
Then $\card{C_i \cap C_j} \geq \card{C_i} + \card{C_j} - (\Delta + 1) \geq 6$.  It follows from Lemma
\ref{K6Possibilities} that $\card{C_i - C_j} \leq 1$.

Now suppose $C_i$ intersects $C_j$ and $C_k$.  By the above,
$\card{C_i \cap C_j} \geq \frac{\Delta(G) + 5}{2}$ and similarly $\card{C_i \cap
C_k} \geq \frac{\Delta(G) + 5}{2}$.  Hence $\card{C_i \cap C_j \cap C_k} \geq
\Delta(G) + 5 - (\Delta(G) - 1) = 6$.  By Lemma \ref{K6Possibilities} we see that $\omega(G[C_i \cup C_j \cup C_k]) \geq \card{C_i \cup C_j \cup C_k} - 1$ which is impossible since each of $C_i, C_j, C_k$ are maximal.

\smallskip

\noindent The existence of the required partition is immediate. 
\end{proof}

This can quickly be turned into a decomposition for $d$-dense graphs. Call $v \in V(G)$ $d$-sparse if it has more than $d\Delta$ non-edges in its neighborhood. 

\begin{lem}\label{improved}
Let $0 \leq d \leq \frac{\Delta}{10} - \frac32$. We can partition $V(G)$ into $S,D_1, \ldots, D_r$ so that

\begin{enumerate}
\item each vertex in $S$ is $d$-sparse,
\item each $D_i$ contains a vertex $w_i$ such that $D_i - w_i$ is a clique of size at least $\Delta - 3d + 1$,
\item no vertex outside of $D_i$ has more than $\frac{3\Delta}{4}$ neighbors in $D_i$ and $w_i$ has at least $\frac{3\Delta}{4}$ neighbors in $D_i$.
\end{enumerate}
\end{lem}
\begin{proof}
Put $t \DefinedAs \frac34 \Delta + 1$, $B \DefinedAs \bigcup \CC_t$ and $S \DefinedAs V(G) - B$. Apply Lemma \ref{partition} to get $D_1, \ldots, D_r$ partitioning $B$. We claim that some subset of $\set{D_1, \ldots, D_r}$ works.  For item (i), we need to check that each $v \in S$ is $d$-sparse.  By an easy computation like the one below, each $v \in S$ has more than $\binom{\Delta-1}{2} - \frac25 \Delta^2 \geq ( \frac{\Delta}{10} - \frac32)\Delta$ non-edges in its neighborhood, so $v$ is $d$-sparse.

Item (iii) follows by the definition of the $D_i$.  Now item (ii).  If for any $i$, all vertices of $D_i$ are $d$-sparse, then just move all of $D_i$ into $S$.  So now we may assume that each $D_i$ contains a non-sparse vertex $v_i$.  Clearly, the largest clique in $G$ containing $v_i$ is contained in $D_i$.  Hence it will be enough to show that $v_i$ is in a $\Delta - 3d + 1$ clique.  Let $x$ be some $v_i$. Suppose $x$ is in no $\Delta - 3d + 1$ clique, then using Lemma \ref{neighborhood}, we get a sequence $y_1,
\ldots, y_{3d} \in N(x)$ such that 
\[\card{N(y_i) \cap (N(x) - \set{y_1, \ldots, y_{i-1}})} \leq \frac12 (\Delta + 1 - i).\]

Hence $x$ is $d$-sparse since it has at least 

\[\frac12\sum_{i=1}^{3d} (\Delta - i) > d\Delta.\]

non-edges in its neighborhood.
\end{proof}

\section{Lemma 5.3}
We need the following result from \cite{mules}.

\begin{lem}\label{NeighborhoodPotShrink}
Let $H$ be a $d_0$-choosable graph such that $G \DefinedAs \join{K_1}{H}$ is not
$d_1$-choosable and $L$ a minimal bad $d_1$-assignment on $G$.  If some
nonadjacent pair in $H$ have intersecting lists, then $\card{Pot(L)} \leq \card{H} - 1$.
\end{lem}

\begin{lem}\label{AtMostFourIn}
Each $v \in C_i$ has at most one
neighbor outside of $C_i$ with more than $4$ neighbors in $C_i$ and no such
neighbor if $v$ is low.
\end{lem}
\begin{proof}
Suppose otherwise that we have $v \in C_i$ with two neighbors $w_1, w_2 \in V(G)
- C_i$ each with $6$ or more neighbors in $C_i$.  Put $Q \DefinedAs
G[\set{w_1,w_2} \cup C_i - v]$, then $v$ is joined to $Q$ and hence
$\join{K_1}{Q} \unlhd G$.  We show that $\join{K_1}{Q}$ must be $d_1$-choosable.  Note that $Q$ is $d_0$-choosable since it contains a diamond. Let $L$ be a bad $d_1$-assignment on $\join{K_1}{Q}$ minimizing $\card{Pot(L)}$.

First, suppose there are different $z_1,z_2 \in C_i$ such that $\set{w_1, z_1}$
and $\set{w_2, z_2}$ are independent.  By the Small Pot Lemma, $\card{Pot(L)} \leq \card{Q}$.  We have $\card{L(w_1)} + \card{L(z_1)} \geq 4 + \card{Q} - 3 > \card{Pot(L)}$.  Hence $w_1$ and $w_2$ have intersecting lists.  Applying Lemma \ref{NeighborhoodPotShrink} shows that $\card{Pot(L)} \leq \card{Q} - 1$.

Now $\card{L(w_j)} + \card{L(z_j)} \geq 4 + \card{Q} - 3 \geq \card{Pot(L)} + 2$.  Hence $\card{L(w_j) \cap L(z_j)} \geq 2$.  Pick $x \in N(w_1) \cap \set{C_i - v - z_2}$.  Then after coloring each pair $\set{w_1, z_1}$ and $\set{w_2, z_2}$ with a different color, we can finish the coloring because we saved a color for $x$ and two colors for $v$.

By maximality of $C_i$, neither $w_1$ nor $w_2$ can be adjacent to all of $C_i$
hence it must be the case that there is $y \in C_i$ such that $w_1$ and $w_2$
are joined to $C_i - y$.  If $w_1$ and $w_2$ aren't adjacent, then $G$ contains $\join{K_6}{E_3}$ contradicting Corollary \ref{K6Possibilities}.  Hence $C_i$ intersects the larger clique $\set{w_1, w_2} \cup C_i - \set{y}$, this is impossible by the definition of $C_i$.

When $v$ is low, an argument similar to the above shows that there can be no
$z_1$ in $C_i$ so that $\set{w_1, z_1}$ is independent, and hence $C_i \cup
\set{w_1}$ is a clique contradicting maximality of $C_i$.
\end{proof}

\bibliographystyle{amsplain}
\bibliography{GraphColoring}
\end{document}
