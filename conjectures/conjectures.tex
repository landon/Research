\documentclass[12pt]{article}
\usepackage{amsmath, amsthm, amssymb}
\usepackage{hyperref}
\usepackage{verbatim}
\usepackage[top=1.0in, bottom=1.0in, left=1.0in, right=1.0in]{geometry}
\usepackage{color}
\pagestyle{plain}

\usepackage{sectsty}
\allsectionsfont{\sffamily}

\setcounter{secnumdepth}{5}
\setcounter{tocdepth}{5}

\makeatletter
\newtheorem*{rep@theorem}{\rep@title}
\newcommand{\newreptheorem}[2]{
\newenvironment{rep#1}[1]{
 \def\rep@title{#2 \ref{##1}}
 \begin{rep@theorem}}
 {\end{rep@theorem}}}
\makeatother

\theoremstyle{plain}
\newtheorem{thm}{Theorem}
\newreptheorem{thm}{Theorem}
\newtheorem{prop}[thm]{Proposition}
\newreptheorem{prop}{Proposition}
\newtheorem{lem}[thm]{Lemma}
\newreptheorem{lem}{Lemma}
\newtheorem{conjecture}[thm]{Conjecture}
\newreptheorem{conjecture}{Conjecture}
\newtheorem{cor}[thm]{Corollary}
\newreptheorem{cor}{Corollary}
\newtheorem{prob}[thm]{Problem}

\newtheorem*{KernelLemma}{Kernel Lemma}
\newtheorem*{BK}{Borodin-Kostochka Conjecture}
\newtheorem*{BK2}{Borodin-Kostochka Conjecture (restated)}
\newtheorem*{Reed}{Reed's Conjecture}
\newtheorem*{ClassificationOfd0}{Classification of $d_0$-choosable graphs}


\theoremstyle{definition}
\newtheorem{defn}{Definition}
\theoremstyle{remark}
\newtheorem*{remark}{Remark}
\newtheorem*{problem}{Problem}
\newtheorem{example}{Example}
\newtheorem*{question}{Question}
\newtheorem*{observation}{Observation}

\newcommand{\fancy}[1]{\mathcal{#1}}
\newcommand{\C}[1]{\fancy{C}_{#1}}


\newcommand{\IN}{\mathbb{N}}
\newcommand{\IR}{\mathbb{R}}
\newcommand{\G}{\fancy{G}}
\newcommand{\CC}{\fancy{C}}
\newcommand{\D}{\fancy{D}}
\newcommand{\T}{\fancy{T}}
\newcommand{\B}{\fancy{B}}
\renewcommand{\L}{\fancy{L}}
\newcommand{\HH}{\fancy{H}}

\newcommand{\inj}{\hookrightarrow}
\newcommand{\surj}{\twoheadrightarrow}

\newcommand{\set}[1]{\left\{ #1 \right\}}
\newcommand{\setb}[3]{\left\{ #1 \in #2 : #3 \right\}}
\newcommand{\setbs}[2]{\left\{ #1 : #2 \right\}}
\newcommand{\card}[1]{\left|#1\right|}
\newcommand{\size}[1]{\left\Vert#1\right\Vert}
\newcommand{\ceil}[1]{\left\lceil#1\right\rceil}
\newcommand{\floor}[1]{\left\lfloor#1\right\rfloor}
\newcommand{\func}[3]{#1\colon #2 \rightarrow #3}
\newcommand{\funcinj}[3]{#1\colon #2 \inj #3}
\newcommand{\funcsurj}[3]{#1\colon #2 \surj #3}
\newcommand{\irange}[1]{\left[#1\right]}
\newcommand{\join}[2]{#1 \mbox{\hspace{2 pt}$\ast$\hspace{2 pt}} #2}
\newcommand{\djunion}[2]{#1 \mbox{\hspace{2 pt}$+$\hspace{2 pt}} #2}
\newcommand{\parens}[1]{\left( #1 \right)}
\newcommand{\brackets}[1]{\left[ #1 \right]}
\newcommand{\DefinedAs}{\mathrel{\mathop:}=}

\newcommand{\mic}{\operatorname{mic}}
\newcommand{\AT}{\operatorname{AT}}
\newcommand{\col}{\operatorname{col}}
\newcommand{\ch}{\operatorname{ch}}
\newcommand{\type}{\operatorname{type}}
\newcommand{\nonsep}{\bar{S}}

\def\adj{\leftrightarrow}
\def\nonadj{\not\!\leftrightarrow}

\newcommand\restr[2]{{% we make the whole thing an ordinary symbol
  \left.\kern-\nulldelimiterspace % automatically resize the bar with \right
  #1 % the function
  \vphantom{\big|} % pretend it's a little taller at normal size
  \right|_{#2} % this is the delimiter
  }}

\def\D{\fancy{D}}
\def\C{\fancy{C}}
\def\A{\fancy{A}}

\newcommand{\case}[2]{{\bf Case #1.}~{\it #2}~~}

\title{Conjectures that should be true\thanks{(dis)proofs $\Rightarrow$ \texttt{landon.rabern@gmail.com}}}

\begin{document}
\maketitle


\section{Edges in list-critical graphs}
A graph $G$ is \emph{$k$-list-critical} if $G$ is not $(k-1)$-choosable, but every
proper subgraph of $G$ is $(k-1)$-choosable.  Replace `$(k-1)$-' with `online $(k-1)$-' and `\emph{$k$-}' with `\emph{online $k$-}' in the previous sentence and read it.

\begin{conjecture}\label{C1}
Every incomplete $k$-list-critical graph has average degree at least \[k-1 + \frac{k-3}{(k-1)^2}.\]
\end{conjecture}
\begin{proof}[\textbf{Background}]
The connected graphs in which each block is a complete graph or an odd cycle are called \emph{Gallai trees}.  Gallai \cite{gallai1963kritische} proved that in a $k$-critical graph, 
the vertices of degree $k-1$ induce a disjoint union of Gallai trees.  The same is true for $k$-list-critical graphs \cite{borodin1977criterion, erdos1979choosability}. 
This quickly implies a lower bound on the average degree of $k$-list-critical graphs of \[k-1 + \frac{k-3}{k^2-3}.\]
In \cite{rabern2016better}, R. improved this to \[k-1 + \frac{k-3}{k^2-2k+2}\] using a lemma from Kierstead and R. \cite{KernelMagic} that generalizes a kernel technique of Kostochka and Yancey \cite{kostochkayancey2012ore}.
As noted at the end of \cite{rabern2016better}, a small improvement to the argument would yield Conjecture \ref{C1}.  
\end{proof}

\begin{conjecture}
Every incomplete online $k$-list-critical graph $G$ has \[2\size{G} \ge (k-1)\card{G} + k-3.\]
\end{conjecture}\label{C2}
\begin{proof}[\textbf{Background}]
Dirac \cite{dirac1957theorem} proved this for $k$-critical graphs.  Kostochka and Stiebitz \cite{kostochka2002list} proved it for $k$-list-critical graphs.  Their proof does not seem to generalize.
When $\card{G}$ is large compared with $k$, the conjecture holds by Gallai-type bounds on the average degree of online $k$-list-critical graphs \cite{OreVizing,DischargingLowerBound}.
\end{proof}

\subsection{The $\frac56$ bound}
\begin{conjecture}
Every vertex-transitive graph has $\chi \le \max \set{\omega, \ceil{\frac{5\Delta + 3}{6}}}$.
\end{conjecture}
\begin{proof}[\textbf{Background}]
In \cite{vertextransitive}, the following was proved
\begin{itemize}
\item the conjecture holds for the fractional chromatic number $\chi^*$,
\item the conjecture holds with $\ceil{\frac{5\Delta + 3}{6}}$ replaced by $\epsilon(\Delta + 1)$ for some $\epsilon < 1$,
\item the conjecture holds both if Reed's $\omega, \Delta, \chi$ conjecture and the strong $2\Delta$-colorability conjecture hold for vertex-transitive graphs (only strong $\frac52\Delta$-colorability is required),
\end{itemize}
Does the conjecture hold for Cayley graphs?
\end{proof}

\begin{conjecture}
Every line graph (of a multigraph) has $\chi \le \max \set{\omega, \ceil{\frac{5\Delta + 3}{6}}}$.
\end{conjecture}
\begin{proof}[\textbf{Background}]
In \cite{rabern2011strengthening} this was proved with $\ceil{\frac{5\Delta + 3}{6}}$ replaced by $\frac{7\Delta + 10}{8}$.  
Conjecture 14 in \cite{rabern2011strengthening} that implies this conjecture is now known to be false.
\end{proof}

\section{Planar graphs}
\begin{conjecture}
Every planar graph with no $K_5$-subdivision is $2$-fold $9$-colorable.
\end{conjecture}
\begin{proof}[\textbf{Background}]
In \cite{planar92}, Cranston and R. gave a short proof of this conjecture with $K_5$-minors excluded instead of $K_5$-subdivisions (which also follows from the Four Color Theorem).  Haj{\'o}s conjectured that every graph is $(k-1)$-colorable unless it contains a subdivision of
$K_k$. This is known to be true for $k \le 4$ and false for $k \ge 7$. The cases $k = 5$ and $k = 6$ remain unresolved.
\end{proof}

\section{Maximum degree, clique number and colorings}
\subsection{Around Borodin-Kostochka}
\begin{conjecture}\label{C3}
Every graph with $\chi \ge \Delta \ge 8$ contains a $K_3 \vee H$ where $H$ is some graph on $\Delta - 3$ vertices.
\end{conjecture}
\begin{proof}[\textbf{Background}]
By results in \cite{mules}, for $\Delta \ge 9$ the existence of $K_3 \vee H$ implies the existence of $K_\Delta$.  
So, this (seemingly weaker) conjecture for $\Delta \ge 9$ implies the Borodin-Kostochka conjecture.  The one known connected counterexample to 
the Borodin-Kostochka conjecture for $\Delta=8$ is a 5-cycle with each vertex blown up to a triangle.  This graph is not a counterexample to Conjecture \ref{C3}.
\end{proof}

\begin{conjecture}\label{C4}
Every graph with $\chi \ge \Delta$ contains $K_{\Delta-3}$.
\end{conjecture}
\begin{proof}[\textbf{Background}]
Results in \cite{bigcliques} show that this holds with $K_{\Delta-4}$ instead of $K_{\Delta-3}$.  Moreover, \cite{bigcliques} proves the conjecture for all but $\Delta \in \set{6,8,9,11,12}$.
Reed's conjecture \cite{reed1998omega} that every graph satisfies $\chi \le \ceil{\frac{\omega + \Delta + 1}{2}}$ implies this conjecture with $K_{\Delta-2}$ instead of $K_{\Delta-3}$.
\end{proof}

\begin{conjecture}\label{C5}
Every graph with $\chi \ge \Delta$ either contains $K_{\Delta}$ or contains a $K_{\Delta-4}$ with all $\Delta$-vertices.
\end{conjecture}
\begin{proof}[\textbf{Background}]
Results in \cite{bigcliques} show that this holds with $K_{\Delta-5}$ instead of $K_{\Delta-4}$. For $\Delta \le 7$, the conjecture holds by \cite{rabern2010a, krs_one}.
Also by \cite{bigcliques}, it holds when $\Delta = 3r+1$ for $r \ge 3$.
\end{proof}

\begin{conjecture}\label{C6}
Every graph with $\Delta \ge 8$ and $\omega < \Delta$ is $2$-fold $(2\Delta - 1)$-colorable.
\end{conjecture}
\begin{proof}[\textbf{Background}]
The one known connected counterexample to the Borodin-Kostochka conjecture for $\Delta=8$ is a 5-cycle with each vertex blown up to a triangle.  This graph is not a counterexample to Conjecture \ref{C6}.
\end{proof}

\begin{conjecture} 
Every graph with $\theta\ge10$ and $\omega\leq\frac{\theta}{2}$ is $\floor{\frac{\theta}{2}}$-choosable. 
\end{conjecture}
\begin{proof}[\textbf{Background}]
Here $\theta$ is the \emph{Ore degree} give by $\theta(G)\DefinedAs\max_{xy\in E(G)}d(x) + d(y)$.  This conjecture holds for ordinary coloring \cite{kierstead2009ore, rabern2010a, krs_one, rabern2012partitioning}.
In \cite{KernelMagic}, the conjecture is proved for $\theta \ge 18$ for both list-coloring and online list-coloring.  Further lowering of $\theta$ would follow from improved bounds on average degree of list-critical graphs \cite{OreVizing}.
\end{proof}

\begin{conjecture}
Every claw-free graph with $\Delta \ge 9$ and $\omega < \Delta$ is $(\Delta-1)$-choosable.
\end{conjecture}
\begin{proof}[\textbf{Background}]
In \cite{cranstonrabernclaw}, this was proved for ordinary coloring. In \cite{BkClawList}, the conjecture was proved for $\Delta \ge 69$.  Also, \cite{BkClawList} proved that the full conjecture follows from the line-graph case.
\end{proof}

\begin{conjecture}
There is a polynomial time graph algorithm that finds either a $(\Delta-1)$-coloring or a $K_{\Delta-3}$.
\end{conjecture}
\begin{proof}[\textbf{Background}]
In \cite{bigcliques}, the following was proved
\begin{itemize}
\item the conjecture holds with $K_{\Delta-3}$ replaced by $K_{\Delta-4}$,
\item the conjecture holds for $\Delta \ge 25$ (the proof uses algorithmic versions of the local lemma),
\item the conjecture holds when $\Delta = 3r+1$.
\end{itemize}
\end{proof}

\bibliographystyle{amsplain}
\bibliography{GraphColoring1}
\end{document}

 
