\documentclass[12pt]{article}
\usepackage{amssymb, amsmath, amsthm}
\usepackage[top=1.0in, bottom=1.0in, left=1.5in, right=1.5in]{geometry}

\pagestyle{plain}

\usepackage{tkz-graph}
\usetikzlibrary{arrows}
\usepackage[position=bottom]{subfig}

\usepackage{sectsty}
\allsectionsfont{\sffamily}

\usepackage[pdftex, colorlinks, hyperfootnotes]{hyperref}

\makeatletter
\newtheorem*{rep@theorem}{\rep@title}
\newcommand{\newreptheorem}[2]{
\newenvironment{rep#1}[1]{
 \def\rep@title{#2 \ref{##1}}
 \begin{rep@theorem}}
 {\end{rep@theorem}}}
\makeatother

\theoremstyle{plain}
\newtheorem{thm}{Theorem}[section]
\newreptheorem{thm}{Theorem}
\newtheorem{prop}[thm]{Proposition}
\newreptheorem{prop}{Proposition}
\newtheorem{lem}[thm]{Lemma}
\newreptheorem{lem}{Lemma}
\newtheorem{conjecture}[thm]{Conjecture}
\newreptheorem{conjecture}{Conjecture}
\newtheorem{cor}[thm]{Corollary}
\newreptheorem{cor}{Corollary}
\newtheorem{prob}[thm]{Problem}
\newtheorem{claim}{Claim}
\newtheorem*{unnumberedClaim}{Claim}
\newtheorem*{SmallPotLemma}{Small Pot Lemma}
\newtheorem*{BK}{Borodin-Kostochka Conjecture}
\newtheorem*{BK2}{Borodin-Kostochka Conjecture (restated)}
\newtheorem*{Reed}{Reed's Conjecture}
\newtheorem*{ClassificationOfd0}{Classification of $d_0$-choosable graphs}

\theoremstyle{definition}
\newtheorem{defn}{Definition}[section]
\newtheorem*{CliqueGraph}{Clique Graph}

\theoremstyle{remark}
\newtheorem*{remark}{Remark}
\newtheorem{example}{Example}
\newtheorem*{question}{Question}
\newtheorem*{observation}{Observation}

\newcommand{\fancy}[1]{\mathcal{#1}}
\newcommand{\C}[1]{\fancy{C}_{#1}}
\newcommand{\IN}{\mathbb{N}}
\newcommand{\IR}{\mathbb{R}}
\newcommand{\G}{\fancy{G}}

\newcommand{\inj}{\hookrightarrow}
\newcommand{\surj}{\twoheadrightarrow}

\newcommand{\set}[1]{\left\{ #1 \right\}}
\newcommand{\setb}[3]{\left\{ #1 \in #2 \mid #3 \right\}}
\newcommand{\setbs}[2]{\left\{ #1 \mid #2 \right\}}
\newcommand{\card}[1]{\left|#1\right|}
\newcommand{\size}[1]{\left\Vert#1\right\Vert}
\newcommand{\ceil}[1]{\left\lceil#1\right\rceil}
\newcommand{\floor}[1]{\left\lfloor#1\right\rfloor}
\newcommand{\func}[3]{#1\colon #2 \rightarrow #3}
\newcommand{\funcinj}[3]{#1\colon #2 \inj #3}
\newcommand{\funcsurj}[3]{#1\colon #2 \surj #3}
\newcommand{\irange}[1]{\left[#1\right]}
\newcommand{\join}[2]{#1 \mbox{\hspace{2 pt}$\ast$\hspace{2 pt}} #2}
\newcommand{\djunion}[2]{#1 \mbox{\hspace{2 pt}$+$\hspace{2 pt}} #2}
\newcommand{\parens}[1]{\left( #1 \right)}

\newcommand{\DefinedAs}{\mathrel{\mathop:}=}

\title{Independent transversals via entropy compression (18)}
\author{landon rabern}

\begin{document}
\maketitle
\section{Introduction}
\section{The setup}
Let $G$ be a graph and $\func{\pi}{V(G)}{\irange{k}}$ be a (proper) coloring of $G$.  For $i \in \irange{k}$, put $P_i \DefinedAs \pi^{-1}(i)$.  For each $xy \in E(G)$, recursively define $T_G(xy)$ to be the tree with root $xy$ having as children $T_G(e)$ for all $e \in E(G)$ such that $e$ intersects $P_{\pi(x)} \cup P_{\pi(y)}$.  We call $T_G(xy)$ the \emph{dependency tree} rooted at $xy$.  Note that, by definition, $T_G(xy)$ is a child of $xy$ and in particular $T_G(xy)$ is an infinite tree.

\section{Balanced partitions}
\begin{lem}
If $\card{P_i} \geq 2\Delta(G)$ for all $i \in \irange{k}$, then $\pi$ has an independent transversal.
\end{lem}
\begin{proof}
Suppose the lemma is false. Then $\pi$ has no independent transversal and we may as well have $\card{P_i} = 2\Delta(G)$ for all $i \in \irange{k}$.  Put $\Delta \DefinedAs \Delta(G)$.  Number the vertices of each $P_i$ with $1, \ldots, 2\Delta$. Then a transversal of $\pi$ is represented by an element of $\irange{2\Delta}^k$.  Let $A$ be an arbitrary transversal of $\pi$ and $xy$ an edge contained in $A$.  For $h \in \IN$, let $R_h$ be the collection of subtrees of $T_G(xy)$ rooted at $xy$ having $h$ edges. 

For each $h \in \IN$, we construct an injection $\funcinj{f_h}{\irange{2\Delta}^{2h}}{R_h \times \irange{2\Delta}^k}$.  Fix $\parens{s_1, t_1, s_2, t_2, \ldots, s_h, t_h} \in \irange{2\Delta}^{2h}$.  Put $A_0 \DefinedAs A$, $x_0 \DefinedAs x$ and $y_0 \DefinedAs y$.  For $j \in \irange{h}$, to form $A_j$ replace the $\pi(x_{j-1})$th slot of $A_{j-1}$ with $s_j$ and the $\pi(y_{j-1})$th slot of $A_{j-1}$ with $t_j$.  
\end{proof}

\section{Lopsided partitions}
\end{document}
