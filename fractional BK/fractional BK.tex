\documentclass[12pt]{article}
\usepackage{amsmath, amsthm, amssymb}
\usepackage{hyperref}
\usepackage{verbatim}
\usepackage[top=1.0in, bottom=1.0in, left=1.0in, right=1.0in]{geometry}

\pagestyle{plain}

\usepackage{tkz-graph}
\usetikzlibrary{arrows}
\usetikzlibrary{shapes}
\usepackage[position=bottom]{subfig}

\usepackage{longtable}
\usepackage{array}

\usepackage{sectsty}
\allsectionsfont{\sffamily}

\setcounter{secnumdepth}{5}
\setcounter{tocdepth}{5}

\makeatletter
\newtheorem*{rep@theorem}{\rep@title}
\newcommand{\newreptheorem}[2]{
\newenvironment{rep#1}[1]{
 \def\rep@title{#2 \ref{##1}}
 \begin{rep@theorem}}
 {\end{rep@theorem}}}
\makeatother

\theoremstyle{plain}
\newtheorem{thm}{Theorem}[section]
\newreptheorem{thm}{Theorem}
\newtheorem{prop}[thm]{Proposition}
\newreptheorem{prop}{Proposition}
\newtheorem{lem}[thm]{Lemma}
\newreptheorem{lem}{Lemma}
\newtheorem{conjecture}[thm]{Conjecture}
\newreptheorem{conjecture}{Conjecture}
\newtheorem{cor}[thm]{Corollary}
\newreptheorem{cor}{Corollary}
\newtheorem{prob}[thm]{Problem}
\newtheorem{observation}{Observation}
\newtheorem*{mainconj}{Main Conjecture}
\newtheorem*{mainthm}{Main Theorem}

\theoremstyle{definition}
\newtheorem{defn}{Definition}
\theoremstyle{remark}
\newtheorem*{remark}{Remark}
\newtheorem*{problem}{Problem}
\newtheorem{example}{Example}
\newtheorem*{question}{Question}


\newcommand{\fancy}[1]{\mathcal{#1}}
\newcommand{\C}[1]{\fancy{C}_{#1}}
\newcommand{\IN}{\mathbb{N}}
\newcommand{\IR}{\mathbb{R}}
\newcommand{\G}{\fancy{G}}
\newcommand{\CC}{\fancy{C}}
\newcommand{\D}{\fancy{D}}

\newcommand{\inj}{\hookrightarrow}
\newcommand{\surj}{\twoheadrightarrow}

\newcommand{\set}[1]{\left\{ #1 \right\}}
\newcommand{\setb}[3]{\left\{ #1 \in #2 \mid #3 \right\}}
\newcommand{\setbs}[2]{\left\{ #1 \mid #2 \right\}}
\newcommand{\card}[1]{\left|#1\right|}
\newcommand{\size}[1]{\left\Vert#1\right\Vert}
\newcommand{\ceil}[1]{\left\lceil#1\right\rceil}
\newcommand{\floor}[1]{\left\lfloor#1\right\rfloor}
\newcommand{\func}[3]{#1\colon #2 \rightarrow #3}
\newcommand{\funcinj}[3]{#1\colon #2 \inj #3}
\newcommand{\funcsurj}[3]{#1\colon #2 \surj #3}
\newcommand{\irange}[1]{\left[#1\right]}
\newcommand{\join}[2]{#1 \mbox{\hspace{2 pt}$\ast$\hspace{2 pt}} #2}
\newcommand{\djunion}[2]{#1 \mbox{\hspace{2 pt}$+$\hspace{2 pt}} #2}
\newcommand{\parens}[1]{\left( #1 \right)}
\newcommand{\brackets}[1]{\left[ #1 \right]}
\newcommand{\nint}[1]{\widetilde{N}\left(#1\right)}
\newcommand{\DefinedAs}{\mathrel{\mathop:}=}

\def\adj{\leftrightarrow}
\def\nonadj{\not\!\leftrightarrow}

\def\D{\fancy{D}}
\def\C{\fancy{C}}
\def\Q{\fancy{Q}}
\def\Z{\fancy{Z}}

\newcommand{\bigclique}[1]{\frac{2}{3}\Delta(#1) + 5}
\newcommand{\bigcliqueraw}{\frac{2}{3}\Delta + 5}
\newcommand{\cliqueparts}{\frac{2}{3}\Delta}

% any changes to \claim should be mirrored in \claimnonum and \subclaim
\newcommand{\claim}[2]{{\bf Claim #1.}~{\it #2}~~}
\newcommand{\claimnonum}[1]{{\bf Claim.}~{\it #1}~~}
\newcommand{\subclaim}[2]{{\bf Subclaim #1.}~{\it #2}~~}

\begin{document}
\title{fractional BK}
\maketitle

King, Lu and Peng \cite{king2012fractional} showed that $\chi_f(G) \le 4 - \frac{2}{67}$ when $\Delta(G) \le 4$ and $G$ does not contain $K_4$ or $C_8^2$.  Moreover, they showed that this bound of $4 - \frac{2}{67}$ lifts to larger $\Delta$ by hitting maximum cliques (and a few other structures).  Edwards and King improved this bound for $\Delta \ge 6$ using a probabilistic method, getting an upper bound of $6 - \frac{2}{45}$ for $\Delta=6$.  The goal is to improve these bounds by just improving the $\Delta=4$ case.

Here is the basic form of the argument.  I am going to loosen the bounds for simplicity.  Suppose $\Delta(G) = 4$ and $G$ doesn't contain $K_4$ or $C_8^2$.  Take a $161$-coloring of $G^4$ and let $X$ be a color class.  In $G$, blow up each vertex of $X$ to $K_2$.  The resulting graph $Q$ has $\Delta(Q) \le 5$, where the vertices of degree $5$ are exactly the blown-up vertices and their neighbors.  So, by our choice of $X$, the high vertex subgraph of $Q$ is the disjoint union of graphs of the form $\join{K_2}{T}$ where $|T| = 4$ (or $|T| = 3$ if the blown-up vertex was low, but that case is easier).  Call these components of the high vertex subgraph $H_1, \ldots, H_k$.

\begin{lem}\label{FourColorability}
$Q$ is $4$-colorable.
\end{lem}

First, let's see what Lemma \ref{FourColorability} gets us.

\begin{thm}
$\chi_f(G) \le 4 - \frac{2}{81}$.
\end{thm}
\begin{proof}
For each of the $161$ color classes of $G^4$, Lemma \ref{FourColorability} gives a $4$-coloring of $G$ where each vertex in $X$ gets $2$ colors.  Putting together such colorings for each of the $161$ color classes gives a $162$-fold coloring of $G$ from a pot of $4 * 161$ colors.
\end{proof}

\begin{proof}[Proof of Lemma \ref{FourColorability}]
Suppose not and let $P$ be a $5$-critical subgraph of $Q$.  Let $L$ be the low vertex subgraph of $P$ and $H$ the high vertex subgraph. Since $X$ is independent in $G^4$, for each $v \in L$, there is a $j$ such that $N(v) - L \subseteq V(H_j)$.  Note that there may be vertices from some $H_i$ that are now in $L$, if so they must have all their neighbors in $L$.   Also note that for the blown-up vertices that are still high, all their neighbors are high.

By renumbering if necessary, the components of $H$ are $H_1, \ldots, H_s$ for some $s \le k$.   Say $H_i = \join{K_2}{T_i}$ for each $i$.  We are going to color the $K_2$ from each $H_i$ in such a way that we can complete the coloring on $L$.  There are $6$ subsets of $\irange{4}$ of size $2$, assign one $\set{a,b}$ to each $H_i$, color the $K_2$ with $a$ and $b$ and then color $T_i$ with $\irange{4} - \set{a,b}$ so that the color classes are as imbalanced as possible (so if $T_i$ is $E_4$, only one color is used, if $\alpha(T_i) = 3$, one color is used on three vertices, the other on one).  We will show that there is a way to make such an assignment of $2$-sets to the $H_i$ so that the coloring is completable on $L$.

Remember that each low vertex has neighbors in at most one $H_i$, for $v \in V(L)$, let $h(v)$ be the $i$ such that $v$ has a neighbor in $H_i$ (or $0$ if $v$ has no high neighbors). Given such an assignment to the $H_i$, a component $A$ of $L$ is \emph{good} if 

\begin{enumerate}
\item $A$ has a noncutvertex $v$ with at least two high neighbors colored the same; or
\item $A$ has adjacent noncutvertices $v, w$ such that $H_{h(v)}$ and $H_{h(w)}$ are assigned different $2$-sets.
\end{enumerate}

Such a component is good because we can color greedily towards $v$ or $v,w$ and finish.  Note that if $A$ has only one vertex, say $v$, then it is good for every assignment to the $H_i$ since $v$ will always satisfy (1).

Take an assignment of $2$-sets to the $H_i$ giving the maximum number of good components in $L$.  
If every component is good, then we are good.  So suppose we have a component $A$ of $L$ that is not good.

First, suppose $A$ has a block $B$ with adjacent noncutvertices $v$ and $w$ such that $h(v) \ne h(w)$.



\end{proof}


\bibliographystyle{amsplain}
\bibliography{GraphColoring}
\end{document}
