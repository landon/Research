\documentclass[10pt]{article}
\usepackage[empty]{fullpage}
\usepackage[hidelinks]{hyperref}
\usepackage{enumerate}
\usepackage{graphicx}
\usepackage{color}
\usepackage{hyperref}

\raggedbottom
\raggedright
\setlength{\tabcolsep}{0in}

\addtolength{\oddsidemargin}{-0.5in}
\addtolength{\evensidemargin}{-0.5in}
\addtolength{\textwidth}{1.0in}
\addtolength{\topmargin}{-0.5in}
\addtolength{\textheight}{1.0in}

\def\CC{{C\nolinebreak[4]\hspace{-.05em}\raisebox{.4ex}{\tiny\bf ++}}}
\newcommand{\CS}{C\includegraphics{sharp}}
\newcommand{\FS}{F\includegraphics{sharp}}

\usepackage[svgnames]{xcolor}
\usepackage{framed}
\definecolor{shadecolor}{rgb}{0.91, 0.97, 1.0}
 
\newcommand{\resheading}[1]{
  \parbox{\textwidth}{
    \begin{shaded}
      \textcolor{darkgray}{\hspace{-.05in}\sffamily{\mbox{~}{\large #1}}}
    \end{shaded}
  }
}

\newcommand{\squishlist}{
   \begin{list}{$\circ$}
    { \setlength{\itemsep}{0pt}    \setlength{\parsep}{0pt}
      \setlength{\topsep}{4.5pt}     \setlength{\partopsep}{0pt}
      \setlength{\leftmargin}{2em} \setlength{\labelwidth}{1.5em}
      \setlength{\labelsep}{0.5em} } }
      
\newcommand{\squishend}{
    \end{list}  }

\begin{document}



\begin{tabular*}{7.5in}{l@{\extracolsep{\fill}}r}
& {\footnotesize \textcolor{darkgray}{(805) 403-8185}} \\
& {\footnotesize  \textcolor{darkgray}{landon.rabern@gmail.com}} \\
\textbf{\Large  \sffamily landon rabern} & {\footnotesize  \href{https://landon.github.io/}{\textcolor{darkgray}{https://landon.github.io/}}}\\
\end{tabular*}
\line(1,0){544}

\vspace{0.1in}

\resheading{objective}
to improve the well-being of all concious beings. to get there, teaching myself the needed tools: generative adversarial networks, dna nanotech, chemistry as legos, cryptocontracts, novelty search.
democratize, distribute, and anonymize wealth and information.  enhance and augment human intelligence through nootropics, meditation, prediction of future thoughts/output/actions based on neurofeedback and biofeedback.
non-intrusive man-machine interfaces to amplify the power of human intuitive understanding. make implicit human knowledge explicit through a patchwork of always-evolving domain-specific formal languages.
get more human minds involved by meeting basic survival needs worldwide. an initial project. build a device to simulate some aspects of lucid dreaming inside virtual reality. the necessary components
to do this exist already, they have just not been put together (VR, neurofeedback, biofeedback, generative adversarial networks).
this will be used as a workspace in which to understand and design at the speed of the user's intuition. think of it like the ability some human's have of visualizing and manipulating complex environments.
except now we can record it, multiple users can work collaboratively in the same visualization space, and we can apply general computer science algorithms to the visualized objects.
\vspace{0.1in}

\resheading{work history}
\begin{tabular*}{7.5in}{l@{\extracolsep{\fill}}r}
	\textbf{data scientist: facebook} & 2018 - 2019\\
\end{tabular*}
\begin{minipage}{15cm}
worked with the world.ai team to ingest and digest open street map diffs (python, presto, giraph)\end{minipage}

\vspace{.1in}

\smallskip
 \begin{tabular*}{7.5in}{l@{\extracolsep{\fill}}r}
        \textbf{senior staff engineer: IQVIA} & 2017 - 2018\\
  \end{tabular*}
\begin{minipage}{15cm}
built general diagram of things charting engine with arbitrary depth axis-aligned recursively nested, interactive, 
  animated charts.  chart components and databinding specified by an xml-based markup language used by a team of 100+ 
  engineers in india to build client specific applications (\CS{}, JavaScript)\end{minipage}
 
 \vspace{.1in}
 
	\smallskip
 \begin{tabular*}{7.5in}{l@{\extracolsep{\fill}}r}
        \textbf{cto, co-founder: lbd data} & 2008 - 2018\\
  \end{tabular*}
\begin{minipage}{15cm}
 built a suite of mobile video software for police and public transit. the suite is used throughout the united states. 
(\CS{}, WinForms, WPF, libavcodec, OpenStreetMap, OpenCV, Amazon S3, DynamoDB, SQL, JavaScript, HTML5, \CC{})\end{minipage}

 \vspace{.1in}
 
\smallskip
	 \begin{tabular*}{7.5in}{l@{\extracolsep{\fill}}r}
        \textbf{adjunct assistant professor, mathematics: franklin \& marshall college} & 2014 - 2017 \\
		\end{tabular*}
		\begin{minipage}{15cm} taught math!\end{minipage}
	
	 \vspace{.1in}
	 
		\smallskip
    \begin{tabular*}{7.5in}{l@{\extracolsep{\fill}}r}
        \textbf{senior software engineer: markit on demand} & 2010 - 2011 \\
    \end{tabular*}
\begin{minipage}{15cm} optimized middleware supporting hundreds of developers (\CS{}, \CC{})\end{minipage}
	
	 \vspace{.1in}
	\smallskip
  \begin{tabular*}{7.5in}{l@{\extracolsep{\fill}}r}
        \textbf{kernel engineer: synaptics} & 2009 - 2010 \\
  \end{tabular*}
\begin{minipage}{15cm} improved reliability of touchpad (\CC{})\end{minipage}

 \vspace{.1in}
\smallskip
\begin{tabular*}{7.5in}{l@{\extracolsep{\fill}}r}
	 \textbf{software engineer: markit on demand} & 2007 - 2009 \\
\end{tabular*}
\begin{minipage}{15cm} charts, reports, and tools for the financial services industry (\CS{}, \CC{}, HTML5, JavaScript)\end{minipage}

 \vspace{.1in}
 
\smallskip

\begin{tabular*}{7.5in}{l@{\extracolsep{\fill}}r}
	\textbf{scientific programmer: titan national security} & 2006 - 2007 \\
\end{tabular*}
\begin{minipage}{15cm} created software to model the effects of electromagnetic pulses on military systems (\CC{}, \CS{})\end{minipage}

 \vspace{.1in}
 
\smallskip
\resheading{education}
	\begin{tabular*}{7.5in}{l@{\extracolsep{\fill}}r}
        \textbf{phd, mathematics: arizona state university} & 2011 - 2013 \\
       
  $\circ$ research: discrete math, combinatorics, graph coloring, games and algorithms\\
  $\circ$ dissertation on \href{http://www.openproblemgarden.org/op/the_borodin_kostochka_conjecture}{the borodin-kostochka conjecture}\\
 $\circ$ advisor: hal kierstead
       
    \end{tabular*}
	
	 \vspace{.1in}
	 
\smallskip


 
\begin{tabular*}{7.5in}{l@{\extracolsep{\fill}}r}
	\textbf{ma, mathematics: uc santa barbara} & 2003 - 2005 \\
\end{tabular*}
	$\circ$ research: noncommutative noetherian rings, quantum groups, low dimensional topology
	
	 \vspace{.1in}
	 
\smallskip

\begin{tabular*}{7.5in}{l@{\extracolsep{\fill}}r}
	\textbf{ba, mathematics: washington university in st. louis} & 1999 - 2003 \\
        $\circ$ ross middlemiss prize for top graduating mathematics major\\
	$\circ$ study abroad in the netherlands at utrecht university & 2001 - 2002 \\
\end{tabular*}

\resheading{honors \& activities}
    \begin{tabular*}{7.5in}{l@{\extracolsep{\fill}}r}
        $\circ$ \textbf{erd\H{o}s number 2} & 2011 
    \end{tabular*}
    
    \smallskip
    
	\begin{tabular*}{7.5in}{l@{\extracolsep{\fill}}r}
        $\circ$ \textbf{\$80k in grants from the nsa to extend my phd research} & 2015 - 2017
    \end{tabular*}
    
    \smallskip
    
    \begin{tabular*}{7.5in}{l@{\extracolsep{\fill}}r}
       $\circ$ \textbf{$\mathbf{1^{st}}$ place, mentor graphics state programming competition} & 1997, 1998\\
    \end{tabular*}
    
    \smallskip
    
    \begin{tabular*}{7.5in}{l@{\extracolsep{\fill}}r}
        $\circ$ \textbf{developed betsy, a strong chess ai, in C and x86 assembly}  & 1998 - 2003\\
    \end{tabular*}
    
    \smallskip
    
    \begin{tabular*}{7.5in}{l@{\extracolsep{\fill}}r}
        $\circ$ \textbf{built tesla coils and produced massive lightning bolts} & 1997 - 1999 \\
    \end{tabular*}    

\resheading{research}
\begin{minipage}{17cm}$30^+$ \href{https://landon.github.io/#math}{\color{blue}{publications}} in top-tier discrete mathematics and philosophy journals---including journal of graph theory, journal of combinatorial theory, combinatorica, discrete mathematics, journal of philosophical logic, and analysis.\end{minipage}

\vspace{.2in}

\textbf{favorites:}

\vspace{.1in}

{
- 
\newblock \href{https://landon.github.io/graphdata/Papers/planar%209%20halves.pdf}{\color{blue}{planar graphs are $\frac92$-colorable}}
\newblock {\em  journal of combinatorial theory}, 2017, (with d.w.~cranston).}

\begin{quote}
\begin{minipage}{17cm}this article is about coloring countries on a map so that adjacent countries receive distinct colors. it was conjectured
in 1852 that any map could be colored thusly using only 4 colors. this was finally proved in 1976, but the proof is not human-checkable;
it requires many hours of computer time to check thousands of cases. finding a human-checkable proof is still an open problem.
to prove that 5 colors suffice is relatively simple.  we gave a human-checkable proof that 4.5 colors suffice; this means that
we get to use 9 colors, but have to assign each country 2 colors.\end{minipage}
\squishlist
	\item settled a 20-year old conjecture on the existence of such a proof.
	\item featured on 
\href{http://blog.computationalcomplexity.org/2015/10/a-human-readable-proof-that-every.html}{\color{blue}{computational complexity}}, a popular computer science blog by lance fortnow \& bill gasarch.
\squishend
\end{quote}


\vspace{.2in}

{

- \newblock \href{http://brianrabern.net/sshlpe.pdf}{\color{blue}{a simple solution to the hardest logic puzzle ever}}. 
\newblock {\em analysis}, 68\penalty0 (2), 2008, (with b.~rabern).}

\begin{quote}
\begin{minipage}{17cm} three gods A, B, and C are called, in no particular order, true, false, and random. 
true always speaks truly, false always speaks falsely, but whether random speaks truly or falsely is a completely random matter. 
your task is to determine the identities of A, B, and C by asking three yes-no questions; each question must be put to exactly one god. 
the gods understand english, but will answer all questions in their own language, in which the words for yes and no are da and ja, in some order. 
you do not know which word means which.\end{minipage}
	  \squishlist
		\item showed how to trivialize the puzzle by asking questions that elicit meaningful answers from random.
		\item showed how to solve the puzzle in only two questions by using paradoxes to explode god-heads.
		\item this article led to the problem getting a lot of \href{http://brianrabern.net/New_Scientist_HLPE.pdf}{\color{blue}{press}} 
		and many \href{https://scholar.google.com/scholar?oi=bibs&hl=en&cites=14941881349355280851}{\color{blue}{follow-up papers}} have been written.
      \squishend
\end{quote}

\vspace{1.75in}
\begin{center}
\emph{\textcolor{darkgray}{\small -- references and further information available upon request  --}}\end{center}
\end{document}