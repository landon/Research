\documentclass{article}

\begin{document}
My interest is intelligence. The brain is the only intelligent computer we know of. I want to reverse engineer it. To have any chance of doing so, I need to understand how the brain works in all its gooey details. In particular, the neocortex.

Why do i care about intelligence? More intelligence means more innovation which leads to superabundance and transcending Maslow's hierarchy. Personally, I want more intelligence to help me prove all the discrete math and combinatorics conjectures I have worked on but not solved. I have had some success here. After leaving my math phd program with a master's degree to work in industry, I taught myself graph theory and eventually proved a coloring conjecture of two prominent graph theorists (Kierstead and Kostochka, two of my recommenders). Followup work, embellishing my original idea, led to me getting a free math phd and an Erdős number of 2.  Since then I have published over thirty papers in top tier math journals as well as two philosophy papers. One paper settled a twenty year old conjecture on the existence of a human-grokable proof that planar maps can be 4.5-colored. This was featured on Computational Complexity, a popular computer science blog by Lance Fortnow \& Bill Gasarch.

Two years ago, I stopped work on math and started teaching myself machine learning and data science. I worked with Facebook as a data scientist using machine learning to understand satellite imagery. Now I am working at the research lab of Tokyo Electron. They make the giant machines that deposit atom-thick layers onto silicon wafers for chip manufacturers. One project I am working on is an autonomous controller to replace all the human technicians. 

My data science work has made it clear that we need better techniques. Most techniques in use boil down to just nonlinear function approximation. We need techniques that work more like the cortex. A memory-prediction framework that is efficient. Jeff Hawkins and Numenta have made progress with their Hierarchical Temporal Memory, but there are a lot of missing pieces and it is nowhere near as efficient as the brain. I need to understand the brain better. 

I have a particular interest in the visual cortex. I have aphantasia, a blind mind's eye. I want to figure out the differences between an aphantasia brain and a neurotypical brain. This is a good initial research project. To this end, I have an Emotiv portable EEG that I have been playing with. Just on myself so far, trying to see what happens when I meditate. With better hardware and test subjects, an aphantasia study appears doable.

I can help with any computer programming that needs to be done in the lab. In high-school, I won the state computer programming competition a couple times. Since then I wrote my own graph theory software and in industry worked on simulating electromagnetic pulse from nuclear weapons, stock market analytics, touchpad drivers, video analytics for the police, and data science. 

\end{document}