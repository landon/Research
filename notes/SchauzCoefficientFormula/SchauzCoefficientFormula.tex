\documentclass[12pt]{article}
\usepackage{amsmath, amsthm, amssymb}
\usepackage{hyperref}
\usepackage[top=1.0in, bottom=1.0in, left=1.0in, right=1.0in]{geometry}
\usepackage{hyperref}
\usepackage{mathscinet}
\usepackage{tikz,tkz-graph,tkz-berge}
\usetikzlibrary{positioning}
\usepackage{verbatim, xcolor}

\usetikzlibrary{arrows}
\usetikzlibrary{decorations.markings}
\newcommand{\boundellipse}[3]% center, xdim, ydim
{(#1) ellipse (#2 and #3)
}

\pagestyle{plain}

\usepackage{sectsty}
\allsectionsfont{\sffamily}

\setcounter{secnumdepth}{5}
\setcounter{tocdepth}{5}

\makeatletter
\newtheorem*{rep@theorem}{\rep@title}
\newcommand{\newreptheorem}[2]{
\newenvironment{rep#1}[1]{
 \def\rep@title{#2 \ref{##1}}
 \begin{rep@theorem}}
 {\end{rep@theorem}}}
\makeatother

\theoremstyle{plain}
\newtheorem{thm}{Theorem}[section]
\newreptheorem{thm}{Theorem}
\newtheorem{prop}[thm]{Proposition}
\newreptheorem{prop}{Proposition}
\newtheorem{lem}[thm]{Lemma}
\newreptheorem{lem}{Lemma}
\newtheorem{conjecture}[thm]{Conjecture}
\newreptheorem{conjecture}{Conjecture}
\newtheorem{cor}[thm]{Corollary}
\newreptheorem{cor}{Corollary}
\newtheorem{prob}[thm]{Problem}

\newtheorem*{Theorem}{Theorem}
\newtheorem*{Lemma}{Lemma}
\newtheorem*{Nullstellensatz}{Combinatorial Nullstellensatz}
\newtheorem*{CoefficientFormula}{Coefficient Formula}
\newtheorem*{EulerianOrientationsLemma}{Eulerian Reduction}



\theoremstyle{definition}
\newtheorem{defn}{Definition}
\theoremstyle{remark}
\newtheorem*{remark}{Remark}
\newtheorem*{problem}{Problem}
\newtheorem{example}{Example}
\newtheorem*{question}{Question}
\newtheorem*{observation}{Observation}

\newcommand{\fancy}[1]{\mathcal{#1}}
\newcommand{\C}[1]{\fancy{C}_{#1}}
\newcommand{\IN}{\mathbb{N}}
\newcommand{\IR}{\mathbb{R}}
\newcommand{\G}{\fancy{G}}
\newcommand{\CC}{\fancy{C}}
\newcommand{\D}{\fancy{D}}
\newcommand{\T}{\fancy{T}}
\newcommand{\B}{\fancy{B}}
\renewcommand{\L}{\fancy{L}}
\newcommand{\HH}{\fancy{H}}

\newcommand{\inj}{\hookrightarrow}
\newcommand{\surj}{\twoheadrightarrow}

\newcommand{\set}[1]{\left\{ #1 \right\}}
\newcommand{\setb}[3]{\left\{ #1 \in #2 \mid #3 \right\}}
\newcommand{\setbs}[2]{\left\{ #1 \mid #2 \right\}}
\newcommand{\card}[1]{\left|#1\right|}
\newcommand{\size}[1]{\left\Vert#1\right\Vert}
\newcommand{\ceil}[1]{\left\lceil#1\right\rceil}
\newcommand{\floor}[1]{\left\lfloor#1\right\rfloor}
\newcommand{\func}[3]{#1\colon #2 \rightarrow #3}
\newcommand{\funcinj}[3]{#1\colon #2 \inj #3}
\newcommand{\funcsurj}[3]{#1\colon #2 \surj #3}
\newcommand{\irange}[1]{\left[#1\right]}
\newcommand{\join}[2]{#1 \mbox{\hspace{2 pt}$\ast$\hspace{2 pt}} #2}
\newcommand{\djunion}[2]{#1 \mbox{\hspace{2 pt}$+$\hspace{2 pt}} #2}
\newcommand{\parens}[1]{\left( #1 \right)}
\newcommand{\brackets}[1]{\left[ #1 \right]}
\newcommand{\DefinedAs}{\mathrel{\mathop:}=}

\newcommand{\mic}{\operatorname{mic}}
\newcommand{\AT}{\operatorname{AT}}
\newcommand{\col}{\operatorname{col}}
\newcommand{\ch}{\operatorname{ch}}

\newcommand{\erdos}{Erd\H{o}s}
\newcommand{\CondOne}{\hyperref[cond1]{Condition (1)}}
\newcommand{\CondTwo}{\hyperref[cond2]{Condition (2)}}

\def\adj{\leftrightarrow}
\def\nonadj{\not\!\leftrightarrow}

\newcommand\restr[2]{{% we make the whole thing an ordinary symbol
  \left.\kern-\nulldelimiterspace % automatically resize the bar with \right
  #1 % the function
  \vphantom{\big|} % pretend it's a little taller at normal size
  \right|_{#2} % this is the delimiter
  }}

\def\D{\fancy{D}}
\def\C{\fancy{C}}
\def\A{\fancy{A}}
\def\L{\fancy{L}}
\def\H{\fancy{H}}

\newcommand{\case}[2]{{\bf Case #1.}~{\it #2}~~}
\newcommand{\claim}[2]{{\bf Claim #1.}~{\it #2}~~}
\newcommand{\subclaim}[2]{{\bf Subclaim #1.}~{\it #2}~~}

\title{graph theory notes\thanks{clarifications, errors, simplifications $\Rightarrow$ \texttt{landon.rabern@gmail.com}}\\ \bigskip
The combinatorial nullstellensatz and Schauz's coefficient formula}
\date{}
\begin{document}
\maketitle

In \cite{Alon1992125}, Alon and Tarsi introduced a beautiful algebraic technique for proving the existence of list colorings.  Alon \cite{Alon19997} further developed this technique into the \emph{Combinatorial Nullstellensatz}. Fix an arbitrary field $\mathbb{F}$. We write $f_{k_1, \ldots, k_n}$ for the coefficient of $x_1^{k_1}\cdots x_n^{k_n}$ in the polynomial $f \in \mathbb{F}[x_1, \ldots, x_n]$. 

\begin{Nullstellensatz}[Alon]
Suppose $f \in \mathbb{F}[x_1, \ldots, x_n]$ and $k_1, \ldots, k_n \in \IN$ with $\sum_{i \in \irange{n}} k_i = \deg(f)$.  If $f_{k_1, \ldots, k_n} \ne 0$, then for any $A_1, \ldots, A_n \subseteq \mathbb{F}$ with $\card{A_i} \ge k_i + 1$, there exists $(a_1, \ldots, a_n) \in A_1 \times \cdots \times A_n$ with $f(a_1, \ldots, a_n) \ne 0$.
\end{Nullstellensatz}

Micha{\l}ek \cite{michalek} gave a very short proof of the Combinatorial Nullstellensatz just using long division.  Schauz \cite{schauz2008algebraically} sharpened the Combinatorial Nullstellensatz by proving the following coefficient formula.  Versions of this result were also proved by Hefetz \cite{hefetz2011two} and Laso{\'n} \cite{lason2010generalization}. Our presentation is similar to Laso{\'n}'s.

\begin{CoefficientFormula}[Schauz]
Suppose $f \in \mathbb{F}[x_1, \ldots, x_n]$ and $k_1, \ldots, k_n \in \IN$ with $\sum_{i \in \irange{n}} k_i = \deg(f)$.  For any $A_1, \ldots, A_n \subseteq \mathbb{F}$ with $\card{A_i} = k_i + 1$, we have
\[f_{k_1, \ldots, k_n} = \sum_{(a_1, \ldots, a_n) \in A_1 \times \cdots \times A_n} \frac{f(a_1, \ldots, a_n)}{N(a_1, \ldots, a_n)},\]
where
\[N(a_1, \ldots, a_n) \DefinedAs \prod_{i \in \irange{n}} \prod_{b \in A_i \setminus \set{a_i}} (a_i - b).\]
\end{CoefficientFormula}

We first give Micha{\l}ek's proof of the Combinatorial Nullstellensatz and use this to derive the coefficient formula.

\begin{proof}[Proof of  Combinatorial Nullstellensatz]
Suppose the result is false and choose $f \in \mathbb{F}[x_1, \ldots, x_n]$ for which it fails minimizing $\deg(f)$. Then $\deg(f) \ge 2$ and we have $k_1, \ldots, k_n \in \IN$ with $\sum_{i \in \irange{n}} k_i = \deg(f)$ and $A_1, \ldots, A_n \subseteq \mathbb{F}$ with $\card{A_i} \ge k_i + 1$ such that $f(a_1, \ldots, a_n) = 0$ for all $(a_1, \ldots, a_n) \in A_1 \times \cdots \times A_n$.  By symmetry, we may assume that $k_1 > 0$.  Fix $a \in A_1$ and divide $f$ by $x_1 - a$ to get $f = (x_1 - a)Q + R$ where the degree of $x_1$ in $R$ is zero.  Then the coefficient of $x_1^{k_1-1}x_2^{k_2} \cdots x_n^{k_n}$ in $Q$ must be non-zero and $\deg(Q) < \deg(f)$.  So, by minimality of $\deg(f)$ there is $(a_1, \ldots, a_n) \in (A_1 \setminus \set{a}) \times \cdots \times A_n$ such that $Q(a_1,\ldots, a_n) \ne 0$.  Since $0 = f(a_1,\ldots, a_n) = (a_1 - a)Q(a_1,\ldots, a_n) + R(a_1,\ldots, a_n)$ we must have $R(a_1,\ldots, a_n) \ne 0$.  But $x_1$ has degree zero in $R$, so $R(a,\ldots, a_n) = R(a_1,\ldots, a_n) \ne 0$.  Finally, this means that $f(a,\ldots, a_n) = (a-a)Q(a,\ldots, a_n) + R(a,\ldots, a_n) \ne 0$, a contradiction.
\end{proof}

\begin{proof}[Proof of Coefficient Formula]
Let $f \in \mathbb{F}[x_1, \ldots, x_n]$ and $k_1, \ldots, k_n \in \IN$ with $\sum_{i \in \irange{n}} k_i = \deg(f)$. Also, let $A_1, \ldots, A_n \subseteq \mathbb{F}$ with $\card{A_i} = k_i + 1$.  For each $(a_1, \ldots, a_n) \in A_1 \times \cdots \times A_n$, let $\chi_{(a_1, \ldots, a_n)}$ be the characteristic function of the set $\set{(a_1, \ldots, a_n)}$; that is $\func{\chi_{(a_1, \ldots, a_n)}}{A_1 \times \cdots \times A_n}{\mathbb{F}}$ with $\chi_{(a_1, \ldots, a_n)}(x_1, \ldots, x_n) = 1$ when $(x_1, \ldots, x_n) = (a_1, \ldots, a_n)$ and $\chi_{(a_1, \ldots, a_n)}(x_1, \ldots, x_n) = 0$ otherwise.  Consider the function
\[F = \sum_{(a_1, \ldots, a_n) \in A_1 \times \cdots \times A_n} f(a_1, \ldots, a_n)\chi_{(a_1, \ldots, a_n)}.\]
Then $F$ agrees with $f$ on all of $A_1 \times \cdots \times A_n$ and hence $f - F$ is zero on $A_1 \times \cdots \times A_n$.  We will apply the Combinatorial Nullstellensatz to $f - F$ to conclude that $(f-F)_{k_1, \ldots, k_n} = 0$ and hence $f_{k_1, \ldots, k_n} = F_{k_1, \ldots, k_n}$ where  $F_{k_1, \ldots, k_n}$ will turn out to be our desired sum.  To apply the Combinatorial Nullstellensatz, we need to represent $F$ as a polynomial, we can do so by representing each $\chi_{(a_1, \ldots, a_n)}$ as a polynomial as follows.  For $(a_1, \ldots, a_n) \in A_1 \times \cdots \times A_n$, let 
\[N(a_1, \ldots, a_n) \DefinedAs \prod_{i \in \irange{n}} \prod_{b \in A_i \setminus \set{a_i}} (a_i - b).\]
\noindent Then it is readily verified that
\[\chi_{(a_1, \ldots, a_n)}(x_1, \ldots, x_n) = \frac{\prod_{i \in \irange{n}} \prod_{b \in A_i \setminus \set{a_i}} (x_i - b)}{N(a_1, \ldots, a_n)}.\]
\noindent Using this to define $F$ we get 
\[F(x_1, \ldots, x_n) = \sum_{(a_1, \ldots, a_n) \in A_1 \times \cdots \times A_n} f(a_1, \ldots, a_n)\frac{\prod_{i \in \irange{n}} \prod_{b \in A_i \setminus \set{a_i}} (x_i - b)}{N(a_1, \ldots, a_n)}.\]
Now  $\deg(F) = \sum_{i \in \irange{n}} (|A_i| - 1) = \sum_{i \in \irange{n}} k_i = \deg(f)$.  Since $f - F$ is zero on $A_1 \times \cdots \times A_n$, applying the Combinatorial Nullstellensatz to $f - F$ with $k_1, \ldots, k_n$ and sets $A_1, \ldots, A_n$ gives $(f-F)_{k_1, \ldots, k_n} = 0$ and hence 
\[f_{k_1, \ldots, k_n} = F_{k_1, \ldots, k_n} = \sum_{(a_1, \ldots, a_n) \in A_1 \times \cdots \times A_n} \frac{f(a_1, \ldots, a_n)}{N(a_1, \ldots, a_n)}.\]
\end{proof}

\section{Applications to graph coloring}
Let $G$ be a loopless multigraph with vertex set $V \DefinedAs \set{x_1, \ldots, x_n}$ and edge multiset $E$.  The \emph{graph polynomial} of $G$ is
\[p_G(x_1,\ldots,x_n) \DefinedAs \prod_{\substack{\set{x_i,x_j} \in E\\i < j}} (x_i - x_j).\]
To each orientation $\vec{G}$ of $G$, there is a corresponding monomial $m_{\vec{G}}(x_1,\ldots, x_n)$ given by choosing either $x_i$ or $-x_j$ from each factor $(x_i - x_j)$ according to $\vec{G}$. Precisely, given an orientation $\vec{G}$ of $G$ with edge multiset $\vec{E}$, put
\[m_{\vec{G}}(x_1,\ldots, x_n) \DefinedAs \parens{\prod_{\substack{(x_i,x_j) \in \vec{E}\\i < j}} x_i}\parens{\prod_{\substack{(x_j,x_i) \in \vec{E}\\i < j}} -x_j}.\]
Then $p_G(x_1,\ldots,x_n) = \sum_{\vec{G}} m_{\vec{G}}(x_1,\ldots, x_n)$, where the sum is over all orientations $\vec{G}$ of $G$.  

Each $m_{\vec{G}}(x_1,\ldots, x_n)$ has coefficient either $1$ or $-1$.   We are interested in collecting up all monomials of the form $x_1^{k_1}\cdots x_n^{k_n}$.  Let $DE_{k_1,\ldots, k_n}(G)$ be the orientations of $G$ where $m_{\vec{G}}(x_1,\ldots, x_n) = x_1^{k_1}\cdots x_n^{k_n}$ and $DO_{k_1,\ldots, k_n}(G)$ the orientations of $G$ where $m_{\vec{G}}(x_1,\ldots, x_n) = -x_1^{k_1}\cdots x_n^{k_n}$.  Write $p_{k_1, \ldots, k_n}(G)$ for the coefficient of $x_1^{k_1}\cdots x_n^{k_n}$ in $p_G(x_1,\ldots,x_n)$.  Then we have
\[p_{k_1, \ldots, k_n}(G) = |DE_{k_1,\ldots, k_n}(G)| - |DO_{k_1,\ldots, k_n}(G)|.\]
This gives a combinatorial interpretation of the coefficients of $p_G$, but unfortunately it quantifies over all orientations of $G$.  For applying the Combinatorial Nullstellensatz, it is useful to have a single orientation of $G$ as a certificate that $p_{k_1, \ldots, k_n}(G) \ne 0$.  This can be achieved in terms of Eulerian subgraphs.  A digraph is Eulerian if the in-degree and out-degree are equal at every vertex.  Let $EE(\vec{G})$ be the spanning Eulerian subgraphs of $\vec{G}$ with an even number of edges and let $EO(\vec{G})$ be the spanning Eulerian subgraphs of $\vec{G}$ with an odd number of edges.  

\begin{EulerianOrientationsLemma}
If $\vec{G}$ is an orientation of $G$, then we have
\[\card{|EE(\vec{G})| - |EO(\vec{G})|} = \card{|DE_{d_{\vec{G}}^+(x_1), \ldots, d_{\vec{G}}^+(x_n)}(G)|- |DO_{d_{\vec{G}}^+(x_1), \ldots, d_{\vec{G}}^+(x_n)}(G)|}.\]
\end{EulerianOrientationsLemma}
\begin{proof}
	For $D \in DE_{d_{\vec{G}}^+(x_1), \ldots, d_{\vec{G}}^+(x_n)}(G) \cup DO_{d_{\vec{G}}^+(x_1), \ldots, d_{\vec{G}}^+(x_n)}(G)$, let $\vec{G} \oplus D$ be the spanning subgraph of $\vec{G}$ with edge set 
	\[\setb{(x_i,x_j)}{E(\vec{G})}{(x_j,x_i) \in E(D)}.\]
	Then $\vec{G} \oplus D$ is Eulerian since all vertices have the same out-degree in $\vec{G}$ and $D$.  If $\vec{G}$ is even, this gives bijections between $DE_{d_{\vec{G}}^+(x_1), \ldots, d_{\vec{G}}^+(x_n)}(G)$ and $EE(\vec{G})$ as well as between $DO_{d_{\vec{G}}^+(x_1), \ldots, d_{\vec{G}}^+(x_n)}(G)$ and $EO(\vec{G})$.  When $\vec{G}$ is odd, the bijections are between $DE_{d_{\vec{G}}^+(x_1), \ldots, d_{\vec{G}}^+(x_n)}(G)$ and $EO(\vec{G})$ as well as between $DO_{d_{\vec{G}}^+(x_1), \ldots, d_{\vec{G}}^+(x_n)}(G)$ and $EE(\vec{G})$.  In either case, we have
\[\card{|EE(\vec{G})| - |EO(\vec{G})|} = \card{|DE_{d_{\vec{G}}^+(x_1), \ldots, d_{\vec{G}}^+(x_n)}(G)|- |DO_{d_{\vec{G}}^+(x_1), \ldots, d_{\vec{G}}^+(x_n)}(G)|}.\]
\end{proof} 
Therefore, if $\vec{G}$ is an orientation of $G$ with $d_{\vec{G}}^+(x_i) = k_i$ for all $i \in \irange{n}$, then
\[|p_{k_1, \ldots, k_n}(G)| = \card{|EE(\vec{G})| - |EO(\vec{G})|}.\]

\bibliographystyle{amsplain}
\bibliography{GraphColoring}
\end{document}

 
