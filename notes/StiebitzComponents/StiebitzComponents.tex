\documentclass[12pt]{article}
\usepackage{amsmath, amsthm, amssymb}
\usepackage{hyperref}
\usepackage{verbatim}
\usepackage[top=1.0in, bottom=1.0in, left=1.0in, right=1.0in]{geometry}

\pagestyle{plain}

\usepackage{sectsty}
\allsectionsfont{\sffamily}

\setcounter{secnumdepth}{5}
\setcounter{tocdepth}{5}

\makeatletter
\newtheorem*{rep@theorem}{\rep@title}
\newcommand{\newreptheorem}[2]{
\newenvironment{rep#1}[1]{
 \def\rep@title{#2 \ref{##1}}
 \begin{rep@theorem}}
 {\end{rep@theorem}}}
\makeatother

\theoremstyle{plain}
\newtheorem{thm}{Theorem}
\newreptheorem{thm}{Theorem}
\newtheorem{prop}[thm]{Proposition}
\newreptheorem{prop}{Proposition}
\newtheorem{lem}[thm]{Lemma}
\newreptheorem{lem}{Lemma}
\newtheorem{conjecture}[thm]{Conjecture}
\newreptheorem{conjecture}{Conjecture}
\newtheorem{cor}[thm]{Corollary}
\newreptheorem{cor}{Corollary}
\newtheorem{prob}[thm]{Problem}

\newtheorem*{KernelLemma}{Kernel Lemma}
\newtheorem*{Theorem}{Theorem}
\newtheorem*{Lemma}{Lemma}
\newtheorem*{BK2}{Borodin-Kostochka Conjecture (restated)}
\newtheorem*{Reed}{Reed's Conjecture}
\newtheorem*{ClassificationOfd0}{Classification of $d_0$-choosable graphs}


\theoremstyle{definition}
\newtheorem{defn}{Definition}
\theoremstyle{remark}
\newtheorem*{remark}{Remark}
\newtheorem*{problem}{Problem}
\newtheorem{example}{Example}
\newtheorem*{question}{Question}
\newtheorem*{observation}{Observation}

\newcommand{\fancy}[1]{\mathcal{#1}}
\newcommand{\C}[1]{\fancy{C}_{#1}}
\newcommand{\IN}{\mathbb{N}}
\newcommand{\IR}{\mathbb{R}}
\newcommand{\G}{\fancy{G}}
\newcommand{\CC}{\fancy{C}}
\newcommand{\D}{\fancy{D}}
\newcommand{\T}{\fancy{T}}
\newcommand{\B}{\fancy{B}}
\renewcommand{\L}{\fancy{L}}
\newcommand{\HH}{\fancy{H}}

\newcommand{\inj}{\hookrightarrow}
\newcommand{\surj}{\twoheadrightarrow}

\newcommand{\set}[1]{\left\{ #1 \right\}}
\newcommand{\setb}[3]{\left\{ #1 \in #2 \mid #3 \right\}}
\newcommand{\setbs}[2]{\left\{ #1 \mid #2 \right\}}
\newcommand{\card}[1]{\left|#1\right|}
\newcommand{\size}[1]{\left\Vert#1\right\Vert}
\newcommand{\ceil}[1]{\left\lceil#1\right\rceil}
\newcommand{\floor}[1]{\left\lfloor#1\right\rfloor}
\newcommand{\func}[3]{#1\colon #2 \rightarrow #3}
\newcommand{\funcinj}[3]{#1\colon #2 \inj #3}
\newcommand{\funcsurj}[3]{#1\colon #2 \surj #3}
\newcommand{\irange}[1]{\left[#1\right]}
\newcommand{\join}[2]{#1 \mbox{\hspace{2 pt}$\ast$\hspace{2 pt}} #2}
\newcommand{\djunion}[2]{#1 \mbox{\hspace{2 pt}$+$\hspace{2 pt}} #2}
\newcommand{\parens}[1]{\left( #1 \right)}
\newcommand{\brackets}[1]{\left[ #1 \right]}
\newcommand{\DefinedAs}{\mathrel{\mathop:}=}

\newcommand{\mic}{\operatorname{mic}}
\newcommand{\AT}{\operatorname{AT}}
\newcommand{\col}{\operatorname{col}}
\newcommand{\ch}{\operatorname{ch}}

\def\adj{\leftrightarrow}
\def\nonadj{\not\!\leftrightarrow}

\newcommand\restr[2]{{% we make the whole thing an ordinary symbol
  \left.\kern-\nulldelimiterspace % automatically resize the bar with \right
  #1 % the function
  \vphantom{\big|} % pretend it's a little taller at normal size
  \right|_{#2} % this is the delimiter
  }}

\def\D{\fancy{D}}
\def\C{\fancy{C}}
\def\A{\fancy{A}}
\def\L{\fancy{L}}
\def\H{\fancy{H}}

\newcommand{\case}[2]{{\bf Case #1.}~{\it #2}~~}
\newcommand{\claim}[2]{{\bf Claim #1.}~{\it #2}~~}
\newcommand{\subclaim}[2]{{\bf Subclaim #1.}~{\it #2}~~}

\title{graph theory notes\thanks{clarifications, errors, simplifications $\Rightarrow$ \texttt{landon.rabern@gmail.com}}\\ \bigskip
Stiebitz's proof of Gallai's conjecture on the number of components in the high and low vertex subgraphs of critical graphs}
\date{}
\begin{document}
\maketitle

Tibor Gallai conjectured the following in 1963 \cite{gallai1963kritische, gallai1963kritische2} and Michael Stiebitz proved it in 1982 \cite{stiebitz1982proof}.  For a graph $G$,
let $\L(G)$ be the subgraph of $G$ induced on the vertices of degree $\delta(G)$ and let $\H(G)$ be the subgraph of $G$ induced on the vertices of degree larger than $\delta(G)$.

\begin{Theorem}[Stiebitz]
If $G$ is a color-critical graph with $\delta(G) = \chi(G) - 1$, then $\H(G)$ has at most as many components as $\L(G)$.
\end{Theorem}

In fact, Stiebitz proved a stronger statement.  Theorem follows immediately from Lemma \ref{MainLemma} using $X = V(\L(G))$.  The main induction step in the proof requires a non-trivial fact about bipartite graphs, we save the proof of this fact until later but state it here.  A bipartite graph $G$ with parts $A$ and $B$ has \emph{positive surplus} with respect to $A$ if $|N(S)| > |S|$ for all $\emptyset \ne S \subseteq A$.  Note that $A$ could be empty here, in which case the graph has positive surplus vacuously. 

\begin{lem}[Stiebitz]
Let $G$ be a bipartite graph with parts $A \ne \emptyset$ and $B$ such that $G$ has positive surplus with respect to $A$.   Then there is $x \in A$ such that for any different $y,z \in N(x)$, the bipartite graph $G'$ formed by contracting $\set{y,z}$ and removing $x$ has positive surplus with respect to $A \setminus \set{x}$.
\label{BipartiteLemma}
\end{lem}

Lemma \ref{BipartiteLemma} is used in Claim 2 of the proof of Lemma \ref{WithPositiveSurplus}. Note that following Lemma \ref{WithPositiveSurplus} is trivially true in the $X = \emptyset$ case, we allow this to avoid having to handle a base case where $G[X]$ has one component separately.
\begin{lem}[Stiebitz]
Let $G$ be a graph and $X \subseteq V(G)$ such that
\begin{itemize}
\item $d_G(x) \le k - 1$ for all $x \in X$; and
\item $\chi(G-X) \le k-1$; and
\item for each component $C$ of $G[X]$, we have $\chi(G - V(C)) \le k - 1$.
\end{itemize}
Suppose $G-X$ is the disjoint union of (possibly not connected) graphs $M_1, \ldots, M_{\ell + 1}$ such that the bipartite graph $\B$ formed by contracting each $M_i$ and each component of $G[X]$ has positive surplus with respect to the $G[X]$ side.  If $f_i$ is a $(k-1)$-coloring of $M_i$ for each $i \in \irange{\ell + 1}$, then there are permutations $\pi_1, \ldots, \pi_{\ell + 1}$ of $\irange{k-1}$ such that the $(k-1)$-coloring of $G-X$ given by $(\pi_1 \circ f_1) \cup \cdots \cup (\pi_{\ell + 1} \circ f_{\ell + 1})$ extends to a $(k - 1)$-coloring of $G$.
\label{WithPositiveSurplus}
\end{lem}
\begin{proof}
Suppose the lemma is false and choose a counterexample $G$ and nonempty $X \subseteq V(G)$ so that $|X|$ is as small as possible.  So, $G-X$ is the disjoint union of graphs $M_1, \ldots, M_{\ell + 1}$ and we have $(k-1)$-colorings $f_i$ of $M_i$ for each $i \in \irange{\ell + 1}$ so that no permutations allow us to extend to a $(k - 1)$-coloring of $G$.

\claim{1}{Each non-separating vertex in $G[X]$ has neighbors in at least two of the $M_i$.}
Suppose to the contrary that we have a component $C$ of $G[X]$ and $x \in V(C)$ a non-separating vertex that has neighbors in at most one of the $M_i$.  Let $X' = X \setminus \set{x}$.  The hypotheses of the lemma are satisfied with $X'$ in place of $X$ and $M_i' = G[V(M_i) \cup \set{x}]$ in place of $M_i$ since $\B$ remains the same and hence still has positive surplus. Since $\B$ has positive surplus, we must have $|C| \ge 2$ and hence $x$ has at most $k-2$ neighbors in $G-X$, so we can greedily complete the given $(k-1)$-coloring of $G-X$ to $G-X'$. Applying minimality of $|X|$ to this coloring of $G-X'$, we get permutations that allow us to extend to a $(k - 1)$-coloring of $G$.  But these same permutations also allow us to extend the given $(k-1)$-coloring of $G-X$ to $G$, a contradiction.

\claim{2}{There is a component $C$ of $G[X]$ and $i \ne j$ such that some non-separating $x \in V(C)$ has neighbors in $M_i$ and $M_j$ and the bipartite graph formed from $\B$ by contracting $\set{M_i, M_j}$ and removing $C$ has positive surplus.}

By Lemma \ref{BipartiteLemma}, $G[X]$ has a component $C$ such that for any neighbors $M_i, M_j$ of $C$ in $\B$, the bipartite graph formed from $\B$ by contracting $\set{M_i, M_j}$ and removing $C$ has positive surplus.  Let $x$ be a non-separating vertex in $C$.  By Claim 1, there are different $M_i, M_j$ in which $x$ has neighbors, so $C$ with $x$ and $i,j$ works. 

\claim{3}{The lemma is true.}
Let $C, x,i,j$ be as in Claim 2, by symmetry we may assume $i=1$ and $j=2$. Let $G' = G - V(C)$ and $X' = X \setminus V(C)$.  Then $G' - X'$ is the disjoint union of the $\ell$ graphs $M_1 \cup M_2, M_3, \ldots, M_{\ell + 1}$.  Pick $y_1 \in N(x) \cap V(M_1)$ and $y_2 \in N(x) \cap V(M_2)$. We permute the colors in the coloring of $f_2$ so that $y_1$ and $y_2$ get the same color.  This will save one color for $x$ so that we can greedily color $C$, ending at $x$. Formally, let $\tau$ be a permutation of $\irange{k-1}$ such that $(\tau \circ f_2)(y_2) = f_1(y_1)$ and let $f_* = f_1 \cup (\tau \circ f_2)$.   
By Claim 2 and minimality of $|X|$, we can apply the lemma to $G'$ with $M_1 \cup M_2, M_3, \ldots, M_{\ell + 1}$ and colorings $f_*, f_3, \ldots, f_{\ell +1}$ to get permutations $\pi_*, \pi_3, \ldots, \pi_{\ell+1}$ such that the $(k-1)$-coloring of $G'-X'$ given by $(\pi_* \circ f_*) \cup (\pi_3 \circ f_3) \cup \cdots \cup (\pi_{\ell + 1} \circ f_{\ell + 1})$ extends to a $(k - 1)$-coloring of $G'$.  But this is the same as the $(k-1)$-coloring $(\pi_* \circ f_1) \cup (\pi_* \circ \tau \circ f_2) \cup (\pi_3 \circ f_3) \cup \cdots \cup (\pi_{\ell + 1} \circ f_{\ell + 1})$, so using the permutations $\pi_*, \pi_* \circ \tau, \pi_3, \ldots, \pi_{\ell + 1}$ we get a coloring of $G - X$ that extends to $G - V(C)$.  
In these colorings, $y_1$ and $y_2$ receive the same color. This means that $x$ has $k - 1 - (d_G(x) - d_C(x)) + 1 \ge d_C(x) + 1$ colors available and each other vertex $v$ in $C$ has $k - 1 - (d_G(v) - d_C(v)) + 1 \ge d_C(v) \ge d_C(v)$ colors available.  So, coloring $C$ greedily in order of decreasing distance from $x$ gives an extension to a $(k - 1)$-coloring of $G$, a contradiction.
\end{proof}


Note that, like the previous one, the following lemma is trivially true in the $X = \emptyset$ case, again we allow this to avoid having to handle a base case where $G[X]$ is connected separately.
\begin{lem}[Stiebitz]
Let $G$ be a connected graph and $X \subseteq V(G)$ such that
\begin{itemize}
\item $d_G(x) \le k - 1$ for all $x \in X$; and
\item $\chi(G-X) \le k-1$; and
\item for each component $C$ of $G[X]$, we have $\chi(G - V(C)) \le k - 1$; and
\item $G[X]$ has $\ell$ components and $G-X$ has at least $\ell + 1$ components.
\end{itemize}
Then $G$ is $(k-1)$-colorable.
\label{MainLemma}
\end{lem}
\begin{proof}
Suppose not and choose a counterexample minimizing $|X|$.  If $|X| = 0$, the lemma is trivially true, so we must have $|X| \ge 1$.
Let $\B$ be the bipartite graph formed by contracting each component of $G-X$ and each component of $G[X]$.  If $\B$ has positive surplus with respect to the $G[X]$ side, then applying Lemma \ref{WithPositiveSurplus} to any $(k-1)$-coloring of $G-X$ gives a $(k-1)$-coloring of $G$.  So, we may assume that $\B$ does not have positive surplus.  That is, for some $t$, $G[X]$ has a set of $t$ components $\set{C_1, \ldots, C_t}$ which together have neighbors in at most $t$ components of $G-X$.  But then the other $\ell - t$ components of $G[X]$ together have neighbors in at least $\ell + 1 - t$ components of $G-X$ since $G$ is connected.  Let $X' = X \setminus \bigcup_{i \in \irange{t}} V(C_i)$.  Then the hypotheses of the lemma are satisfied with $X'$ in place of $X$, so minimality of $|X|$ shows that $G$ is $(k-1)$-colorable, a contradiction.
\end{proof}

\section*{Bipartite graphs}
A bipartite graph $G$ with parts $A$ and $B$ is a \emph{2-forest} with respect to $A$ if $G$ is a forest where each vertex in $A$ has degree $2$ in $G$.  Plainly, any $2$-forest has positive surplus.  We first need a few lemmas about bipartite graphs with positive surplus.  It is well-known that the edge-minimal positive surplus bipartite graphs are exactly the $2$-forests (see \cite{LovaszPlummer)}).  More precisely,

\begin{lem}\label{PositiveSurplusIsTwoForest}
Let $G$ be a bipartite graph with parts $A \ne \emptyset$ and $B$ such that $G$ has positive surplus with respect to $A$.  If $G-e$ does not have positive surplus for each $e \in E(G)$, then $G$ is a $2$-forest with respect to $A$.
\end{lem}

The next lemma says that a positive surplus bipartite graph always has a special vertex in $A$ such that removing most of its incident edges leaves a positive surplus graph.

\begin{lem}[Stiebitz]
Let $G$ be a bipartite graph with parts $A \ne \emptyset$ and $B$ such that $G$ has positive surplus with respect to $A$.   Then there is $x \in A$ such that removing any set of $d_G(x) - 2$ edges incident to $x$ leaves a bipartite graph having positive surplus with respect to $A$.
\label{MagicVertexExists}
\end{lem}
\begin{proof}
Suppose not and choose a counterexample $G$ minimizing $||G||$.  First, suppose there is $\emptyset \ne A' \subsetneq A$ such that $|N(A')| \le |A'| + 1$.  Let $G' = G[A' \cup N(A')]$.  Then $G'$ has positive surplus with respect to $A'$ and hence by minimality of $||G||$, there is $x \in A'$ such that removing any set of $d_G(x) - 2$ edges incident to $x$ leaves a bipartite graph having positive surplus with respect to $A'$.  If $W \subseteq A \setminus A'$, then $|W| + |A'| + 1 \le |N_G(W) \cup N_G(A')| \le |N_G(W) \setminus N_G(A')| + |N_G(A')| \le |N_G(W) \setminus N_G(A')| + |A'| + 1$ since $G$ has positive surplus.  Therefore $|N_G(W) \setminus N_G(A')| \ge |W|$ for each $W \subseteq A \setminus A'$.  Since $G$ is a counterexample, we have $y,z \in N(x)$ such that the graph $H$ made by removing all edges incident to $x$ except $xy,xz$ does not have positive surplus with respect to $A$.  That is, we have $S \subseteq A$ such that $|N_H(S)| \le |S|$.  Let $S' = S \cap A'$ and $W = S \setminus S'$.  Since $H[A' \cup N(A')]$ has positive surplus with respect to $A'$, we have $|N_H(S')| \ge |S'| + 1$.  But also, $|N_H(W) \setminus N_H(A')| = |N_G(W) \setminus N_G(A')| \ge |W|$, so $|N_H(S)| = |N_H(S')| + |N_H(W) \setminus N_H(A')| \ge |S'| + 1 + |W| = |S| + 1$, a contradiction.

So, for every $\emptyset \ne A' \subsetneq A$ we must have $|N(A')| \ge |A'| + 2$.  Pick $x \in A$ arbitrarily. Since $G$ is a counterexample, we have $y,z \in N(x)$ such that the graph $H$ made by removing all edges incident to $x$ except $xy, xz$ does not have positive surplus with respect to $A$.  Since $d_G(x) \ge 3$, there is $w \in N_G(x) \setminus \set{y,z}$.  Suppose the bipartite graph $G - xw$ has positive surplus with respect to $A$.  Then, by minimality of $||G||$, there is $u \in A$ such that removing any set of $d_{G-xw}(u) - 2$ edges incident to $u$ from $G-xw$ leaves a bipartite graph having positive surplus with respect to $A$.  If $u \ne x$, then $d_{G-xw}(u) - 2 = d_G(u) - 2$, so $u$ works as the special vertex for $G$, a contradiction.  So, $u=x$ and thus $H$ has positive surplus, a contradiction.

Therefore, there is $\emptyset \ne S \subseteq A$ such that $|N_{G-xw}(S)| \le |S|$.  But $|N_{G-xw}(S)| \ge |N_G(S)| - 1$, so if $S \ne A$, then $|N_{G-xw}(S)| \ge |S| + 2 - 1$.  So, we must have $S=A$ and $|N(A)| = |A| + 1$ as well as $|N_{G-xw}(A)| = |N_G(A)| - 1$ and hence $d_G(w) = 1$.  This was for arbitrary $x \in A$, so we conclude that every $x \in A$ has a neighbor in $B$ of degree $1$ and hence all but at most one vertex in $B$ has degree $1$ in $G$.  But then every vertex in $A$ has degree $2$ in $G$ and hence will work for the special vertex, giving the final contradiction.
\end{proof}

\begin{proof}[Proof of Lemma \ref{BipartiteLemma}]
By Lemma \ref{MagicVertexExists}, there is $x \in A$ such that for any different $y,z \in N(x)$, the bipartite graph $H$ formed from $G$ by removing all edges incident to $x$ except $xz,yz$ has positive surplus with respect to $A$. But then, by Lemma \ref{PositiveSurplusIsTwoForest}, $H$ contains a $2$-forest $F$ with respect to $A$ containing the edges $xy$ and $xz$.
Consider the graph $F'$ formed from $F$ by contracting $\set{y,z}$ and removing $x$.  Then $F'$ is a $2$-forest with respect to $A$ since the degree of vertices in $A$ don't change and $x$ is a separating vertex in $F$.  Now $G'$ contains $F'$ and hence has positive surplus, so we are done.
\end{proof}

\bibliographystyle{amsplain}
\bibliography{GraphColoring}
\end{document}

 
