\documentclass[12pt]{amsart}
\usepackage{amsmath, amsthm, amssymb}
\usepackage[top=1.25in, bottom=1.25in, left=1.0in, right=1.0in]{geometry}

\allowdisplaybreaks
\pagestyle{headings}

\usepackage{tkz-graph}
\usetikzlibrary{arrows}
\usetikzlibrary{shapes}
\usepackage[position=bottom]{subfig}

\makeatletter
\newtheorem*{rep@theorem}{\rep@title}
\newcommand{\newreptheorem}[2]{
\newenvironment{rep#1}[1]{
 \def\rep@title{#2 \ref{##1}}
 \begin{rep@theorem}}
 {\end{rep@theorem}}}
\makeatother

\theoremstyle{plain}
\newtheorem{thm}{Theorem}
\newreptheorem{thm}{Theorem}
\newtheorem{prop}[thm]{Proposition}
\newreptheorem{prop}{Proposition}
\newtheorem{lem}[thm]{Lemma}
\newreptheorem{lem}{Lemma}
\newtheorem{conjecture}[thm]{Conjecture}
\newreptheorem{conjecture}{Conjecture}
\newtheorem{cor}[thm]{Corollary}
\newreptheorem{cor}{Corollary}
\newtheorem{prob}[thm]{Problem}
\theoremstyle{definition}
\newtheorem{defn}{Definition}
\theoremstyle{remark}
\newtheorem*{remark}{Remark}
\newtheorem{example}{Example}
\newtheorem*{question}{Question}
\newtheorem*{observation}{Observation}

\newcommand{\fancy}[1]{\mathcal{#1}}
\newcommand{\C}[1]{\fancy{C}_{#1}}
\newcommand{\IN}{\mathbb{N}}
\newcommand{\IR}{\mathbb{R}}
\newcommand{\G}{\fancy{G}}
\newcommand{\PP}{\mathcal{P}}

\newcommand{\inj}{\hookrightarrow}
\newcommand{\surj}{\twoheadrightarrow}

\newcommand{\set}[1]{\left\{ #1 \right\}}
\newcommand{\setb}[3]{\left\{ #1 \in #2 \mid #3 \right\}}
\newcommand{\setbs}[2]{\left\{ #1 \mid #2 \right\}}
\newcommand{\card}[1]{\left|#1\right|}
\newcommand{\size}[1]{\left\Vert#1\right\Vert}
\newcommand{\ceil}[1]{\left\lceil#1\right\rceil}
\newcommand{\floor}[1]{\left\lfloor#1\right\rfloor}
\newcommand{\func}[3]{#1\colon #2 \rightarrow #3}
\newcommand{\funcinj}[3]{#1\colon #2 \inj #3}
\newcommand{\funcsurj}[3]{#1\colon #2 \surj #3}
\newcommand{\irange}[1]{\left[#1\right]}
\newcommand{\join}[2]{#1 \mbox{\hspace{2 pt}$\ast$\hspace{2 pt}} #2}
\newcommand{\djunion}[2]{#1 \mbox{\hspace{2 pt}$+$\hspace{2 pt}} #2}
\newcommand{\parens}[1]{\left( #1 \right)}
\newcommand{\brackets}[1]{\left[ #1 \right]}
\newcommand{\DefinedAs}{\mathrel{\mathop:}=}
\newcommand{\im}{\operatorname{im}}

\newcommand{\F}{\mathfrak{F}}
\renewcommand{\S}{\mathfrak{S}}
\newcommand{\E}{\mathfrak{E}}
\newcommand{\W}{\fancy{W}}
\title{}

\begin{document}
\maketitle

\textbf{Efficieny as the measure.} In the moment, impotent worry over outcomes might be exchanged for learning to use a general tool, say Redfield enumeration.

\textbf{Semigroup.} A \emph{semigroup} is a set $X$ together with a set of functions $\F$ from $X$ itself.

\textbf{Monoid.} A \emph{monoid} is a semigroup where $\F$ contains the identity function.

\textbf{Group.} A \emph{group} is a monoid where for each $f \in \F$, there is $g \in \F$ so that applying $f$ and then applying $g$ gives the identity function.

\textbf{Danger.} A semigroup is \emph{dangerous} if it has a fixed-point-free element.

\textbf{Graphical.} A semigroup $\S \DefinedAs (X, \F)$ is \emph{graphical} if, for some directed graph $\G \DefinedAs (V, \E)$, there is a representation of $X$ as $X = \prod_{v\in V} \mathbb{Z}_2$ such that $f$ is in $\F$ just in case for each $v \in V$, the function $f^v$ given by $f^v(x) \DefinedAs f(x)_v$ is independent of $V - N^+(v)$.  Such a $\G$ is a \emph{witness} for $\S$.

\textbf{Witnesses}. For each semigroup $\S$, write $\W(\S)$ for the witnesses of $\S$.  Note that if $\S$ is not graphical, then $\W(\S) = \emptyset$.

\textbf{Problem.} Classify the dangerous graphical semigroups by their witness sets.
\end{document}
