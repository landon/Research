\documentclass[12pt]{amsart}
\usepackage{amsmath, amsthm, amssymb}
\usepackage[top=1.25in, bottom=1.25in, left=1.0in, right=1.0in]{geometry}

\pagestyle{headings}

\makeatletter
\newtheorem*{rep@theorem}{\rep@title}
\newcommand{\newreptheorem}[2]{
\newenvironment{rep#1}[1]{
 \def\rep@title{#2 \ref{##1}}
 \begin{rep@theorem}}
 {\end{rep@theorem}}}
\makeatother

\theoremstyle{plain}
\newtheorem{thm}{Theorem}
\newreptheorem{thm}{Theorem}
\newtheorem{prop}[thm]{Proposition}
\newreptheorem{prop}{Proposition}
\newtheorem{lem}[thm]{Lemma}
\newreptheorem{lem}{Lemma}
\newtheorem{conjecture}[thm]{Conjecture}
\newreptheorem{conjecture}{Conjecture}
\newtheorem{cor}[thm]{Corollary}
\newreptheorem{cor}{Corollary}
\newtheorem{prob}[thm]{Problem}
\theoremstyle{definition}
\newtheorem{defn}{Definition}
\theoremstyle{remark}
\newtheorem*{remark}{Remark}
\newtheorem{example}{Example}
\newtheorem*{question}{Question}
\newtheorem*{observation}{Observation}

\newcommand{\fancy}[1]{\mathcal{#1}}
\newcommand{\C}[1]{\fancy{C}_{#1}}
\newcommand{\IN}{\mathbb{N}}
\newcommand{\IR}{\mathbb{R}}
\newcommand{\G}{\fancy{G}}

\newcommand{\inj}{\hookrightarrow}
\newcommand{\surj}{\twoheadrightarrow}

\newcommand{\set}[1]{\left\{ #1 \right\}}
\newcommand{\setb}[3]{\left\{ #1 \in #2 \mid #3 \right\}}
\newcommand{\setbs}[2]{\left\{ #1 \mid #2 \right\}}
\newcommand{\card}[1]{\left|#1\right|}
\newcommand{\size}[1]{\left\Vert#1\right\Vert}
\newcommand{\ceil}[1]{\left\lceil#1\right\rceil}
\newcommand{\floor}[1]{\left\lfloor#1\right\rfloor}
\newcommand{\func}[3]{#1\colon #2 \rightarrow #3}
\newcommand{\funcinj}[3]{#1\colon #2 \inj #3}
\newcommand{\funcsurj}[3]{#1\colon #2 \surj #3}
\newcommand{\irange}[1]{\left[#1\right]}
\newcommand{\join}[2]{#1 \mbox{\hspace{2 pt}$\ast$\hspace{2 pt}} #2}
\newcommand{\djunion}[2]{#1 \mbox{\hspace{2 pt}$+$\hspace{2 pt}} #2}
\newcommand{\parens}[1]{\left( #1 \right)}
\newcommand{\DefinedAs}{\mathrel{\mathop:}=}

\begin{document}

\begin{lem}\label{OrderingLemma}
For any connected graph $G$ and any $z \in V(G)$, there exists a total ordering $\leq$ of $V(G)$ with $z$ minimum, such that $G\left[x \mid x \leq y\right]$ is connected for each $y \in V(G)$.
\end{lem}
\begin{proof}
Let $G$ be a connected graph and $z \in V(G)$.  Let $H$ be a maximal induced subgraph of $G$ which has such an ordering with $z$ minimum.  If $H \neq G$, then, since $G$ is connected, some $w \in V(G) - V(H)$ has an edge into $H$. Thus we may add $w$ to $H$ as the last vertex in the ordering contradicting the maximality of $H$.  Hence $H = G$ and we have our ordering.
\end{proof}

\begin{lem}\label{TwoConnectedHasGoodP3}
Let $G$ be an incomplete $2$-connected graph with $\delta(G) \geq 3$. Then $G$ contains an induced $P_3$, say $abc$, such that $G - a - c$ is connected.
\end{lem}
\begin{proof}
Since $G$ is connected and not complete, it contains induced $P_3$'s. If $G$ is $3$-connected, any induced $P_3$ will do.  Otherwise, let $\set{b, x} \subseteq V(G)$ be a cutset.  Since $G - b$ is not $2$-connected, it has at least two endblocks $B_1, B_2$.  But $G$ is $2$-connected, so $b$ must be adjacent to 
noncutvertices $a \in B_1$ and $c \in B_2$.  Thus $G - a - c$ is connected since $d(b) \geq 3$.  Whence $abc$ is our desired $P_3$.
\end{proof}

\begin{thm}[Brooks 1941]
Every graph with $\Delta \geq 3$ satisfies $\chi \leq \max\set{\omega, \Delta}$.
\end{thm}
\begin{proof}
Suppose not and choose a counterexample $G$ minimizing $\card{G}$.  Plainly, $G$ must be regular, $2$-connected and not complete.  Let $abc$ be the induced $P_3$ guaranteed by Lemma \ref{TwoConnectedHasGoodP3}.  By Lemma \ref{OrderingLemma}, we have an ordering $b, x_1, x_2, \ldots, x_k$ of $V(G - a - c)$ such that $G\left[b, x_1 ,\ldots, x_i\right]$ is connected for each $1 \leq i \leq k$.  Thus, greedily coloring with $V(G)$ ordered $a, c, x_k, x_{k-1}, \ldots, x_1, b$ uses only $\Delta(G)$ colors.
\end{proof}
\end{document}
