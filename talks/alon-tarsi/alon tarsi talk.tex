\documentclass{beamer}
\usepackage{tkz-graph}
\usetikzlibrary{shapes}

%===============BEGIN GLOBAL BEAMER OPTIONS=====================

%THIS CHANGES THE BACKGROUND COLOUR OF BOXES, AND ROUNDS OFF THEIR EDGES
%\setbeamercolor{block title}{bg=yellow!50}
%\setbeamercolor{block body}{}
%\setbeamertemplate{blocks}[default]

%THIS REMOVES NAVIGATION BAR ON THE BOTTOM
\setbeamertemplate{navigation symbols}{}


%%THIS ADDS FRAME NUMBERS TO THE Right (x/y)
%\newcommand*\oldmacro{}
%\let\oldmacro\insertshorttitle
%\renewcommand*\insertshorttitle{
 %\oldmacro\hfill
 %\insertframenumber\,/\,\inserttotalframenumber}

%+++++++++++++ MODES AND THEMES +++++++++
\mode<presentation>

\usetheme{Madrid}

\usecolortheme{whale}
%\usecolortheme{crane}
%\usecolortheme{default}
%\usecolortheme{albatross}
%\usecolortheme{beetle} 
%\usecolortheme{dove} 
%\usecolortheme{fly} 
%\usecolortheme{seagull}
%\usecolortheme{rose}
%\usecolortheme{orchid}
%\usecolortheme{seahorse}



%\useoutertheme{split}
%\useoutertheme{shadow}

% USE TO CREATE PRINTER FRIENDLY HANDOUT VERSION, TOGETHER WITH [HANDOUT]
%ALSO USE ALTERNATE FIRST SLIDE
%\usecolortheme{dove}

%THIS CHANGES THE DEFAULT TRANSPARENCY GRADE FOR UNCOVERING COMMANDS
%\setbeamercovered{transparent=30}

%\beamertemplateshadingbackground{yellow!70}{black!20}

%====================================

\title[Extending Alon-Tarsi Orientations]{Extending Alon-Tarsi Orientations}
\author[landon rabern]{{\large landon rabern}}
\institute[]{~~\\~~\\
Joint with Hal Kierstead\\
Arizona State University\\
\bigskip
\bigskip
\bigskip
AMS Special Session on Structural and Extremal Problems\\
January 15, 2014}
\date{}

\theoremstyle{plain}
\newtheorem{thm}{Theorem}
\newtheorem{prop}[thm]{Proposition}
\newtheorem{lem}[thm]{Lemma}
\newtheorem{cor}[thm]{Corollary}
\newtheorem*{OreBrooks}{Theorem}
\newtheorem*{DeltaTwo}{Theorem}
\newtheorem*{Spectrum}{Theorem}
\newtheorem*{OreBrooksKK}{Theorem}
\newtheorem*{krs1}{Theorem}
\newtheorem*{krs2}{Theorem}
\newtheorem*{BrooksTheorem}{Theorem}
\newtheorem*{MozhansLemma}{Lemma}
\newtheorem*{conjecture}{Conjecture}
\newtheorem*{PartyGame}{A prison problem}
\newtheorem{claim}{Claim}
\theoremstyle{definition}
\newtheorem{defn}{Definition}
\theoremstyle{remark}
\newtheorem*{remark}{Remark}
\newtheorem*{goal}{Goal}
\newtheorem*{question}{Question}
\newtheorem*{observation}{Observation}

\newcommand{\fancy}[1]{\mathcal{#1}}
\newcommand{\C}[1]{\fancy{C}_{#1}}
\newcommand{\IN}{\mathbb{N}}
\newcommand{\IR}{\mathbb{R}}

\newcommand{\inj}{\hookrightarrow}
\newcommand{\surj}{\twoheadrightarrow}

\newcommand{\set}[1]{\left\{ #1 \right\}}
\newcommand{\setb}[3]{\left\{ #1 \in #2 \mid #3 \right\}}
\newcommand{\setbs}[2]{\left\{ #1 \mid #2 \right\}}
\newcommand{\card}[1]{\left|#1\right|}
\newcommand{\size}[1]{\left\Vert#1\right\Vert}
\newcommand{\ceil}[1]{\left\lceil#1\right\rceil}
\newcommand{\floor}[1]{\left\lfloor#1\right\rfloor}
\newcommand{\func}[3]{#1\colon #2 \rightarrow #3}
\newcommand{\funcinj}[3]{#1\colon #2 \inj #3}
\newcommand{\funcsurj}[3]{#1\colon #2 \surj #3}
\newcommand{\irange}[1]{\left[#1\right]}
\newcommand{\join}[2]{#1 \mbox{\hspace{2 pt}$\ast$\hspace{2 pt}} #2}
\newcommand{\djunion}[2]{#1 \mbox{\hspace{2 pt}$+$\hspace{2 pt}} #2}
\newcommand{\parens}[1]{\left( #1 \right)}

\newcommand{\DefinedAs}{\mathrel{\mathop:}=}

%\AtBeginSubsection
%{
%  \begin{frame}<beamer>{Outline}
%    \tableofcontents[currentsection,currentsubsection]
%  \end{frame}
%}

%\AtBeginSection
%{
%  \begin{frame}<beamer>{Outline}
%    \tableofcontents[currentsection,currentsubsection]
%  \end{frame}
%}

\newcommand{\1}{\item<1-> }
\newcommand{\2}{\item<2-> }
\newcommand{\3}{\item<3-> }
\newcommand{\4}{\item<4-> }
\newcommand{\5}{\item<5-> }
\newcommand{\6}{\item<6-> }
\newcommand{\7}{\item<7-> }
\newcommand{\8}{\item<8-> }
\newcommand{\9}{\item<9-> }
\newcommand{\ten}{\item<10-> }
\newcommand{\ele}{\item<11-> }
\newcommand{\twe}{\item<12-> }
\newcommand{\thi}{\item<13-> }
\newcommand{\fou}{\item<14-> }
\newcommand{\fif}{\item<15-> }
\newcommand{\six}{\item<16-> }
\newcommand{\sev}{\item<17-> }
\newcommand{\eig}{\item<18-> }

\newenvironment{dbluenv}{\color[rgb]{.2,.2,.6}}{\normalcolor}
\newcommand<>{\dblue}[1]{\begin{dbluenv}#1\end{dbluenv}}
\newcommand{\bhead}[1]{\textbf{\dblue{#1}}}

\newenvironment{whiteenv}{\color[rgb]{1,1,1}}{\normalcolor}
\newcommand<>{\white}[1]{\begin{whiteenv}#1\end{whiteenv}}
\newenvironment{redenv}{\color[rgb]{.9,0,0}}{\normalcolor}
\newcommand<>{\red}[1]{\begin{redenv}#1\end{redenv}}
\newenvironment{purenv}{\color[rgb]{.8,0,.9}}{\normalcolor}
\newcommand<>{\purple}[1]{\begin{purenv}#1\end{purenv}}
\newenvironment{greenv}{\color[rgb]{0,.8,0}}{\normalcolor}
\newcommand<>{\green}[1]{\begin{greenv}#1\end{greenv}}
\newenvironment{bluenv}{\color[rgb]{0,0,.8}}{\normalcolor}
\newcommand<>{\blue}[1]{\begin{bluenv}#1\end{bluenv}}
\newenvironment{broenv}{\color[rgb]{.58,.35,.2}}{\normalcolor}
\newcommand<>{\brown}[1]{\begin{broenv}#1\end{broenv}}
\newenvironment{oraenv}{\color[rgb]{1,.41,.12}}{\normalcolor}
\newcommand<>{\orange}[1]{\begin{oraenv}#1\end{oraenv}}
\newenvironment{aquenv}{\color[cmyk]{1,0,0,0}}{\normalcolor}
\newcommand<>{\aqua}[1]{\begin{aquenv}#1\end{aquenv}}
\newenvironment{blaenv}{\color[rgb]{0,0,0}}{\normalcolor}
\newcommand<>{\black}[1]{\begin{blaenv}#1\end{blaenv}}
\newenvironment{yelenv}{\color[cmyk]{0,0,1,0}}{\normalcolor}
\newcommand<>{\yellow}[1]{\begin{yelenv}#1\end{yelenv}}

\definecolor{cf9f9f9}{RGB}{249,249,249}

\begin{document}
\setbeamertemplate{caption}{\insertcaption}
\begin{frame}
\titlepage
\end{frame}

\section{Introduction}
\begin{frame}{a motivating problem}{coloring a graph with around $\Delta$ colors}
\begin{overprint} 
\onslide<1> \bhead{by greed:} every graph is \green{$(\Delta + 1)$-}\green{colorable}
\onslide<2> \bhead{by greed:} every graph is \green{$(\Delta + 1)$-}\orange{list-colorable}
\onslide<3> \bhead{by greed:} every graph is \green{$(\Delta + 1)$-}\red{online-list-colorable}
\onslide<4> \bhead{by greed:} every graph is \green{$(\Delta + 1)$-}\red{online-list-colorable}
\onslide<5> \bhead{by greed:} every graph is \green{$(\Delta + 1)$-}\red{online-list-colorable}
\onslide<6> \bhead{by greed:} every graph is \green{$(\Delta + 1)$-}\red{online-list-colorable}
\onslide<7> \bhead{by greed:} every graph is \green{$(\Delta + 1)$-}\red{online-list-colorable}
\onslide<8> \bhead{by greed:} every graph is \green{$(\Delta + 1)$-}\red{online-list-colorable}
\end{overprint}

\bigskip
\uncover<4->{\bhead{with more work:}}

\begin{itemize}
\5 can \green{$\Delta$-color} when no $K_{\Delta+1}$ \scriptsize{(Brooks 1941)}
\6 can \green{$\Delta$-}\orange{list-color} when no $K_{\Delta+1}$ \scriptsize{(Vizing 1976)}
\7 can \green{$\Delta$-}\red{online-list-color} when no $K_{\Delta+1}$ \scriptsize{(Hladk{\`y}-Kr{\'a}l-Schauz 2010)}
\end{itemize}

\bigskip

\uncover<8->{\bhead{hard conjecture:} can \green{$(\Delta-1)$-color} when no $K_{\Delta}$ and $\Delta \ge 9$ \scriptsize{(Borodin-Kostochka 1977)}}
\end{frame}

\begin{frame}{a motivating problem}{between $\Delta$-coloring and $(\Delta-1)$-coloring}
\uncover<1->{\bhead{def:} the \green{ore-degree} \purple{$\theta(G)$} of a graph \green{$G$} is \purple{$\max_{xy \in E(G)} d(x) + d(y)$}}
\begin{itemize}
\2 can \green{$\floor{\frac{\theta}{2}}$-color} when \green{$\theta \ge 12$} and no $K_{\floor{\frac{\theta}{2}} + 1}$ \scriptsize{(Kierstead-Kostochka 2009)}
\3 can \green{$\floor{\frac{\theta}{2}}$-color} when \orange{$\theta \ge 10$} and no $K_{\floor{\frac{\theta}{2}} + 1}$ \scriptsize{(R. 2010)}
\4 can \green{$\floor{\frac{\theta}{2}}$-color} when \red{$\theta \ge 8$} and no $K_{\floor{\frac{\theta}{2}} + 1}$ nor \purple{$O_5$} \scriptsize{(Kostochka-Stiebtiz-R. 2011)}
\end{itemize}
\uncover<4->{
\begin{figure}[h]
\centering
\begin{tikzpicture}[scale = 5]
\tikzstyle{VertexStyle}=[shape = circle,	
								 minimum size = 1pt,
								 inner sep = 1.2pt,
                         draw]
\Vertex[x = 0.270266681909561, y = 0.890800006687641, L = \tiny {}]{v0}
\Vertex[x = 0.336266696453094, y = 0.962799992412329, L = \tiny {}]{v1}
\Vertex[x = 0.334666579961777, y = 0.821199983358383, L = \tiny {}]{v2}
\Vertex[x = 0.56306654214859, y = 0.890800006687641, L = \tiny {}]{v3}
\Vertex[x = 0.244666695594788, y = 0.731600046157837, L = \tiny {}]{v4}
\Vertex[x = 0.417866677045822, y = 0.732000052928925, L = \tiny {}]{v5}
\Vertex[x = 0.243866696953773, y = 0.543200016021729, L = \tiny {}]{v6}
\Vertex[x = 0.415866762399673, y = 0.542800068855286, L = \tiny {}]{v7}
\Vertex[x = 0.0926666706800461, y = 0.890000000596046, L = \tiny {}]{v8}
\tikzstyle{EdgeStyle}=[]
\Edge[](v1)(v0)
\tikzstyle{EdgeStyle}=[]
\Edge[](v2)(v0)
\tikzstyle{EdgeStyle}=[]
\Edge[](v3)(v0)
\tikzstyle{EdgeStyle}=[]
\Edge[](v2)(v1)
\tikzstyle{EdgeStyle}=[]
\Edge[](v3)(v1)
\tikzstyle{EdgeStyle}=[]
\Edge[](v2)(v3)
\tikzstyle{EdgeStyle}=[]
\Edge[](v5)(v4)
\tikzstyle{EdgeStyle}=[]
\Edge[](v6)(v4)
\tikzstyle{EdgeStyle}=[]
\Edge[](v6)(v5)
\tikzstyle{EdgeStyle}=[]
\Edge[](v6)(v7)
\tikzstyle{EdgeStyle}=[]
\Edge[](v7)(v4)
\tikzstyle{EdgeStyle}=[]
\Edge[](v7)(v5)
\tikzstyle{EdgeStyle}=[]
\Edge[](v4)(v8)
\tikzstyle{EdgeStyle}=[]
\Edge[](v6)(v8)
\tikzstyle{EdgeStyle}=[]
\Edge[](v5)(v3)
\tikzstyle{EdgeStyle}=[]
\Edge[](v7)(v3)
\tikzstyle{EdgeStyle}=[]
\Edge[](v0)(v8)
\tikzstyle{EdgeStyle}=[]
\Edge[](v1)(v8)
\tikzstyle{EdgeStyle}=[]
\Edge[](v2)(v8)
\end{tikzpicture}
\caption{the graph \purple{$O_5$}}
\end{figure}}
\end{frame}

\begin{frame}{a motivating problem}{for (online) list coloring}
\uncover<1->{\bhead{problem:} \purple{none} of these methods work for list coloring}\\
\bigskip
\uncover<2->{\bhead{a new method:} combine lower bounds on edges in critical graphs together with Kernel Lemma \scriptsize{(Kostochka-Yancey 2012)}\normalsize }\\
\bigskip
\uncover<3->{\bhead{known bounds:} combining with best known lower bound on edges in list-critical graphs \scriptsize{(Kostochka-Stiebitz 2003)} \normalsize can \green{$\floor{\frac{\theta}{2}}$-}\orange{list-color} when $\theta \ge 56$ and no $K_{\floor{\frac{\theta}{2}} + 1}$}\\
\bigskip
\uncover<4->{\bhead{our improvement:} can \green{$\floor{\frac{\theta}{2}}$}-\red{online-list-color} when \purple{$\theta \ge 18$} and no $K_{\floor{\frac{\theta}{2}} + 1}$}\\
\smallskip

\begin{itemize}
\5 small improvement of Kernel Lemma application
\6 new lower bound on edges in online-list-critical graphs proved via \purple{extending Alon-Tarsi orientations}
\end{itemize}
\end{frame}
\begin{frame}{extending Alon-Tarsi orientations}{many edges or nicely orientable subgraph}
\uncover<1->{\bhead{def:}  A digraph \green{$D$} is \purple{$f$-Alon-Tarsi} for \purple{$\func{f}{V(D)}{\IN}$} if}
\begin{itemize}
\2 $d^{+}(v) < f(v)$ for all $v$; and
\3 $D$ has differing numbers of even and odd spanning eulerian subgraphs
\end{itemize}
\bigskip

\uncover<4->{\bhead{the point:} if \green{$D$} is \purple{$f$-Alon-Tarsi}, then \green{$D$} is \purple{online $f$-choosable} (Schauz 2010)}

\bigskip
\uncover<5->{\bhead{main result:} for a graph \green{$G$} with $\delta(G) \ge 5$ and $K_{\delta(G) + 1} \not \subseteq G$, either}
\begin{itemize}
\6 \green{$G$} has ``lots'' of edges; or
\7 there is an orientation of some induced subgraph \green{$H$} which is \purple{$f$-Alon-Tarsi} where $\purple{f(v) \DefinedAs \delta(G) + d_H(v) - d_G(v)}$
\end{itemize}
\bigskip
\uncover<8->{\bhead{corollary:}  since we can complete a $\delta(G)$-coloring of \green{$G-H$} to such an \green{$H$}, this implies that online-list-critical graphs have ``lots'' of edges}
\end{frame}

\begin{frame}{extending Alon-Tarsi orientations}{key lemma}
\uncover<1->{
\begin{lem}
Let $G$ be a multigraph without loops and $\func{f}{V(G)}{\IN}$. If there are $F \subseteq G$ and
$Y \subseteq V(G)$ such that:
\begin{enumerate}
\item any multiple edges in $G$ are contained in $G[Y]$; and
\item $f(v) \geq d_G(v)$ for all $v \in V(G - Y)$; and
\item $f(v) \geq d_{G[Y]}(v) + d_F(v) + 1$ for all $v \in Y$; and
\item For each component $T$ of $G-Y$ there are different $x_1, x_2 \in V(T)$ where $N_T[x_1] = N_T[x_2]$ and $T - \set{x_1, x_2}$ is connected such that either:
	\begin{enumerate}
	\item there are $x_1y_1, x_2y_2 \in E(F)$ where $y_1 \neq y_2$ and $N(x_i) \cap Y = \set{y_i}$ for $i \in \irange{2}$; or
	\item $\card{N(x_2) \cap Y} = 0$ and there is $x_1y_1 \in E(F)$ where $N(x_1) \cap Y = \set{y_1}$,
	\end{enumerate}
\end{enumerate}

\noindent then $G$ is $f$-Alon-Tarsi.
\end{lem}}

\uncover<2->{\red{\large \bf we need a picture}}
\end{frame}

\begin{frame}{extending Alon-Tarsi orientations}{key lemma in pictures}
\resizebox{\textwidth}{!}{
\includegraphics<1>{extend1}
\includegraphics<2>{extend2}
\includegraphics<3>{extend3}
\includegraphics<4>{extend4}
\includegraphics<5>{extend5}
\includegraphics<6>{extend6}
\includegraphics<7>{extend7}
\includegraphics<8>{extend8}
\includegraphics<9>{extend9}
\includegraphics<10>{extend10}
\includegraphics<11>{extend11}
\includegraphics<12>{extend12}
\includegraphics<13>{extend13}
}
\end{frame}

\begin{frame}{conjectures}{please prove}

\uncover<1->{can \red{$\floor{\frac{\theta}{2}}$-online-list-color} when \orange{$\theta \ge 10$} and no $K_{\floor{\frac{\theta}{2}} + 1}$}\\
\bigskip
\uncover<2->{or even better:}\\
\bigskip
\uncover<3->{can \red{$\floor{\frac{\theta}{2}}$-online-list-color} when \red{$\theta \ge 8$} and no $K_{\floor{\frac{\theta}{2}} + 1}$ nor \purple{$O_5$}}\\
\bigskip
\bigskip
\bigskip
\bigskip
\bigskip
\bigskip
\bigskip
\uncover<4->{\centering \bf{\huge thanks for watching}}
\end{frame}
\end{document}
