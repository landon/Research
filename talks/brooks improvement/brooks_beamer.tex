\documentclass[handout]{beamer}
\usepackage{tkz-graph}
%===============BEGIN GLOBAL BEAMER OPTIONS=====================

%THIS CHANGES THE BACKGROUND COLOUR OF BOXES, AND ROUNDS OFF THEIR EDGES
%\setbeamercolor{block title}{bg=yellow!50}
%\setbeamercolor{block body}{}
%\setbeamertemplate{blocks}[default]

%THIS REMOVES NAVIGATION BAR ON THE BOTTOM
\setbeamertemplate{navigation symbols}{}


%%THIS ADDS FRAME NUMBERS TO THE Right (x/y)
%\newcommand*\oldmacro{}
%\let\oldmacro\insertshorttitle
%\renewcommand*\insertshorttitle{
 %\oldmacro\hfill
 %\insertframenumber\,/\,\inserttotalframenumber}

%+++++++++++++ MODES AND THEMES +++++++++
\mode<presentation>

\usetheme[hideothersubsections]{Hannover}
\addtobeamertemplate{footline}
{
  \usebeamercolor[fg]{author in sidebar}
  \vskip-1cm\hskip2pt
  %\insertpagenumber\,/\,\insertpresentationendpage\kern1em\vskip2pt%
  \insertframenumber\,/\,\inserttotalframenumber\kern1em\vskip2pt%
}

%\usecolortheme{beaver}
%\usecolortheme{crane}
%\usecolortheme{default}
%\usecolortheme{albatross}
%\usecolortheme{beetle} 
%\usecolortheme{dove} 
%\usecolortheme{fly} 
%\usecolortheme{seagull}
%\usecolortheme{rose}
\usecolortheme{orchid}
%\usecolortheme{seahorse}



%\useoutertheme{split}
%\useoutertheme{shadow}

% USE TO CREATE PRINTER FRIENDLY HANDOUT VERSION, TOGETHER WITH [HANDOUT]
%ALSO USE ALTERNATE FIRST SLIDE
%\usecolortheme{dove}

%THIS CHANGES THE DEFAULT TRANSPARENCY GRADE FOR UNCOVERING COMMANDS
%\setbeamercovered{transparent=30}

%\beamertemplateshadingbackground{yellow!70}{black!20}

%====================================

\title[An improvement on Brooks' theorem]{An improvement on Brooks' theorem}
\author{Landon Rabern}
\institute{landon.rabern@gmail.com}
%\date{February 28, 2011}

\theoremstyle{plain}
\newtheorem{thm}{Theorem}
\newtheorem{prop}[thm]{Proposition}
\newtheorem{lem}[thm]{Lemma}
\newtheorem{cor}[thm]{Corollary}
\newtheorem*{OreBrooks}{Theorem}
\newtheorem*{DeltaTwo}{Theorem}
\newtheorem*{Spectrum}{Theorem}
\newtheorem*{OreBrooksKK}{Theorem}
\newtheorem*{krs1}{Theorem}
\newtheorem*{krs2}{Theorem}
\newtheorem*{BrooksTheorem}{Theorem}
\newtheorem*{MozhansLemma}{Lemma}
\newtheorem*{conjecture}{Conjecture}
\newtheorem{claim}{Claim}
\theoremstyle{definition}
\newtheorem{defn}{Definition}
\theoremstyle{remark}
\newtheorem*{remark}{Remark}
\newtheorem*{question}{Question}
\newtheorem*{observation}{Observation}
\newtheorem*{TemptingThought}{A tempting thought}

%\AtBeginSubsection
%{
%  \begin{frame}<beamer>{Outline}
%    \tableofcontents[currentsection,currentsubsection]
%  \end{frame}
%}

%\AtBeginSection
%{
%  \begin{frame}<beamer>{Outline}
%    \tableofcontents[currentsection,currentsubsection]
%  \end{frame}
%}

\newcommand{\1}{\item<1-> }
\newcommand{\2}{\item<2-> }
\newcommand{\3}{\item<3-> }
\newcommand{\4}{\item<4-> }
\newcommand{\5}{\item<5-> }
\newcommand{\6}{\item<6-> }
\newcommand{\7}{\item<7-> }
\newcommand{\8}{\item<8-> }
\newcommand{\9}{\item<9-> }
\newcommand{\ten}{\item<10-> }
\newcommand{\ele}{\item<11-> }
\newcommand{\twe}{\item<12-> }
\newcommand{\thi}{\item<13-> }
\newcommand{\fou}{\item<14-> }
\newcommand{\fif}{\item<15-> }
\newcommand{\six}{\item<16-> }
\newcommand{\sev}{\item<17-> }
\newcommand{\eig}{\item<18-> }

\begin{document}
\begin{frame}
\titlepage
\end{frame}

\begin{frame}{Outline}
  \tableofcontents
\end{frame}

\section{Introduction}
\begin{frame}{Introduction}
\uncover<1->{
\begin{BrooksTheorem}[Brooks 1941]
Every graph with $\Delta \geq 3$ satisfies $\chi \leq \max\{\omega, \Delta\}$.
\end{BrooksTheorem}}

\uncover<2->{
\begin{defn}
The \emph{Ore-degree} of an edge $xy$ in a graph $G$ is $\theta(xy) = d(x) + d(y)$.  The \emph{Ore-degree} of a graph $G$ is $\theta(G) = \max_{xy \in E(G)}\theta(xy)$.
\end{defn}}

\begin{itemize}
\3 every graph satisfies $\left\lfloor\frac{\theta}{2} \right \rfloor \leq \Delta$ \\
\4 greedy coloring (in any order) shows that every graph satisfies 
$\chi \leq \left\lfloor\frac{\theta}{2} \right\rfloor + 1$
\end{itemize}
\end{frame}

\begin{frame}
\uncover<1->{
\begin{OreBrooksKK}[Kierstead and Kostochka 2009]
Every graph with $\theta \geq 12$ satisfies $\chi \leq \max \left\{\omega, \left\lfloor\frac{\theta}{2} \right \rfloor\right\}$.
\end{OreBrooksKK}}

\uncover<2->{
Kierstead and Kostochka \cite{kierstead2009ore} conjectured that the $12$ could be reduced to $10$.  That this would be best possible can be seen from the following example which has $\theta = 9$, $\omega = 4$ and $\chi = 5$.}

\uncover<3->{
\begin{figure}[h]
\centering
\begin{tikzpicture}[scale = 5]
\tikzstyle{VertexStyle}=[shape = circle,	
								 minimum size = 1pt,
								 inner sep = 1.2pt,
                         draw]
\Vertex[x = 0.270266681909561, y = 0.890800006687641, L = \tiny {4}]{v0}
\Vertex[x = 0.336266696453094, y = 0.962799992412329, L = \tiny {4}]{v1}
\Vertex[x = 0.334666579961777, y = 0.821199983358383, L = \tiny {4}]{v2}
\Vertex[x = 0.56306654214859, y = 0.890800006687641, L = \tiny {5}]{v3}
\Vertex[x = 0.244666695594788, y = 0.731600046157837, L = \tiny {4}]{v4}
\Vertex[x = 0.417866677045822, y = 0.732000052928925, L = \tiny {4}]{v5}
\Vertex[x = 0.243866696953773, y = 0.543200016021729, L = \tiny {4}]{v6}
\Vertex[x = 0.415866762399673, y = 0.542800068855286, L = \tiny {4}]{v7}
\Vertex[x = 0.0926666706800461, y = 0.890000000596046, L = \tiny {5}]{v8}
\tikzstyle{EdgeStyle}=[]
\Edge[](v1)(v0)
\tikzstyle{EdgeStyle}=[]
\Edge[](v2)(v0)
\tikzstyle{EdgeStyle}=[]
\Edge[](v3)(v0)
\tikzstyle{EdgeStyle}=[]
\Edge[](v2)(v1)
\tikzstyle{EdgeStyle}=[]
\Edge[](v3)(v1)
\tikzstyle{EdgeStyle}=[]
\Edge[](v2)(v3)
\tikzstyle{EdgeStyle}=[]
\Edge[](v5)(v4)
\tikzstyle{EdgeStyle}=[]
\Edge[](v6)(v4)
\tikzstyle{EdgeStyle}=[]
\Edge[](v6)(v5)
\tikzstyle{EdgeStyle}=[]
\Edge[](v6)(v7)
\tikzstyle{EdgeStyle}=[]
\Edge[](v7)(v4)
\tikzstyle{EdgeStyle}=[]
\Edge[](v7)(v5)
\tikzstyle{EdgeStyle}=[]
\Edge[](v4)(v8)
\tikzstyle{EdgeStyle}=[]
\Edge[](v6)(v8)
\tikzstyle{EdgeStyle}=[]
\Edge[](v5)(v3)
\tikzstyle{EdgeStyle}=[]
\Edge[](v7)(v3)
\tikzstyle{EdgeStyle}=[]
\Edge[](v0)(v8)
\tikzstyle{EdgeStyle}=[]
\Edge[](v1)(v8)
\tikzstyle{EdgeStyle}=[]
\Edge[](v2)(v8)
\end{tikzpicture}

\caption{$O_5$, a counterexample with $\theta = 9$.}
\end{figure}}
\end{frame}

\section{Rephrasing the problem}
\begin{frame}{Rephrasing the problem}

\begin{itemize}
\1 let $G$ be a critical graph with $\chi = \left\lfloor\frac{\theta}{2} \right\rfloor + 1$ \\
\2 it follows that $G$ must satisfy $\theta \leq 2\chi - 1$ \\
\3 if $\Delta < \chi$ we are done by Brooks' theorem \\
\4 otherwise we have $\theta \geq \delta + \Delta \geq 2\chi - 1$ giving $\theta = 2\chi - 1$ \\
\5 thus, $\chi = \Delta$ and no two vertices of max degree in $G$ can be adjacent \\
\end{itemize}
\end{frame}

\begin{frame}
\uncover<1->{
\begin{defn}
Let $G$ be a graph.  The low vertex subgraph $\mathcal{L}(G)$ is the graph induced on the vertices of degree $\chi(G) - 1$.  The high vertex subgraph $\mathcal{H}(G)$ is the graph induced on the vertices of degree at least $\chi(G)$.
\end{defn}}

\uncover<2->{
\begin{problem}
Prove that $K_{\Delta(G) + 1}$ is the only critical graph $G$ with $\chi(G) \geq \Delta(G) \geq 6$ such that $\mathcal{H}(G)$ is edgeless.
\end{problem}}
\end{frame}

\section{Solving the rephrased problem}
\subsection{Kierstead and Kostochka's proof}
\begin{frame}{Kierstead and Kostochka's proof}
\begin{itemize}
\1 take a minimal counterexample $G$ and use minimality to prove some structural properties \\
\2 $\mathcal{H}(G)$ has at most as many components as $\mathcal{L}(G)$ by a result of Stiebitz \cite{stiebitz1982proof} \\
\3 since $\mathcal{H}(G)$ is edgeless it has at most as many vertices as $\mathcal{L}(G)$ has components \\
\4 apply Alon and Tarsi's algebraic list coloring theorem to an auxilliary bipartite graph \\
\5 do some counting and get a contradiction \\
\6 it only works for $\theta \geq 12$ \\
\end{itemize}
\end{frame}
\subsection{Problem solved}
\begin{frame}
\uncover<1->{
In \cite{rabern2010a} we solved the problem in a more general fashion.}
\uncover<2->{
\begin{OreBrooks}[Rabern 2010]
$K_{\Delta(G) + 1}$ is the only critical graph $G$ with $\chi(G) \geq \Delta(G) \geq 6$ and $\omega(\mathcal{H}(G)) \leq \left \lfloor \frac{\Delta(G)}{2} \right \rfloor - 2$.
\end{OreBrooks}

Setting $\omega(\mathcal{H}(G)) = 1$ proves the conjecture of Kierstead and Kostochka.}
\end{frame}
\subsection{Proof outline}
\begin{frame}{Proof outline}
\begin{itemize}
\1 take a minimal counterexample $G$ and use minimality to prove some structural properties  \\
\2 run a carefully chosen recoloring algorithm to construct a large ``dense'' subgraph $H$ \\
\3 inductively $\Delta - 1$ color $G - H$ \\
\4 use minimality of $G$ to show that the $\Delta - 1$ coloring can be completed to $H$ \\
\end{itemize}
\end{frame}
\subsection{Mozhan's lemma}
\begin{frame}{Partitioned colorings}
\begin{defn}
Let $G$ be a vertex critical graph.  Let $a \geq 1$ and $r_1, \ldots, r_a$ be such that $1 + \sum_i r_i = \chi(G)$. By a \emph{$(r_1, \ldots, r_a)$-partitioned coloring} of $G$ we mean a proper coloring of $G$ of the form
\[\left\{\{x\}, L_{11}, L_{12}, \ldots, L_{1r_1}, L_{21}, L_{22}, \ldots, L_{2r_2}, \ldots, L_{a1}, L_{a2}, \ldots, L_{ar_a}\right\}.\]

Here $\{x\}$ is a singleton color class and each $L_{ij}$ is a color class.
\end{defn}
\end{frame}

\begin{frame}{Mozhan's Lemma}
\begin{MozhansLemma}[Mozhan 1983]
Let $G$ be a vertex critical graph.   Let $a \geq 1$ and $r_1, \ldots, r_a$ be such that $1 + \sum_i r_i = \chi(G)$. Of all $(r_1, \ldots, r_a)$-partitioned colorings of $G$ pick one minimizing

\[\sum_{i = 1}^a \left|E\left(G\left[\bigcup_{j = 1}^{r_i} L_{ij}\right]\right)\right|.\]

Remember that $\{x\}$ is a singleton color class in the coloring. Put $U_i = \bigcup_{j = 1}^{r_i} L_{ij}$ and let $Z_i(x)$ be the component of $x$ in $G[\{x\} \cup U_i]$.  If $d_{Z_i(x)}(x) = r_i$, then $Z_i(x)$ is complete if $r_i \geq 3$ and $Z_i(x)$ is an odd cycle if $r_i = 2$.
\end{MozhansLemma}
\end{frame}

\subsection{The recoloring algorithm}
\begin{frame}{The recoloring algorithm}
\begin{itemize}
\1 take a  $(\left \lfloor \frac{\Delta - 1}{2} \right \rfloor, \left \lceil \frac{\Delta - 1}{2} \right \rceil)$-partitioned coloring minimizing the above function \\
\2 prove that we may assume that $x$ is a low vertex \\
\3 by Mozhan's lemma, the neighborhood of $x$ in each part induces a clique or an odd cycle \\
\4 swap $x$ with a low vertex $x_1$ in the right part \\
\5 swap $x_1$ with a low vertex $x_2$ in the left part \\
\6 continue swapping back and forth until you wrap around \\
\7 use the fact that you wrapped around to show that there are many edges between the two induced cliques (odd cycles) \\
\8 we have now constructed the desired large ``dense'' subgraph \\
\end{itemize}
\end{frame}
\section{A spectrum of generalizations}
\subsection{Generalizing maximum degree}
\begin{frame}{Generalizing maximum degree}
\uncover<1->{
\begin{defn}
For $0 \leq \epsilon \leq 1$, define $\Delta_\epsilon(G)$ as

\[\left\lfloor\max_{xy \in E(G)} (1 - \epsilon)\min\{d(x), d(y)\} + \epsilon\max\{d(x), d(y)\}\right\rfloor.\]
\end{defn}
}
\uncover<2->{
Note that $\Delta_1 = \Delta$, $\Delta_{\frac12} = \left\lfloor\frac{\theta}{2} \right\rfloor$.
}
\end{frame}
\subsection{The generalized bound}
\begin{frame}{The generalized bound}
\uncover<1->{
\begin{Spectrum}[Rabern 2010]
For every $0 < \epsilon \leq 1$, there exists $t_\epsilon$ such that every graph with $\Delta_\epsilon \geq t_\epsilon$ satisfies 

\[\chi \leq \max\{\omega, \Delta_\epsilon\}.\]
\end{Spectrum}
}
\begin{itemize}
\2 the proof uses a recoloring algorithm similar to the above \\
\3 it would be interesting to determine, for each $\epsilon$, the smallest $t_\epsilon$ that works \\
\4 that $t_1 = 3$ is smallest is Brooks' theorem \\
\5 the graph $O_5$ shows that $t_\epsilon = 6$ is smallest for $\frac12 \leq \epsilon < 1$ \\
\6 we will see below that if \emph{P} $\neq$ \emph{NP}, then $t_0$ does not exist and hence $t_\epsilon \rightarrow \infty$ as $\epsilon \rightarrow 0$
\end{itemize}
\end{frame}
\subsection{What about $\Delta_0$?}
\begin{frame}{What about $\Delta_0$?}
\begin{itemize}
\1 the above proofs only work for $\epsilon > 0$ \\
\2 what happens when $\epsilon = 0$? \\
\3 the parameter $\Delta_0$ has already been investigated by Stacho \cite{stacho2001new} under the name $\Delta_2$ \\
\end{itemize}
\uncover<4->{
\begin{defn}[Stacho 2001]
For a graph $G$ define

\[\Delta_2(G) = \max_{xy \in E(G)} \min\{d(x), d(y)\}.\]
\end{defn}
}
\end{frame}
\begin{frame}{Facts about $\Delta_2$}
\begin{itemize}
\1 $\Delta_2 = \Delta_0$ \\
\2 greedy coloring (in any order) shows that every graph satisfies $\chi \leq \Delta_2 + 1$ \\
\3 for any fixed $t \geq 3$, the problem of determining whether or not $\chi(G) \leq \Delta_2(G)$ for graphs with $\Delta_2(G) = t$ is \emph{NP}-complete (see \cite{stacho2001new})
\end{itemize}
\end{frame}

\begin{frame}{A tempting thought}
\uncover<1->{
\begin{TemptingThought}
There exists $t$ such that every graph with $\Delta_2 \geq t$ satisfies $\chi \leq \max \{\omega, \Delta_2\}$.
\end{TemptingThought}}

\begin{itemize}
\2 unfortunately, the tempting thought cannot hold for any $t$ if \emph{P}$\neq$\emph{NP} \\
\3 to show this, we use Lov\'{a}sz's $\vartheta$ parameter \cite{gr�tschel1981ellipsoid} which can be appoximated in polynomial time and has the property that $\omega(G) \leq \vartheta(G) \leq \chi(G)$ \\
\end{itemize}
\end{frame}

\begin{frame}{A polynomial-time algorithm}
\begin{itemize}
\1 assume the tempting thought holds for some $t \geq 3$ \\
\2 take any arbitrary graph with $\Delta_2 \geq t$ \\
\3 first, compute $\Delta_2$ in polynomial time \\
\4 second, compute $x$ such that $x - \frac12 < \vartheta < x + \frac12$ in polynomial time \\
\5 if $x \geq \Delta_2 + \frac12$, then $\chi \geq \vartheta > \Delta_2$ and hence $\chi = \Delta_2 + 1$ \\
\6 if $x < \Delta_2 + \frac12$, then $\omega \leq \vartheta < \Delta_2 + 1$, and hence $\omega \leq \Delta_2$ \\
\7  now, $\chi \leq \max \{\omega, \Delta_2\} \leq \Delta_2$ \\
\8 we just gave a polynomial time algorithm to determine whether or not $\chi \leq \Delta_2$ for graphs with $\Delta_2 \geq t$ \\
\9 this is impossible unless \emph{P}=\emph{NP} \\
\end{itemize}
\end{frame}

\begin{frame}{What we can prove about $\Delta_0$ (aka $\Delta_2$)}
\uncover<1->{
\begin{DeltaTwo}[Rabern 2010]
Every graph with $\Delta \geq 3$ satisfies \[\chi \leq \max \left\{\omega, \Delta_2, \frac{5}{6}(\Delta + 1)\right\}.\]
\end{DeltaTwo}}

\begin{itemize}
\2 the proof uses a recoloring algorithm similar to the above \\
\3 actually, all the above results about $\Delta_\epsilon$ follow from this result \\
\end{itemize}
\end{frame}

\section{Further improvements}
\begin{frame}
\uncover<1->{
In joint work with Kostochka and Stiebitz \cite{krs_one} similar techniques were used to improve the bounds further.  We give some highlights.}

\uncover<2->{
\begin{krs1}[Kostochka, Rabern and Stiebitz 2010]
Every graph with $\theta \geq 8$, except $O_5$, satisfies $\chi \leq \max \left\{\omega, \left\lfloor\frac{\theta}{2} \right \rfloor\right\}$.
\end{krs1}}

\uncover<3->{
\begin{krs2}[Kostochka, Rabern and Stiebitz 2010]
Every graph satisfies \[\chi \leq \max \left\{\omega, \Delta_2, \frac{3}{4}(\Delta + 2)\right\}.\]
\end{krs2}}
\end{frame}

\nocite{rabern2010b}
\bibliographystyle{amsplain}
\tiny
\bibliography{GraphColoring}
\end{document}
