\documentclass{beamer}
\usepackage{tkz-graph}
%===============BEGIN GLOBAL BEAMER OPTIONS=====================

%THIS CHANGES THE BACKGROUND COLOUR OF BOXES, AND ROUNDS OFF THEIR EDGES
%\setbeamercolor{block title}{bg=yellow!50}
%\setbeamercolor{block body}{}
%\setbeamertemplate{blocks}[default]

%THIS REMOVES NAVIGATION BAR ON THE BOTTOM
\setbeamertemplate{navigation symbols}{}


%%THIS ADDS FRAME NUMBERS TO THE Right (x/y)
%\newcommand*\oldmacro{}
%\let\oldmacro\insertshorttitle
%\renewcommand*\insertshorttitle{
 %\oldmacro\hfill
 %\insertframenumber\,/\,\inserttotalframenumber}

%+++++++++++++ MODES AND THEMES +++++++++
\mode<presentation>

\usetheme[hideothersubsections]{Hannover}
\addtobeamertemplate{footline}
{
  \usebeamercolor[fg]{author in sidebar}
  \vskip-1cm\hskip2pt
  %\insertpagenumber\,/\,\insertpresentationendpage\kern1em\vskip2pt%
  \insertframenumber\,/\,\inserttotalframenumber\kern1em\vskip2pt%
}

%\usecolortheme{beaver}
%\usecolortheme{crane}
%\usecolortheme{default}
%\usecolortheme{albatross}
%\usecolortheme{beetle} 
%\usecolortheme{dove} 
%\usecolortheme{fly} 
%\usecolortheme{seagull}
%\usecolortheme{rose}
\usecolortheme{orchid}
%\usecolortheme{seahorse}



%\useoutertheme{split}
%\useoutertheme{shadow}

% USE TO CREATE PRINTER FRIENDLY HANDOUT VERSION, TOGETHER WITH [HANDOUT]
%ALSO USE ALTERNATE FIRST SLIDE
%\usecolortheme{dove}

%THIS CHANGES THE DEFAULT TRANSPARENCY GRADE FOR UNCOVERING COMMANDS
%\setbeamercovered{transparent=30}

%\beamertemplateshadingbackground{yellow!70}{black!20}

%====================================

\title[An Improvement On Brooks' Theorem]{An Improvement On Brooks' Theorem}
\author{Landon Rabern}
\date{February 28, 2011}

\theoremstyle{plain}
\newtheorem{thm}{Theorem}
\newtheorem{prop}[thm]{Proposition}
\newtheorem{lem}[thm]{Lemma}
\newtheorem{cor}[thm]{Corollary}
\newtheorem*{OreBrooks}{Rabern (2010)}
\newtheorem*{DeltaTwo}{Rabern (2010)}
\newtheorem*{OreBrooksKK}{Kierstead and Kostochka (2009)}
\newtheorem*{krs1}{Kostochka, Rabern and Stiebitz (2010)}
\newtheorem*{krs2}{Kostochka, Rabern and Stiebitz (2010)}
\newtheorem*{BrooksTheorem}{Brooks (1941)}
\newtheorem*{MozhansLemma}{Mozhan (1983)}
\newtheorem*{conjecture}{Conjecture}
\newtheorem{claim}{Claim}
\theoremstyle{definition}
\newtheorem{defn}{Definition}
\theoremstyle{remark}
\newtheorem*{remark}{Remark}
\newtheorem*{question}{Question}
\newtheorem*{observation}{Observation}
\newtheorem*{TemptingThought}{Tempting Thought}

\AtBeginSubsection
{
  \begin{frame}<beamer>{outline}
    \tableofcontents[currentsection,currentsubsection]
  \end{frame}
}

\AtBeginSection
{
  \begin{frame}<beamer>{outline}
    \tableofcontents[currentsection,currentsubsection]
  \end{frame}
}



\begin{document}


\begin{frame}
\titlepage
\end{frame}

\section{Introduction}
\begin{frame}{Introduction}
\uncover<1->{Brooks' Theorem \cite{brooks1941colouring} gives an upper bound on a graph's chromatic number in terms of its maximum degree and clique number.}  

\uncover<2->{
\begin{BrooksTheorem}
Every graph with $\Delta \geq 3$ satisfies $\chi \leq \max\{\omega, \Delta\}$.
\end{BrooksTheorem}}

\uncover<3->{
In \cite{kierstead2009ore}, Kierstead and Kostochka showed that Brooks' theorem can be tightened by looking at the maximum Ore-degree of edges in a graph.}

\uncover<4->{
\begin{defn}
The \emph{Ore-degree} of an edge $xy$ in a graph $G$ is $\theta(xy) = d(x) + d(y)$.  The \emph{Ore-degree} of a graph $G$ is $\theta(G) = \max_{xy \in E(G)}\theta(xy)$.
\end{defn}

Note that every graph satisfies $\left\lfloor\frac{\theta}{2} \right \rfloor \leq \Delta$.}
\end{frame}

\begin{frame}
\uncover<1->{
By coloring greedily (in any order), we see that every graph satisfies 
$\chi \leq \left\lfloor\frac{\theta}{2} \right\rfloor + 1$. Kierstead and Kostochka improved this as follows and conjectured that the $12$ could be reduced to $10$.}

\uncover<2->{
\begin{OreBrooksKK}
Every graph with $\theta \geq 12$ satisfies $\chi \leq \max \left\{\omega, \left\lfloor\frac{\theta}{2} \right \rfloor\right\}$.
\end{OreBrooksKK}}

\uncover<3->{
That $10$ would be best possible can be seen from the following example which has $\theta = 9$, $\omega = 4$ and $\chi = 5$.
\begin{figure}[h]
\centering
\begin{tikzpicture}[scale = 5]
\tikzstyle{VertexStyle}=[shape = circle,	
								 minimum size = 1pt,
								 inner sep = 1.2pt,
                         draw]
\Vertex[x = 0.270266681909561, y = 0.890800006687641, L = \tiny {4}]{v0}
\Vertex[x = 0.336266696453094, y = 0.962799992412329, L = \tiny {4}]{v1}
\Vertex[x = 0.334666579961777, y = 0.821199983358383, L = \tiny {4}]{v2}
\Vertex[x = 0.56306654214859, y = 0.890800006687641, L = \tiny {5}]{v3}
\Vertex[x = 0.244666695594788, y = 0.731600046157837, L = \tiny {4}]{v4}
\Vertex[x = 0.417866677045822, y = 0.732000052928925, L = \tiny {4}]{v5}
\Vertex[x = 0.243866696953773, y = 0.543200016021729, L = \tiny {4}]{v6}
\Vertex[x = 0.415866762399673, y = 0.542800068855286, L = \tiny {4}]{v7}
\Vertex[x = 0.0926666706800461, y = 0.890000000596046, L = \tiny {5}]{v8}
\tikzstyle{EdgeStyle}=[]
\Edge[](v1)(v0)
\tikzstyle{EdgeStyle}=[]
\Edge[](v2)(v0)
\tikzstyle{EdgeStyle}=[]
\Edge[](v3)(v0)
\tikzstyle{EdgeStyle}=[]
\Edge[](v2)(v1)
\tikzstyle{EdgeStyle}=[]
\Edge[](v3)(v1)
\tikzstyle{EdgeStyle}=[]
\Edge[](v2)(v3)
\tikzstyle{EdgeStyle}=[]
\Edge[](v5)(v4)
\tikzstyle{EdgeStyle}=[]
\Edge[](v6)(v4)
\tikzstyle{EdgeStyle}=[]
\Edge[](v6)(v5)
\tikzstyle{EdgeStyle}=[]
\Edge[](v6)(v7)
\tikzstyle{EdgeStyle}=[]
\Edge[](v7)(v4)
\tikzstyle{EdgeStyle}=[]
\Edge[](v7)(v5)
\tikzstyle{EdgeStyle}=[]
\Edge[](v4)(v8)
\tikzstyle{EdgeStyle}=[]
\Edge[](v6)(v8)
\tikzstyle{EdgeStyle}=[]
\Edge[](v5)(v3)
\tikzstyle{EdgeStyle}=[]
\Edge[](v7)(v3)
\tikzstyle{EdgeStyle}=[]
\Edge[](v0)(v8)
\tikzstyle{EdgeStyle}=[]
\Edge[](v1)(v8)
\tikzstyle{EdgeStyle}=[]
\Edge[](v2)(v8)
\end{tikzpicture}

\caption{$O_5$, a counterexample with $\theta = 9$.}
\end{figure}}
\end{frame}

\section{Rephrasing the Problem}
\begin{frame}{Rephrasing the Problem}
\uncover<1->{
Let $G$ be a graph with $\chi = \left\lfloor\frac{\theta}{2} \right\rfloor + 1$. Then $G$ must satisfy $\theta \leq 2\chi - 1$.  We may also assume that $G$ is critical and hence satisfies $\delta \geq \chi - 1$.  But then $G$ must have $\Delta \leq \chi$.  Also, no two vertices of degree $\Delta$ in $G$ can be adjacent.  These considerations give us an equivalent formulation of the problem that is easier to work with.}

\uncover<2->{
\begin{defn}
Given a graph $G$, let $\mathcal{H}(G)$ be the subgraph of $G$ induced on the vertices of degree at least $\chi(G)$.
\end{defn}}

\uncover<3->{
\begin{problem}
Prove that $K_{\Delta(G)}$ is the only critical graph $G$ with $\chi(G) \geq \Delta(G) \geq 6$ such that $\mathcal{H}(G)$ is edgeless.
\end{problem}}
\end{frame}

\section{A Simpler Proof via Mozhan's Lemma}
\subsection{Problem Solved}
\begin{frame}
\uncover<1->{
In \cite{rabern2010a} we solved the problem in a more general fashion.}
\uncover<2->{
\begin{OreBrooks}
$K_{\Delta(G)}$ is the only critical graph $G$ with $\chi(G) \geq \Delta(G) \geq 6$ and $\omega(\mathcal{H}(G)) \leq \left \lfloor \frac{\Delta(G)}{2} \right \rfloor - 2$.
\end{OreBrooks}

Setting $\omega(\mathcal{H}(G)) = 1$ proves the conjecture of Kierstead and Kostochka. In the following sections we will outline the proof.}
\end{frame}

\subsection{Mozhan's Lemma}
\begin{frame}{Kierstead and Kostochka's Proof}
\uncover<1->{
Kierstead and Kostochka's proof combines a list coloring argument on an auxiliary bipartite graph with Gallai's characterization of the low vertex subgraph $L(G)$ of a critical graph $G$ and a result from \cite{stiebitz1982proof} saying that if $G$ is a critical graph, then the high vertex subgraph $H(G)$ has at most as many components as $L(G)$ has.  It is an elegant use of known results, but unfortunately only works for $\theta \geq 12$.\newline}

\uncover<2->{
We obtained the proof that also works for $\theta \geq 10$ by employing the following useful observation of Mozhan \cite{mozhan1983} together with a carefully chosen recoloring algorithm.}
\end{frame}

\begin{frame}{Partitioned Colorings}
\begin{defn}
Let $G$ be a vertex critical graph.  Let $a \geq 1$ and $r_1, \ldots, r_a$ be such that $1 + \sum_i r_i = \chi(G)$. By a \emph{$(r_1, \ldots, r_a)$-partitioned coloring} of $G$ we mean a proper coloring of $G$ of the form
\[\left\{\{x\}, L_{11}, L_{12}, \ldots, L_{1r_1}, L_{21}, L_{22}, \ldots, L_{2r_2}, \ldots, L_{a1}, L_{a2}, \ldots, L_{ar_a}\right\}.\]

Here $\{x\}$ is a singleton color class and each $L_{ij}$ is a color class.
\end{defn}
\end{frame}

\begin{frame}{Mozhan's Lemma}
\begin{MozhansLemma}
Let $G$ be a vertex critical graph.   Let $a \geq 1$ and $r_1, \ldots, r_a$ be such that $1 + \sum_i r_i = \chi(G)$. Of all $(r_1, \ldots, r_a)$-partitioned colorings of $G$ pick one (call it $\pi$) minimizing

\[\sum_{i = 1}^a \left|E\left(G\left[\bigcup_{j = 1}^{r_i} L_{ij}\right]\right)\right|.\]

Remember that $\{x\}$ is a singleton color class in the coloring. Put $U_i = \bigcup_{j = 1}^{r_i} L_{ij}$ and let $Z_i(x)$ be the component of $x$ in $G[\{x\} \cup U_i]$.  If $d_{Z_i(x)}(x) = r_i$, then $Z_i(x)$ is complete if $r_i \geq 3$ and $Z_i(x)$ is an odd cycle if $r_i = 2$.
\end{MozhansLemma}
\end{frame}

\subsection{The Recoloring Algorithm}
Add stuff.

\section{A Stronger Bound}
\subsection{A Useful Observation}
\begin{frame}{A Useful Observation}
\uncover<1->{
Our proof of the above relied heavily on the high vertex subgraph being edgeless.  Intuitively, if this property could be captured by a numerical graph invariant $\phi$, then we should be able to get an upper bound on $\chi$ in terms of $\phi$. It turns out that such a graph parameter has already been studied in the literature.\newline}

\uncover<2->{
In \cite{stacho2001new} Stacho introduced the graph parameter $\Delta_2$ as the largest degree that a vertex $v$ can have subject to the condition that $v$ is adjacent to a vertex whose degree is at least as large as its own. By coloring greedily (in any order), we see that every graph satisfies $\chi \leq \Delta_2 + 1$.}

\uncover<3->{
\begin{observation}
For any graph $G$, $\chi(G) > \Delta_2(G)$ if and only if $\mathcal{H}(G)$ is edgeless.  
\end{observation}}
\end{frame}

\subsection{A Tempting Thought}
\begin{frame}{A Tempting Thought}
\uncover<1->{
From our experiences above with Brooks' Theorem and the Ore-degree version of Brook's Theorem, it is very tempting to conjecture the following.}

\uncover<2->{
\begin{TemptingThought}
There exists $t$ such that every graph with $\Delta_2 \geq t$ satisfies $\chi \leq \max \{\omega, \Delta_2\}$.
\end{TemptingThought}}

\uncover<3->{
Unfortunately, the tempting thought cannot hold for any $t$ unless \emph{P}=\emph{NP}. To see this, note that if the tempting thought held for some $t$, then using Lov\'{a}sz's $\vartheta$ parameter \cite{gr�tschel1981ellipsoid} which can be computed in polynomial time and has the property that $\omega(G) \leq \vartheta(\overline{G}) \leq \chi(G)$, we could determine whether or not $\chi(G) \leq \Delta_2(G)$ in polynomial time.  But Stacho proved in \cite{stacho2001new} that for any fixed $t \geq 3$, the problem of determining whether or not $\chi(G) \leq \Delta_2(G)$ for graphs with $\Delta_2(G) = t$ is \emph{NP}-complete.}
\end{frame}

\subsection{What We Can Prove}
\begin{frame}
\uncover<1->{
Even though the tempting thought likely fails, if we limit how far from $\Delta + 1$ our upper bound can stray, we can get a generalization of Brooks' theorem involving $\Delta_2$.}

\uncover<2->{
\begin{DeltaTwo}
Every graph with $\Delta \geq 3$ satisfies \[\chi \leq \max \left\{\omega, \Delta_2, \frac{5}{6}(\Delta + 1)\right\}.\]
\end{DeltaTwo}

Note that this also generalizes the Ore-degree version of Brooks' Theorem.  The proof uses a recoloring algorithm similar to the above.}
\end{frame}

\section{Further Improvements}
\begin{frame}
\uncover<1->{
In joint work with Kostochka and Stiebitz \cite{krs_one} similar techniques were used to improve the bounds further.  We give some highlights.}

\uncover<2->{
\begin{krs1}
Every graph with $\theta \geq 8$, except $O_5$, satisfies $\chi \leq \max \left\{\omega, \left\lfloor\frac{\theta}{2} \right \rfloor\right\}$.
\end{krs1}}

\uncover<3->{
\begin{krs2}
Every graph with $\Delta \geq 3$ satisfies \[\chi \leq \max \left\{\omega, \Delta_2, \frac{3}{4}(\Delta + 2)\right\}.\]
\end{krs2}}
\end{frame}

\bibliographystyle{amsplain}
\bibliography{GraphColoring}
\end{document}
