\documentclass[12pt]{article}
\usepackage{amsmath, amsthm, amssymb, fullpage}

\title{Hitting maximum cliques}
\author{Landon Rabern}

\theoremstyle{plain}
\newtheorem{thm}{Theorem}
\newtheorem{prop}[thm]{Proposition}
\newtheorem*{lem}{Lemma}
\newtheorem{cor}[thm]{Corollary}
\newtheorem*{conjecture}{Conjecture}
\newtheorem{claim}{Claim}
\theoremstyle{definition}
\newtheorem{defn}{Definition}
\newtheorem*{CliqueGraph}{Clique graph}
\theoremstyle{remark}
\newtheorem*{remark}{Remark}
\newtheorem*{question}{Question}
\newtheorem*{observation}{Observation}

\newcommand{\fancy}[1]{\mathcal{#1}}
\newcommand{\C}[1]{\fancy{C}_{#1}}
\newcommand{\IN}{\mathbb{N}}
\newcommand{\IR}{\mathbb{R}}

\newcommand{\inj}{\hookrightarrow}
\newcommand{\surj}{\twoheadrightarrow}

\newcommand{\set}[1]{\left\{ #1 \right\}}
\newcommand{\setb}[3]{\left\{ #1 \in #2 \mid #3 \right\}}
\newcommand{\setbs}[2]{\left\{ #1 \mid #2 \right\}}
\newcommand{\card}[1]{\left|#1\right|}
\newcommand{\size}[1]{\left\Vert#1\right\Vert}
\newcommand{\ceil}[1]{\left\lceil#1\right\rceil}
\newcommand{\floor}[1]{\left\lfloor#1\right\rfloor}
\newcommand{\func}[3]{#1\colon #2 \rightarrow #3}
\newcommand{\funcinj}[3]{#1\colon #2 \inj #3}
\newcommand{\funcsurj}[3]{#1\colon #2 \surj #3}
\newcommand{\irange}[1]{\left[#1\right]}
\newcommand{\join}[2]{#1 \mbox{\hspace{2 pt}$\ast$\hspace{2 pt}} #2}
\newcommand{\djunion}[2]{#1 \mbox{\hspace{2 pt}$+$\hspace{2 pt}} #2}
\newcommand{\parens}[1]{\left( #1 \right)}

\newcommand{\DefinedAs}{\mathrel{\mathop:}=}

\begin{document}
\section*{What we are proving}
\begin{lem}[Rabern 2009]
There exists a positive constant $c < 1$ such that every graph satisfying $\omega > c(\Delta + 1)$ has a stable set hitting every maximum clique.
\end{lem}
\section*{Paper memory}
\begin{lem}[Hajnal 1965]
For a collection $\fancy{Q}$ of maximum cliques in a graph $G$ we have
\[\card{\bigcup \fancy{Q}} + \card{\bigcap \fancy{Q}} \geq 2\omega(G).\]
\end{lem}
\begin{CliqueGraph}
For a collection of cliques $\mathcal{Q}$ in a graph, let $X_{\mathcal{Q}}$ be the intersection graph of $\mathcal{Q}$; that is, the vertex set of $X_{\mathcal{Q}}$ is $\mathcal{Q}$ and there is an edge between $Q_1 \neq Q_2 \in \mathcal{Q}$ iff $Q_1$ and $Q_2$ intersect.
\end{CliqueGraph}
\begin{lem}[Kostochka 1980]
Let $G$ be a graph satisfying $\omega > \frac{2}{3}(\Delta + 1)$.   If $\mathcal{Q}$ is a collection of maximum cliques in $G$ such that $X_{\mathcal{Q}}$ is connected, then $\cap \mathcal{Q} \neq \emptyset$.
\end{lem}
\begin{lem}[Alon 1988]
A partition $\set{V_1, \ldots, V_r}$ of the vertex set of a graph $G$ has an independent transversal if $\card{V_i} \geq 2e\Delta(G)$ for each $i$.
\end{lem}

\nocite{haxell2001note}
\nocite{aharoni2007independent}
\nocite{KingAXiv}
\nocite{king2007upper}
\nocite{2009arXiv0907.3705R}
\nocite{alon1988linear}
\nocite{borodin1977upper}
\nocite{kostochkaRussian}
\nocite{hajnaltheorem}
\nocite{rabernhitting}
\bibliographystyle{amsplain}
\tiny
\bibliography{GraphColoring}
\end{document}
