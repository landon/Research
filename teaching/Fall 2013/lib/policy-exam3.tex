%STANDARD THREE EXAM POLICY
\Topic{Course Requirements and Assessments}
Each student is assessed based on a computed point score from 0 to 100. The
score is based on exams, assigned papers, and bonus assignments.
\begin{compactenum}
\item
\textit{Assignments/Projects}: One project (50 points).
\item
\textit{Examinations}: 1 quiz (5 bonus points), and three exams (100 points each).
\item
\textit{Participation}: Active engagement in classwork and class discussion
is a necessary part of this class (please review the Assessment section).
\end{compactenum}

\Subtopic{Exams and Projects}
During the semester there is one quiz and three exams.
The quiz, worth 5 bonus points, is
given early in the semester before the drop date to allow student to determine
whether the course is appropriate for continued study.
The three exams (each worth 100 points) consist of a short in class exam,
midterm examination paper and a comprehensive final exam.
In addition there is one projects assigned as part of this class,
worth 50 points.

The first exam is closed book and notes, and it covers
only one specific topic.
The midterm paper is intended to provide an extensive assessment of the
student's ability to work with the material and to reason over a broader
range of problems, and is open book and notes.
Because of this, the level and quality of work is expected to be high.
The comprehensive final exam is closed book and notes
and covers all material which is discussed in the
course and is given on the scheduled exam day.

The course grade is based on the
average score obtained on all exams and projects (for a
total of at most 350 points). The student's total score is obtained by
averaging these 350 points and then adding the bonus points
to the average score. The total point  score (TPS) is the basis for grade
assignment.

\Subtopic{Bonus Assignments}
Bonus assignments are typically given as extra problems on exams,
or else are given during the course of
the semester, to be turned in at the required time. Bonus problems are more
difficult and challenging problems which are available for students to gain
further in-depth experience, pushing the limits of the course.
These problems usually have point values from 1 to 5. Students are
not required to attempt these problems, however it is to the student's benefit
to attempt these since they are counted after the all other assessment
results are averaged. Thus bonus assignments serve to increase a
student's score. Note that the first quiz is counted as a bonus score.

\Subtopic{Scoring and Course Grades}
The letter grade is based on the student's total point score (TPS) for the
semester. The TPS is computed using the average of all points
earned on all exams and projects plus all of the bonus points
which the student has earned.
The class letter grade is assigned based on

\begin{center}
\begin{tabular}{rcr|l}
\multicolumn{3}{c}{\textbf{TPS}} & \textbf{Grade} \\
\hline
& & &  \\[-0.8em]
    $97.5$ &-- &  $> 100.0$ & \hspace*{0.8em} \ec A+ \\
    $92.5$ &-- &  $97.5$ & \hspace*{0.8em} \ec A \\
    $90.0$ &-- &  $92.5$ & \hspace*{0.8em} \ec A- \\
    $87.5$ &-- &  $90.0$ & \hspace*{0.8em} \ec B+ \\
    $82.5$ &-- &  $87.5$ & \hspace*{0.8em} \ec B \\
    $80.0$ &-- &  $82.5$ & \hspace*{0.8em} \ec B- 
\end{tabular} \hspace*{1.8em}
\begin{tabular}{rcr|l}
\multicolumn{3}{c}{\textbf{TPS}} & \textbf{Grade} \\
\hline
& & &  \\[-0.8em]
    $77.5$ &-- &  $80.0$ & \hspace*{0.8em} \ec C+ \\
    $72.5$ &-- &  $77.5$ & \hspace*{0.8em} \ec C \\
    $70.0$ &-- &  $72.5$ & \hspace*{0.8em} \ec C- \\
    $67.5$ &-- &  $70.0$ & \hspace*{0.8em} \ec D+ \\
    $60.0$ &-- &  $67.5$ & \hspace*{0.8em} \ec D \\
    $0$ &-- & $60$ & \hspace*{0.8em} \ec F
\end{tabular}

\end{center}


All computations use single precision arithmetic, with scores rounded
to the nearest tenth of a point.
A TPS which falls on or below a grade boundary is assigned 
a grade based on a valuation of the extent to which a student
actively participated in the class
during the semester, including the number of bonus problems attempted
(meaning that even a score of 0 on a bonus problem is significant).
For example, a rounded score of 90 can be assigned an A- or B+ grade.
For this reason active class participation is an important part of the
assessment process.
Information about university grades can be found at \Urls{\GradepolicyURL}.

\Subtopic{Class Participation (Assignments and Homework)}
There is no credit
given for homework assignments (i.e., homework is not graded),
however since the comprehensive
final exam is based on the material and problems which are to be found in the
homework assignments, it is beneficial to attempt all assigned problems.
Students are required to be provide solutions to homework problems in class
on the day the assignments are due.

Because of the importance
attached to problem solving and because exam results are
not curved or normalized, students are strongly encourage to
attempt as many bonus assignments as possible. These provide a mechanism for
improving performance; however, unlike `curving' exam results, this approach
requires that the students take the initiative to improve their scores.
It is important to attempt bonus problems since these add to your final
score.

\smallskip
Example: A student earns the following grades in \Course: $3$ bonus points for
the quiz, $72$ points on the first exam, $78$ points on the midterm exam,
and $83$ points on the comprehensive final exam.
In addition the student
earns $37$ points on the projects. In addition
the student attempts thee bonus assignments worth $5$ points each, earning
$0$, $3$, and $5$ points on each of these. The student's score is then computed as
%
%
%
\begin{equation*}
\ec
\frac{(70 + 78 + 83) + (37)}{3.5} = \frac{562}{7} = 80.3,
\end{equation*}
%
%
%
then the bonus points earned, i.e., $\ec (3 + 0+ 3 + 5) = 11$ are added to
this score, yielding a final score of $\ec 91.3$.
Based on this final score, the student is assigned a {\ec A-} letter grade.
% END THREE EXAM POLICY
