\documentclass[10pt]{article}
\usepackage{amssymb, amsmath, amsthm}
\usepackage{fullpage}
\usepackage[hidelinks]{hyperref}
\usepackage{epigraph}
\usepackage{lmodern}
\usepackage[T1]{fontenc}
\usepackage{xspace}
\usepackage{termcal}
\usepackage{color}
\usepackage{fmtcount}

\makeatletter
\DeclareRobustCommand{\maybefakesc}[1]{%
  \ifnum\pdfstrcmp{\f@series}{\bfdefault}=\z@
    {\fontsize{\dimexpr0.8\dimexpr\f@size pt\relax}{0}\selectfont\uppercase{#1}}%
  \else
    \textsc{#1}%
  \fi
}
\newcommand\AM{\,\maybefakesc{am}\xspace}
\newcommand\PM{\,\maybefakesc{pm}\xspace}
\makeatother


\newcommand{\MWFClass}{%
\calday[Monday]{\classday}
\skipday % Tuesday (no class)
\calday[Wednesday]{\classday}
\skipday % Thursday (no class)
\calday[Friday]{\classday}
\skipday\skipday % weekend (no class)
}

\newcommand{\MTRClass}{%
	\calday[Monday]{\classday}
	\calday[Tuesday]{\classday}
	\skipday
	\calday[Thursday]{\classday}
	\skipday
	\skipday\skipday % weekend (no class)
}

\newcommand{\Holiday}[2]{%
\options{#1}{\noclassday}
\caltext{#1}{#2}
}


\pagestyle{plain}
\title{Math 109: Calculus 1\\ \bigskip\small{Fall 2015}}
\date{}
\begin{document}
\maketitle

\begin{tabular}{r l}
\textbf{instructor:}& landon rabern\\
\textbf{email:}& \href{mailto:lrabern@fandm.edu}{\nolinkurl{lrabern@fandm.edu}}\\
\textbf{class webpage:}& \href{http://bit.ly/109Fall2015}{\nolinkurl{bit.ly/109Fall2015}}\\
\textbf{office hours:}& TBA in STA 231\\
\textbf{textbook:}&\textit{Calculus, Concepts and Contexts 4E}, by James Stewart\\
\end{tabular}

\section*{Webwork} 
You will continue to be assigned Webwork problems. This will still count for 15\% of your grade.
\section*{Test 1 replacement}
On Wednesday, November \ordinalnum{4} we will have a quiz on the material from Chapter 2 to replace the first exam. This will count for 10\% of your grade; however, to account for the work you already put in on the first exam, you will receive 9\% of the 10\% automatically.  Said differently, the worst score you can get for the first exam is a 90\%; the remaining 10\% will be decided by this quiz.  On the original syllabus, the first exam was worth 15\%, the remaining 5\% has been moved to the final which will include material from Chapter 2.

\section*{Worksheet replacement}
Worksheets will be replaced by Friday quizzes. For your past effort on the worksheets, you will automatically receive 5\%.  The quizzes will count for the remaining 10\%.

\section*{Exams}
There will be in-class exams on Friday, October \ordinalnum{13} and Wednesday, December \ordinalnum{2}.  In addition, there will be a final exam during finals week (at the same time and place as currently scheduled).
Since you can use your textbook as well as any other written material, no memorization is required; however, these exams occur in a finite time period, so rapid recall of facts will serve you well. 

\section*{Computing devices}
You may use a calculator on exams and quizzes, but it is unlikely that it will help you. No devices that are capable of wireless communication.

\section*{Graded work breakdown (you have already earned points in \textcolor{blue}{blue})}
\begin{tabular}{r | r | l}
what & \% & when \\
\hline
past attendance/worksheets & \textcolor{blue}{\textbf{5}} & \\
webwork & 15 & weekly \\
quizzes & 10 & weekly on Fridays\\
chapter 2 quiz & 1 + \textcolor{blue}{\textbf{9}} & Friday, October \ordinalnum{30}\\
in-class exam \#2 & 15 & Friday, October \ordinalnum{13} \\
in-class exam \#3 & 15 & Wednesday, December \ordinalnum{2} \\
final exam & 30 & TBA, in finals week \\
\end{tabular}

\section*{Help}
If you need help or just want to know more about something, please come to my scheduled office hours or set up another time to meet. 
In addition to my office hours, there are several undergraduate mathematics teaching assistants who hold regular hours (schedule on the webpage).

\section*{Tentative Schedule}
\begin{center}
\begin{calendar}{10/26/2015}{8}
\setlength{\calboxdepth}{.3in}
\MWFClass

\caltextnext{3.3 and Appendix C, derivatives of trig functions}
\caltextnext{3.4 chain rule}
\caltextnext{3.5 implicit differentiation}
\caltextnext{\textcolor{blue}{\textbf{chapter 2 quiz}}}
\caltextnext{1.6, 3.6 derivatives of inverse trig functions}
\caltextnext{3.7 derivatives of logarithms}
\caltextnext{\textcolor{blue}{exam \#2 review}}
\caltextnext{\textcolor{blue}{\textbf{in-class exam \#2}}}
\caltextnext{4.2, 4.3 max and min values}
\caltextnext{4.2, 4.3 max and min values}
\caltextnext{4.6 optimization}
\caltextnext{4.8 antiderivatives}
\caltextnext{\textcolor{blue}{exam \#3 review}}
\caltextnext{\textcolor{blue}{\textbf{in-class exam \#3}}}
\caltextnext{5.1 areas and distances}
\caltextnext{5.2 definite integrals}
\caltextnext{5.3 evaluating definite integrals}
\caltextnext{\textcolor{blue}{final exam review}}
\caltextnext{\textcolor{blue}{final exam review}}

\Holiday{11/25/2015}{Thanksgiving}
\Holiday{11/26/2015}{Thanksgiving}
\Holiday{11/27/2015}{Thanksgiving}
\Holiday{11/28/2015}{Thanksgiving}
\Holiday{11/29/2015}{Thanksgiving}

\Holiday{12/12/2015}{Reading Day}
\Holiday{12/13/2015}{Reading Day}
\Holiday{12/14/2015}{Reading Day}
\Holiday{12/15/2015}{Reading Day}

\Holiday{12/16/2015}{Exam Week}
\Holiday{12/17/2015}{Exam Week}
\Holiday{12/18/2015}{Exam Week}
\Holiday{12/19/2015}{Exam Week}
\Holiday{12/20/2015}{Exam Week}

\end{calendar}
\end{center}
\end{document}