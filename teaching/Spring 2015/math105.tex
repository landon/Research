\documentclass[12pt]{article}
\usepackage{amssymb, amsmath, amsthm}
\usepackage{fullpage}
\usepackage[hidelinks]{hyperref}
\usepackage{epigraph}
\usepackage{lmodern}
\usepackage[T1]{fontenc}
\usepackage{xspace}
\usepackage{termcal}
\usepackage{color}
\usepackage{fmtcount}

\makeatletter
\DeclareRobustCommand{\maybefakesc}[1]{%
  \ifnum\pdfstrcmp{\f@series}{\bfdefault}=\z@
    {\fontsize{\dimexpr0.8\dimexpr\f@size pt\relax}{0}\selectfont\uppercase{#1}}%
  \else
    \textsc{#1}%
  \fi
}
\newcommand\AM{\,\maybefakesc{am}\xspace}
\newcommand\PM{\,\maybefakesc{pm}\xspace}
\makeatother

\newcommand{\MWFClass}{%
\calday[Monday]{\classday}
\skipday % Tuesday (no class)
\calday[Wednesday]{\classday}
\skipday % Thursday (no class)
\calday[Friday]{\classday}
\skipday\skipday % weekend (no class)
}

\newcommand{\Holiday}[2]{%
\options{#1}{\noclassday}
\caltext{#1}{#2}
}


\pagestyle{plain}
\title{Math 105: Preparation for College Mathematics\\ \bigskip\small{Spring 2015}}
\date{}
\begin{document}
\maketitle

\begin{tabular}{r l}
\textbf{instructor:}& landon rabern\\
\textbf{email:}& \href{mailto:lrabern@fandm.edu}{\nolinkurl{lrabern@fandm.edu}}\\
\textbf{class webpage:}& \href{http://bit.ly/precalculus2015}{\nolinkurl{bit.ly/precalculus2015}}\\
\textbf{class meetings:}& 8:00\AM--8:50\AM MWF in STA 215 (lecture / group work)\\
& 8:00\AM--8:50\AM R in STA 215 (working problems)\\
\textbf{office hours:}& 10:00\AM--10:50\AM MWF in STA 108\\
\textbf{textbook:}& \textit{Pre-Calculus, 6E}, by Stewart, Redlin, and Watson\\
\end{tabular}

\section*{What is this class about?}
We will look at functions and how to play with them safely.  Our investigations will include polynomial, rational, exponential, logarithmic and trigonometric functions.  We'll encounter numbers with attributes such as being ``irrational'', ``transcendental'', and even ``imaginary''.  By the end of the course you'll be able to bisect an angle using only a compass and straightedge just like the ancient Greeks.  Unlike the ancient Greeks, you will also know why you can't trisect an angle in such a manner.

\section*{Homework} 
\epigraph{I can only show you the door. You're the one that has to walk through it.}{}
To achieve fluency in this subject, you will need to immerse yourself in the material.  
Working tons of problems is a great way to do this.  How many problems?  
My recommendation would be to work problems of a given type until they become easy for you.

I will put a list of practice problems for each class period on the class webpage.  These will not be collected.  
Each Friday, I will select a couple of the more interesting problems and assign them---due the following Friday.
These will be graded both for correctness and clarity of exposition.

\section*{Quizzes}
\epigraph{Pop quiz, hotshot.}{}
There will be tiny quizzes at random times throughout the course.  I will set the random number generator so that the expected number of quizzes is 10.
Quizzes are intended to reinforce basic concepts as well as encourage attendance.
Unlike exams, quizzes will be closed-book.  Your lowest three quiz scores will be dropped.

\section*{Exams}
There will be two in-class exams and then a final exam during finals week. 
The purpose of the exams is to test your understanding of, and ability to reason about, the mathematical concepts. Since you can use your textbook as well as any other written material, no memorization is required; however, these exams occur in a finite time period, so rapid recall of facts will serve you well.  All electronic devices should be stowed in your bag for the duration of the exam and any brain implants should be turned off.

\section*{Graded work breakdown}
\begin{tabular}{r | r | l}
what & \% & when \\
\hline
homework & 20 & weekly\\
quizzes & 10 & random times\\
in-class exam \#1 & 15 & Friday, February \ordinalnum{13} \\
in-class exam \#2 & 25 &  Friday, March \ordinalnum{13} \\
final exam & 30 & TBA, in finals week \\
\end{tabular}

\section*{Help}
If you need help or just want to know more about something, please come to my scheduled office hours or set up another time to meet. In addition to my office hours, there are several undergraduate mathematics teaching assistants who hold regular hours.

\section*{Attendance}
Please be advised that Math Department and F\&M policy state that penalties (including
grade reduction and/or dismissal from the course) may be assessed for
excessive, unexcused absences.

\section*{Tentative Schedule}
\begin{center}
\begin{calendar}{1/12/2015}{16}
\setlength{\calboxdepth}{.3in}
\MWFClass

\options{1/12/2015}{\noclassday} 

% schedule
\caltexton{1}{1.1 real numbers}
\caltextnext{1.2, 1.3 exponents, algebraic expressions}
\caltextnext{1.4, 1.5 rational expressions, equations}
\caltextnext{1.7 inequalities}
\caltextnext{1.10 lines}
\caltextnext{2.1, 2.2 functions and their graphs}
\caltextnext{2.3 what we can learn from the graph}
\caltextnext{2.5 transformations of funtions}
\caltextnext{2.6 composition of functions}
\caltextnext{2.7 one-to-one functions and inverses}
\caltextnext{\textcolor{blue}{exam \#1 review}}
\caltextnext{\textcolor{blue}{exam \#1 review}}
\caltextnext{\textcolor{blue}{\textbf{in-class exam \#1}}}
\caltextnext{3.1 quadratic functions}
\caltextnext{3.2 polynomial functions}
\caltextnext{3.3 long division}
\caltextnext{3.4 zeroes of polynomials}
\caltextnext{3.5 imaginary numbers}
\caltextnext{3.6 the fundamental theorem of algebra}
\caltextnext{solving cubics and quartics by radicals, Galois theory}
\caltextnext{3.7 rational functions}
\caltextnext{Lagrange interpolation and IQ tests}
\caltextnext{\textcolor{blue}{exam \#2 review}}
\caltextnext{\textcolor{blue}{exam \#2 review}}
\caltextnext{\textcolor{blue}{\textbf{in-class exam \#2}}}
\caltextnext{4.1, 4.2 exponential functions}
\caltextnext{4.3, 4.4 logarithms}
\caltextnext{5.1 the unit circle}
\caltextnext{5.2 trigonometric functions}
\caltextnext{5.3 trigonometric graphs}
\caltextnext{5.5 inverse trigonometric functions}
\caltextnext{compass-and-straightedge constructions}
\caltextnext{compass-and-straightedge constructions}
\caltextnext{attempt to trisect an angle}
\caltextnext{7.1 trigonometric identities}
\caltextnext{7.2, 7.3 more identities}
\caltextnext{why we didn't succeed in trisecting the angle}
\caltextnext{\textcolor{blue}{final exam review}}
\caltextnext{\textcolor{blue}{final exam review}}

% Holidays
\Holiday{1/19/2015}{Martin Luther King Day}
\Holiday{3/14/2015}{Spring Break}
\Holiday{3/15/2015}{Spring Break}
\Holiday{3/16/2015}{Spring Break}
\Holiday{3/17/2015}{Spring Break}
\Holiday{3/18/2015}{Spring Break}
\Holiday{3/19/2015}{Spring Break}
\Holiday{3/20/2015}{Spring Break}
\Holiday{3/21/2015}{Spring Break}
\Holiday{3/22/2015}{Spring Break}

\Holiday{4/24/2015}{Reading Day}
\Holiday{4/25/2015}{Reading Day}
\Holiday{4/26/2015}{Reading Day}
\Holiday{4/27/2015}{Reading Day}

\Holiday{4/28/2015}{Finals week}
\Holiday{4/29/2015}{Finals week}
\Holiday{4/30/2015}{Finals week}
\Holiday{5/1/2015}{Finals week}
\Holiday{5/2/2015}{Finals week}
\end{calendar}
\end{center}
\end{document}