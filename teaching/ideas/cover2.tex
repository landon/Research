\documentclass[10pt,stdletter,dateno]{newlfm}
\PassOptionsToPackage{hyphens}{url}\usepackage{hyperref}

\widowpenalty=1000
\clubpenalty=1000

\newlfmP{headermarginskip=0pt}
\newlfmP{sigsize=12pt}
\newlfmP{dateskipafter=0pt}

\namefrom{Landon Rabern}
\addrfrom{%
    \today\\[10pt]
    497 Mine Road\\
    Lebanon, PA 17042}

\addrto{%
F\&M Search Committee\\
Department of Computer Science \\
Franklin \& Marshall College \\
Lancaster, PA 17604}

\greetto{Dear F\&M Search Committee,}
\closeline{Sincerely,}
\begin{document}
\begin{newlfm}

I am writing to apply for Franklin and Marshall's Visiting Assistant Professor position in Computer Science.
I was glad to learn of the listing; 
I have taught multiple sections of MATH 105 and MATH 109 at F\&M for the past two years, and would look forward to strengthening my connection with the college. 
My general areas of research interest include structural and extremal graph theory, machine learning and philosophical logic. In particular, I am 
drawn to problems at intersections of computer science, math and philosophy. Given my research agenda, it is a pleasure to apply to a college that values interdisciplinarity, 
and to a department that houses both math and computer science.  

As a researcher, I can offer an active publishing record as well as a genuine interest in mentoring and potentially collaborating with undergraduate students. 
One of my favorite recent results is a short human-readable proof that every possible arrangement of countries on a planar map can be colored using only nine crayons so that each country
gets two colors and adjacent countries get disjoint sets of colors 
(see \url{http://blog.computationalcomplexity.org/2015/10/a-human-readable-proof-that-every.html} for a 
discussion of this result on Lance Fortnow and Bill Gasarch's blog \emph{Computational Complexity}).  This $\frac92$-color map theorem sits halfway between the nearly trivial $5$-color map theorem and the difficult
$4$-color map theorem.  The latter theorem was an open conjecture for over a hundred years before finally succumbing to a massive computer-assisted proof in 1976.
This and other research appears in numerous journals including Combinatorica, Journal of Combinatorial Theory Series B, Journal of Graph Theory, Electronic Journal of Combinatorics,
Discrete Mathematics, Analysis, Journal of Philosophical Logic, and Journal of Pure and Applied Algebra. Much of my work emerges from collaborations, an ethos I would 
extend to interested students at Franklin and Marshall. In my primary research area of graph coloring, there are many open problems that would be amenable to collaborative 
work with undergraduate students, and likely the opportunity for co-authored articles. 

I am interested in teaching courses in discrete math and computer science.
I do have some familiarity with F\&M’s Connections curriculum, and can also imagine contributing a course exploring the history of cryptography, 
for example, if that would be desirable. 

In sum, I would look forward to bringing my experience to an inspiring liberal arts community that values rigorous scholarship while centralizing pedagogy. Through 
teaching, collaborative research with students, and other mentorship, I would enjoy contributing to a department and college community that I have come to respect.
\end{newlfm}
\end{document}
