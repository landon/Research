\documentclass[12pt]{article}
\usepackage{amsmath, amssymb}


\newtheorem{acknowledgement}{Acknowledgement}
\newtheorem{algorithm}{Algorithm}
\newtheorem{axiom}{Axiom}
\newtheorem{case}{Case}
\newtheorem{claim}{Claim}
\newtheorem{conclusion}{Conclusion}
\newtheorem{condition}{Condition}
\newtheorem{conjecture}{Conjecture}
\newtheorem{corollary}{Corollary}
\newtheorem{criterion}{Criterion}
\newtheorem{definition}{Definition}
\newtheorem{example}{Example}
\newtheorem{exercise}{Exercise}
\newtheorem{lemma}{Lemma}
\newtheorem{notation}{Notation}
\newtheorem{problem}{Problem}
\newtheorem{proposition}{Proposition}
\newtheorem{remark}{Remark}
\newtheorem{solution}{Solution}
\newtheorem{summary}{Summary}
\newtheorem{theorem}{Theorem}


\newcommand{\fancy}[1]{\mathcal{#1}}
\newcommand{\C}[1]{\fancy{C}_{#1}}


\newcommand{\IN}{\mathbb{N}}
\newcommand{\IZ}{\mathbb{Z}}
\newcommand{\IR}{\mathbb{R}}
\newcommand{\G}{\fancy{G}}
\newcommand{\CC}{\fancy{C}}
\newcommand{\D}{\fancy{D}}
\newcommand{\T}{\fancy{T}}
\newcommand{\B}{\fancy{B}}
\renewcommand{\L}{\fancy{L}}
\newcommand{\HH}{\fancy{H}}

\newcommand{\inj}{\hookrightarrow}
\newcommand{\surj}{\twoheadrightarrow}

\newcommand{\set}[1]{\left\{ #1 \right\}}
\newcommand{\setb}[3]{\left\{ #1 \in #2 : #3 \right\}}
\newcommand{\setbs}[2]{\left\{ #1 : #2 \right\}}
\newcommand{\card}[1]{\left|#1\right|}
\newcommand{\size}[1]{\left\Vert#1\right\Vert}
\newcommand{\ceil}[1]{\left\lceil#1\right\rceil}
\newcommand{\floor}[1]{\left\lfloor#1\right\rfloor}
\newcommand{\func}[3]{#1\colon #2 \rightarrow #3}
\newcommand{\funcinj}[3]{#1\colon #2 \inj #3}
\newcommand{\funcsurj}[3]{#1\colon #2 \surj #3}
\newcommand{\irange}[1]{\left[#1\right]}
\newcommand{\join}[2]{#1 \mbox{\hspace{2 pt}$\ast$\hspace{2 pt}} #2}
\newcommand{\djunion}[2]{#1 \mbox{\hspace{2 pt}$+$\hspace{2 pt}} #2}
\newcommand{\parens}[1]{\left( #1 \right)}
\newcommand{\brackets}[1]{\left[ #1 \right]}
\newcommand{\DefinedAs}{\mathrel{\mathop:}=}

%\newcommand{\mic}{\operatorname{mic}}
%\newcommand{\AT}{\operatorname{AT}}
%\newcommand{\col}{\operatorname{col}}
%\newcommand{\ch}{\operatorname{ch}}
%\newcommand{\type}{\operatorname{type}}
%\newcommand{\nonsep}{\bar{S}}
%\newcommand{\type}{\operatorname{type}}
%\def\adj{\leftrightarrow}
%\def\nonadj{\not\!\leftrightarrow}
%\newcommand{\gcd}{\operatorname{gcd}}

\newcommand\restr[2]{{% we make the whole thing an ordinary symbol
  \left.\kern-\nulldelimiterspace % automatically resize the bar with \right
  #1 % the function
  \vphantom{\big|} % pretend it's a little taller at normal size
  \right|_{#2} % this is the delimiter
  }}

\def\D{\fancy{D}}
\def\C{\fancy{C}}
\def\A{\fancy{A}}

%\newcommand{\claim}[2]{{\bf Claim #1.}~{\it #2}~~}
%\newcommand{\case}[2]{{\bf Case #1.}~{\it #2}~~}
%\newcommand\numberthis{\addtocounter{equation}{1}\tag{\theequation}}

%\def\gcd{\bigtriangledown}
%\def\lcm{\bigtriangleup}
%\def\no{\natural}


\usepackage{tikz}
\usetikzlibrary{calc}

\pgfdeclarelayer{background}
\pgfsetlayers{background,main}
\newcommand{\Bond}[6]%
% start, end, thickness, incolor, outcolor, iterations
{ \begin{pgfonlayer}{background}
        \colorlet{InColor}{#4}
        \colorlet{OutColor}{#5}
        \foreach \I in {#6,...,1}
        {   \pgfmathsetlengthmacro{\r}{#3/#6*\I}
            \pgfmathsetmacro{\C}{sqrt(1-\r*\r/#3/#3)*100}
            \draw[InColor!\C!OutColor, line width=\r] (#1.center) -- (#2.center);
        }
    \end{pgfonlayer}
}

\newcommand{\BlackBond}[2]%
% start, end
{   \Bond{#1}{#2}{0.7071mm}{black!25}{black!25!black}{10}
}

\title{Teaching Philosophy}
\author{Landon Rabern}
\date{}
\begin{document}
\maketitle
Teaching addresses a core question: How can I help students to balance
their own enjoyment and motivation against the rigorous and sometimes repetitive
work needed to understand mathematics?
Further, how can I help them retain the material beyond the next
exam?  My students enter the classroom with a range of interests and
ability levels.  For each of them, I seek to create positive experiences that
affirm why mathematics matters.

To excite passion about concepts, I explore applications that I hope will
resonate with the group.  In a multivariable calculus class, for example, 3D
games provide an ideal backdrop for a discussion of euler angles and their
susceptibility to gimbal lock (a loss of a degree of freedom in a
three-dimensional mechanism when axes of two of the three gimbals become
parallel).  This can lead naturally into studying
quaternions and their advantages in performing smooth 3D rotations. 
Seeing applications like these can lead students to dramatically change their
attitudes, and help them to embrace the rigor of the discipline, with more
curiosity. 

My background in industry also informs my approach in the classroom, in particular
my emphasis on group work and peer review. As a Senior Software Engineer and
Scientific Programmer, I've mentored many junior software engineers. 
The most effective way I've seen to help an individual improve is through
communicating to and with a group, with a clear shared goal in mind. 
For example, how can we as a team write code that is the purest logical
expression of the idea/algorithm in the given language? 
How can each individual contribute the most to this shared goal? 
At Wall Street On Demand in Boulder, we initiated code reviews: 
a small subset of the engineers sat together and reviewed one another's code,
line by line. This tedious-sounding practice was extremely effective at
improving both code and engineer quality, 
and it also produced a positive ethos of teamwork. 

Inspired by the results of the code reviews, I tried a similar idea in a
graduate graph theory course at Arizona State University. 
A student would present a homework solution on the board and the rest of the
class would give feedback. 
To foster a positive atmosphere, we took the ``Yes and...'' approach espoused
by theatrical improvisation: all feedback must build on the
good parts of the solution instead of just pointing out the bad. 
Students were initially shy, but by the end of the semester they could better 
articulate their own reasoning and communicate with others, and they also
had a greater sense of pride and focus on finding the best possible
collective solution. 

When I teach a class at any level, I like to experiment with classroom
mechanics such as the balance between working alone and in groups. 
%moving toward best possible practices. 
Last year, I had students keep a \emph{Practice Journal} in which they worked
suggested exercises. I gave them a large list of exercises of 
each type and simply instructed them to keep working problems in their journals
until they became easy. I said that I would 
be collecting journals for inspection every few weeks, to better understand
their individual processes and progress.  It turned out to be
unfeasible to collect and review journals this often, and initially I thought
that the journal idea was a total failure. But at the end of the course, 
a large fraction of the students remarked on the efficacy of this requirement. 
Allowing students ownership over the number of problems had inspired more rigor
and repetition than I would have asked of them. As with the peer review, I
continually look for ways that I can facilitate students' self-motivation, 
which is essential for them to deeply understand the concepts.

Beyond these mechanics, I bring to the classroom a great love of and enthusiasm
for discrete math and algorithms. Because my path within the discipline is unique to me, 
I seek to mentor the infinite possible paths that my individual students might take, from a standpoint of interest in and respect for their work. 
I aim to balance rigor, which is necessary for mastery of the discipline, with
pleasure, which will motivate students to think beyond a single course. 
Though rigor is often uncomfortable, I want my students to see that the payoff
is worth it, because the work opens up crucial windows into perceiving 
the structure of the world. 

\end{document}
