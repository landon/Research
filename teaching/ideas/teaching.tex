
\documentclass[12pt]{article} 
\usepackage{amssymb, amsmath, amsthm, mathrsfs, stmaryrd, color, verbatim, natbib, tikz, verbatim}

\addtolength{\oddsidemargin}{-.875in}
	\addtolength{\evensidemargin}{-.875in}
	\addtolength{\textwidth}{1.75in}

	\addtolength{\topmargin}{-.875in}
	\addtolength{\textheight}{1.75in}

                                                                           

%\date{}

\begin{document}



%+++++++++++++++++++++++++++++++++++++++++++++++++++++


\begin{center}
{\Large \sc Teaching Statement}\\
{\sc Landon Rabern}
\end{center}

\vspace{.4in}

I approach teaching as an ongoing engagement with a core problem: how can one enable students to balance the rigorous and sometimes repetitive work that is necessary to achieve thorough understanding, with motivation and enjoyment? Further, how can I maximize students' retention of material beyond the context of an upcoming exam? I seek to transmit to my students, who will come into the classroom with a range of interest and ability levels depending on the course, positive experiences that affirm why mathematics matters.

In my experience, I am best able to address the questions above when I treat students as individuals; I find that students are motivated by a wide variety of different stimuli. For example, I was the sort of student who responded to a challenge and did some of my best individual work on problems my instructors described as ``impossible''. In contrast, when I taught undergraduates at Arizona State University, I noticed that some students responded especially well to opportunities to work in groups. These students displayed the best understanding of the material when I asked them to solve a small subset of problems individually and then present and explain solutions to each other. I found that the act of explaining demanded a high level of attention (no one was able to check out) and accountability. Generally, I find small group work punctuated by short lectures to be more effective than lecture alone, as it demands that students be active agents throughout the learning process.

Additionally, I look for opportunities to apply mathematical concepts to the students' everyday lives. Sometimes, this takes the form of a hook at the beginning of class. For example, one of my favorite ice breaking exercises involves asking the students if any of them know the day of the week they were born on. I then mentally compute the day of the week from the date using Conway's Doomsday algorithm. Students often have a lot of fun during this exercise and I welcome the opportunity to show them that there is no magic involved, just simple arithmetic. Throughout the process of teaching different courses, I actively search out applications that I hope will resonate with the group. In a multivariable calculus class, for example, 3D games give a perfect backdrop for a discussion of Euler angles and their susceptibility to gimbal lock.  This leads naturally into the topic of quaternions and their advantages in performing smooth 3D rotations.  Just one ``Whoa that's cool'' or ``Ah, I see how this could be useful'' moment for a student can bring about a drastic change in their attitude.

I also find I can increase student engagement when I make connections between mathematics and other disciplines. Currently, I am working on an interdisciplinary course in theater and graph theory I was asked to co-design for Arizona State University. The course investigates intersections between the aesthetics of graph theory and the aesthetics of theatrical staging and attempts to provide students with a basic grounding in introductory graph theory through creative rehearsal processes. One reason I'm interested in disciplinary partnerships like this one is that I perceive mathematics as a continuing story about the potential of the human mind and the relationship between the mind and the societies we build. In the classroom, I introduce historical context to accompany each new topic, in order to establish the stakes for the development of innovative ideas throughout history and the impacts on physics, warfare and society as a whole. Part of the story of course includes the innovators themselves and I share specific examples including Newton's solution of Bernoulli's mathematical challenge and Archimedes' inventions for the defense of the city of Syracuse.

My background in industry also informs my approach to teaching and my emphasis on rigor. As a Senior Software Engineer and Scientific Programmer, I've mentored many junior software engineers. I've learned that, at least in this realm, the single most effective way to help a student improve is to get them to take great pride in their work. In my view, this pride derives from writing code that is the purest logical expression of the idea/algorithm in the given language. At Wall Street On Demand in Boulder, we instilled this pride through code reviews---a small subset of the engineers get together and look over each others code line by line.  This tedious-sounding practice is extremely effective in improving both code and engineer quality. Inspired by these results, I tried a similar idea in a graduate graph theory course at Arizona State University. A student would present a homework solution on the board and the rest of the class would give feedback.  To prevent progress being quashed by negative feedback, we took the ``Yes and...'' approach espoused by theatrical improvisation, wherein feedback must build on the good parts of the solution instead of just pointing out the bad.
 
Beyond these mechanics, I bring a great love of and enthusiasm for mathematics to the classroom. My path into and within the discipline is unique to me; I seek to mentor the infinite possible paths that my individual students might take, from a standpoint of interest in and respect for their work. I aim to balance the rigor that is necessary for mastery of the discipline with the pleasure that will motivate students to think beyond the confines of a single class. Though practicing rigor is not always comfortable, I endeavor to communicate to my students that the payoff is worth it, because the work opens up crucial windows into perceiving the structure of the world. 
\end{document}
