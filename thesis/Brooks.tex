\chapter{BROOKS' THEOREM}\label{BrooksChapter}

In Chapter \ref{APrioriWowChapter} we will rely heavily on the technique of
adding edges to a proper induced subgraph of a minimum counterexample.  We first
learned of this technique when reading Reed's proof of the Borodin-Kostochka
conjecture for large $\Delta$ (see \cite{reed1999strengthening}). To introduce
the idea we give a short proof of Brooks' theorem. The proof is completely
different from Lov\'{a}sz's short proof in \cite{Lovasz1975269}. We first reduce to the cubic case and then add edges to a proper induced
subgraph to get a coloring we can complete. The reduction to the cubic case is an immediate consequence of more
general lemmas on hitting all maximum cliques with an independent set that we
prove in Chapter \ref{ReductionChapter} (see also \cite{kostochkaRussian},
\cite{rabernhitting} and \cite{KingHitting}). Additionally, this reduction was
demonstrated by Tverberg in \cite{tverberg1983brooks}.  One
interesting feature of the proof is that it doesn't use any connectivity
concepts.
We'll give two versions of the proof, the first is shorter but uses the extra idea of excluding diamonds ($K_4$ less an edge).

\begin{thm}[Brooks \cite{brooks1941colouring}]
Every graph satisfies $\chi \leq \max\set{3, \omega, \Delta}$.
\end{thm}
\begin{proof}
Suppose the theorem is false and choose a counterexample $G$ minimizing
$\card{G}$.  Put $\Delta \DefinedAs \Delta(G)$. Using minimality of $\card{G}$,
we see that $\chi(G - v) \leq \Delta$ for all $v \in
V(G)$. In particular, $G$ is $\Delta$-regular.

First, suppose $G$ is $3$-regular.  If $G$ contains a diamond $D$, then we may $3$-color $G-D$ and easily extend the coloring to $D$ by first coloring the nonadjacent vertices in $D$ the same.  So, $G$ doesn't contain diamonds. Since $G$ is not a forest it contains an induced cycle $C$. Since $K_4 \not
\subseteq G$ we have $\card{N(C)} \geq 2$. So, we may take different $x, y \in N(C)$ and put $H \DefinedAs G - C$ if $x$ is adjacent to $y$ and $H \DefinedAs (G-C) + xy$ otherwise.  Then, $H$ doesn't contain $K_4$ as $G$ doesn't contain diamonds. By minimality of $\card{G}$, $H$ is $3$-colorable. That is, we have a $3$-coloring of $G - C$ where $x$ and $y$ receive different colors.  We can easily extend this partial
coloring to all of $G$ since each vertex of $C$ has a set of two available
colors and some pair of vertices in $C$ get different sets.  

Hence we must have $\Delta \geq 4$. Consider a $\Delta$-coloring of $G-v$ for some $v \in V(G)$.  Each color must be used on every $K_{\Delta}$ in $G-v$ and hence some color must be used on every $K_{\Delta}$ in $G$.  Let $M$ be such a color class expanded to a maximal independent set.  Then $\chi(G-M) = \chi(G) - 1 = \Delta > \max\set{3, \omega(G-M), \Delta(G-M)}$, a contradiction.
\end{proof}

Here is the other version, not excluding diamonds and doing the reduction differently.

\begin{thm}[Brooks \cite{brooks1941colouring}]
Every graph $G$ with $\chi(G) = \Delta(G) + 1 \geq 4$ contains
$K_{\Delta(G) + 1}$.
\end{thm}
\begin{proof}
Suppose the theorem is false and choose a counterexample $G$ minimizing
$\card{G}$.  Put $\Delta \DefinedAs \Delta(G)$. Using minimality of $\card{G}$,
we see that $\chi(G - v) \leq \Delta$ for all $v \in
V(G)$. In particular, $G$ is $\Delta$-regular.

First, suppose $\Delta \geq 4$.  Pick $v \in V(G)$ and let $w_1, \ldots,
w_\Delta$ be $v$'s neighbors. Since $K_{\Delta + 1} \not \subseteq G$, by
symmetry we may assume that $w_2$ and $w_3$ are not adjacent. Choose a $(\Delta
+ 1)$-coloring $\set{\set{v}, C_1, \ldots, C_\Delta}$ of $G$ where $w_i \in
C_i$ so as to maximize $\card{C_1}$.  Then $C_1$ is a maximal independent set in
$G$ and in particular, with $H \DefinedAs G - C_1$, we have $\chi(H) =
\chi(G) - 1 = \Delta = \Delta(H) + 1 \geq 4$.  By minimality of $\card{G}$, we
get $K_\Delta \subseteq H$.  But $\set{\set{v}, C_2, \ldots, C_\Delta}$ is a
$\Delta$-coloring of $H$, so any $K_\Delta$ in $H$ must contain $v$ and hence
$w_2$ and $w_3$, a contradiction.

Therefore $G$ is $3$-regular.  Since $G$ is not a forest it contains an induced
cycle $C$.  Put $T \DefinedAs N(C)$.  Then $\card{T} \geq 2$ since $K_4 \not
\subseteq G$.  Take different $x, y \in T$ and put $H_{xy} \DefinedAs G - C$ if
$x$ is adjacent to $y$ and $H_{xy} \DefinedAs (G-C) + xy$ otherwise.  Then, by
minimality of $\card{G}$, either $H_{xy}$ is $3$-colorable or adding $xy$
created a $K_4$ in $H_{xy}$.

Suppose the former happens.  Then we have a $3$-coloring of $G - C$
where $x$ and $y$ receive different colors.  We can easily extend this partial
coloring to all of $G$ since each vertex of $C$ has a set of two available
colors and some pair of vertices in $C$ get different sets. 

Whence adding $xy$ created a $K_4$, call it $A$, in $H_{xy}$.  We conclude that
$T$ is independent and each vertex in $T$ has exactly one neighbor in $C$.  Hence
$\card{T} \geq \card{C} \geq 3$. Pick $z \in T - \set{x,y}$.  Then $x$ is
contained in a $K_4$, call it $B$, in $H_{xz}$.  Since $d(x) = 3$, we must have
$A - \set{x,y} = B - \set{x, z}$.  But then any $w \in A - \set{x,y}$ has degree
at least $4$, a contradiction.
\end{proof}
