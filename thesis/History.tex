\chapter{INTRODUCTION}
Graph coloring is instantiated by many partitioning problems that arise in
practice.  Any time we encounter a relation between things of some sort and wish
to group those things so that no group contains a related pair, we have to solve
a graph coloring problem.  On some such encounters, the possible relations
between our sorted things are restricted in a way that allows a small number of
groups to be used.  This is good.  If the sort in question were
\emph{a sort of tasks to be performed} and the relation were \emph{can't be
performed at the same time}, then such a grouping, taken in some order, gives an
order in which to perform the tasks.  Fewer groups means faster task completion.
How many groups are needed for a certain sort of task collection?  How can we
find a grouping with the minimum number of groups?  If we can't find such a
grouping, how can we at least determine the number of groups in a minimum
grouping?  This dissertation is primarily concerned with the first type
of question---but an answer to the first type of question can have bearing
on the other questions when coupled with a method of sorting.  When we move from
the instantiation to abstract graph coloring we call the things \emph{vertices}
and represent the relation of two vertices by an \emph{edge} between them.  The
vertices together with the edges constitute a \emph{graph} and one of our
groupings corresponds to a labeling of the vertices with $1, 2, \ldots, k$ so
that there are no edges between vertices receiving the same label.  Such a labeling is called a
\emph{coloring} or more precisely a \emph{$k$-coloring}. For a graph $G$ we
write $\chi(G)$ for the minimum $k$ for which $G$ has a $k$-coloring---this
corresponds to the number of groups in a minimum grouping.  This study concerns
the relation of $\chi$ to other \emph{graph parameters}.  Our
terminology and notation are basically standard (for all notation, see
Appendix \ref{NotationAppendix}).

\section{A short history}
Here we collect statements of the results and conjectures that have bearing on this inquiry woven together with some historical remarks and our improvements.  
The first non-trivial result about coloring graphs with around $\Delta$ colors is Brooks' theorem from 1941.

\begin{thm}[Brooks \cite{brooks1941colouring}]
Every graph with $\Delta \geq 3$ satisfies $\chi \leq \max\{\omega, \Delta\}$.
\end{thm}

In 1977, Borodin and Kostochka conjectured that a similar result holds for $\Delta - 1$ colorings.  Counterexamples exist showing that the $\Delta \geq 9$ condition is tight (see Figure \ref{fig:SmallCE}).
\bigskip
\begin{figure}[htb]
\centering
\subfloat[$\Delta=6$]{
{\parbox{5cm}{
\begin{tikzpicture}[scale = 10]
\tikzstyle{VertexStyle}=[shape = circle, minimum size = 1pt, inner sep = 3pt,
draw]
\Vertex[x = 0.25085711479187, y = 0.92838092893362, L = \tiny {}]{v0}
\Vertex[x = 0.0932380929589272, y = 0.929142817854881, L = \tiny {}]{v1}
\Vertex[x = 0.250571489334106, y = 0.781142801046371, L = \tiny {}]{v2}
\Vertex[x = 0.092571459710598, y = 0.781142845749855, L = \tiny {}]{v3}
\Vertex[x = 0.355238050222397, y = 0.889142841100693, L = \tiny {}]{v4}
\Vertex[x = 0.353904783725739, y = 0.853142827749252, L = \tiny {}]{v5}
\Vertex[x = 0.353238135576248, y = 0.818476170301437, L = \tiny {}]{v6}
\Vertex[x = 0.476000010967255, y = 0.968571435660124, L = \tiny {}]{v7}
\Vertex[x = 0.537999987602234, y = 0.853238105773926, L = \tiny {}]{v8}
\Vertex[x = 0.444666564464569, y = 0.77590474486351, L = \tiny {}]{v9}
\Vertex[x = 0.475333333015442, y = 0.731904745101929, L = \tiny {}]{v10}
\Vertex[x = 0.451999962329865, y = 0.916571423411369, L = \tiny {}]{v11}
\Edge[](v0)(v1)
\Edge[](v2)(v1)
\Edge[](v3)(v1)
\Edge[](v0)(v3)
\Edge[](v2)(v3)
\Edge[](v2)(v0)
\Edge[](v4)(v2)
\Edge[](v5)(v2)
\Edge[](v6)(v2)
\Edge[](v4)(v0)
\Edge[](v5)(v0)
\Edge[](v6)(v0)
\Edge[](v4)(v1)
\Edge[](v5)(v1)
\Edge[](v6)(v1)
\Edge[](v4)(v3)
\Edge[](v5)(v3)
\Edge[](v6)(v3)
\Edge[](v8)(v7)
\Edge[](v9)(v7)
\Edge[](v10)(v7)
\Edge[](v11)(v7)
\Edge[](v9)(v8)
\Edge[](v10)(v8)
\Edge[](v11)(v8)
\Edge[](v10)(v9)
\Edge[](v11)(v9)
\Edge[](v11)(v10)
\Edge[](v4)(v7)
\Edge[](v4)(v11)
\Edge[](v6)(v9)
\Edge[](v6)(v10)
\Edge[](v5)(v8)
\Edge[](v5)(v9)
\end{tikzpicture}}}}\qquad\qquad
\subfloat[$\Delta=7$]{
{\parbox{5cm}{

\begin{tikzpicture}[scale = 5]
\tikzstyle{VertexStyle}=[shape = circle, minimum size = 1pt, inner sep = 3pt,
draw]
\Vertex[x = 0.751999914646149, y = 0.724000096321106, L = \tiny {}]{v0}
\Vertex[x = 0.751999974250793, y = 0.590000092983246, L = \tiny {}]{v1}
\Vertex[x = 0.652000069618225, y = 0.38400000333786, L = \tiny {}]{v2}
\Vertex[x = 0.578000009059906, y = 0.51800012588501, L = \tiny {}]{v3}
\Vertex[x = 0.572000086307526, y = 0.808000028133392, L = \tiny {}]{v4}
\Vertex[x = 0.0419999808073044, y = 0.742000013589859, L = \tiny {}]{v5}
\Vertex[x = 0.0399999916553497, y = 0.612000048160553, L = \tiny {}]{v6}
\Vertex[x = 0.163999989628792, y = 0.569999992847443, L = \tiny {}]{v7}
\Vertex[x = 0.25600004196167, y = 0.676000028848648, L = \tiny {}]{v8}
\Vertex[x = 0.159999996423721, y = 0.782000005245209, L = \tiny {}]{v9}
\Vertex[x = 0.653999924659729, y = 0.921999998390675, L = \tiny {}]{v10}
\Vertex[x = 0.379999995231628, y = 0.771999999880791, L = \tiny {}]{v11}
\Vertex[x = 0.381999999284744, y = 0.682000011205673, L = \tiny {}]{v12}
\Vertex[x = 0.386000007390976, y = 0.592000007629395, L = \tiny {}]{v13}
\Edge[](v0)(v4)
\Edge[](v1)(v4)
\Edge[](v2)(v4)
\Edge[](v3)(v4)
\Edge[](v0)(v3)
\Edge[](v1)(v3)
\Edge[](v2)(v3)
\Edge[](v0)(v2)
\Edge[](v1)(v2)
\Edge[](v0)(v1)
\Edge[](v5)(v6)
\Edge[](v5)(v7)
\Edge[](v5)(v8)
\Edge[](v5)(v9)
\Edge[](v6)(v7)
\Edge[](v6)(v8)
\Edge[](v6)(v9)
\Edge[](v7)(v8)
\Edge[](v7)(v9)
\Edge[](v8)(v9)
\Edge[](v0)(v10)
\Edge[](v1)(v10)
\Edge[](v2)(v10)
\Edge[](v3)(v10)
\Edge[](v4)(v10)
\Edge[](v5)(v11)
\Edge[](v6)(v11)
\Edge[](v7)(v11)
\Edge[](v8)(v11)
\Edge[](v9)(v11)
\Edge[](v5)(v12)
\Edge[](v6)(v12)
\Edge[](v7)(v12)
\Edge[](v8)(v12)
\Edge[](v9)(v12)
\Edge[](v5)(v13)
\Edge[](v6)(v13)
\Edge[](v7)(v13)
\Edge[](v8)(v13)
\Edge[](v9)(v13)
\Edge[](v11)(v10)
\Edge[](v11)(v4)
\Edge[](v12)(v0)
\Edge[](v12)(v1)
\Edge[](v13)(v3)
\Edge[](v13)(v2)
\end{tikzpicture}}}}\qquad\qquad
\subfloat[$\Delta=7$~~~~~~~~~~~~~~~]{
{\parbox{5cm}{

\begin{tikzpicture}[scale = 10]
\tikzstyle{VertexStyle}=[shape = circle, minimum size = 1pt, inner sep = 3pt,
draw]
\Vertex[x = 0.257401078939438, y = 0.729450404644012, L = \tiny {}]{v0}
\Vertex[x = 0.232565611600876, y = 0.681758105754852, L = \tiny {}]{v1}
\Vertex[x = 0.383801102638245, y = 0.820650428533554, L = \tiny {}]{v2}
\Vertex[x = 0.358965694904327, y = 0.772958129644394, L = \tiny {}]{v3}
\Vertex[x = 0.40850430727005, y = 0.773111909627914, L = \tiny {}]{v4}
\Vertex[x = 0.506290018558502, y = 0.730872631072998, L = \tiny {}]{v5}
\Vertex[x = 0.530993163585663, y = 0.683334112167358, L = \tiny {}]{v6}
\Vertex[x = 0.440956592559814, y = 0.589494824409485, L = \tiny {}]{v7}
\Vertex[x = 0.416121125221252, y = 0.541802525520325, L = \tiny {}]{v8}
\Vertex[x = 0.46565979719162, y = 0.541956305503845, L = \tiny {}]{v9}
\Vertex[x = 0.317756593227386, y = 0.592694818973541, L = \tiny {}]{v10}
\Vertex[x = 0.292921125888824, y = 0.545002520084381, L = \tiny {}]{v11}
\Vertex[x = 0.342459797859192, y = 0.545156300067902, L = \tiny {}]{v12}
\Edge[](v1)(v0)
\Edge[](v3)(v2)
\Edge[](v4)(v2)
\Edge[](v4)(v3)
\Edge[](v6)(v5)
\Edge[](v8)(v7)
\Edge[](v9)(v7)
\Edge[](v9)(v8)
\Edge[](v11)(v10)
\Edge[](v12)(v10)
\Edge[](v12)(v11)
\Edge[](v2)(v0)
\Edge[](v3)(v0)
\Edge[](v4)(v0)
\Edge[](v2)(v1)
\Edge[](v3)(v1)
\Edge[](v4)(v1)
\Edge[](v2)(v5)
\Edge[](v3)(v5)
\Edge[](v4)(v5)
\Edge[](v2)(v6)
\Edge[](v3)(v6)
\Edge[](v4)(v6)
\Edge[](v5)(v7)
\Edge[](v6)(v7)
\Edge[](v5)(v9)
\Edge[](v6)(v9)
\Edge[](v5)(v8)
\Edge[](v6)(v8)
\Edge[](v10)(v8)
\Edge[](v11)(v8)
\Edge[](v12)(v8)
\Edge[](v10)(v7)
\Edge[](v11)(v7)
\Edge[](v12)(v7)
\Edge[](v10)(v9)
\Edge[](v11)(v9)
\Edge[](v12)(v9)
\Edge[](v10)(v1)
\Edge[](v11)(v1)
\Edge[](v12)(v1)
\Edge[](v10)(v0)
\Edge[](v11)(v0)
\Edge[](v12)(v0)
\end{tikzpicture}}}}\qquad\qquad
\subfloat[$\Delta=8$~~~~~~~~~~~~~~~]{
{\parbox{5cm}{

\begin{tikzpicture}[scale = 10]
\tikzstyle{VertexStyle}=[shape = circle, minimum size = 1pt, inner sep = 3pt,
draw]
\Vertex[x = 0.257401078939438, y = 0.729450404644012, L = \tiny {}]{v0}
\Vertex[x = 0.232565611600876, y = 0.681758105754852, L = \tiny {}]{v1}
\Vertex[x = 0.282104313373566, y = 0.681911885738373, L = \tiny {}]{v2}
\Vertex[x = 0.383801102638245, y = 0.820650428533554, L = \tiny {}]{v3}
\Vertex[x = 0.358965694904327, y = 0.772958129644394, L = \tiny {}]{v4}
\Vertex[x = 0.40850430727005, y = 0.773111909627914, L = \tiny {}]{v5}
\Vertex[x = 0.506290018558502, y = 0.730872631072998, L = \tiny {}]{v6}
\Vertex[x = 0.48145455121994, y = 0.683180332183838, L = \tiny {}]{v7}
\Vertex[x = 0.530993163585663, y = 0.683334112167358, L = \tiny {}]{v8}
\Vertex[x = 0.440956592559814, y = 0.589494824409485, L = \tiny {}]{v9}
\Vertex[x = 0.416121125221252, y = 0.541802525520325, L = \tiny {}]{v10}
\Vertex[x = 0.46565979719162, y = 0.541956305503845, L = \tiny {}]{v11}
\Vertex[x = 0.317756593227386, y = 0.592694818973541, L = \tiny {}]{v12}
\Vertex[x = 0.292921125888824, y = 0.545002520084381, L = \tiny {}]{v13}
\Vertex[x = 0.342459797859192, y = 0.545156300067902, L = \tiny {}]{v14}
\Edge[](v2)(v1)
\Edge[](v2)(v0)
\Edge[](v1)(v0)
\Edge[](v4)(v3)
\Edge[](v5)(v3)
\Edge[](v5)(v4)
\Edge[](v7)(v6)
\Edge[](v8)(v6)
\Edge[](v8)(v7)
\Edge[](v10)(v9)
\Edge[](v11)(v9)
\Edge[](v11)(v10)
\Edge[](v13)(v12)
\Edge[](v14)(v12)
\Edge[](v14)(v13)
\Edge[](v3)(v0)
\Edge[](v4)(v0)
\Edge[](v5)(v0)
\Edge[](v3)(v2)
\Edge[](v4)(v2)
\Edge[](v5)(v2)
\Edge[](v3)(v1)
\Edge[](v4)(v1)
\Edge[](v5)(v1)
\Edge[](v3)(v6)
\Edge[](v4)(v6)
\Edge[](v5)(v6)
\Edge[](v3)(v7)
\Edge[](v4)(v7)
\Edge[](v5)(v7)
\Edge[](v3)(v8)
\Edge[](v4)(v8)
\Edge[](v5)(v8)
\Edge[](v6)(v9)
\Edge[](v7)(v9)
\Edge[](v8)(v9)
\Edge[](v6)(v11)
\Edge[](v7)(v11)
\Edge[](v8)(v11)
\Edge[](v6)(v10)
\Edge[](v7)(v10)
\Edge[](v8)(v10)
\Edge[](v12)(v10)
\Edge[](v13)(v10)
\Edge[](v14)(v10)
\Edge[](v12)(v9)
\Edge[](v13)(v9)
\Edge[](v14)(v9)
\Edge[](v12)(v11)
\Edge[](v13)(v11)
\Edge[](v14)(v11)
\Edge[](v12)(v2)
\Edge[](v13)(v2)
\Edge[](v14)(v2)
\Edge[](v12)(v1)
\Edge[](v13)(v1)
\Edge[](v14)(v1)
\Edge[](v12)(v0)
\Edge[](v13)(v0)
\Edge[](v14)(v0)
\end{tikzpicture}}}}
\caption{Counterexamples to the Borodin-Kostochka Conjecture for small
$\Delta$.}
\label{fig:SmallCE}
\end{figure}


\begin{conjecture}[Borodin and Kostochka \cite{borodin1977upper}]
Every graph with $\Delta \geq 9$ satisfies $\chi \leq \max\{\omega, \Delta - 1\}$.
\end{conjecture}

Note that another way of stating this is that for $\Delta \geq 9$, the only obstruction to $(\Delta-1)$-coloring is a $K_\Delta$. In the same paper, Borodin and Kostochka prove the following weaker statement.

\begin{thm}[Borodin and Kostochka \cite{borodin1977upper}]\label{BorodinKostochkaBK}
Every graph satisfying $\chi \geq \Delta \geq 7$ contains a $K_{\floor{\frac{\Delta + 1}{2}}}$.
\end{thm}

The proof is quite simple once you have a decomposition lemma of Lov\'{a}sz from the 1960's \cite{lovasz1966decomposition}.

\begin{lem}[Lov\'{a}sz
\cite{lovasz1966decomposition}]\label{LovaszDecomposition} Let $G$ be a graph and $r_1, \ldots, r_k \in \IN$ such that $\sum_{i=1}^k r_i \geq \Delta(G) + 1 - k$. 
Then $V(G)$ can be partitioned into sets $V_1, \ldots, V_k$ such that $\Delta(G[V_i]) \leq r_i$ for each $i \in \irange{k}$.
\end{lem}
\begin{proof}
For a partition $P \DefinedAs \parens{V_1, \ldots, V_k}$ of $V(G)$ let
			\[f(P) \DefinedAs \sum_{i=1}^k \parens{\size{G[V_i]} - r_i\card{V_i}}.\]

\noindent Let $P \DefinedAs \parens{V_1, \ldots, V_k}$ be a partition of $V(G)$
minimizing $f(P)$.  Suppose there is $i \in \irange{k}$ and $x \in V_i$ with
$d_{V_i}(x) > r_i$. Since $\sum_{i=1}^k r_i \geq \Delta(G) + 1 - k$, there is
some $j \neq i$ such that $d_{V_j}(x) \leq r_j$ and thus moving $x$ from $V_i$ to
$V_j$ gives a new partition violating minimality of $f(P)$.  Hence
$\Delta(G[V_i]) \leq r_i$ for each $i \in \irange{k}$.
\end{proof}

Now to prove Borodin and Kostochka's result, let $G$ be a graph with $\chi \geq
\Delta \geq 7$ and use $r_1 \DefinedAs \ceil{\frac{\Delta - 1}{2}}$ and $r_2
\DefinedAs \floor{\frac{\Delta - 1}{2}}$ in Lov\'{a}sz's lemma to get a
partition $(V_1, V_2)$ of $V(G)$ with $\Delta(G[V_i]) \leq r_i$ for each $i \in
\irange{2}$.  Since $r_1 + r_2 = \Delta - 1$ and $\chi \geq \Delta$, it
must be that $\chi(G[V_i]) \geq r_i + 1$ for some $i \in \irange{2}$. But
$\Delta \geq 7$, so $r_i \geq 3$ and hence by Brooks' theorem $G[V_i]$ contains
a $K_{\floor{\frac{\Delta + 1}{2}}}$.

A decade later, Catlin \cite{CatlinAnotherBound} showed that bumping the $\Delta(G) + 1$ to $\Delta(G) + 2$ allowed for shuffling vertices from one partition set to another and thereby 
proving stronger decomposition results. A few years later Kostochka \cite{KostochkaTriangleFree} modified Catlin's algorithm to show that every triangle-free graph $G$ can be 
colored with at most $\frac23 \Delta(G) + 2$ colors.  In \cite{rabern2010destroying}, we generalized Kostochka's modification to prove the following.

\begin{replem}{DestroyLemma}[Rabern \cite{rabern2010destroying}]
Let $G$ be a graph and $r_1, \ldots, r_k \in \IN$ such that $\sum_{i=1}^k r_i \geq \Delta(G) + 2 - k$. 
Then $V(G)$ can be partitioned into sets $V_1, \ldots, V_k$ such that $\Delta(G[V_i]) \leq r_i$ and $G[V_i]$ contains no incomplete $r_i$-regular components for each $i \in \irange{k}$.
\end{replem}

Setting $k = \left \lceil \frac{\Delta(G) + 2}{3} \right \rceil$ and $r_i = 2$
for each $i$ gives a slightly more general form of Kostochka's triangle-free
coloring result.

\begin{repcor}{TrianglesAndPaths}[Rabern \cite{rabern2010destroying}]
The vertex set of any graph $G$ can be partitioned into $\left \lceil \frac{\Delta(G) + 2}{3} \right \rceil$ sets, each of which induces a disjoint union of triangles and paths.
\end{repcor}

For coloring, this actually gives the bound $\chi(G) \leq 2  \left \lceil \frac{\Delta(G) + 2}{3} \right \rceil$ for triangle free graphs.  
To get $\frac23 \Delta(G) + 2$, just use $r_k = 0$ when $\Delta \equiv 2 (\text{mod } 3)$. 
Similarly, for any $r \geq 2$, setting $k = \left \lceil \frac{\Delta(G) + 2}{r + 1} \right \rceil$ and $r_i = r$ for each $i$ gives the following.
\begin{repcor}{NoKrPlusOne}[Rabern \cite{rabern2010destroying}]
Fix $r \geq 2$.  The vertex set of any $K_{r + 1}$-free graph $G$ can be partitioned into $\left \lceil \frac{\Delta(G) + 2}{r + 1} \right \rceil$ sets each inducing an $(r-1)$-degenerate subgraph with maximum degree at most $r$.
\end{repcor}
In fact, we proved a lemma stronger than Lemma \ref{DestroyLemma} allowing us
to forbid a larger class of components coming from any so-called \emph{$r$-permissible collection}. In section \ref{ShuffleHeightSection} we will explore a result that both simplifies and generalizes this latter result.

Also in the 1980's, Kostochka proved the following using a complicated recoloring argument together with a technique for reducing $\Delta$ in a counterexample 
based on hitting every maximum clique with an independent set.

\begin{thm}[Kostochka \cite{kostochkaRussian}]\label{KostochkaBK}
Every graph satisfying $\chi \geq \Delta$ contains a $K_{\Delta - 28}$.
\end{thm}

Kostochka \cite{kostochkaRussian} proved the following result which shows that
graphs having clique number sufficiently close to their maximum degree contain
an independent set hitting every maximum clique. In \cite{rabernhitting} we
improved the antecedent to $\omega \geq \frac34(\Delta + 1)$.  Finally, King  \cite{KingHitting} made the result tight.

\begin{replem}{KostochkaHitting}[Kostochka \cite{kostochkaRussian}]
If $G$ is a graph satisfying $\omega \geq \Delta + \frac32 - \sqrt{\Delta}$,
then $G$ contains an independent set $I$ such that $\omega(G - I) < \omega(G)$.
\end{replem}
\begin{replem}{RabernHitting}[Rabern \cite{rabernhitting}]
If $G$ is a graph satisfying $\omega \geq \frac34 (\Delta + 1)$,
then $G$ contains an independent set $I$ such that $\omega(G - I) < \omega(G)$.
\end{replem}
\begin{replem}{HittingMaxCliques}[King \cite{KingHitting}]
If $G$ is a graph satisfying $\omega > \frac23 (\Delta + 1)$, then $G$ contains
an independent set $I$ such that $\omega(G - I) < \omega(G)$.
\end{replem}

If $G$ is a vertex critical graph satisfying $\omega > \frac23 (\Delta + 1)$ and
we expand the independent set $I$ produced by Lemma \ref{HittingMaxCliques} to a
maximal independent set $M$ and remove $M$ from $G$, we see that $\Delta(G-M)
\leq \Delta(G) - 1$, $\chi(G-M) = \chi(G) - 1$ and $\omega(G - M) = \omega(G) - 1$. Using this, the proof of many coloring results can be reduced to the case of the
smallest $\Delta$ for which they work. In Chapter
\ref{ReductionChapter}, we give three such applications.

A little after Kostochka proved his bound, Mozhan \cite{mozhan1983}
used a function minimization and vertex shuffling procedure different than, but related to Catlin's, to prove the following.  

\begin{thm}[Mozhan \cite{mozhan1983}]\label{MozhanTwoThirdsBK}
Every graph satisfying $\chi \geq \Delta \geq 10$ contains a $K_{\floor{\frac{2\Delta + 1}{3}}}$.
\end{thm}

Finally, in his dissertation Mozhan proved the following.  We don't know the
method of proof as we were unable to obtain a copy of his dissertation. 
However, we suspect the method is a more complicated version of the above proof.

\begin{thm}[Mozhan]\label{MozhanBK}
Every graph satisfying $\chi \geq \Delta \geq 31$ contains a $K_{\Delta - 3}$.
\end{thm}

In \cite{rabern2010a}, we used part of Mozhan's method to prove the following result.  For a graph $G$ let $\mathcal{H}(G)$ be the subgraph of $G$ induced on the vertices of degree at least $\chi(G)$.

\begin{thm}[Rabern \cite{rabern2010a}]\label{TheoremM}
$K_{\chi(G)}$ is the only vertex critical graph $G$ with $\chi(G) \geq \Delta(G) \geq 6$ and $\omega(\mathcal{H}(G)) \leq \left \lfloor \frac{\Delta(G)}{2} \right \rfloor - 2$.
\end{thm}

Setting $\omega(\mathcal{H}(G)) = 1$ proved a conjecture of Kierstead and Kostochka \cite{kierstead2009ore}.

\begin{cor}[Rabern \cite{rabern2010a}]\label{CorollaryN}
$K_{\chi(G)}$ is the only vertex critical graph $G$ with $\chi(G) \geq \Delta(G) \geq 6$ such that $\mathcal{H}(G)$ is edgeless.
\end{cor}

In joint work with Kostochka and Stiebitz \cite{krs_one}, we generalized and improved this result, again using Mozhan's technique.  
In section \ref{ShuffleLowsSection}, we will improve these results further and simplify the proofs by using Catlin's vertex shuffling algorithm in place of Mozhan's.

In 1999, Reed used probabilistic methods to prove that the Borodin-Kostochka conjecture holds for graphs with very large maximum degree.

\begin{thm}[Reed \cite{reed1999strengthening}]\label{ReedBK}
Every graph satisfying $\chi \geq \Delta \geq 10^{14}$ contains a $K_\Delta$.
\end{thm}

A lemma from Reed's proof of the above theorem is generally useful.

\begin{lem}[Reed \cite{reed1999strengthening}]\label{ReedsLemma}
Let $G$ be a critical graph satisfying $\chi = \Delta \geq 9$ having the minimum number of vertices.  If $H$ is a $K_{\Delta - 1}$ in $G$, then any vertex in $G - H$ has at most $4$ neighbors in $H$.  In particular, the $K_{\Delta - 1}$'s in $G$ are pairwise disjoint.
\end{lem}

In Chapter \ref{APrioriWowChapter}, we improve this lemma by showing that under
the same hypotheses, any vertex in $G - H$ has at most $1$ neighbor in $H$.  Moreover, we lift the result out of the context of a minimal counterexample to graphs satisfying a certain criticality condition---we refer to such graphs as mules.  This allows meaningful results to be proved for values of $\Delta$ less than $9$.  Also in Chapter \ref{APrioriWowChapter}, we prove that the following, prima facie weaker, conjecture is equivalent to the Borodin-Kostochka conjecture.

\begin{conjecture}[Cranston and Rabern \cite{mules}] 
If $G$ is a graph with $\chi = \Delta = 9$, then $\join{K_3}{\overline{K}_6}
\subseteq G$.
\end{conjecture}

At the core of these results are the list coloring lemmas proved in section
\ref{ListColoringChapter}.  There we classify graphs of the form $\join{A}{B}$
that are not $f$-choosable where $f(v) \DefinedAs d(v) - 1$ for each vertex
$v$.  In Chapter \ref{ClawFreeChapter} we use these list coloring results
together with Chudnovsky and Seymour's decomposition theorem for claw-free graphs \cite{chudnovsky2005structure} and our proof in \cite{rabern2011strengthening} of the Borodin-Kostochka conjecture for line graphs of multigraphs to prove the conjecture for claw-free graphs.

\begin{thm}[Cranston and Rabern \cite{cranstonrabernclaw}]
Every claw-free graph with $\Delta\geq 9$ satisfies $\chi \leq \max\set{\omega, \Delta - 1}$.
\end{thm}

In Chapter \ref{StrongColoringChapter}, we adapt a recoloring trick
previously used for strong coloring and prove the following.

\begin{repcor}{TwoThirdsCliqueCor}
Every graph with $\chi \geq \Delta \geq 9$ such that every
vertex is in a clique on $\frac23\Delta + 2$ vertices contains $K_\Delta$.
\end{repcor}

Using this we show that to prove the Borodin-Kostochka conjecture it is enough
to prove it for irregular graphs; more precisely, we prove the following.

\begin{repthm}{IrregularReduction}
Every graph satisfying $\chi \geq \Delta = k \geq 9$ either
contains $K_k$ or contains an irregular critical subgraph satisfying $\chi
= \Delta = k - 1$.
\end{repthm}

In particular, the Borodin-Kostochka conjecture would follow from the following.
This is at least somewhat plausible since the only known critical (or connected
even) counterexample to Borodin-Kostochka for $\Delta = 8$ is regular (see
Figure \ref{fig:M_8}).

\begin{repconjecture}{EightRegular}
Every critical graph satisfying $\chi \geq \Delta = 8$ is regular.
\end{repconjecture}

As our final application of the recoloring trick, we prove the following bounds on the chromatic number.  The first generalizes the result of Beutelspacher and Hering \cite{beutelspacher1984minimal} that the Borodin-Kostochka conjecture holds for graphs with independence number at most two.  This result was generalized in another direction in \cite{cranstonrabernclaw} (also Chapter \ref{ClawFreeChapter}) where the conjecture was proved for claw-free graphs.

\begin{repthm}{AlphaBound}
Every graph satisfies $\chi \leq \max\set{\omega, \Delta - 1, 4\alpha}$.
\end{repthm}

The second bound shows that the Borodin-Kostochka conjecture holds for graphs with maximum degree on the order of the square root of their order.  This improves on prior bounds of $\Delta > \frac{n + 1}{2}$ from Beutelspacher and Hering \cite{beutelspacher1984minimal} and $\Delta > \frac{n-6}{3}$ of Naserasr \cite{naserasr}.

\begin{repthm}{OrderBound}
Every graph satisfies $\chi \leq \max\set{\omega, \Delta - 1, \ceil{\frac{15 + \sqrt{48n + 73}}{4}}}$.
\end{repthm}

Borodin and Kostochka also conjectured \cite{PersonalComms} that
their conjecture holds for list coloring. In Chapter \ref{ChoiceLargeDeltaChapter}, we prove that this conjecture holds for large $\Delta$.

\begin{conjecture}[Borodin and Kostochka \cite{PersonalComms}]\label{BKList}
Every graph with $\Delta \geq 9$ satisfies $\chi_l \leq \max\{\omega, \Delta -
1\}$.
\end{conjecture}

