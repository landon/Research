
%Peter W.
%Requires the memoir class (as of this date v1.6180339e 2009/02/17)
%I suggest
\documentclass[oneside,12pt]{memoir}
%%% with the wide textblock, 12pt is too small for reading ease, so best not
%%% to use 11pt or 10pt.

%%% Arial
%\usepackage[T1]{fontenc}
%\usepackage[scaled]{uarial}
%\renewcommand*\familydefault{\sfdefault} %% Only if the base font of the document is to be sans serif

%%% Garamond
%\usepackage[T1]{fontenc}
%\usepackage{lmodern}
%\usepackage{garamond}

%%% MS San Serif
%\usepackage[T1]{fontenc}
%\usepackage[scaled]{helvet}
%\renewcommand*\familydefault{\sfdefault} %% Only if the base font of the document is to be sans serif

%%% Times
%\usepackage{mathptmx}  % Times New Roman, but if you have Garamond then use it;
                       % you are writing a book, not a newspaper column
\DoubleSpacing         % memoir's double spacing
\usepackage{pwasu}     % this package



%%%%%%%%%%%%%%%%%%%%%%%%%%%%%%%%%%%%%%%%%%%%%%%%
% Packages

\usepackage{amsmath, amsfonts,amssymb,amsthm}
\usepackage{color}
%\usepackage{epsf}
\usepackage{verbatim}
\usepackage{enumitem}
\usepackage[sharp]{easylist}
\usepackage[labelformat=empty]{subfig}
%\setcounter{MaxMatrixCols}{40}

%\usepackage[showframe]{geometry}
%\usepackage{showframe}



\newtheorem{theorem}{Theorem}[section]
\newtheorem{lemma}[theorem]{Lemma}
\newtheorem{proposition}[theorem]{Proposition}
\newtheorem{corollary}[theorem]{Corollary}
\newtheorem{fact}[theorem]{Fact}
\newtheorem{conjecture}[theorem]{Conjecture}
\newtheorem{claim}[theorem]{Claim}
\newtheorem{definition}[theorem]{Definition}
\newtheorem{problem}{Problem}


% 
% \makeatletter
% \@addtoreset{lemma}{chapter}
% \makeatother




\newcommand{\ep}{\epsilon}
\newcommand{\g}{\gamma}
\newcommand{\floor}[1]{\left\lfloor#1\right\rfloor}
\newcommand{\ceiling}[1]{\left\lceil#1\right\rceil}
\newcommand{\half}{\frac{1}{2}}
\newcommand{\aster}{\textasteriskcentered}
\newcommand{\ssm}{\smallsetminus}
\newcommand{\ext}{\mathrm{ext}}
\newcommand{\croot}[1]{\left\lceil\sqrt{#1}\right\rceil}
\newcommand{\n}{2\times 10^8}
\newcommand{\KSS}{Koml\'{o}s, S\'{a}rk\"{o}zy, Szemer\'{e}di }
\newcommand{\KSaS}{Koml\'{o}s, S\'{a}rk\"{o}zy, and Szemer\'{e}di }


\renewcommand{\labelenumi}{\emph{(\roman{enumi})}}

%\input{psfig}
%\setcounter{MaxMatrixCols}{40}


\setcounter{tocdepth}{3}
\maxsecnumdepth{subsection}






%%%%%%Added by Craig Picone to meet ASU's margin requirements
\usepackage{graphicx}    % needed for including graphics e.g. EPS, PS
\topmargin -0.4125in        % read Lamport p.163
%\oddsidemargin 0.625in   % read Lamport p.163
\evensidemargin 0in  % same as oddsidemargin but for left-hand pages
%\textwidth 6in
\textheight 9.1in 
%\pagestyle{empty}       % Uncomment if don't want page numbers
\parskip 7.2pt           % sets spacing between paragraphs
%\renewcommand{\baselinestretch}{1.5} % Uncomment for 1.5 spacing between lines
\parindent 0pt          % sets leading space for paragraphs
 %%%%%%%%

%    The general sequence in your document, after you have set the data for
%the TITLE and APPROVAL pages, and any other specifics in the preamble is:
\DoubleSpacing
\begin{document}
\maxtocdepth{subparagraph} % put everything into the ToC
\pagestyle{plain}  % pagestyle for the prelims
\frontmatter
\thetitlepage
%%\approvalpage

%% Added by bbailey1
% Macro for List of Symbols
\def\listofsymbols{\input{memoir/symbols} \clearpage}
\def\addsymbol #1: #2#3{$#1$ \> \parbox{5.45in}{#2 \dotfill \pageref{#3}}\\}
\def\newnot#1{\label{#1}}
 
 







%%%%%%%%%%%%%%%%%%%%%%%%%%%%%%%%%%%%%%%%%%%%%%%%%%%%%%%%%%%%%%%%%%%
% here is the main part of your dissertation
 
% put your abstract here

\asuabstract 
\setcounter{page}{1}
\setlength{\parindent}{.5in}

In a large network (graph) it would be desirable to guarantee the existence of some local property based only on global knowledge of the network.  For instance we might want to know how many connections are necessary to guarantee that the network contains three nodes which are pairwise adjacent.  It turns out that more than $n^2/4$ connections are needed, and no smaller number will suffice in general.  Problems of this type fall under the category of ``extremal graph theory.''

Generally speaking, extremal graph theory is the study of how global parameters of a graph are related to local properties.  This dissertation deals with the relationship between minimum degree conditions of a host graph $G$ and the property that $G$ contains a specified spanning subgraph (or class of subgraphs).  The goal is to find the optimal minimum degree which guarantees the existence of a desired spanning subgraph.

The main tools will be Szemer\'edi's Regularity Lemma and some basic probabilistic techniques.  These tools will be used to obtain four main results.  Chapter \ref{posachapter} contains a proof of P\'osa's conjecture about the square of a Hamilton cycle for graphs on at least $\n$ vertices.  Chapter \ref{2factorschapter} is devoted to a problem about the existence of $2$-factors in a bipartite graph with a generalized minimum degree condition.  Finally, in Chapters \ref{mindegtilingchapter} and \ref{sumdegtilingchapter}, analogs of the Hajnal-Szemer\'edi theorem are explored for bipartite graphs.

% In 1962 P\'osa conjectured that every graph $G$ on $n$ vertices with
% minimum degree $\delta(G)\ge\frac{2}{3}n$ contains the square of
% a hamiltonian cycle. In 1996, Koml\'os, S\'ark\"ozy, and Szemer\'edi proved P\'osa's Conjecture,
% using the Regularity and Blow-up Lemmas, for graphs of order $n\ge n_{0}$, where $n_{0}$ %HK
% is a very large constant. Here we show without using these lemmas
% that $n_{0}:=\n$ is sufficient. 
% %We are motivated by the 
% %recent work of Levitt, Szemer\'edi and S\'ark\"ozy, but our methods are
% %based on techniques that were available in the 90's. 
% 
% Let $G$ and $H$ be balanced $U,V$-bigraphs on $2n$ vertices with $\Delta(H)\leq 2$.  Let $k$ be the number of components of $H$, $\delta_U:=\min\{\deg_G(u):u\in U\}$ and  $\delta_V:=\min\{\deg_G(v):v\in V\}$.  We prove that if $n$ is sufficiently large and $\delta_U+\delta_V\geq n+k$ then $G$ contains $H$.  This solves a conjecture of Amar %\cite{A} 
% in the case that $n$ is large. We also show that $G$ contains $H$ even when $\delta_U+\delta_V\geq n+2$ as long as $n$ is sufficiently large in terms of $k$ and $\delta(G)$ is $\Omega(n)$, which partially solves a conjecture of Wang.
% %\cite{W2}.
% %\ge \frac{n}{200k}+1.
% 
% Bipartite graph tiling was studied by Zhao 
% %\cite{Z} 
% who gave the best possible minimum degree conditions for a balanced bipartite graph on $2ms$ vertices to contain $m$ vertex disjoint copies of $K_{s,s}$. Let $s<t$ be fixed positive integers. Hladk\'y and Schacht 
% %\cite{HS} 
% gave minimum degree conditions for a balanced bipartite graph on $2m(s+t)$ vertices to contain $m$ vertex disjoint copies of $K_{s,t}$.  Their results were best possible, except in the case when $m$ is odd and $t> 2s+1$.  We give the best possible minimum degree condition in this case.
% 
% For a balanced bipartite graph $G[U,V]$ we consider the problem of determining the optimal value of $\delta_U+\delta_V$ which guarantees that $G$ can be tiled with $K_{s,s}$.  We show that the optimal value depends on $d:=|\delta_V-\delta_U|$.  When $d$ is small, we show that $\delta_U+\delta_V\geq n+3s-5$ is best possible.  As $d$ becomes larger, we show that $\delta_U+\delta_V$ can be made smaller, but no smaller than $n+2s-2\croot{s}$.  However, when $d=n-C$ for some constant $C$, we show that there exist graphs with $\delta_U+\delta_V\geq n+s^{s^{1/3}}$ which cannot be tiled with $K_{s,s}$.



 
% your acknowledgement

	\newpage
\centering{DEDICATION}
\flushleft{Tom was a good mathematician, and a good man.  He was -- he was one of us.  He was a man who loved the outdoors, and math. And as a teacher he explored the schools of the northeast from New York to Massachusetts, and down to Arizona.
He died, as so many young men of his generation, before his time...

This dissertation is dedicated to the memory of my uncle Tom Wielunski.  Unfortunately, my period of mathematical enlightenment didn't intersect with your life; however, I remain inspired by you.
}







\asuacknowledgements
{Most importantly, I would like to thank my parents.  Their support has helped me achieve goals that, just a short time ago, I never would have imagined.

This dissertation would not be possible without the guidance and assistance of my advisors, Andrzej Czygrinow and Hal Kierstead.  %Andrzej taught me my first course in Graph Theory, then a couple of years later the three of us worked together on what would be my first paper (and Hal's 100th paper).  
%I must also thank Hal for starting my career off with an Erd\H{o}s number of $2$.  
I can't thank them enough for the countless hours they spent working with me.  
I would like to thank the members of my committee: Glenn Hurlbert, Susanna Fishel and Kevin Kadell.  In particular, Glenn has had a major influence on my mathematical development. % and I credit him for beginning my fascination with mathematics.
%; he taught me my first proof based math course with an enthusiasm which was infectious.  In fact his enthusiasm was so infectious that I took as many of his classes as possible, not only for his inspiring teaching, but also for having a great sense of humor.  
In addition, I am extremely grateful to Yi Zhao, Matthias Kawski, Keyong Hah Roh, and Katie Kolossa for taking the time to write letters of recommendation for me; Anne Gelb for her advice and her work with PFMF; and John Jones for his time during my defense.%; and Helen Barcelo for teaching me about Enumerative Combinatorics.

Thank you to Kelsey Vance for her love and encouragement; David Smith for being a great friend and a great source of knowledge; and Phong Ch\^au for the many stimulating discussions.  Additionally, I've had the pleasure of being friends with the following people, some I met in grad school and others long before: Adam Bland, Katie Carter, Ian Dye, Rosalynn Dye, Molly Etchebarren, Ryan Etchebarren, Tacker Frink, Aviva Halani, Ben Hester, Taylor Hines, Vikram Kamat, Marilyn Murphy, Bruce Rogers, Karin Saoub, Susie Seal, Chris Severs, Wei Shen, Matt Smith, and Heather Sollum.

I must thank the ASU math department as a whole for providing me with many opportunities; I have been very fortunate.  
%I have been very fortunate to receive summer research grants, travel funds, awards, and excellent teaching opportunities, all of which have benefited me greatly.  
A special thanks goes to Debbie Olsen for answering all of my questions and helping me out in a big way with the job application process.

Finally, one BIG thank you to my BIG family who have always been so supportive of me.
}

\tableofcontents
% \listoftables   % if you have any tables

\listoffigures  % if you have any figures

%% Added by bbailey1
%% Uncomment the next 3 lines for List of Symbols
% \newpage
% \chapter*{List of Symbols\hfill} \addcontentsline{toc}{chapter}{LIST OF SYMBOLS}
% \listofsymbols

%%
% Mark your variables in your source code with \newnot{YOUR_SYMBOL_LABEL}.
% Example:
% ...Here, if the dimensions of A \newnot{sybmol:A}, B \newnot{symbol:B}, and C \newnot{symbol:C} are
% nxn, nxm and lxn \newnot{symbol:nml} respectfully; then ...
%%

%\newpage
%\chapter*{PREFACE\hfill} \addcontentsline{toc}{chapter}{PREFACE}
%[Enter your text here]
%\clearpage

%% if you have more prelim sections, then
%%%\clearpage
%%%%%\pagestyle{plain}
%%%%%\prelimtitle   text % for sections after the ToC, etc, before main text
\mainmatter
\pagestyle{asu}

\addcontentsline{toc}{chapter}{CHAPTER}

\pagestyle{plain} 
% finally, start of your main text



\chapter{BACKGROUND MATERIAL}

\DoubleSpacing
\setlength{\parindent}{.5in}

A \emph{hypergraph} is a pair of sets $(V,E)$ with the property that $E$ is a family of subsets of $V$.  A \emph{graph} is a hypergraph in which every element of $E$ has order $2$.  Given a hypergraph $G=(V,E)$, we refer to the set $V$ as \emph{vertices}, and the set $E$ as \emph{edges}.  For any graph $G$, we will use the notation $V(G)$ to represent the set of vertices of $G$ and the notation $E(G)$ to represent the edges of $G$.  In this dissertation we will only consider graphs with a finite vertex set.  Given a set $V$ and a nonnegative integer $k$, we let $\binom{V}{k}=\{S\subseteq V:|S|=k\}$.  For a positive integer $k$, let $[k]=\{1,2,\dots, k\}$.  We write edges $\{x,y\}$ as $xy$.  

Let $H=(W,F)$ and $G=(V, E)$ be graphs.  We say $H$ is \emph{isomorphic} to $G$ if there exists a function $f:W\to V$ such that $xy\in F$ if and only if $f(x)f(y)\in E$.  We say $H$ is a \emph{subgraph} of $G$, denoted $H\subseteq G$, if there exists some $V'\subseteq V$ and $E'\subseteq \binom{V'}{2}$ such that $H$ is isomorphic to $(V', E')$.  

The complete graph is a graph $G$ for which $E(G)=\binom{V(G)}{2}$.  We denote the complete graph on $r$ vertices as $K_r$, and we call $K_3$ a \emph{triangle}.  The starting point of extremal graph theory can be captured in the following question: If $G$ is a graph on $n$ vertices, what is the fewest number of edges that $G$ must have in order to guarantee that $G$ contains a triangle?  The answer to this question is Mantel's theorem from 1907.

\begin{theorem}[Mantel \cite{M}]
Let $G$ be a graph on $n$ vertices.  If $|E(G)|\geq \floor{\frac{n^2}{4}}+1$, then $K_3\subseteq G$.  Furthermore, there exists a graph with $\floor{\frac{n^2}{4}}$ edges which is triangle-free.
\end{theorem}

Of course the natural follow-up question is: If $r$ is fixed and $G$ is a graph on $n$ vertices, what is the fewest number of edges that $G$ must have in order to guarantee that $K_r\subseteq G$?  It appears as if Mantel's result was mostly unknown, since it was not until 1941 when Tur\'an independently asked himself that very question and solved it for all $r$ (for an incredible story of how Tur\'an solved this problem while working in a labor camp during World War II, see \cite{Twel}).  Let $T_r(n)$ be the complete $r$-partite graph on $n$ vertices such that the sizes of any two parts differ by at most $1$; it is clear that $T_r(n)$ does not contain a copy of $K_{r+1}$.  Let $t_r(n)$ be the number of edges in $T_r(n)$.  Note that when $r$ divides $n$, we have $t_r(n)=\binom{r}{2}\left(\frac{n}{r}\right)^2=\frac{r-1}{r}\frac{n^2}{2}$. 

\begin{theorem}[Tur\'an \cite{T}]
Let $G$ be a graph on $n$ vertices.  If $|E(G)|\geq t_r(n)+1$, then $K_{r+1}\subseteq G$.  Furthermore, there exists a graph with $t_r(n)$ edges which is $K_{r+1}$-free.
\end{theorem}

Starting with Tur\'an's theorem, the subject of extremal graph theory blossomed into a coherent subject with many interesting theorems and powerful techniques.

For the rest of this dissertation we will be focusing on subgraph problems, but of a slightly different type.  If $H$ is a subgraph of $G$, we say $H$ is \emph{spanning} if $H$ and $G$ have the same number of vertices.  In this case, it is no longer natural to ask how many edges $G$ must have so that $H\subseteq G$.  To see why, let $H$ be any connected graph on $n$ vertices and let $G$ be the complete graph $K_{n-1}$ plus an isolated vertex.  On one hand $G$ has almost every possible edge, but on the other hand $H$ is not a subgraph of $G$.  So when studying sufficient conditions for spanning subgraphs, the most natural thing is to restrict the number of edges at each vertex.  Let $G$ be a graph and $v\in V(G)$. The neighborhood of $v$, denoted $N(v)$, is the set $\{u\in V(G):uv\in E(G)\}$.  The degree of $v$, denoted $\deg(v)$, is the quantity $|N(v)|$. The \emph{minimum degree} of $G$, denoted $\delta(G)$, is the quantity $\min\{\deg(v):v\in V(G)\}$.  For a set $S\subseteq V(G)$, we write $\deg(v, S)$ for the quantity $|N(v)\cap S|$.  We will study the relationship between minimum degree of a graph $G$
and the property $H\subseteq G$.  If $G$ has at least as many vertices
as $H$, there is always a relationship between $\delta(G)$ and the
property $H\subseteq G$: if $\delta(G)\geq n-1$, then $G$ is complete
and $H\subseteq G$.  So the goal is to minimize $\delta(G)$ with
respect to the condition $H\subseteq G$.

We first define two special types of graphs. Let $P_k$ be a graph with vertex set $\{v_1, v_2, \dots, v_k\}$ and edge set $\{v_iv_{i+1}:
i\in[k-1]\}$.  We call $P_k$ a \emph{path} on $k$ vertices and we
denote $P_k$ as $v_1v_2\dots v_k$.  Let $C_k$ be a graph with vertex
set $\{v_1, v_2, \dots, v_k\}$ and edge set $\{v_iv_{i+1}:
i\in[k-1]\}\cup \{v_kv_1\}$.  We call $C_k$ a \emph{cycle} on $k$
vertices and we denote $C_k$ as $v_1v_2\dots v_kv_1$.

Let $G$ be a graph on $n$ vertices.  To illustrate the title ``Optimal Degree Conditions for Spanning Subgraphs'', we will fully examine the (well known) relationship between $\delta(G)$ and the property $C_n\subseteq G$.  We start with the following basic fact.

\begin{proposition}\label{delta+1}
If $\delta(G)\geq 2$, then $G$ contains a cycle on at least
$\delta(G)+1$ vertices.
\end{proposition}

\begin{proof}
Let $P=v_1v_2\dots v_k$ be a path of maximum length in $G$.  Since $P$
is maximum, $N(v_1)\subseteq V(P)$.  Since $\deg(v_1)\geq \delta(G)$,
there exists some $v_i\in N(v_1)$ such that $i\geq \delta(G)+1$.  Thus
$v_1v_2\dots v_iv_1$ is a cycle on at least $\delta(G)+1$ vertices.
\end{proof}

Unfortunately, we cannot use this result to directly conclude anything
about the current problem.  All we get is that $\delta(G)\geq n-1$
implies $C_n\subseteq G$, however we already knew this from the discussion
above.  So we try to do better.

\begin{proposition}
If $\delta(G)\geq \frac{2n}{3}$, then $C_n\subseteq G$.
\end{proposition}

\begin{proof}
Let $C=v_1v_2\dots v_kv_1$ be a cycle of maximum length in $G$.  By
Proposition \ref{delta+1}, we know $k\geq \frac{2n}{3}+1$.  If $k=n$,
we are done, so suppose not.  Let $x\in V(G)\setminus V(C)$.  If
$\deg(x, C)>\frac{k}{2}$, then there exists $i\in [k]$ such that $v_i,
v_{i+1}\in N(x)$, but then $v_1v_2\dots v_ixv_{i+1}\dots v_kv_1$ is
longer cycle than $C$.  So we may suppose that $\deg(x, C)\leq
\frac{k}{2}$.  However, now we have the following contradiction
$$\frac{2n}{3}\leq \deg(x)\leq \deg(x, C)+\deg(x, G-C)\leq
\frac{k}{2}+n-k-1=n-1-\frac{k}{2}\leq \frac{2n}{3}-\frac{3}{2}.$$
\end{proof}

So now we ask ourselves if any lower value of $\delta(G)$ will
suffice.  One thing to do would be to try to construct a graph with
minimum degree less than $\frac{2n}{3}$ which does not contain $C_n$.
After trying for a while, two examples might come to mind.

\begin{proposition}\label{diracexamples}
There exists a graph $G$ on $n$ vertices with $\delta(G)=\ceiling{\frac{n}{2}}-1$ such that $G$ does not contain $C_n$.
\end{proposition}

\begin{proof}

We give two examples of such a graph.  Let $G_1$ be the union of a complete graph of $\ceiling{\frac{n}{2}}$ vertices and a complete graph on $\floor{\frac{n}{2}}+1$ vertices which intersect in one vertex.  First note that $G_1$ has $\ceiling{\frac{n}{2}}+\floor{\frac{n}{2}}+1-1=n$ vertices.  Every vertex in $G_1$ has degree at least $\min\{n-1, \ceiling{\frac{n}{2}}-1, \floor{\frac{n}{2}}\}$.  Since $\floor{\frac{n}{2}}\geq \ceiling{\frac{n}{2}}-1$, we have $\delta(G_1)=\ceiling{\frac{n}{2}}-1$.  Since $G_1$ has a cut vertex, it is not the case that $C_n\subseteq G_1$.

Let $G_2$ be the complete bipartite graph with parts of size $\floor{\frac{n}{2}}+1$ and $\ceiling{\frac{n}{2}}-1$.  Since $\floor{\frac{n}{2}}+1\geq \ceiling{\frac{n}{2}}-1$, we have $\delta(G_2)=\ceiling{\frac{n}{2}}-1$.  Since $G_2$ is bipartite with unequal part sizes, it is not the case that $C_n\subseteq G_2$.




\end{proof}

In each of the examples above we have $\delta(G)=\ceiling{\frac{n}{2}}-1$, so we are not very close to $\frac{2n}{3}$. Perhaps at
this point we try to prove that $\delta(G)\geq \frac{n}{2}$ suffices.

\begin{theorem}[Dirac 1952 \cite{D}]
If $\delta(G)\geq \frac{n}{2}$, then $C_n\subseteq G$.
\end{theorem}

\begin{proof}

Let $P=v_1v_2\dots v_k$ be a path of maximum length in $G$.  Note that $k\geq \frac{n}{2}+1$ by Proposition \ref{delta+1}.  We first show that there exists a cycle $C$ with the property that $|C|=k$ and $V(P)\subseteq V(C)$.  Since $P$ is a maximum length path, $N(v_1)\subseteq V(P)$ and $N(v_k)\subseteq V(P)$.  Let $d_1:=\deg(v_1)$ and $d_k:=\deg(v_k)$.  We assign $d_1$ ``units of charge'' to $v_1$ and $d_k$ ``units of charge'' to $v_k$.  Let $N^+(v_k)=\{v_{i+1}:v_i\in N(v_k)\}$.  We now distribute the charge according the following rule: $v_1$ gives one unit of charge to each vertex in $N(v_1)$ and $v_k$ gives one unit of charge to each vertex in $N^+(v_k)$.  Note that according to the rule, $v_1$ necessarily ends up with $0$ units of charge.  There are now $d_1+d_k\geq n$ units of charge on at most $k-1\leq n-1$ vertices.  So some vertex $v_i\in\{v_2,\dots,v_k\}$ has two units of charge, which translates to $v_i\in N(v_1)$ and $v_{i-1}\in N(v_k)$.  Then $v_1\dots v_{i-1}v_k\dots v_iv_1$ is a cycle with the desired property.  If $k=n$, then we have $C_n\subseteq G$, so suppose not.  Let $x\in V(G)\setminus V(C)$.  Since $k\geq \frac{n}{2}+1$, we have $n-k\leq \frac{n}{2}-1$ and thus there exists $v_i\in V(C)\cap N(x)$.  But now $xv_i\dots v_kv_1\dots v_{i-1}$ is path which is longer than $P$, contradicting our assumption.

\end{proof}

So now we have an optimal result.  If $\delta(G)\geq \frac{n}{2}$, then
$C_n\subseteq G$, but no smaller value will suffice because of Proposition \ref{diracexamples}.  This example illustrates the type of results we will prove throughout the dissertation.

Two of the main threads running through the research presented here (in Chapters \ref{posachapter}, \ref{mindegtilingchapter}, and \ref{sumdegtilingchapter}) can be traced back to Problem 9 of the Proceedings of the Symposium held in Smolenice in June 1963 \cite{E}.  Given a graph $G=(V,E)$ let \emph{$r^\text{th}$ power} of $G$, denoted $G^r$, be the graph obtained by adding an edge between every pair of vertices of distance at most $r$ in $G$.  
%More precisely, given $G$, let $G^r=(V,\{\{x,y\}:\mathrm{dist}_G(x,y)\leq r\})$.  When $r=2$, 
We say $G^2$ is the \emph{square} of $G$.  Erd\H{o}s made the following conjecture: If $G$ is a graph on $n$ vertices with $\delta(G)\geq\frac{rn}{r+1}$, then $G$ contains $\floor{\frac{n}{r+1}}$ vertex disjoint copies of $K_{r+1}$.  Erd\H{o}s goes on to point out that the case $r=1$ is a consequence of Dirac's Theorem since the graph $C_n$ contains $\floor{\frac{n}{2}}$ copies of $K_2$.  He also mentions that the case $r=2$ was solved by Corr\'adi and Hajnal in a paper which appeared in 1963 \cite{CH}.  Furthermore, he goes on to state that P\'osa made the following conjecture which would contain the result of Corr\'adi and Hajnal: If $\delta(G)\geq \frac{2n}{3}$, then $C_n^2\subseteq G$.  Note that the square of $C_n$ contains $\floor{\frac{n}{3}}$ vertex disjoint copies of $K_3$.

In 1970, Hajnal and Szemer\'edi proved Erd\H{o}s' conjecture from 1963 \cite{HSz}.  Then in 1973, Seymour generalized P\'osa's conjecture, making a conjecture which would contain the Hajnal-Szemer\'edi Theorem \cite{Sey}: If $\delta(G)\geq \frac{rn}{r+1}$, then $C_n^r\subseteq G$.  It would be close to $30$ years before there were any results on the P\'osa-Seymour conjecture.

One of the most powerful combinatorial tools is Szemer\'edi's Regularity Lemma \cite{Sz} (here we will discuss the Regularity Lemma somewhat informally, with precise statements given in Chapter \ref{regularitychapter}).  The Regularity Lemma came out of Szemer\'edi's proof of a conjecture of Erd\H{o}s and Tur\'an on arithmetic sequences (for which Szemer\'edi received a \$1000 prize from Erd\H{o}s).  

\begin{theorem}[Szemer\'edi \cite{SzPro}]
For every $d\in(0,1)$ and $k\in \mathbb{N}$ there exists $N$ such that if $S\subseteq \{1,\dots, N\}$ and $|S|\geq dN$, then $S$ contains an $k$-term arithmetic progression.  
\end{theorem}

Here we will only talk about the applications of the Regularity Lemma for graphs.  One of the consequences of the Regularity Lemma is that large dense graphs behave like random graphs from the point of view of bounded degree
subgraphs.  To see what this means more precisely, let $p\in (0,1)$
and let $G_n$ be a graph on $n$ vertices where each edge exists with
probability $p$ -- thus the expected number of edges in $G$ is $\Omega(n^2)$ and we say $G$ is \emph{dense}.  Let $\Delta$ be a positive integer, $\ep\in (0,1)$, and $H$ be a graph on $(1-\ep)n$ vertices with maximum
degree $\Delta(H)\leq \Delta$.  
\begin{claim}
The probability that $H\subseteq G$ goes to $1$ as $n\to \infty$
\end{claim}

\begin{proof}
We embed $H$ one vertex at a time. Since there are always at least $\ep n$ vertices left over, the probability that there is no suitable candidate for the next vertex is
$(1-p^\Delta)^{\ep n}\to 0$.
\end{proof}
 
%In fact as long as $H$ is a graph on $n-f(n)$ vertices and $f(n)$ is $\omega(1)$, we will get the same result.
This shows that it is easy to embed ``almost'' spanning subgraphs in
dense random graphs.  The Regularity Lemma and corresponding ``Key Lemma'' (see \cite{KS}) allows one to obtain the same result in any dense enough large graph.

However, we are still at a loss if we want to find spanning subgraphs,
which is of course the aim of this dissertation.  In the 1990's, \KSaS proved
the Blow-up Lemma \cite{KSSbu}. The abstract of their paper read, ``Regular pairs behave
like complete bipartite graphs from the point of view of bounded
degree subgraphs.''  The Blow-up Lemma works in regular pairs which
satisfy an additional minimum degree condition.  So using the Blow-up
Lemma in conjunction with the Regularity Lemma, it is possible to find
spanning subgraphs.  In fact, one of the first uses of the Blow-up
Lemma was to give a proof of P\'osa's conjecture for large graphs.

\begin{theorem}[\KSS \cite{KSSp}]
Let $G$ be a graph on $n$ vertices.  There exists $N_0\in \mathbb{N}$ such that if $\delta(G)\geq \frac{2n}{3}$ and $n\geq N_0$, then $C_n^2\subseteq G$.
\end{theorem}

They went on to also prove Seymour's conjecture when $n$ is large with respect to $r$.  The Blow-up Lemma has since been used to prove many results
and we will give two applications in Chapters \ref{2factorschapter} and \ref{sumdegtilingchapter}.  One of the unfortunate aspects of the Regularity-Blow-up method is that
the graphs being considered are extremely large. In fact they are so large that
they exceed any physical description, i.e. much larger than the number
of atoms in the universe.  So any result which is proved using the Regularity-Blow-up method leaves open the general statement which has no lower bound on the number of vertices.  Lately,
there has been increasing success in removing Regularity from certain arguments and we begin
with such a result in Chapter \ref{posachapter}.

Finally before getting into the main results, we give an example of how the Regularity-Blow-up method is usually
applied.  Let $G$ be a graph on $n$ vertices with $\delta(G)\geq
\frac{n}{2}$.  Suppose we are trying to prove that $C_n\subseteq G$.
Of course a simple proof was already given above, but imagine for the
moment that we are unaware of such a proof.  We saw in Proposition \ref{diracexamples}, that there exists a graph $G$ with $\delta(G)=\frac{n-1}{2}$ which does not contain $C_n$.  We call $G$ an ``extremal example'', since any increase in the minimum degree will give us the desired cycle. What is the key property which makes $G$ an extremal example? $G$ has an independent set $X$ of size $\frac{n+1}{2}$, and an independent set $Y$ of size $\frac{n-1}{2}$ with every possible edge between them. $G$ doesn't contain $C_n$ because any two vertices in $X$ must be separated on the cycle by a vertex from $Y$, which isn't possible since $|X|>|Y|$. Notice that we can in fact add every possible edge to $Y$ and still have an extremal example for the same reason.  This tells us that the key property is that $G$ has a slightly too large independent set.  Now in the graph with the correct degree condition we can define an appropriate notion of ``closeness'' to the extremal example.  There is no one right way to do this, but we may introduce some parameter $\alpha>0$ and say that $G$ is in the extremal case if $G$ has a set $S$ of size at least $(1-\alpha)\frac{n}{2}$ which contains fewer than $\alpha\binom{|S|}{2}$ edges.  Then we will split the proof into two cases. When $G$ is not in the extremal case, we will apply the Regularity-Blow-up method.  When $G$ is in the extremal case, we will use ad hoc techniques which take advantage of the narrow structure imposed by the extremal condition.



\chapter{P\'OSA'S CONJECTURE FOR GRAPHS ON AT LEAST $\n$ VERTICES}\label{posachapter}
\DoubleSpacing
\setlength{\parindent}{.5in}

This chapter is joint work with Phong Ch\^au and H.A. Kierstead.  

\section{Introduction}

The \emph{square} $H^{2}$ of a graph $H$ is obtained by joining
all pairs $\{x,y\}\subset V(H)$ with distance $dist(x,y)=2$ in $H$.
If $H$ is a path (cycle) then $H^{2}$ is called a \emph{square}
path (cycle). Now fix a graph $G=(V,E)$ on $n$ vertices. We say
that $v_{1}\dots v_{t}$ is a \emph{square} path (cycle) in $G$ if
$v_{1}\dots v_{t}$ is a path (cycle) in $G$ and its square is contained
in $G$. In 1962 P\'osa \cite{E} conjectured: 
\begin{conjecture}
Every graph $G$ with $\delta(G)\ge\frac{2}{3}|G|$ contains a hamiltonian
square cycle. 
\end{conjecture}
During the 90's there were numerous partial results on P\'osa's conjecture.
Here we review a number that have a direct impact on this paper. Fan
and Kierstead \cite{FK1,FK2,FK3} proved the following three theorems. 
The first is a connecting lemma that
immediately yields an approximate version of P\'osa's conjecture. 
\begin{theorem}[Fan
and Kierstead \cite{FK1}]
\label{thm:FK1}For every $\ep>0$ there exists a constant $m$ such that for
every graph $G$ with $\delta(G)\ge(\frac{2}{3}+\ep)|G|+m$
and every pair $e_{1},e_{2}$ of disjoint ordered edges, $G$ has a hamiltonian 
square path starting with $e_{1}$ and ending with $e_{2}$. In particular,
$G$ has a hamiltonian square cycle.
\end{theorem}
We shall need two ideas from this paper---\emph{weak reservoirs}
\footnote{The term reservoir is not mentioned in \cite{FK1}, and the modifiers
\emph{weak}, \emph{strong }and \emph{special} are our own invention.
However, in light of more recent papers this terminology provides
a consistent transition (see Definition \ref{def:Reservoir}).
}, and \emph{optimal} square paths and cycles---which will be presented
in the next section. Roughly, given a graph $G$ on $n$ vertices, a weak reservoir is a small fraction $R$ of the vertex set $V(G)$  such that  $|N\cap R|\approx |N||R|/n$ for any neighborhood $N:=N(v)$.  Weak reservoirs were used to connect long square 
paths contained in  $V(G)\setminus R$. The second theorem is a
path version of P\'osa's Conjecture. 
\begin{theorem}[Fan and Kierstead \cite{FK2}]
\label{thm:FK2}Every graph $G$ with $\delta(G)\ge\frac{2|G|-1}{3}$
contains a hamiltonian square path. 
\end{theorem}
The third theorem shows that $V(G)$ can be partitioned into at most two square cycles.
\begin{theorem}[Fan and Kierstead \cite{FK3}]
\label{thm:FK3}Suppose $G$ is a graph with $\delta(G)\ge\frac{2}{3}|G|$.
If $G$ has a square cycle of length greater than $\frac{2}{3}|G|$
then $G$ has a hamiltonian square cycle. Moreover, $V(G)$ can be
partitioned into at most two square cycles, each of length at least $\frac{1}{3}|G|$. 
\end{theorem}
The proofs of Theorems \ref{thm:FK2} and \ref{thm:FK3} are based
on optimal paths and cycles, but do not use weak reservoirs. Theorem
\ref{thm:FK3} is essential to this paper, because it allows our constructions 
to terminate as soon as we get a square cycle of length greater than
$\frac{2}{3}|G|$.

Next came a major breakthrough. Koml\'os, S\'ark\"ozy and Szemer\'edi proved
their famous Blow-up Lemma \cite{KSSbu}, and used it and the Regularity
Lemma \cite{Sz} to prove: 
\begin{theorem}[Koml\'os, S\'ark\"ozy and Szemer\'edi \cite{KSSp}]
\label{thm:KSSp}There exists a constant $n_0$ such that every graph
$G$ with $|G|\ge n_0$ and $\delta(G)\ge\frac{2}{3}|G|$ has a hamiltonian
square cycle.
\end{theorem}
Their proof has the following structure. First they determine extremal
configurations that are very close to being counterexamples, but because
of the tightness of the degree condition, cannot achieve this status.
(For example, if the independence number $\alpha(G)>\frac{1}{3}|G|$
then $G$ does not have a hamiltonian square cycle, but then also does 
not satisfy $\delta(G)\ge\frac{2}{3}|G|.$ Moreover if $G$ has an
almost independent set of size almost $\frac{1}{3}|G|$ and $\delta(G)\geq \frac{2}{3}|G|$, then we will %HK
see that $G$ does have a hamiltonian square cycle.) Next they proved
that if $|G|$ is sufficiently large, $\delta(G)\ge\frac{2}{3}|G|$,
and $G$ has an extremal configuration, then $G$ has a hamiltonian
square cycle. When there are no extremal configurations, the Regularity
Lemma imposes a pseudo random structure on the graph that can be exploited,
using this lack of extremal configurations and the Blow-up Lemma,
to construct a hamiltonian square cycle. The use of the Regularity
Lemma causes the constant $n_0$ to be extremely large.

Very recently R\"odl, Ruci\'{n}ski and Szemer\'edi have made another 
important advance \cite{RRS1,RRS2}. They proved
the following version of Dirac's Theorem for 3-uniform hypergraphs
(3-graphs). An \emph{open chain} $P:=v_{1}v_{2}v_{3}\dots v_{s-2}v_{s-1}v_{s}$
in a 3-graph $H$ is a sequence of distinct vertices such that 
$v_{i}v_{i+1}v_{i+2}\in E(H)$ for all $i\in[s-2]$; $P$ is a \emph{closed
chain} if in addition $v_{s-1}v_{s}v_{1},v_{s}v_{1}v_{2}\in E(H)$.
\begin{theorem}[R\"odl, Ruci\'{n}ski and Szemer\'edi \cite{RRS2}]
There exists an integer $n_{0}$ such that for every $3$-graph $H$
on at least $n_{0}$ vertices, if every pair of vertices of $H$ is
contained in at least $\lfloor\frac{1}{2}|H|\rfloor$ edges of $H$
then $H$contains a hamiltonian closed chain. 
\end{theorem}
The remarkable proof is very long, but has a similar structure to
the proof of Theorem~\ref{thm:KSSp}. However, a major difference 
is that the non-extremal case does not use any version of the Blow-up Lemma, and regularity
(weak hypergraph regularity) is only used in a quite generic way to
construct various \emph{strong reservoirs}---weak reservoirs with no extreme sets. The Blow-up Lemma is replaced
by a construction based on an ingenious \emph{absorbing path }lemma\emph{,}
and a \emph{connecting} lemma, that uses the strong reservoir.

Levitt, S\'ark\"ozy and Szemer\'edi \cite{LSS} applied similar techniques 
to the non-extremal case of P\'osa's Conjecture without using the Regularity Lemma, and thus proved the result for much smaller graphs than those considered in Theorem \ref{thm:KSSp}. 

Here we show that P\'osa's Conjecture holds for graphs of order at least
$\n$ without using the Regularity-Blow-up
method. In addition, our proof of the extremal case holds for all $n$.
We were influenced by the ideas of \cite{LSS}, but only rely on results
from \cite{FK1,FK2,FK3}, and the idea from \cite{KSSp} of dividing
the problem into an extremal case and a non-extremal case.  We avoid the Blow-up Lemma and absorbing paths by using Theorem~\ref{thm:FK3}. 
 Our approach is explained fully in the
next section.

\subsubsection*{Notation}

Most of our notation is consistent with Diestel's graph theory text
\cite{Di}. In particular note that $P^{n}$ is a path on $n$ edges,
$|G|=|V(G)|$, $\|G\|=|E(G)|$, and $d(v)$ is the degree of the vertex
$v$. For $A,B\subseteq V(G)$, let $\|A,B\|=|E(A,B)|$, where $E(A,B)$ is the set of edges with one end
in $A$ and the other in $B$, in particular we shall write $\|a,B\|$ if $A=\{a\}$. We also use $\overline{\|A,B\|}$ to denote the number of edges  in the complement of $G$ that have one end in  $A$ and the other in  $B$. 
For $a_{1},a_{2},\dots,a_{k}\in V(G)$, let $N(a_{1},a_{2},\dots,a_{k})=N(a_{1})\cap N(a_{2})\dots\cap N(a_{k})$. 


\section{Main theorem and proof strategy}

Here is our main result:
\begin{theorem}
\label{posamain} Let $G$ be a graph on $n$ vertices with $n\geq n_{0}:=\n$.
If $\delta(G)\geq\frac{2}{3}n$, then $G$ has a hamiltonian square cycle.
\end{theorem}
In this section we organize the structure of the proof. The first
step is to define a usable extremal configuration. Our choice is simpler
than the choice in \cite{LSS}, which was much simpler than the several
extremal configurations used in \cite{KSSp}. A priori, this makes the extremal case easier and the non-extremal case harder.
\begin{definition}
Let $G$ be a graph on $n$ vertices.  A set $S\subseteq V(G)$ is $\alpha$-\emph{extreme} if $|S|\geq (1-\alpha)\frac{n}{3}$
and $\|v,S\|<\alpha \frac{n}{3}$ for all $v\in S$. 
\end{definition}
The proof divides into two parts, depending on whether $G$ is $\frac{1}{36}$-extreme,
i.e., contains an $\alpha$-extreme set with $\alpha:=\frac{1}{36}$.
The extreme case is handled in Section~\ref{sec:Ex}, where we prove
the following theorem without assuming anything about the order of
$G$. Its proof only requires elementary graph theory. Notice that
$K_{3t+2}-E(K_{t+1})$ demonstrates that the degree condition is tight.
\begin{theorem}[Extremal Case]
\label{thm:good}Let $G$ be a graph on $n$ vertices with $\delta(G)\geq \frac{2}{3}n$. If $G$ has a $\frac{1}{36}$-extreme set, then $G$ has a hamiltonian square cycle. 
\end{theorem}
The non-extremal case is more complicated. In Section~\ref{sec:nex} we will prove: 
\begin{theorem}[Non-extremal Case]
\label{non-extremal} Let $G$ be a graph on $n$ vertices with $\delta(G)\geq\frac{2}{3}n$
and $n\geq n_{0}:=\n$. If $G$ does not contain a $\frac{1}{36}$-extreme
set, then $G$ has a hamiltonian square cycle. 
\end{theorem}

Note that if $G$ has an $\alpha$-extreme set $S\subseteq V(G)$ for some $\alpha<\frac{1}{36}$, then $S$ is a $\frac{1}{36}$-extreme set.  This explains why we only consider $\frac{1}{36}$-extreme sets in Theorems \ref{thm:good} and \ref{non-extremal}.

The proof of Theorem~\ref{non-extremal} has three parts. First we
use the Reservoir Lemma (Lemma~\ref{reservoir}) to construct a special reservoir
$R$ with $|R|<\frac{1}{3}n$. Then we use the Path Cover Lemma (Lemma~\ref{pathcover}) to construct two disjoint square
paths $P_{1},P_{2}$ in $G-R$ such that $|P_{1}|+|P_{2}|>\frac{2}{3}n$
using techniques and results from \cite{FK1,FK2}. Finally, we use
the properties of the special reservoir $R$, together with our version
of the Connecting Lemma (Lemma~\ref{connecting}), to connect the ends of
$P_{1}$ to the ends of $P_{2}$ by disjoint square paths in $R$
so as to form a square cycle of length greater than $\frac{2}{3}n$.
Thus by Theorem~\ref{thm:FK3} we obtain a hamiltonian square cycle.


\subsection{Reservoirs and the Connecting Lemma}

The bottleneck in this line of attack is in determining properties for
special reservoirs that are strong enough to prove the Connecting
Lemma, yet weak enough to ensure the existence of special reservoirs
in moderately sized graphs. 
In the process of constructing a connecting square path we need to know that 
certain subsets of the reservoir are nonextreme. Since it is too expensive to ensure that all subsets are nonextreme,
we anticipate a limited collection of  \emph{special} subsets that might appear in this construction, and construct
 a reservoir with no extreme special sets.
\begin{definition}
\label{def:special}A set $S\subseteq V(G)$ is \emph{special} if there
exist (not necessarily distinct) vertices $u,v,w,x,y\in V(G)$ such that $S=(N(u,v,w)\cup N(u,v,x))\cap N(y)$.
\end{definition}

A set $S$ of size at least $(1-\alpha)\frac{n}{3}$ that is not $\alpha$-extreme has at least one vertex with ``large'' degree to $S$, but we will need more than one vertex of ``large'' degree, so we define a more general notion of extremity.

\begin{definition}\label{def:alpha-beta}
Let $G$ be a graph with $n$ vertices.  A set $S\subseteq V(G)$ is $(\alpha,\beta)$-\emph{extreme} if $|S|\geq (1-\alpha+\beta)\frac{n}{3}$ and there are fewer than $\floor{\beta\frac{n}{3}}$ vertices $v\in S$ such that $\|v,S\|\geq \alpha \frac{n}{3}$. 
\end{definition}

So a set $S$ of size at least $(1-\alpha+\beta)\frac{n}{3}$ that is not $(\alpha,\beta)$-extreme has at least $\floor{\beta\frac{n}{3}}$ vertices with ``large'' degree to $S$.  In the non-extremal case we know that $G$ contains no $\alpha$-extreme sets, but we must ensure for the Connecting Lemma that the reservoir has no $(\alpha', \beta')$-extreme special sets.  So we use the following simple observation when constructing the reservoir.

\begin{lemma}
\label{nicevertices} Let $G$ be a graph on $n$ vertices and let $\alpha,\beta>0$. If $G$ has no $\alpha$-extreme sets and $S\subseteq V(G)$ with $|S|\geq (1-\alpha+\beta)\frac{n}{3}$, then $S$ is not $(\alpha, \beta)$-extreme. 
\end{lemma}
\begin{proof}
Suppose $S$ is $(\alpha, \beta)$-extreme and let $S'=\{v\in S: \|v, S\|\geq \alpha\frac{n}{3}\}$. Since $S$ is $(\alpha, \beta)$-extreme, we have $|S'|<\floor{\beta\frac{n}{3}}$.  Thus $|S\setminus S'|\geq (1-\alpha)\frac{n}{3}$ and $\|v, S\setminus S'\|<\alpha\frac{n}{3}$ for all $v\in S\setminus S'$, contradicting the fact that $G$ has no $\alpha$-extreme sets.
\end{proof}


Here are the technical definitions of $(\ep, \varrho)$-weak,
$(\alpha, \ep, \varrho)$-strong and $(\alpha, \beta, \ep, \varrho)$-special
reservoir. 
\begin{definition}
[Reservoir] \label{def:Reservoir}Let $G$ be a graph on $n$ vertices. Let $1\geq \varrho\geq 0$ and
$\ep>0$. An \emph{$(\ep, \varrho)$-weak reservoir} is a set $R\subseteq V(G)$
such that $|R|=\ceiling{\varrho n}$ and for all $u\in V(G)$, \[
\left(\frac{d(u)}{n}-\ep\right)|R|\le\|u,R\|\leq\left(\frac{d(u)}{n}+\ep\right)|R|.\]

An \emph{$(\alpha, \ep, \varrho)$-strong reservoir} is an $(\ep,\varrho)$-weak
reservoir $R$ such that $G[R]$ has no $\alpha$-extreme sets.

An \emph{$(\alpha, \beta, \ep,\varrho)$-special reservoir} is an $(\ep, \varrho)$-weak
reservoir $R$ such that for all special sets $S\subseteq V(G)$, $S\cap R$ is not $(\alpha, \beta)$-extreme in $G[R]$.
\end{definition}
A routine application of Chernoff's bound yields $(\ep,\varrho)$-weak
reservoirs $R$ in moderately large graphs. The reason for this is
that we have only polynomially many conditions to preserve. A similar
observation allows us to construct $(\alpha, \beta, \ep,\varrho)$-special
reservoirs. However this standard approach fails for $(\alpha,\ep,\varrho)$-strong
reservoirs, because there are exponentially many conditions to check.

A connecting lemma should state that any two disjoint ordered edges in $V(G)\setminus R$
can be connected by a short square path whose interior vertices are in
$R$. For example, Fan and Kierstead \cite{FK1} proved:
\begin{lemma}
If $\delta(G)>\frac{2}{3}|G|$ then there exists a square path connecting
any two disjoint edges.
\end{lemma}
\noindent In the context of Theorem~\ref{thm:FK1}, $(\ep/2,\varrho)$-weak
reservoirs are sufficient since the degree bounds ensure that $\delta(G[R])>\frac{2}{3}|R|$.
In \cite{LSS,RRS2} the authors prove connecting lemmas for strong
reservoirs. We use a simpler argument and show that it works for special
reservoirs.


\subsection{Optimal paths}

Let $e_{1}:=v_1v_2$ and $e_{2}:=v_{s-1}v_s$ be disjoint ordered edges. A square $(e_{1},e_{2})$-\emph{path} is a square path of the form $v_1v_2\dots v_{s-1}v_s$.
\begin{definition}
An \emph{optimal} square path (or cycle, or $(e_{1},e_{2})$-path)
is a square path (or cycle, or $(e_{1},e_{2})$-path) $P$ such
that among all square paths (or cycles, or $(e_{1},e_{2})$-paths)
(i) $P$ is as long as possible, (ii) subject to (i), $P$ has as
many 3-chords as possible, and (iii) subject to (i) and (ii), $P$
has as many 4-chords as possible. 
\end{definition}
All the work in \cite{FK1,FK2,FK3} starts with lemmas about optimal
square paths. 

\begin{lemma}
[Fan-Kierstead \cite{FK1}, \cite{FK2} Lemma 1]\label{segment} Suppose that
$P$ is a square path in a graph $G$ and $v\in V(G-P)$.
If $P$ is an $(e_{1},e_{2})$-optimal square path then $\|v,Q\|\leq\frac{2}{3}|V(Q)|+1$
for every segment $Q$ of $P$. Moreover, if $P$ is an optimal square
path then $\|v,P\|\leq\frac{2}{3}|P|-\frac{1}{3}$ and if $P$ is
an optimal square cycle then $ \|v,P\|\leq\frac{2}{3}|P|+\frac{1}{3}$.
\end{lemma}

In the extremal case we will take advantage of the following fact.

\begin{corollary}\label{3k}
P\'osa's Conjecture is true, if it holds for all $G$ with $|G|$ divisible by $3$. 
\end{corollary}
\begin{proof}Suppose $|G|=3k+r$, where
$1\le r\le2$. Let $G'$ be $G$  with $r$ vertices deleted. Then $$\delta(G')\ge\lceil\frac{2}{3}(3k+r)\rceil-r=2k=\frac{2}{3}|G'|.$$
Thus by hypothesis, $G'$ has a hamiltonian square cycle $C'$. So an optimal
square cycle $C$ in $G$ has length at least $3k.$ Suppose $C$
is not hamiltonian in $G$. Then there exists $x\in V(G-C)$. By Lemma~\ref{segment},
we have the following contradiction: \[
2k+r\leq\delta(G)\leq\left\Vert v,C\right\Vert +|G|-|C|-1\le|G|-\frac{1}{3}|C|-\frac{2}{3}\leq 2k+r-\frac{2}{3}.\]
\end{proof}

We will also need:
\begin{lemma}
[Fan-Kierstead \cite{FK2}, Lemma 9]\label{edgetopath} Let $P$
be an optimal square path of $G$. Let $xy$ be an edge of $G-P$
such that there are square paths, of at least $q$ vertices, starting
at $xy$ and $yx$ in $G-P$. If $|P|\geq 2q+2$, then $\|xy,P\|\leq\frac{4}{3}|P|-\frac{2}{3}q+2$. 
\end{lemma}

\subsection{Probability}

If $X$ is a random variable with hypergeometric distribution (and 
our experiment consists of drawing $n$ items from a collection of
$N$ total items, $m$ of which are good and $N-m$ of which are bad)
the expected value of $X$ is given by \[
\mathbb{E}X=\sum_{k=0}^{n}k\cdot Pr(X=k)=\sum_{k=0}^{n}k\cdot\frac{\binom{m}{k}\binom{N-m}{n-k}}{\binom{N}{n}}=\frac{nm}{N}.\]

\begin{theorem}
[Chernoff's bound \cite{Ch,JLR}]\label{Chernoff} Let $X$ be a random variable
with binomial or hypergeometric distribution. Then the following hold:\end{theorem}
\begin{enumerate}
\item $Pr(X\geq\mathbb{E}X+t)\leq\exp\left(-\frac{t^{2}}{2(\mathbb{E}X+t/3)}\right),~~t\geq0$;
\item $Pr(X\leq\mathbb{E}X-t)\leq\exp\left(-\frac{t^{2}}{2\mathbb{E}X}\right),~~t\geq0$;
\item If $0<\g\leq3/2$, then $Pr(|X-\mathbb{E}X|\geq\g\mathbb{E}X)\leq2\exp\left(-\frac{\g^{2}}{3}\mathbb{E}X\right)$.
\end{enumerate}

\section{Non-extremal case\label{sec:nex}}

In this section we prove Theorem \ref{non-extremal}.  We have compromised optimality somewhat in our constructions %HK
 and calculations in favor of clarity of exposition.  For instance, we know how to reduce $n_0$ by a factor of $2$.  That being said, we can make the reservoir lemma slightly simpler and we can choose ``nicer'' constants throughout the non-extremal case at the cost of a factor of $3$ in $n_0$.

We first show that if $H$ is a graph with no $(\alpha,\beta)$-extreme special
sets whose minimum degree is almost $\frac{2}{3}|H|$, then any two
disjoint edges in $H$ can be connected by a short square path.  Let $xy\in E(H)$; we say that $P\{xy\}Q$ is a square path if one of $PxyQ$ or $PyxQ$ is a square path.

\begin{lemma}
[Connecting Lemma] \label{connecting} Let $0<\beta<\alpha\leq\frac{1}{36}$,
$0<\ep\leq\frac{\alpha-\beta}{15.1}$, $l:=10$ and suppose $n\geq\max\{\frac{660}{\ep},\frac{69}{\beta}\}$.
Let $H=(V,E)$ be a graph on $n$ vertices with no $(\alpha,\beta)$-extreme
special sets such that $\delta(H)\geq(\frac{2}{3}-\ep)n$. Let $L\subseteq V$
such that $|L|\leq l$. If $ab$, $cd$ are any two disjoint
ordered edges in $H-L$, then there is a square $(ab, cd)$-path $P$
of order at most $14$ for which $V(P)\subseteq V\setminus L$. \end{lemma}
\begin{proof}

Let $ab$, $cd$ be disjoint ordered edges in $H-L$ and set $A:=\{a,b,c,d\}$.
Here is our plan. First (a) we find disjoint edges $a'b',c'd'$
in $H-L-A$ such that $\|ab,a'b'\|=4=\|cd,c'd'\|$. Then, setting $A':=\{a',b',c',d'\}$,
(b) we construct a square path $\{a'b'\}Q\{c'd'\}$ with $Q\subseteq H':=H\setminus(L\cup A\cup A')$
connecting the unordered edges $a'b',c'd'$. This will yield a square
path $ab\{a'b'\}Q\{c'd'\}cd$, where the order of $\{a'b'\}$ and $\{c'd'\}$
is determined by $Q$.

Let $M\subseteq V$ with $|M|\le l+12$. We will often use the following
statement: \begin{equation}
\textrm{If \ensuremath{S} is a special set with \ensuremath{|S|\ge(1-\alpha+\beta)\frac{n}{3}} then \ensuremath{\|S\setminus M\|>0}.}\label{ne}\end{equation}
To see this, note that since $S$ is not $(\alpha,\beta)$-extreme and $n\geq \frac{69}{\beta}$, $S$ has at least $\floor{\beta\frac{n}{3}}>l+12$ vertices with degree at least $\alpha\frac{n}{3}>l+12$.

Consider the special set $N(a,b)=(N(a,a,a)\cup N(a,a,a))\cap N(b)$.
Since $\delta(H)\geq(\frac{2}{3}-\ep)n$, we have \[
|N(a,b)|\geq(1-6\ep)\frac{n}{3}\geq(1-\alpha+\beta)\frac{n}{3}.\]
 By \eqref{ne}, there exists $a'b'\in E(N(a,b)\setminus(L\cup A))$.
Likewise there is an edge $c'd'\in E(N(c,d)\setminus(\{a',b'\}\cup L\cup A))$,
completing the first goal (a).

Next we show (b). Let $V':=V(H')$. Then $|V'|\ge n-l-8$. We must
construct $Q\subseteq H'$. For $i\in[4]$, let $S_{i}:=S_i(A')=\{v\in V:\|v,A'\|=i\}$.
Then 
\begin{equation}
\frac{8}{3}n-4\ep n=4(\frac{2}{3}-\ep)n\le\|A',V\|  =\sum_{i\in[4]}i|S_{i}|\le4|S_{4}|+3|S_{3}|+2(n-|S_{4}|-|S_{3}|),\label{edges1}
\end{equation}
which gives
\begin{equation}
2|S_{4}|+|S_{3}|\geq \frac{2}{3}n-4\ep n.\label{edges2}
\end{equation}
\textbf{Case 1:} $|S_{4}|>l+12$. Looking ahead to an application
in Case 2.a, we will construct $Q\subseteq H'':=H'-A''$, for any fixed
$4$-set $A''$. Set $V'':=V(H'')$. By the case assumption, there exists $x\in S_{4}\cap V''$.
If there exists $u\in N(x)\cap(S_{4}\cup S_{3})\cap V''$ then set
$Q:=\{xu\}$. Otherwise, $|S_{4}|+|S_{3}|\le\frac{1}{3}n+\ep n+l+12$,
since $d(x)\ge\frac{2}{3}n-\ep n$. Thus by \eqref{edges2}, and using
$\alpha-\beta\ge15.1\ep$ and $n\geq \frac{660}{\ep}$, we have \[
|S_{4}|\ge\frac{1}{3}n-5\ep n-l-12\ge(1-\alpha+\beta)\frac{n}{3}.\]
 Moreover, $S_{4}=N(a',b',c',d')=(N(a',b',c')\cup N(a',b',c'))\cap N(d')$
is special. Thus by \eqref{ne}, there exists an edge $uv\in S_{4}\cap V''$,
and we set $Q:=uv$.

\noindent \textbf{Case 2:} $|S_{4}|\le l+12$.  Let 
\begin{align}
T_{1} & :=\{v\in S_{3}\cup S_4:\|v,\{a',b'\}\|=2\}=(N(a',b',c')\cup N(a',b',d'))\cap N(a')\textrm{ and }\\
T_{2} & :=\{v\in S_{3}\cup S_4:\|v,\{c',d'\}\|=2\}=(N(c',d',a')\cup N(c',d',b'))\cap N(c').
\end{align}
Then $T_1$ and $T_2$ are both special sets.  Note that $S_3$ is partitioned as $(T_1\setminus S_4)\cup (T_2\setminus S_4)$ and $T_1\cap T_2=S_4$.  By \eqref{edges2} and the fact that $|T_1|+|T_2|=|S_3|+2|S_4|$, we have
\begin{equation}
|T_1|+|T_2|\geq \frac{2}{3}n-4\ep n \label{T12}.
\end{equation}
Without loss of generality, $|T_{1}|\leq|T_{2}|$, and so $T_{2}\neq\emptyset$.  Finally, note that by \eqref{edges2} and the case assumption we have,
\begin{equation}
|T_1\cup T_2|=|S_{3}\cup S_4|\geq\frac{2}{3}n-4\ep n-l-12.\label{S3L}\end{equation}




\noindent \textbf{Case 2.a:} $|T_{1}|>l+8$. If there exists $xy\in E(T_{1},T_{2})\cap E(H')$,
then set $Q:=xy$. Otherwise, let $x\in T_{1}\cap V'$. Then using,
in order, $d(x)\geq(\frac{2}{3}-\ep)n$, \eqref{T12}, $\alpha-\beta\ge15.1\ep$ and $n\geq \frac{660}{\ep}$
we have \begin{equation}
\frac{n}{3}+\ep n+l+8\ge|T_{2}|\ge|T_{1}|\ge\frac{n}{3}-5\ep n-l-8\ge(1-\alpha+\beta)\frac{n}{3}.\label{T2}\end{equation}

By \eqref{ne} and \eqref{T2}, there exist edges $a''b''\in E(T_{1})$
and $c''d''\in E(T_{2})$ such that $A'':=\{a'',b'',c'',d''\}$ is disjoint from $L\cup A\cup A'$.  Note that $A''\cap S_4=\emptyset$, since $E(T_{1},T_{2})\cap E(H')=\emptyset$ as mentioned above.

Set $U:=V\setminus (T_1\cup T_2)$. By \eqref{S3L}, \begin{equation}
|U|=n-|T_1\cup T_2|\le\frac{n}{3}+4\ep n+l+12.\label{U}\end{equation}

By \eqref{T2}, for any $x\in A''$, \begin{equation}
\|x,U\|\ge\frac{2}{3}n-\ep n-|T_{2}|\ge\frac{n}{3}-2\ep n-l-8.\label{dU}\end{equation}
 By \eqref{U}, \eqref{dU}, and $n\geq\frac{660}{\ep}$, we have $\overline{\|x,U\|}\le6\ep n+3l+32<\frac{1}{5}|U\cap V''|$.
Thus there exist more than $l+12$ vertices in $S_4(A'')$.
Thus by Case 1, there exists a square path $Q:=\{a''b''\}Q'\{c''d''\}$ with $|Q'|\le 2$.

\noindent \textbf{Case 2.b:} $|T_{1}|\le l+8$. Then $|T_{2}|\geq\frac{2}{3}n-4\varepsilon n-l-8$ by \eqref{T12}.
Let $x\in N(a',b')\cap V'$, and note that $S:=T_{2}\cap N(x)=(N(a',c',d')\cup N(b',c',d'))\cap N(x)$
is a special set. Moreover by $\alpha-\beta\ge15.1\ep$ and $n\geq \frac{660}{\ep}$ we have\[
|S|\ge|T_{2}|+|N(x)|-n\ge\frac{n}{3}-5\ep n-l-8\ge(1-\alpha+\beta)\frac{n}{3}.\]
Thus by \eqref{ne}, there exists an edge $yz\in E(S\cap V')$. Let
$Q:=xyz$.

\noindent 
\end{proof}

Now we prove the reservoir lemma.

\begin{lemma}[Reservoir Lemma] \label{reservoir} 
Let $\alpha\geq \frac{1}{36}$, $c\geq\frac{1}{14}$,  $\alpha':=(1-3c)\alpha$, $\beta':=c\alpha$, $\ep\geq \frac{\alpha'-\beta'}{15.1}$, $\varrho\geq %HK---\frac
1-\frac{2/3+\ep}{5/6-2\ep}$ and $n\geq n_{0}:=\n$. If $H$ is a
graph on $n$ vertices such that $\delta(H)\geq \frac{2}{3}n$ and $H$ contains 
no $\alpha$-extreme sets, then $H$ contains an $(\alpha', \beta', \ep, \varrho)$-special
reservoir. \end{lemma}

\begin{proof}
Let $\gamma:=\frac{2\beta'}{1-\alpha'-\beta'}$. We will show that there
exists a set $R\subseteq V(H)$ such that $|R|=\ceiling{\varrho n}$ which satisfies the following three
properties.
\begin{enumerate}
\item[(i)] For all $u\in V(H)$, $\left(\frac{d(u)}{n}-\ep\right)|R|\leq\|u,R\|\leq\left(\frac{d(u)}{n}+\ep\right)|R|$.
\item[(ii)] For all special sets $S\subseteq V(H)$, if $|S|\geq(1-\alpha'+\beta')\varrho\frac{ n}{3}$, then %HKn/3
$|S\cap R|\leq 1.05\varrho|S|$ and for all special sets $S\subseteq V(H)$, if $|S\cap R|\geq(1-\alpha'+\beta')\varrho\frac{ n}{3}$, then $|S\cap R|\leq (1+\gamma)\varrho|S|$.
\item[(iii)] For all special sets $S\subseteq V(H)$, if $|S|\geq(1-\alpha'-\beta')\frac{n}{3}$, then 
there exists a set $T'\subseteq S\cap R$ such that $|T'|\geq \beta'\varrho\frac{n}{3}$
and $\|z,S\cap R\|\geq \alpha'\varrho\frac{n}{3}$ for all $z\in T'$. \end{enumerate}
Then we will show that these three properties imply that $R$ is an 
$(\alpha', \beta', \ep, \varrho)$-special reservoir.

Let $R\subseteq V(H)$ be a set of size $\ceiling{\varrho n}=:r$ chosen
at random from all $\binom{n}{r}$ possibilities. There are five calculations that follow.  In each of these calculations we will need $n$ to be large, specifically $n\geq \n$ is large enough.

Let $u\in V(H)$. The expected value of $\|u,R\|$ is $\frac{rd(u)}{n}\geq\varrho d(u)$.
So by Theorem \ref{Chernoff}(iii), we have \begin{align*}
Pr\left(\left|\|u,R\|-\frac{rd(u)}{n}\right|\geq\frac{\ep n}{d(u)}\frac{rd(u)}{n}\right)&\leq 2\exp\left(-\frac{(\frac{\ep n}{d(u)})^{2}}{3}\frac{rd(u)}{n}\right)\\
&\leq2\exp\left(\frac{-\ep^{2}\varrho n^2}{3d(u)}\right)<\frac{1}{3n}.\end{align*}
There are $n$ vertices in $V(H)$. So by applying Boole's inequality, the probability that there exists a vertex which does not satisfy property (i) is less than $1/3$.

Let $S\subseteq V(H)$ be a special set such that $|S|\geq(1-\alpha'+\beta')\varrho\frac{ n}{3}$. %n/d(u)
The expected value of $|S\cap R|$ is $\frac{r|S|}{n}\geq\varrho|S|\geq (1-\alpha'+\beta')\varrho^2\frac{ n}{3}$.
So by Theorem \ref{Chernoff}(i), we have
\begin{align*}
\log Pr(|S\cap R|\geq1.05\frac{r|S|}{n})&\leq-\frac{(.05\varrho|S|)^{2}}{2(\varrho|S|+.05\varrho|S|/3)}\\
&\leq-\frac{.0025\varrho^{2}(1-\alpha'+\beta')}{2(1+.05/3)}\frac{n}{3}<\log\frac{1}{9n^{5}}.\end{align*}
So with high probability, \begin{equation}\label{WHP}|S\cap R|\leq 1.05\varrho |S|  \text{ for all } S\subseteq V(H) \text{ such that } |S|\geq(1-\alpha'+\beta')\varrho\frac{ n}{3} .\end{equation}  Now let $S\subseteq V(H)$ be a special set such that $|S\cap R|\geq(1-\alpha'+\beta')\varrho\frac{ n}{3}$.  Since $|S|\geq |S\cap R|$ we have $|S|\geq (1-\alpha'+\beta')\varrho\frac{ n}{3}$ and thus by \eqref{WHP}, $|S|\geq \frac{|S\cap R|}{1.05\varrho}\geq \frac{(1-\alpha'+\beta')}{1.05}\frac{n}{3}$. The expected value of $|S\cap R|$ is $\frac{r|S|}{n}\geq\varrho|S|\geq \varrho \frac{(1-\alpha'+\beta')}{1.05}\frac{n}{3}$.  Using Theorem~\ref{Chernoff}(i) again, we have
\begin{align*}
\log Pr(|S\cap R|\geq(1+\gamma)\frac{r|S|}{n})&\leq-\frac{(\gamma\varrho|S|)^2}{2(\varrho|S|+\gamma\varrho|S|/3)}\\
&\leq-\frac{\gamma^{2}\varrho(1-\alpha'+\beta')}{1.05(2+2\gamma/3)}\frac{n}{3}<\log\frac{1}{3n^{5}}.\end{align*}
There are at most $n^5$ special sets $S\subseteq V(H)$. So by applying Boole's inequality, the probability that there exists a set $S$ which does not satisfy property (ii) is less than $4/9$.

Let $S\subseteq V(H)$ be a special set such that $|S|\geq(1-\alpha'-\beta')\frac{n}{3}=(1-\alpha+2c\alpha)\frac{n}{3}$.
Since $H$ has no $\alpha$-extreme sets, we see by Lemma \ref{nicevertices} that $S$ is not $(\alpha, 2c\alpha)$-extreme. So there exists a set $S'\subseteq S$ having
the property that $|S'|=\floor{2c\alpha\frac{n}{3}}$ and for all $v\in S'$,
$\|v,S\|\geq\alpha\frac{ n}{3}$. Let $T':=S'\cap R$.
We first show that with high probability, $|T'|\geq\frac{3\varrho}{4}|S'|\geq \frac{\varrho}{2}(|S'|+1)\geq \beta'\varrho\frac{n}{3}$.
The expected value of $|T'|$ is $\varrho|S'|\geq\varrho(2c\alpha\frac{n}{3}-1)$.  So by Theorem \ref{Chernoff}(ii), we have \begin{align*}
\log Pr(|T'|\leq\varrho|S'|-\frac{\varrho}{4}|S'|)\leq-\frac{(\frac{\varrho}{4}|S'|)^{2}}{2(\varrho|S'|)}=-\frac{\varrho|S'|}{32} \leq-\frac{\varrho(2c\alpha\frac{n}{3}-1)}{32}<\log\frac{1}{9n^{5}}.\end{align*}

Next we show that, with high probability, every vertex in $S'$ has
at least $(1-3c)\varrho \|v,S\|\geq \alpha'\varrho\frac{n}{3}$ neighbors in $S\cap R$.
Let $v\in S'$. The expected value of $\|v,T\|$ is $\varrho\|v,S\|\geq \varrho\alpha\frac{n}{3}$. So by Theorem \ref{Chernoff}(ii), we have
\begin{align*}
\log Pr(\|v,S\cap R\|&\leq(1-3c)\varrho\|v,S\|)\\
&\leq-\frac{(3c\varrho\|v,S\|)^2}{2\varrho\|v,S\|}=-\frac{9c^2\varrho \|v,S\|}{2}\leq -\frac{3c^2\varrho \alpha n}{2}<\log \frac{1}{9n^{6}}.\end{align*}
There are at most $n^5$ special sets $S\subseteq V(H)$ and at most $n^6$ sets defined when we examine the neighborhood of vertices in each special set.  So by applying Boole's inequality, the probability that there exists a set $S$ which does not satisfy property (iii) is less than $2/9$.

The probability that $R$ doesn't satisfy one of the conditions is less than $1$, thus there exists a set $R\subseteq V(H)$ satisfying properties (i)-(iii).

We now show that $R$ is an $(\alpha', \beta', \ep, \varrho)$-special reservoir.  Since $R$ satisfies property (i), $R$ is a $(\ep, \varrho)$-weak reservoir.  Let $S\subseteq V(H)$ be a special set such that $|S\cap R|\geq(1-\alpha'+\beta')\varrho\frac{ n}{3}$.
By property (ii), we have $\varrho|S|(1+\gamma)\geq|S\cap R|\geq(1-\alpha'+\beta')\varrho\frac{ n}{3}$, and thus $$|S|\geq\frac{(1-\alpha'+\beta')}{1+\gamma}\frac{n}{3}=(1-\alpha'-\beta')\frac{n}{3}.$$
Then since $|S|\geq(1-\alpha'-\beta')\frac{n}{3}$ 
there is, by property (iii), a set of vertices $T'\subseteq S\cap R$ with $|T'|\geq\beta'\varrho\frac{ n}{3}$
such that for all $v\in T'$, $\|v,S\cap R\|\geq\alpha'\varrho\frac{ n}{3}$.
Thus $S\cap R$ is not $(\alpha', \beta')$-extreme in $G[R]$.  Therefore $R$ is an $(\alpha', \beta', \ep, \varrho)$-special reservoir.
\end{proof}

We now prove a lemma which allows us to cover most of the complement
of the reservoir with at most two long square paths.

\begin{lemma}[Path Cover Lemma]
\label{pathcover} Suppose  
$\ep\le\frac{1}{500}$ and $n\ge6000$. Let $H$ be a graph on $n$ vertices with $\delta(H)\geq\left(\frac{2}{3}-\ep\right)n$.  %Fixed \ep, n. See later in proof.
Then 

\textup{(a)} $H$ has a square path $P$ with $|P|\geq(\frac{1}{2}-3\ep)n$. 
\label{longpath}

\textup{(b)} $H$ has two vertex disjoint square paths $P_{1}$ and $P_{2}$ so that $|P_{1}|+|P_{2}|>(\frac{5}{6}-2\ep)n$. \label{twopaths} \end{lemma}
\begin{proof}
(a) Let $P:=u_{1}u_{2}...u_{p}$ be an optimal square path in $H$ and suppose that $p<(\frac{1}{2}-3\ep)n$.  We first observe that since $\delta(H)\geq (\frac{2}{3}-\ep)n$ we have $N(u_1,u_2)\geq (\frac{1}{3}-2\ep)n$ and thus $p>(\frac{1}{3}-2\ep)n$.
Let $H':=H-P$ and set $h:=|H'|$. If $\|v,P\|\leq (\frac{2}{3}-4\ep)p$ for all $v\in V(H')$ then we have $\delta(H')\geq(\frac{2}{3}-\ep)n-(\frac{2}{3}-4\ep) p \geq\frac{2}{3}h$.
Thus by Theorem~\ref{thm:FK2},  $H'$ has a hamiltonian square path of length more than than $\frac{1}{2}n$, contradicting the optimality of $P$. Thus there is a vertex $x\in V(H')$ such that $\|x,P\|>(\frac{2}{3}-4\ep)p>\frac{1}{2}p+1$. It follows that $x$ is adjacent to two consecutive vertices of $P$. Choose $i\in[p]$ as small as possible such that $u_{i},u_{i+1}\in N(x)$. Let $Q:=u_{1}u_{2}...u_{i-1}$ and set $q:=i-1$. Then $\|x,Q\|\leq\frac{1}{2}q$. We claim that $q<(\frac{1}{6}-2\ep)n$.
Otherwise, \begin{align*}
\|x,P-Q\|>(\frac{2}{3}-4\ep)p-\frac{1}{2}q & =\frac{2}{3}(p-q)+\frac{1}{6}q-4\ep p\\
 & >\frac{2}{3}|P-Q|+\frac{1}{6}(\frac{1}{6}-2\ep)n-4\ep(\frac{1}{2}-3\ep)n\\
 & >\frac{2}{3}|P-Q|+\frac{1}{36}n-\frac{7}{3}\ep n\\
 & >\frac{2}{3}|P-Q|+1,  \end{align*}
contradicting Lemma \ref{segment}. On the other hand, since $|N(x,u_{i})|\geq(\frac{1}{3} -2\ep)n=\frac{2}{3} (\frac{1}{2}-3\ep)n>\frac{2}{3}p$, Lemma~\ref{segment} implies $x$ and $u_{i}$ have a common neighbor $y$ in $H'$. Also, by Lemma \ref{segment} we have \[
\delta(H')\geq(\frac{2}{3}-\ep)n-(\frac{2}{3}p-\frac{1}{3})>\frac{2}{3}h-\ep n,\] and thus for any edge $uv$ in $H'$, $|N_{H'}(u,v)|\geq\frac{1}{3}h-2\ep n>(\frac{1}{6}-2\ep)n$. Hence, we can find a square path $P'$ of length at least $(\frac{1}{6}-2\ep)n$ starting at $xy$. Since $|P'|>q$, the square path $P'yxu_{i}u_{i+1}...u_{p}$ is longer than $P$, a contradiction. This completes the proof of part (a).

(b) Let $P_{1}$ be an optimal square path in $H$ and let $p:=|P_{1}|$. Note that $p\geq(\frac{1}{2}-3\ep)n$ by Lemma \ref{pathcover}(a).  If $p>(\frac{5}{6}-2\ep)n$, then set $P_2=\emptyset$ and we are done. So we may assume that $p\leq (\frac{5}{6}-2\ep)n$. Set $H':=H-P_{1}$ and $h:=|H'|>n/6$. If $\|v,P_{1}\|\leq(\frac{2}{3}-3\ep)p$ for all $v\in V(H')$
then $\delta(H')\geq(\frac{2}{3}-\ep)n-(\frac{2}{3}-3\ep) p\geq\frac{2}{3}h$. Thus  $H'$ has a hamiltonian square path $P_{2}$ by Theorem~\ref{thm:FK2}, and we are done. Otherwise, let $x\in V(H')$ such that $\|x,P_{1}\| >(\frac{2}{3}-3\ep)p$. Note that by Lemma \ref{segment}, we have $\delta(H')\geq(\frac{2}{3}-\ep)n -(\frac{2}{3}p -\frac{1}{3}) > \frac{2}{3}h-\ep n$, and thus there is a square path of length at least $\frac{1}{3}h-2\ep n$ starting at any ordered edge in $H'$. Set $H'':=G[N_{H'}(x)]$ and $h':=|H''|$. Note that by Lemma \ref{edgetopath}, we have that for all
$y\in V(H'')$, 
\begin{align*}
\|y,P_1\|<\frac{4}{3}p-\frac{2}{3}(\frac{1}{3}h-2\ep n)+2-(\frac{2}{3}-3\ep)p=\frac{2}{3}p-\frac{2}{9}h+\frac{4}{3}\ep n+3\ep p+2,\end{align*}
so \begin{align*}
\|y,H'\|>(\frac{2}{3}-\ep)n-(\frac{2}{3}p-\frac{2}{9}h+\frac{4}{3}\ep n+3\ep p+2)=\frac{8}{9}h-\frac{7}{3}\ep n-3\ep p-2.\end{align*}
So every vertex in $H''$ has at most $\frac{1}{9}h+\frac{7}{3}\ep n+3\ep p+1$ nonneighbors in $H'$. Therefore \begin{align*}
\delta(H'')\geq\frac{\frac{2}{3}h-\ep n-(\frac{1}{9}h+\frac{7}{3}\ep n+3\ep p+1)}{\frac{2}{3}h-\ep n}h' >\frac{2}{3}h',\end{align*}  %This uses fix.
 since $\ep\leq\frac{1}{500}$, $n\ge6000$, and $h>n/6$. Therefore $H''$ has a hamiltonian square path $P_{2}$. Thus \begin{align*}
|P_{1}|+|P_{2}| > p+\frac{2}{3}h-\ep n = n-\frac{1}{3}h-\ep n \geq n-\frac{1}{3}(\frac{1}{2}+3\ep)n-\ep n = (\frac{5}{6}-2\ep)n.\end{align*}

\end{proof}

Now we are ready to finish the nonextreme case.
\begin{proof}[Proof of Theorem \ref{non-extremal}]
Let $\alpha:=\frac{1}{36}$ and let $G$ be a graph on $n$ vertices. Suppose $G$ has no $\alpha$-extreme sets, $n\geq n_{0}:=\n$, and $\delta(G)\geq \frac{2}{3}n$.  Let $c:=\frac{1}{14}$, $\ep:=\frac{50}{1057}\alpha$, and $\varrho:=1-\frac{2/3+\ep}{5/6-2\ep}$.  Apply Lemma~\ref{reservoir} to obtain an $(\frac{11}{14}\alpha, \frac{1}{14}\alpha, \ep, \varrho)$-special reservoir $R$. Let $H:=G-R$ and let $h:=|H|$. Since $R$ is a special
reservoir we have $\delta(H)\geq(\frac{2}{3}-\ep)h$. Now we apply
Lemma~\ref{pathcover} to $H$, to get disjoint square paths $P_{1}$
and $P_{2}$ so that $$|P_{1}|+|P_{2}|>(\frac{5}{6}-2\ep)h=(\frac{5}{6}-2\ep)(n-\ceiling{\varrho n})\geq (\frac{2}{3}+\ep)n-1>\frac{2}{3}n.$$
Since $R$ is a special reservoir, every special set $S\subseteq V(G)$ has the property that $S\cap R$ is not $(\frac{11}{14}\alpha, \frac{1}{14}\alpha)$-extreme in $G[R]$. So we apply Lemma \ref{connecting} at most twice to connect the paths $P_{1}$ and $P_{2}$ through $R$.
On the second application, we set $L:=V(P_1)\cap R$ to make sure that we avoid the vertices used
in the first application. This gives us a square cycle $C$ with $V(P_{1})\cup V(P_{2})\subseteq V(C)$
and thus $|C|>\frac{2}{3}n$. Therefore $G$ has a hamiltonian square cycle by Theorem \ref{thm:FK3}.
\end{proof}

\section{Extremal Case\label{sec:Ex}}

In this section we prove Theorem \ref{thm:good}. First we need two
propositions. Note that the length of an (ordinary) path $P$ is the size $\|P\|$ of its edge set.
\begin{proposition}
\label{pro:LPath}Every connected graph $H$ with $|H|\geq 3$ has a path or cycle of
length $\min(2\delta(H),|H|)$. 
\end{proposition}
\begin{proof}
Let $P$ be a maximum length path in $H$.  If we are not done, then $\|P\| < 2\delta(H)$. So, as in the proof of Dirac's Theorem \cite{D}, 
$G$ has a cycle $C$ that spans $V(P)$. If $C$ is hamiltonian then we are done; otherwise, using connectivity, we can extend $C$ to a path longer than $P$, a contradiction.
\end{proof}
\begin{proposition}
\label{pro:LCyc}If $H$ is a graph with circumference $l>|H|-\delta(H)$,
then $l\geq\min(2\delta(H),|H|)$, and moreover, if $|H|$
is also even, then $H$ has an even cycle of length at least $\min(2\delta(H),|H|)$.\end{proposition}
\begin{proof}
Let $C\subseteq H$ be a cycle of length $l$, and
fix an orientation of $C$. If $|C|=|H|$ then we are done, even if
$|H|$ is even. Otherwise, let $P:=v_{1}\dots v_{p}$ be a maximum
path in $H-C$. Then all neighbors of $v_{p}$ are on $P\cup C$.
By hypothesis $\delta(H)>|H|-l\geq p$, and so $v_{1}$ has a neighbor
$x\in C$ and $v_{p}$ has a neighbor on $C-x$. Let $y,z\neq x$ be neighbors
of $v_{p}$ on $C$ with $y$ as close as possible to $x$ in the
forward direction and $z$ as close as possible in the backward direction
(possibly $y=z$). Then $\left\Vert zCx\right\Vert ,\left\Vert xCy\right\Vert \geq p+1$,
as otherwise we could replace the interior vertices of one of these segments
with $P$ to obtain a longer cycle, which would yield a contradiction. Moreover, since $C$ has maximum length, any two neighbors of $v_{p}$ are separated by
at least one vertex on $C$. Since $v_{p}$ has at least $d(v_{p})-p$
neighbors on $C-x$, \[
|C|=\left\Vert xCy\right\Vert +\left\Vert yCz\right\Vert +\left\Vert zCx\right\Vert \geq(p+1)+2(d(v_{p})-p-1)+(p+1)\geq2\delta(H).\]


Now suppose $|H|$ is even. If $|C|$ is even we are done, so suppose
$|C|$ is odd. Consider the path $P$ and vertices $x,y,z$ defined
above. If $\|xCy\|$ and $\|zCx\|$ have different parity, then replace
$xCy$ with $xPy$ or replace $zCx$ with $zPx$ to get an even cycle
of length at least $2\delta(H)$. So assume $\|xCy\|$ and $\|zCx\|$
have the same parity, and thus $\|yCz\|$ is odd. Now $v_{p}$ has
$k\geq d(v_{p})-p$ neighbors on $yCz$. Let $y=a_{1},a_{2},\dots,a_{k}=z$
be the neighbors of $v_{p}$ on $yCz$ in their natural order. Since
$\|yCz\|$ is odd, some segment $a_{i}Ca_{i+1}$ must have odd length.
By replacing $a_{i}Ca_{i+1}$ with $a_{i}v_{p}a_{i+1}$, we get a
cycle $C'$ with even length such that $|C'|\geq(p+1)+(p+1)+2(d(v_{p})-p-1)\geq2\delta(H)$ as before. 
\end{proof}

\begin{proof}[Proof of Theorem~\ref{thm:good}]
Let $G=(V,E)$ be a graph on $n$ vertices with $\delta(G)\geq \frac{2}{3}n$.  By Corollary \ref{3k} we may assume $n=3k$, which gives $\delta(G)\geq 2k$. Set $\alpha:=\frac{1}{36}$, and suppose $G$ has an $\alpha$-extreme
subset.  Let $S\subseteq V$ be an $\alpha$-extreme set of minimal order, so $|S|= \lceil (1-\alpha)k\rceil$. Set $T:=V\setminus S$. If $k<1/\alpha$, then $|S|=k$, $|T|=2k$, $G[S,T]$ is complete and $\delta(G[T])\geq k$.  So by Dirac's theorem $T$ has a hamiltonian cycle $C:=y_1\dots y_{2k}y_1$. Since $G[S,T]$ is complete we can insert the vertices $x_1, x_2, \dots, x_k$ of $S$ into $C$ so that $y_1y_2x_1y_3y_4x_2\dots y_{2k-1}y_{2k}x_ky_1y_2$ is a hamiltonian square cycle.  So for the rest of the proof assume $k\geq 1/\alpha$.  Choose $T_{0}\subseteq T$ such that $|V\setminus (S\cup T_0)|$ is even, 
$2\floor{\sqrt{\alpha}k}-1\leq|T_0|\le 2\floor{\sqrt{\alpha}k}$, and subject to this, $\left\Vert T_{0},S\right\Vert $
is as small as possible. Set $T_{1}:=T\setminus T_{0}$, and note that $|T_1|$ is even. We have, 
\begin{equation}
\forall x\in S,~\overline{\left\Vert x,T\right\Vert }\le k-(|S|-\left\Vert x,S\right\Vert) \leq 2\floor{\alpha k}.\label{eq:degx}\end{equation}
Every vertex in $T_1$ has at most as many nonneighbors in $S$ as every vertex in $T_0$. Thus, using $\alpha=\frac{1}{36}$, and expressing $k$ as $k=36q+r$ with $q,r\in \mathbb{Z}$ and $0\le r \le 35$, we have %HK
\begin{equation}
\forall y\in T_{1},~\overline{\left\Vert y,S\right\Vert }\leq \floor{\frac{2\floor{\alpha k}|S|}{|T_{0}\cup \{y\}|}}\leq \floor{\frac{2\floor{\alpha k}(k-\floor{\alpha k})}{2\floor{\sqrt{\alpha}k}}} \le \floor{\frac{(35q+r)}{6}}{\leq} \floor{\sqrt{\alpha}k}.
\label{eq:degy}
\end{equation}

Set $m:=k-|T_0|+\floor{\alpha k}$ and note that since $k\geq 36$,
\begin{equation}\label{eq:m}
m\geq \frac{2}{3}k+\floor{\alpha k}\geq \frac{2}{3}k+1. %HK                      
\end{equation}
Thus we have \begin{equation}
\delta(G[T_{1}])\geq 2k-|S\cup T_{0}|=k-|T_0|+\floor{\alpha k}=m\ge\frac{2}{3}k+1.\label{eq:T_1}
\end{equation}

\noindent 
\textbf{Case 1: }There exists an even cycle \textbf{$C\subseteq G[T_{1}]$}
of length $2l\geq2m$; say $C:=y_{1}\dots y_{2l}y_{1}$.  Looking ahead to an application in Case 2, we prove something slightly more general than what is needed for Case 1.  For some $t\leq |T_1|/2$, let $T_1'\subseteq T_1$ such that $|T_1'|=2t$.  Enumerate the vertices of $T_1'$ as $z_1,\dots,z_{2t}$.  Let $P:=\{p_{1},\dots,p_{t}\}$ be a set of \emph{ports}, where $p_{i}:=\{z_{2i-1},z_{2i},z_{2i+1},z_{2i+2}\}$ and addition of indices is modulo $t$.  We say that a vertex $x\in S$ can be inserted into port $p_{i}$ if $p_{i}\subseteq N(x)$.

\begin{claim}\label{ports}
For $S'\subset S$ with $|S'|\ge |S|-4$, let $\Gamma$ be the $S',P$-bigraph with $xp\in E(\Gamma)$ if and only if $x$ can be inserted into $p$. Then $\Gamma$ has a matching $M:=\{x_{i}p_{i}:i\in[t]\}$ that saturates $P$.
\end{claim}

\begin{proof}

Using Hall's Theorem, since $|S'|\geq |T_1|/2\geq |P|$, it suffices to show that  %HK
\begin{equation}
\|x,P\|_{\Gamma}+\|S',p\|_{\Gamma}\ge|P|\textrm{ for all \ensuremath{x\in S' } and \ensuremath{p\in P}.}\label{H}\end{equation} 

If $x\in S'$, then $\overline{\|x,T\|}_{G}\leq 2\floor{\alpha k}$ by \eqref{eq:degx}.
Since each $y\in T_1'$ is in two ports, each nonedge $xy$ contributes
to two nonedges in $\Gamma$. So $\overline{\|x,P\|}_{\Gamma}\leq 4\floor{\alpha k}$.
Thus \begin{equation}
\|x,P\|_{\Gamma}\geq|P|-\overline{\|x,P\|}_{\Gamma}\geq |P|-4\alpha k.\label{SP}\end{equation}
 If $p\in P$, then $\overline{\|S',y\|}_{G}\leq \floor{\sqrt{\alpha}k}$ for each
$y\in p$ by \eqref{eq:degy}. Thus $\overline{\|S',p\|}_{\Gamma}\leq 4\floor{\sqrt{\alpha}k}$. 
So\begin{equation}
\|S',p\|_{\Gamma}\geq|S'|-\overline{\|S',p\|}_{\Gamma}\ge (1-\alpha-\frac{4}{k}-4\sqrt{\alpha})k.\label{PS}\end{equation}
Since $4\sqrt{\alpha}+5\alpha+\frac{4}{k}\leq \frac{33}{36}< 1$, summing \eqref{SP} and \eqref{PS} %HK
yields \eqref{H}.
\end{proof}

Let $S':=S$ and $P:=\{p_{1},\dots,p_{l}\}$, where $p_{i}:=\{y_{2i-1},y_{2i},y_{2i+1},y_{2i+2}\}$ and addition of indices is modulo $2l$. 
By Claim \ref{ports}, there exist $x_{1}, \dots, x_l$ such that $y_{1}y_{2}x_{1}y_{3}y_{4}x_{2}\dots y_{2l-1}y_{2l}x_{l}y_{1}y_{2}$
is a square cycle of length $3l$. By (\ref{eq:T_1}), $3l\geq 3m>2k$, and so Theorem~\ref{thm:FK3} implies that $G$ has a hamiltonian square
cycle.

\noindent 
\textbf{Case 2:} Not Case 1. Since $|T_1|$ is even, using Proposition~\ref{pro:LCyc} and \eqref{eq:T_1}, 
 \begin{equation}
|D|\leq|T_{1}|-\delta(G[T_{1}])\leq k,\text{ for every cycle }D\subseteq G[T_{1}].\label{eq:con}\end{equation}
First suppose $G[T_{1}]$ is connected.  By Proposition \ref{pro:LPath}, there exists a path in $G[T_1]$ of length at least $2m$.
\begin{claim}\label{nocross}
Let $P=y_1\dots y_l$ be a path of maximum length in $G[T_1]$.  If $y_i\in N(y_1)$ and $y_j\in N(y_l)$, then $i\leq j$.
\end{claim}

\begin{proof}
Suppose there exists $y_i\in N(y_1)$, $y_j\in N(y_l)$ such that $i>j$.  With respect to this condition, choose $y_i$ and $y_j$ such that $i-j$ is minimum.  If $i-j-1\leq \frac{1}{3}k$, 
set $D:=y_1\dots y_jy_l\dots y_iy_1$.  By \eqref{eq:m}, %HK
$|D|\geq 2m-\frac{1}{3}k>k$, which contradicts \eqref{eq:con}.  If $i-j-1>\frac{1}{3}k$, let $h$ be maximum such that $y_h\in N(y_1)$ and set $D:= y_1y_2\dots y_hy_1$.  Since $i-j-1>\frac{1}{3}k$ and $i-j$ is minimum, we have $|D|\geq h\geq m+i-j-1>k$, which contradicts \eqref{eq:con}.
\end{proof}

Let $P:=y_1\dots y_l$ be a path of maximum length in $G[T_1]$ and with respect to this condition, choose $P$ so that  $j-i$ is minimum, where $y_j$ is the smallest indexed neighbor of $y_l$ and $y_i$ the largest indexed neighbor  of $y_1$.  %HK
Note that by Claim \ref{nocross}, $j-i\geq 0$. By \eqref{eq:con} we have,
\begin{equation}
 N(y_{1})\subseteq \{y_2,\dots y_k\} \textrm{ and } N(y_{l})\subseteq \{y_{l-k+1},\dots,y_{l-1}\}. \label{dis}\end{equation}

Set \[A:=\{y_{1},\dots,y_{i-1}\},\ B:=\{y_{i},\dots,y_{j}\},\ C:=\{y_{j+1},\dots,y_{l}\}.\]
Without loss of generality we may suppose $|A|\geq |C|$ and thus we have \begin{equation}\label{ACbounds}m\leq\delta(G[T_{1}])\leq|C|\leq |A|< k\end{equation} and $|B|= j-i+1\leq l-2m$.

Next we show that \begin{equation}
\left\Vert A,C\right\Vert =0.\label{dj}\end{equation}
 Suppose $a<i\leq j<b$ and $y_{a}y_{b}\in E$. Choose $y_{a'}\in N(y_{1})$
and $y_{b'}\in N(y_{l})$ such that $a<a'\le i\leq j\le b'<b$ and
both $a'-a$ and $b-b'$ are minimal.  Now $D:=y_{1}Py_{a}y_{b}Py_{l}y_{b'}Py_{a'}y_{1}$ is a cycle having the property that $N(y_1)\cup N(y_l)\subseteq V(D)$ and thus $|D|\geq|N(y_1)\cup N(y_l)|\geq 2m-1>k$, contradicting (\ref{eq:con}).

Set $A':=\{y_h\in A:y_{h+1}\in N(v_1)\}$ and $C':=\{y_h\in C: y_{h-1}\in N(y_l)\}$.  Note that $|A'|\geq m$ and $|C'|\geq m$.  
We claim that the vertices in $A'\cup C'$ are good in the sense that
\begin{equation}\label{A'C'}
\forall a\in A', N(a)\cap (T_1\setminus (A\cup \{y_i\}))= \emptyset \text{ and }  \forall c\in C', N(c)\cap (T_1\setminus (C\cup \{y_j\}))= \emptyset.                                                                                                                                                                                                                                                                                                                                                                                            \end{equation}
Without loss of generality, suppose some $y_h\in A'$ has a neighbor $y'\in T_1\setminus (A\cup \{y_i\})$.  %HK
If $y'\notin V(P)$, then $y'y_h\dots y_1y_{h+1}\dots y_l$ is longer than $P$ which is a contradiction.  Otherwise, by \eqref{dj}, $y'\in B$.  However, $y_h\dots y_1y_{h+1}\dots y_l$ is a path for which $j-i$ is smaller, contradicting the minimality of $j-i$.

Now suppose $G[T_1]$ is not connected.  Since $\delta(G[T_1])\geq m$ and $|T_1|< 3m$, $G[T_1]$ has exactly two components.  Call these components $A$ and $C$, then set $A':=A$ and $C':=C$.  Without loss of generality, suppose $|A|\geq |C|$.  Since $\delta(G[T_1])\geq m$, we have $m+1\leq |C|$ which implies $|A|<k$, by \eqref{eq:m} and the fact that $|T_1|=2k+\floor{\alpha k}-|T_0|$.  So regardless of whether $G[T_1]$ is connected or not, all of the following hold: \eqref{ACbounds}, \eqref{dj}, \eqref{A'C'}, and  \begin{equation}\label{eq:ACnonneighbors} \forall a\in A', \overline{\|a, A\|}\leq|A|-m 
~\text{ and }~ \forall c\in C', \overline{\|c, C\|}\leq|C|-m.\end{equation}
For $Y\in\{A, C\}$, let $Y'=A'$ if $Y=A$ and let $Y'=C'$ if $Y=C$.
\begin{claim}\label{Y'}
For all $v\in V\setminus (A\cup C)$, there exists $Y\in\{A, C\}$ such that for all $y\in Y'$, $|(N(v)\cap N(y))\cap Y|\geq 3$.
\end{claim}

\begin{proof}
For all $v\in V\setminus (A\cup C)$, we have 
\begin{equation}
\|v, A\cup C\|\geq 2k-(|V|-(|A|+|C|))=|A|+|C|-k.
\end{equation}
Suppose there exists $v\in V\setminus (A\cup C)$ and $c\in C'$ such that $|(N(v)\cap N(c))\cap C|\leq 2$.  This implies that $\|v, C\|\leq |C|-m+2$ by \eqref{eq:ACnonneighbors}.  So we have $$\|v, A\|\geq |A|+|C|-k-(|C|-m+2)= |A|+m-k-2.$$  Let $a\in A'$, then by \eqref{eq:m}, $$|(N(v)\cap N(a))\cap A|\geq (|A|+m-k-2)+m-|A|= 2m-k-2\geq \frac{1}{3}k\geq 3.$$

\end{proof}

\begin{claim}
There exist two disjoint square $P^{5}$'s connecting edges of $A$ to edges of $C$.\label{con}
\end{claim}

\begin{proof}
Set $s:=\lfloor\frac{|A|}{2}\rfloor$ and $t:=\lfloor\frac{|C|}{2}\rfloor$.
Choose nonadjacent vertices $x,x'\in S$ and $a_{2s},c_{1}\in N(x)$
with $a_{2s}\in A'$ and $c_{1}\in C'$. Since $a_{2s}$ and $c_{1}$
are nonadjacent they have at least $k+1$ common neighbors distinct
from $x$, and these common neighbors are not in $A\cup C$. One of
them $v$ must also be adjacent to $x$. By Claim \ref{Y'} there exists, without loss of generality, $a_{2s-1}\in A$ such that $a_{2s},v\in N(a_{2s-1})$.  
Since $x\in S$, there exists $c_{2}\in C$ such that $x,c_{1}\in N(c_{2})$. Thus
$Q:=a_{2s-1}a_{2s}vxc_{1}c_{2}$ is a square $P^{5}$ connecting $a_{2s-1}a_{2s}$
to $c_{1}c_{2}$. Similarly, we can choose $a_{1},c_{2t}\in N(x')$
with $a_{1}\in A'-a_{2s-1}-a_{2s}$ and $c_{2t}\in C'-c_{1}-c_{2}$.
Since $a_{1}$ and $c_{2t}$ are nonadjacent, there exist $k$ common
neighbors of $a_{1}$ and $c_{2t}$ that are distinct from $x'$ and
$v$. One of them $v'$ is adjacent to $x'$, and $v'\ne x$ by the
choice of $x,x'$. Moreover, $v'\notin A\cup C$. So as above, we can choose $a_{2}\in A$ and $c_{2t-1}\in C$ so that
$Q':=c_{2t-1}c_{2t}\{v'x'\}a_{1}a_{2}$, $Q\cap Q'=\emptyset$ and $Q'$ is a square $P^{5}$
connecting $c_{2t-1}c_{2t}$ to $a_{1}a_{2}$ (note that we cannot specify the order of $v'$ and $x'$). 
\end{proof}

Finally we claim that there exist paths $$R:=a_{1}a_{2}\dots a_{2s-1}a_{2s}
\subseteq G[A] \textrm{ and } R':=c_{1}c_{2}\dots c_{2t-1}c_{2t}\subseteq G[C],$$ such that $|R|=2s$ and $|R'|=2t$. If $|A|=m$, then $A=A'$ %$C=C'$ 
and thus $G[A]$ is complete by \eqref{eq:ACnonneighbors}.
%and $G[C]$ are complete.  
Otherwise $|A|\geq m+1$ and thus by \eqref{eq:m} we have
\begin{equation}\label{halfA}
\frac{1}{3}k+1\leq \frac{1}{2}|A|.
\end{equation}
By \eqref{dj} and \eqref{halfA}, we have 
$$\delta(G[A])\geq 2k-(|V|-(|A|+|C|))=|A|+|C|-k\geq |A|+2-(\frac{k}{3}+1)\geq \frac{1}{2}|A|+2.$$ 
Thus for all $a,a',a''\in A$, 
\begin{equation*}G[A\setminus \{a,a',a''\}]
\text{ is hamiltonian connected,}
\end{equation*} 
since $\delta(G[A\setminus\{a,a',a''\}])\geq \frac{1}{2}|A|-1>\frac{1}{2}(|A|-3)$. If $|A|=2s$, then we use the fact that $G[A\setminus\{a_1,a_{2s}\}]$ is hamiltonian connected to get $R$.  If $|A|=2s+1$ we let $a'\in A\setminus\{a_1,a_2,a_{2s-1}, a_{2s}\}$, and we use the fact that $G[A\setminus\{a_1, a_{2s}, a'\}]$ is hamiltonian connected to get $R$.  Since $|A|\geq |C|$, the same argument gives us $R'$ in $G[C]$. 
%LD

So by Claim \ref{con}, $D:=RQR'Q'$ is an even cycle of length $2s+2t+4\geq 2m+2$ (note that $D\not\subseteq G[T_1]$). Recall that $V(D)\cap S\subseteq\{x,v,x',v'\}$ and set $S':=S\setminus D$. As in Case 1, let $P:=\{p_{1},\dots,p_{s}, p_1',\dots, p_t'\}$
be a set of \emph{ports}, where $p_{i}:=\{a_{2i-1},a_{2i},a_{2i+1},a_{2i+2}\}$ for $1\leq i\leq s-1$ and $p_j':=\{c_{2j-1},c_{2j},c_{2j+1},c_{2j+2}\}$ for $1\leq j\leq t-1$. 
  By Claim \ref{ports}, 
  there exist $x_1,\dots,x_{s-1},x'_1,\dots,x'_{t-1}$ such that
    $$a_{1}a_{2}x_{1}a_{3}a_{4}x_{2}\dots x_{s-1} a_{2s-1}a_{2s}vxc_{1}c_{2}x_1'c_3c_4x_2'\dots x_{t-1}'c_{2t-1}c_{2t}\{v'x'\}a_1a_2$$ is a square cycle of length at least $2s+2t+4+s-1+t-1\geq 3m-1>2k$. Thus by Theorem~\ref{thm:FK3}, $G$ has a hamiltonian square
cycle.

\end{proof}


\section{Conclusion}
We have established a concrete threshold $n_0:=\n$ such that P\'osa's Conjecture holds for all graphs of order at least  $n_0$, using methods essentially from prior to 1996. It seems in retrospect, that we were blinded by the brilliance of the Regularity-Blow-up method, and missed that the crucial idea of \cite{KSSp} was just to divide the problem into extremal and non-extremal cases.  However P\'osa's Conjecture remains open. We suspect that our probabilistic methods cannot be used to obtain an improvement of more than a factor of 1000. On the other hand we believe that ordinary graph theoretic methods have not yet been exhausted.

We have also developed the method of special reservoirs, for removing regularity from certain arguments.  We believe that this could be used on other problems.
The paper \cite{LSS} was written with the goal of developing methods for a more general set of problems. 
In particular they used an
\emph{absorbing path} lemma which contributes to a much larger value of
$n_0$. However other problems do not (yet) have an analog of Theorem \ref{thm:FK3}, while the absorbing technique is quite adaptable. %LD
Here are some other possible candidates
for applying these new techniques, the first of which was discussed in
\cite{LSS}.

\begin{conjecture}[Seymour \cite{Sey}]
For all positive integers $k$, every graph $G$ with  $\delta(G)\ge \frac{k}{k+1}|G|$ contains the $k^\text{th}$ power of a hamiltonian cycle.
\end{conjecture}

Koml\'os, S\'ark\"ozy and Szemer\'edi \cite{KSSps,KSSs} used the Regularity and Blow-up Lemmas to prove that there exists a function $n(k)$ such that Seymour's Conjecture holds for all $k$ and graphs of order at least $n(k)$.

Ch\^au also used the Regularity and Blow-up Lemmas to prove the following Ore-type version of P\'osa's Conjecture for graphs of large order.
\begin{theorem}[Ch\^au \cite{C}]
Let $G$ be a graph on $n$ vertices such that $d(x)+d(y)\geq
\frac{4}{3}n-\frac{1}{3}$ for all $xy\notin E(G)$.

\textup{(a)} If $\delta(G)=\frac{1}{3}n+2$ or $\delta(G)=\frac{1}{3}n+\frac{5}{3},$ then
$G$ contains a hamiltonian square path.

\textup{(b)} If $\delta(G)>\frac{1}{3}n+2,$ then for sufficiently large $n,$ $G$
contains a hamiltonian square cycle.
\end{theorem}

For a directed graph $G$, the \emph{minimum semi-degree} of $G$, denoted $\delta^0(G)$, is the minimum of the minimum in-degree $\delta^-(G)$ and the minimum out-degree $\delta^+(G)$.  An \emph{oriented graph} is a directed graph with no $2$-cycles. Keevash, K\"uhn, and Osthus proved the following oriented version of Dirac's theorem using the Regularity-Blow-up method (with a directed version of the Regularity Lemma).

\begin{theorem}[Keevash, K\"uhn, Osthus \cite{KKO}] Let $G$ be an oriented graph on $n$ vertices.  If $\delta^0(G)\geq \frac{3n-4}{8}$ and $n$ is sufficiently large, then $G$ contains a hamiltonian cycle.
\end{theorem}

Finally Treglown conjectured the following oriented version of P\'osa's conjecture.

\begin{conjecture}[Treglown \cite{Tr}]
Let $G$ be an oriented graph on $n$ vertices.  If $\delta^0(G)\geq \frac{5n}{12}$, then $G$ contains a the square of a hamiltonian cycle.
\end{conjecture}

% 
% \subsection*{Acknowledgements}
% 
% We thank the referees for their thoughtful comments which improved the presentation of this paper.




\chapter{REGULARITY-BLOW-UP METHOD}\label{regularitychapter}

\DoubleSpacing
\setlength{\parindent}{.5in}


In this section we review the Regularity and Blow-up Lemmas and state all the facts needed for our applications in Chapters \ref{2factorschapter} and \ref{sumdegtilingchapter} (see \cite{KS} for a nice reference). Let $\varGamma$ be a simple graph on $n$ vertices. For two disjoint, nonempty subsets $U$ and $V$ of $V(\varGamma)$, define the density of the pair $(U,V)$ as
\[
d(U,V)=\frac{e(U,V)}{|U||V|}.
\]

\begin{definition}
A pair $(U,V)$ is called $\ep$-\emph{regular} if for every $%
U^{\prime }\subseteq U$ with $|U^{\prime }|\geq \ep |U|$ and every $%
V^{\prime }\subseteq V$ with $|V^{\prime }|\geq \ep |V|$, $
|d(U^{\prime },V^{\prime })-d(U,V)|\leq \ep $. The pair $\left(
U,V\right) $ is $( \ep ,\delta ) $-\emph{super-regular}
if it is $\ep $-regular and for all $u\in U$, 
$\deg\left(
u,V\right) \geq \delta \left| V\right| $ and for all $v\in V$, $\deg\left(
v,U\right) \geq \delta \left| U\right| $.
\end{definition}

First we note the following facts that we will need about $\ep$-regular pairs.

\begin{fact}[Intersection Property]\label{interprop}
\label{deg} If $(U,V)$ is an $\ep $-regular pair with density $d$,
then for any $Y\subseteq V$ with $(d-\ep)^{k-1}|Y|\geq \ep |V|$ there are less than
$k\ep |U|^k$ $k$-tuples of vertices $(u_1, u_2, \dots, u_k)$, $u_i\in U$, such that $|Y\cap N(u_1, u_2, \dots, u_k)|\leq (d-\ep)^k|Y|$.
\end{fact}


\begin{fact}[Slicing Lemma]\label{slicing}
\label{slice} Let $(U,V)$ be an $\ep $-regular pair with density $d$, and for some $\lambda >\ep $ let $U^{\prime }\subseteq U$, $V^{\prime }\subseteq V$, with $|U^{\prime }|\geq \lambda |U|$, $|V^{\prime }|
\geq \lambda |V|$. Then $(U^{\prime },V^{\prime })$ is an $\ep
^{\prime }$-regular pair of density $d^{\prime }$ where $\ep
^{\prime }=\max \{\frac{\ep }{\lambda },2\ep \}$ and $
d^{\prime}\ge d-\ep $.
\end{fact}

% 
% \begin{lemma}
% \label{deg} If $(U,V)$ is an $\ep $-regular pair with density $\delta$,
% then for any $Y\subseteq V$ with $\left| Y\right| \geq \ep \left|
% V\right| $ there are less than
% $\ep |U|$ vertices $u\in U$ such that $\deg(u,Y)<(\delta-\ep )|Y|$.
% \end{lemma}

\begin{proposition}
\label{C4} If $(U,V)$ is an 
%balanced 
$\ep$-regular pair with density $\delta\ge 2\sqrt\ep>0$ and subsets $A,C\subseteq U$, $B,D\subseteq V$ of size at least $\frac{1}{2}\delta|U|$ then there exist $a\in A,b\in B,c\in C,d\in D$ with  $abcda=C_4$.
\end{proposition}
%LD15 


\begin{lemma}[Augmenting Lemma]
\label{aug}Let $\left( U,V\right) $ be an $\ep $-regular pair.
Suppose that $U^{\prime }=U\cup S$ and $V^{\prime }=V\cup T$, where $\left|
S\right| \leq \mu \left| U\right| $, $\left| T\right| \leq \mu \left|
V\right| $, $S\cap V^{\prime }=\emptyset =T\cap U^{\prime }$, and $0<\mu
<\ep $. Then $\left( U^{\prime },V^{\prime }\right) $ is an $%
\ep ^{\prime }$-regular pair, where $\ep ^{\prime }=\max
\left\{ \frac{\mu }{\ep },6\ep \right\} $.
\end{lemma}


We will use the Regularity Lemma of Szemer\'{e}di \cite{Sz} which we state in its multipartite form.

\begin{lemma}[Regularity Lemma - Bipartite Version]\label{bireg} 
For every $\ep>0$ there exists $M:=M(\ep)$ such that if $G:=G[U,V]$ is a balanced bipartite graph on $2n$ vertices and $d\in[0,1]$, then there is a partition of $U$ into clusters $U_0, U_1,\dots, U_t$, a partition of $V$ into clusters $V_0, V_1,\dots, V_t$, and a subgraph $G':=G'[U,V]$ with the following properties: 

\begin{enumerate}
\item $t\leq M$,
\item $|U_0|\leq \ep n$, $|V_0|\leq \ep n$,
\item $|U_i|=|V_i|=\ell\leq \ep n$ for all $i\in [t]$,
\item $\deg_{G'}(x)>\deg_G(x)-(d+\ep)n$ for all $x\in V(G)$,
\item All pairs $(U_i, V_i)$, $i,j\in [t]$, are $\ep$-regular in $G'$ each with density either $0$ or exceeding $d$.

\end{enumerate}
\end{lemma}

We will also use the following stronger version of the Blow-up Lemma of Koml\'{o}s, S\'{a}rk\"{o}zy, and Szemer\'{e}di \cite{KSSbu}.

% \begin{lemma}[Blow-up Lemma]\label{blowup}
% Given $\delta >0$, $\Delta >0$ there exists $\ep >0$ such that the following holds. Let $(U, V)$ be an $(\ep ,\delta )$-super-regular pair. %with $|U|=n_{1}$ and $|W_2|=n_{2}$. 
% If $T$ is a $U',V'$-bigraph with maximum degree $\Delta (T)\leq\Delta $ and $T$ is embeddable into the complete bipartite graph $K_{|U|},{|V|}$ then it is also embeddable into $(U,V)$.
% \end{lemma}

\begin{lemma}[Blow-up Lemma]\label{blowup}
Given $\delta >0$, $\Delta >0$ and $\varrho>0$ there exist $\ep >0$ and $\eta>0$ such that the following holds. Let $S=(X_1, X_2)$ be an $(\ep ,\delta )$-super-regular pair. with $|X_1|=n_{1}$ and $|X_2|=n_{2}$. 
If $T$ is a $Y_1, Y_2$-bigraph with maximum degree $\Delta (T)\leq\Delta $ and $T$ is embeddable into the complete bipartite graph $K_{n_1},{n_2}$ then it is also embeddable into $S$.  Moreover, for all $\eta |X_i|$-subsets $X_i'\subseteq X_i$ and functions $f_{i}:X_{i}'\rightarrow \binom{X_{i}}{\varrho n_{i}}$, $i=1,2$, $T$ can be embedded into $S$ so that the image of each $x_{i}\in X_{i}'$ is in the set $f_{i}\left( x_{i}\right)$.
\end{lemma}


%%%%%%%%%%%%%%%%%%%%%%%%%%%%%%%%%%%%%%%%

% 
% In this section we review the Regularity and Blow-up Lemmas. Let $\varGamma$ be a simple graph on $n$ vertices. For two disjoint, nonempty subsets $U$
% and $V$ of $V(\varGamma)$, define the density of the pair $(U,V)$ as
% \[
% d(U,V)=\frac{e(U,V)}{|U||V|}.
% \]
% 
% \begin{definition}
% A pair $(U,V)$ is called $\ep $-\emph{regular} if for every $%
% U^{\prime }\subseteq U$ with $|U^{\prime }|\geq \ep |U|$ and every $%
% V^{\prime }\subseteq V$ with $|V^{\prime }|\geq \ep |V|$, $
% |d(U^{\prime },V^{\prime })-d(U,V)|\leq \ep $. The pair $\left(
% U,V\right) $ is $\left( \ep ,\delta \right) $-\emph{super-regular}
% if it is $\ep $-regular and for all $u\in U$, 
% $\deg\left(
% u,V\right) \geq \delta \left| V\right| $ and for all $v\in V$, $\deg\left(
% v,U\right) \geq \delta \left| U\right| $.
% \end{definition}
% 
% 
% \begin{definition}
% A partition $\{V_{0},V_{1},\dots, V_{t}\}$ of $V(\varGamma)$ is called $\ep$-regular if the following conditions are satisfied:
% 
% \begin{enumerate}
% \item  $|V_{0}|\leq \ep |V|$.
% 
% \item  For all $i,j\in[t]$, $|V_{i}|=|V_{j}|$.
% 
% \item  All but at most $\ep t^{2}$ of pairs $(V_{i},V_{j})$, $1\leq i,j\leq t$, are $\ep$-regular.
% \end{enumerate}
% \end{definition}
% 
% The parts of the partition are called \emph{clusters}. Note that the cluster
% $V_{0}$ plays a distinguished role in the above definitions and is usually
% called the exceptional cluster (or class). Our main tool in the proof will
% be the Regularity Lemma of Szemer\'{e}di \cite{Sz} which asserts that for
% every $\ep >0$ every graph which is large enough admits an $\ep$-regular partition into a bounded number of clusters.
% 
% \begin{lemma}[Regularity Lemma]  %%Should we change?Change
% \label{regularity} For every $\ep >0$ 
% %and every positive integer $m$ 
% there exists $N:=N(\ep,m)$ and $M:=M(\ep,m)$ such that every graph on at least $N$ vertices admits an $\ep $-regular partition $\{V_{0},V_{1},\dots, V_{t}\}$ with $m\leq t\leq M$.
% \end{lemma}
% %LD15 (remove m?)
% 
% In the next section we will want a regular partition of a bipartite graph so we will use the following formulation (see for example \cite{CK}).
% 
% %Let $\{U_{0},U_{1},\dots,U_{t_{1}}\}\cup\{V_{0},V_{1},\dots,V_{t_{2}}\}$
% %be an $e_{0}$-regular partition of $G$, with $m\le t_{1},t_{2}\le M(\ep,m)$,
% %starting from the initial partition $\{U,V\}$ of $G$. As in \cite{CK}
% %we {}``clean up'' this partition to obtain an $\ep_{1}$-regular
% %partition $\{U_{0},U_{1},\dots,U_{t}\}\cup\{V_{0},V_{1},\dots,V_{t}\}$
% 
% \begin{corollary}[Regularity Lemma - Bipartite Case]
% \label{bireg} For every $\ep>0$ there exists $N:=N(\ep)$ and $M:=M(\ep)$ such that every balanced $U,V$-bigraph on at least $2N$ vertices admits an $\ep$-regular partition $\{U_0,U_1,\dots,U_t\}\cup\{V_0,V_1,\dots,V_t\}$ with $t\leq M$ satisfying
% 
% \begin{enumerate}
% \item $|U_{0}|=|V_{0}|\leq\ep n$,
% \item for all $i,j\in[t]$, $(1-\ep)\frac{n}{t}\leq|U_{i}|=|V_{j}|\leq\frac{n}{t}$ and
% \item for all $U_i\in\{U_1,\dots,U_t\}$ there are at most $\ep t$ sets $V_j\in\{V_1,\dots,V_t\}$ such that $(U_i,V_j)$ is not $\ep$-regular and for all $V_i\in\{V_1,\dots,V_t\}$ there are at most $\ep t$ sets $U_j\in\{U_1,\dots,U_t\}$ such that $(V_i,U_j)$ is not $\ep$-regular.
% %\item for any $U_i\in\{U_1,\dots,U_t\}$ $(V_i\in\{V_1,\dots,V_t\})$ there are at most $\ep t$ sets $V_j\in\{V_1,\dots,V_t\}$ $(U_j\in \{U_1,\dots,U_t\})$ such that $(U_i,V_j)$ $((V_i,U_j))$ is not $\ep$-regular.
% \end{enumerate}
% \end{corollary}
% 
% %LD15





\chapter{2-FACTORS OF BIPARTITE GRAPHS WITH ASYMMETRIC MINIMUM DEGREES}\label{2factorschapter}

\DoubleSpacing
\setlength{\parindent}{.5in}

This chapter is joint work with H.A. Kierstead and Andrzej Czygrinow and was published in SIAM Journal on Discrete Mathematics \cite{CDK}.


\section{Introduction}

This paper is motivated by several lines of research. Let $C_n^r$ ($P_n^r$) be the $r$-th power of a cycle (path) on $n$
%mistake
vertices $C_n$ ($P_n$). In attempt to inspire a new proof of the Hajnal-Szemer\'edi theorem, Seymour made the following conjecture:
\begin{conjecture}[Seymour \cite{Sey}]
If $G$ is a graph on $n$ vertices with $\delta(G)\geq \frac{r}{r+1}n$, then $C_n^r\subseteq G$.
%mistake
\end{conjecture}
Note that the case $r=1$ is Dirac's Theorem and the case $r=2$ is P\'osa's Conjecture.  Koml\'os, S\'ark\"{o}zy and Szemer\'edi \cite{KSSps,KSSs} have used Szemer\'edi's Regularity Lemma \cite{Sz} and their own Blow-up Lemma \cite{KSSbu} to prove Seymour's conjecture for huge graphs, however even P\'osa's Conjecture remains open for small graphs.

Chau generalized the minimum degree condtion in Seymour's conjecture to an Ore-type degree condition.
\begin{conjecture}[Chau \cite{C}]
\label{Ore-seymour}Suppose $G$ is a graph on $n$ vertices such that $\deg(x)+\deg(y)\geq \frac{2r}{r+1}n-\frac{r-1}{r+1}$ for all non-adjacent pairs of vertices $x,y\in V(G)$. 
\begin{enumerate}
\item If $\delta(G)=\frac{r-1}{r+1}n+2$ or $\delta(G)=\frac{r-1}{r+1}n+\frac{5}{3}$, then $P_n^r\subseteq G$.
\item If $\delta(G)>\frac{r-1}{r+1}n+2$, then $C_n^r\subseteq G$.
\end{enumerate}
\end{conjecture}
When $r=1$, the condition $\deg(x)+\deg(y)\geq \frac{2r}{r+1}n-\frac{r-1}{r+1}$ is Ore's condition and thus $C_n^r\subseteq G$ with no further restrictions on the minimum degree.  Chau proved Conjecture \ref{Ore-seymour} for huge graphs when $r=2$.


The following fundamental graph packing conjecture was made independently by Bollob\'as-Eldridge \cite{BE1} and Catlin \cite{Cat}.  We state it here in a complementary form.

\begin{conjecture}[Bollob\'as-Eldridge \cite{BE1}, Catlin \cite{Cat}]
\label{con:BE} If $G$ and $H$ are graphs on $n$ vertices with $\Delta(H)\leq r$ and $\delta(G)\geq \frac{rn-1}{r+1}$, then $H\subseteq G$.
\end{conjecture}
Call a graph on $n$ vertices $r$-\emph{universal} if it contains
every graph $H$ on $n$ vertices with $\Delta(H)\leq r$, then Conjecture \ref{con:BE} states that $G$ is $r$-universal if $\delta(G)\geq \frac{rn-1}{r+1}$.  The case $r=1$ follows from the path version of Dirac's Theorem:
Since $\delta(G)\ge\frac{n-1}{2}$, $G$ contains the $1$-universal
graph $P_{n}$. Aigner and Brandt \cite{AB} proved Conjecture \ref{con:BE}
for the case $r=2$. Fan and Kierstead \cite{FK2} proved the path version
of P\'osa's Conjecture: If $\delta(G)\ge\frac{2n-1}{3}$
then $G$ contains the square $P_{n}^{2}$ of $P_{n}$.
Since $P_{n}^2$ is $2$-universal, we have a stronger version of the
Aigner-Brandt Theorem: If $\delta(G)\ge\frac{2n-1}{3}$ then $G$ contains a $2$-universal graph with maximum degree 4.  Csaba, Shokoufandeh and Szemer\'edi \cite{CSS} have proved Conjecture \ref{con:BE} for large graphs when $r=3$.

Kostochka and Yu generalized the minimum degree condition in the Bollob\'as-Eldridge conjecture to an Ore-type degree condition.
\begin{conjecture}[Kostochka-Yu \cite{KY1}]
\label{con:KY}
If $G$ and $H$ are graphs on $n$ vertices with $\Delta(H)\leq r$ and $\deg(x)+\deg(y)\geq \frac{2(rn-1)}{r+1}$ for all non-adjacent pairs of vertices $x,y\in V(G)$, then $H\subseteq G$.  
\end{conjecture}
The case $r=1$ follows from the path version of Ore's theorem: Since $\deg(x)+\deg(y)\geq n-1$ for all non-adjacent pairs of vertices $x,y\in V(G)$, $G$ contains the $1$-universal graph $P_n$. Kostochka and Yu \cite{KY2} proved Conjecture \ref{con:KY} for the case $r=2$.

El-Zahar made the following conjecture.
\begin{conjecture}[El-Zahar \cite{EZ}]
If $G$ is a graph on $n$ vertices with $\delta(G)\geq\sum_{i=1}^{k}\ceiling{\frac{1}{2}n_{i}}$
where $n_{i}\geq3$ and $n=\sum_{i=1}^{k}n_{i}$, then $G$ contains
$k$ disjoint cycles of lengths $n_{1},\dots,n_{k}$. 
\end{conjecture}
El-Zahar proved that if $G$ is a graph on $n$ vertices
with $\delta(G)\geq\ceiling{\frac{1}{2}n_{1}}+\ceiling{\frac{1}{2}n_{2}}$,
where $n_{1},n_{2}\geq3$ and $n=n_{1}+n_{2}$, then $G$ contains
two disjoint cycles of lengths $n_{1}$ and $n_{2}$.  Abassi \cite{Abas} used the Blow-up and Regularity Lemmas to prove El-Zahar's Conjecture for huge $n$.

Now we focus our attention on bipartite graphs. A $U,V$-bigraph
is \emph{balanced} if $|U|=|V|$. We will call a balanced bipartite
graph on $2n$ vertices \emph{bi-universal} if it contains every balanced
bipartite graph $H$ with $|H|=2n$ and $\Delta(H)=2$. Wang made the following conjecture.

\begin{conjecture}[Wang \cite{W1}]
\label{con:W1}Every balanced bipartite graph $G$ on $2n$ vertices with
$\delta(G)\geq n/2+1$ is bi-universal.
\end{conjecture}
An $n$-\emph{ladder}, denoted by $L_n$, is a balanced bipartite graph with vertex sets
$A=\{ a_{1},\dots,a_{n}\}$ and $B=\{ b_{1},\dots,b_{n}\}$
such that $a_{i}\sim b_{j}$ if and only if $\left|i-j\right|\leq1$. We refer
to the edges $a_{i}b_{i}$ as rungs and the edges $a_{1}b_{1},a_{n}b_{n}$
as the first and last rung respectively. It is easily checked 
that an $n$-ladder is a bi-universal graph with maximum degree $3$.  In this sense, a ladder in a bipartite graph is analogous to a square path in a graph. The first and last author \cite{CK} used the Blow-up and Regularity
Lemmas to prove Conjecture \ref{con:W1} for huge graphs by
proving that such graphs contain a spanning ladder.

Finally we consider bipartite graphs with asymmetric minimum degrees.
For a $U,V$-bigraph $G$, let $\delta_U:=\delta_U(G)$ and $\delta_V:=\delta_V(G)$ denote the minimum
degrees of vertices in $U$ and $V$ respectively. The number of components
of $G$ is denoted by $\mathrm{comp}(G)$. Moon and Moser \cite{MM}
proved that if $G$ is a balanced bipartite graph on $2n$ vertices
with $\delta_{U}+\delta_{V}\geq n+1$, then $G$ is hamiltonian. Amar \cite{A} proved the following result about more general 2-factors.  If $G$ and $H$ are balanced
$U,V$-bigraphs on $2n$ vertices with $\delta_{U}+\delta_{V}\geq n+2$, $\Delta(H)\leq 2$ and $\mathrm{comp}(H)\leq2$ then $G$ contains $H$. As noted in \cite{A}, when $\mathrm{comp}(H)\leq 2$ this result is best possible. Amar then made the following conjecture.

\begin{conjecture}[Amar \cite{A}]
\label{con:A}Let $G$ and $H$ be balanced $U,V$-bigraphs on $2n$
vertices with $\Delta(H)\leq 2$. If $\delta_{U}+\delta_{V}\geq n+\mathrm{comp}(H)
$ then $G$ contains $H$.
\end{conjecture}
We will prove the following theorems, strengthening Conjecture \ref{con:A}
for huge graphs.

\begin{theorem}\label{main} Let $G$ and $H$ be balanced $U,V$-bigraphs
on $2n$ vertices with $\Delta(H)\le2$. For every integer $k$ there
exists $N_0(k)$ such that if $n\geq N_0(k)$, $\delta_{U}+\delta_{V}\geq n+2$,
and $\mathrm{comp}(H)\le k$, then $G$ contains $H$. Furthermore,
if $\delta(G)\geq\frac{1}{200k}n+1$ then $G$ contains a spanning
ladder. 
\end{theorem}

\begin{theorem}\label{mainconstant}There exists a constant $C$ such
that every balanced $U,V$-bigraph $G$ on $2n$ vertices satisfying
$\delta_{U}+\delta_{V}\geq n+C$ contains a spanning ladder.
\end{theorem}

\begin{theorem}\label{amarlarge} Let $G$ and $H$ be balanced $U,V$-bigraphs
on $2n$ vertices with $\Delta(H)\le2$. There exists an integer $N_{0}$
such that if $n\geq N_{0}$ and $\delta_{U}+\delta_{V}\geq n+\mathrm{comp}(H)$
then $G$ contains $H$. 
\end{theorem}

We note that there are no known counterexamples to show that the bound in Amar's conjecture is tight when $k\geq 3$.  In fact, Wang made the following stronger conjecture:
\begin{conjecture}[Wang \cite{W2}]
\label{con:Wang2}
Every balanced $U,V$-bigraph on $2n$ vertices with $\delta_U+\delta_V\geq n+2$ is bi-universal.
\end{conjecture}

In Theorem \ref{amarlarge} we prove Amar's conjecture for huge graphs, but Theorem \ref{main} gives evidence to suggest that a proof of Conjecture \ref{con:Wang2} should ultimately be the goal.

We use the following notation. 
%Given a path $P=x_{1}\dots x_{k}$ we let $x_{i}Px_{j}=x_{i}\dots x_{j}$. 
For $A,B\subseteq V(G)$, $E(A,B)$ is the set of edges with one end in $A$ and the other in $B$. By $E(A)$ we mean $E(A,V(G)\ssm A)$ and instead of $E(\{a\},B)$ we will write $E(a,B)$. Let $e(A,B)=|E(A,B)|$, and we will sometimes write $e(a,B)$ as $\deg(a,B)$. For a subgraph $H\subseteq G$, $e(a,H)$ means $e(a,V(H))$.
%LD15
Let $\Delta(A,B):=\max\{e(a,B):a\in A\}$ and $\delta(A,B):=\min\{e(a,B):a\in A\}$. We denote the graph induced by $A$ as $G[A]$. Given a tree $T$, we write $xTy$ for the unique path in $T$ between vertices $x$ and $y$.  We will use the symbol $\oplus$ to denote modular addition, where the modulus will be clear in context.
%Let $H'\subseteq G$, and $v\in V(G)$. We let $G-H=G[V(G)\ssm V(H)]$, %$G-A=G-G[A]$, $H+H'=G[V(H)\cup V(H')]$, $G-v=G[V(G)\ssm \{v\}]$, %$H+v=G[V(H)\cup\{v\}]$.


\section{Auxiliary facts}

We begin with some facts that we will need throughout the paper.

\begin{lemma}
\label{AC-lem-1} Let $G$ be a connected balanced $U,V$-bigraph on $2n$ vertices. Then $G$ contains a path of order
$t=\min\{2(\delta_{U}+\delta_{V}),2n\}$.
\end{lemma}

\begin{proof}
Let $P$ be any maximal path with $\left\vert
P\right\vert <t$. It suffices to show that $G$ has a path $Q$ with $|Q|>|P|$.
Since $P$ is maximal, the neighborhoods of the ends of $P$ are contained in
$P$. We consider two cases depending on the parity of $P$.

\textbf{Case 1:} $P=x_{1}y_{1}\dots x_{l}y_{l}$ is an even path. Then
$e(x_{1},P)+e(y_{l},P)\geq\delta_{U}+\delta_{V}>l$.
Thus there exists an index $i\in\lbrack l]$ such that $x_{1}\sim y_{i}$ and
$y_{l}\sim x_{i}$. So $C=x_{1}y_{i}Py_{l}x_{i}Px_{1}$ is a cycle of length
$2l$. Since $t\leq2n$ and $G$ is connected, some vertex $z\in P$ has a
neighbor $r\in G-C$. Then $Q=rz(C-z)$ is a longer path.

\textbf{Case 2:} $P=x_{1}y_{1}\dots x_{l}y_{l}x_{l+1}$ is an odd path.
Without loss of generality, let $x_{1}\in U$. Set $P^{\prime}=P-x_{l+1}$ and consider
the components of $G^{\prime}=G-P^{\prime}$. The component containing
$x_{l+1}$ has order $1$ and thus more vertices from $U$ than $V$. Since
$G^{\prime}$ is balanced it also has a component $D$ with more vertices from
$V$ than $U$. Since $G$ is connected, there exists a vertex $r\in D$ that is
adjacent to a vertex $z\in\{x_j,y_j\}\subseteq V(P^{\prime})$. If possible, we choose $r\in V$ and with respect to this condition, choose $r$ so that $j$ is maximized. Let $w$ be the predecessor of $z$ on $P^{\prime}$. If $|D|=1$ then $e(r,P')+e(x_1,P')\geq \delta_U+\delta_V>l$, so there exists an index $i\in\lbrack l]$ such that $x_{1}\sim y_{i}$ and $r\sim x_{i}$.  Thus $Q=rx_iPx_1y_iPx_{l+1}$ is a path with $|Q|>|P|$. So we may assume that $|D|\geq 3$. Fix a depth first search tree $T$
of $D$ that is rooted at $r$. Let $b$ be the number of leaves of $T$ in $V$. Note that
\[
2|T\cap V|-b\leq |E(T)|=|T|-1=|D\cap U|+|D\cap V|-1
\]
which implies $b\geq|D\cap V|-|D\cap U|+1\geq 2$. Let $y$ be a leaf of $T$ in $V$ that is distinct from
$r$. Since $T$ is a depth first search tree, $N(y)\subseteq V(yTr\cup P')$. Let $m=|V(yTr)\cap U|$ and let $i$ be the largest index with $x_1 \sim y_i$. If $j> l-m$ then $Q=yTrzPx_1$ is a path with $|Q|= 2(j+m)\geq 2(l+1)>|P|$. So suppose $j\leq l-m$. If $i>l-m$ then $Q=yTrzPy_ix_1Pw$ is a path with $|Q|\geq 2(i+m)\geq 2(l+1)>|P|$.
%%Change
Otherwise $i\leq l-m$. By choice of $r$ we have $e(x_{1},Py_{l-m})+e(y,Px_{l-m})\geq\delta_{U}+\delta_{V}-m>l-m$.
So there exists an index $h\in\lbrack l-m]$ such that $x_{1}\sim
y_{h}$ and $y\sim x_{h}$. Thus $Q=rTyx_{h}Px_{1}y_{h}Px_{l+1}$ is a path with $|Q|>|P|$.
\end{proof}

\begin{lemma}
\label{ell}Let $G$ be a balanced $U,V$-bigraph on $2n$ vertices.

\begin{enumerate}
\item If $e_{s}$ and $e_{t}$ are independent edges and $\delta(G)\geq\frac
{3}{4}n+1$ then $G$ contains a spanning ladder, starting with $e_{s}$ and
ending with $e_{t}$.

\item If $\Lambda=\{L^{1},\dots,L^{s}\} $ is a set of disjoint ladders in $G$ such
that $\sum_{L\in \Lambda}|L| =2t$ and $\delta
(G)\geq\frac{3n+s+t}{4}+1$ then $G$ has a spanning ladder starting
with the first rung $e_{1}$ of $L^{1}$, ending with the last rung $e_{2}$ of $L^{s}$, and containing each $L\in \Lambda$.
\end{enumerate}
\end{lemma}

\begin{proof}
(i) Let $M$ be a $1$-factor of $G$ with $e_{s},e_{t}\in M$. Define an
auxiliary graph $H=(M,F)$ on $M$ as follows. If $uv,xy\in M$ with $u,x\in U$
then $uv\sim_{H}xy$ if and only if $u\sim_{G}y$ and $v\sim_{G}x$. There is a
natural one-to-one correspondence between ladders $u_{1}v_{1}\dots u_{h}v_{h}$
in $G$, whose rungs are in $M$, and paths in $H$. Also $\left\vert
H\right\vert =n$ and $\delta(H)\geq\frac{1}{2}n+1$. So $H$ is hamiltonian
connected and thus has a Hamilton path, starting with $e_{s}$ and ending with $e_{t}$.
This path corresponds to the required ladder in $G$.

(ii) Note that $\delta(G)$ is large enough to insure that $G$ has a $1$-factor $M$ containing all the rungs of the ladders $L^i$. Form $H$ as in (i). Then each
ladder $L^{i}$ corresponds to a path $P_{i}$ in $H$ and $\delta(H)\geq\frac{n+s+t}{2}+1$. Thus any two vertices of $H$ share $s$ non-path neighbors. For $i\in[s-1]$, connect the end $c_i$ of each $P_i$ to the start $b_{i+1}$ of each $P_{i+1}$ with a non-path vertex $x_i$ to form a path $P\subseteq H$ with $|P|=t+s-1$. Let $H'=H-(P-\{c_{s-1},x_{s-1}\})$. Then $\delta(H^{\prime})\geq\frac{1}{2}|H^{\prime}|+1$ and so $H'$ is hamiltonian connected. It
follows that $H'$ contains a Hamilton path $Q$ starting at $c_{s-1}$ and ending
at $x_{s-1}$. Then the Hamilton path $b_1Pc_{s-1}Qx_{s-1}Pc_s$ of $H$ corresponds to
the required ladder in $G$.
\end{proof}

Observe that in the proof of Lemma \ref{ell}(ii) we do not need the degrees of ``interior'' vertices of $L^i$ to be large. More precisely, given a ladder $L$ we define the partition $V(L)=\ext(L)\cup \mathring{L}$, where $\ext(L)$ is the set of \emph{exterior} vertices, and $\mathring{L}$ is the set of \emph{interior} vertices.  If $L$ is an \emph{initial} ladder, let $\ext(L)$ be the vertices in the last rung.  If $L$ is a \emph{terminal} ladder, let $\mathrm{ext}(L)$ be the vertices in the first rung.  If $L$ is not an initial or terminal ladder, let $\mathrm{ext}(L)$ be the vertices in the first and last rung of $L$.
%Finally, in each case let $\mathring{L}=V(L)\ssm\mathrm{ext}(L)$. 
Note that if $L\in\{L_1,L_2\}$, then it is possible for $\mathring{L}=\emptyset$. Set $I:=I(\Lambda)=\bigcup_{L\in \Lambda}\mathring{L}$. Then Lemma \ref{ell}(ii) still holds if we only require $\deg(v)\ge \frac{3n+s+t}{4}+1$ for $v\in V(G)\smallsetminus I$.
%LD15

\begin{lemma}\label{elll}
Let $G$ be a balanced $U,V$-bigraph on $2n$ vertices and let $\Lambda=\{L^{1},\dots,L^{s}\}$ be a set of disjoint ladders with initial ladder $L^{1}$ and if $s>1$, terminal ladder $L^s$
%LD15
such
that $\sum_{L\in \Lambda}|L|=2t$. Suppose $\deg(v)\ge d$ for all $v\notin I(\Lambda)$ and there exists $Q\subseteq U\cup V$ with $|Q|\le q $ such that $\deg(v)\ge D$ for every $v\notin Q\cup I(\Lambda)$. If 
$$\emph{(i)}~~ D\geq\frac{3n+3s+t+4q}{4}+1 ~~\text{ and }~~ \emph{(ii)}~~ d>t+3q+2s+n-D.$$
then $G$ has a spanning ladder that starts
with the first rung $e_{1}$ of $L^{1}$, contains each $L\in \Lambda$, and, if $s>1$, ends with the last rung $e_{2}$ of $L^{s}$.
\end{lemma}
\begin{proof}
Let $M$ be a matching that saturates $Q'=Q\smallsetminus I$ and avoids the ladders in $\Lambda$. This is possible since $q'=|Q'|\le d-t$ by (ii). We view each edge of $M$ as a $1$-ladder. Let $\Lambda^+=\Lambda\cup M$, $s'=s+q'$ and $t'=t+q'$. Next we extend each ladder $L\in \Lambda^+$ to a new ladder $\phi(L)$ as follows: let $\phi(L^1)=L^1y_1z_1$, $\phi(L^s)=a_sb_sL^s$, and $\phi(L^i)=a_ib_iL^iy_iz_i$ for $i\in [s']\smallsetminus\{1,s\}$ such that $a_h,b_h,y_h,z_h \notin R\cup R'$ for $h\in[s']$, where $R= \bigcup_{L\in \Lambda^+}V(L)$ and $R'$ is the set of all previously chosen extension vertices. For example, suppose we want to find $y_{s'}z_{s'}$ after finding all previous extensions. Let $uv$ be the rung of $L^{s'}$ that we wish to extend, where $u,v\in \mathrm{ext}(L^{s'})$. We have $|(R \cup R')\cap N(v)|<2s'+t'$, and so it is possible by (ii) to choose $y_{s'}\in N(v)\smallsetminus(R \cup R')$.  Note that $Q\cup I(\Lambda)\subseteq R$, and so $\deg(u)\ge D$. 
Now since $D\leq n$ we have $3s+t+4q+4\leq n$ and thus 
\begin{equation}
|(N(u)\cap N(y_{s'}))\smallsetminus (R\cup R')|\geq \half[n-(s+t+2q)]+2\geq 1. \label{intersect}
\end{equation}
So by (i) and (\ref{intersect}) we may choose $z_{s'}\in (N(u)\cap N(y_s'))\smallsetminus (R\cup R')$.

Set $\Lambda'=\{\phi(L):L\in \Lambda^+\}$ and $t''=t'+2s'-2$. Then $s'=|\Lambda'|$ and $2t''=\sum_{L'\in \Lambda'}|L'| $. By (i)
$$D \geq\frac{3n+3s+t+4q}{4}+1\geq \frac{3n+(s+q')+(t+q' +2(s+q'))} {4}+1 \geq\frac{3n+s'+t''}{4}+1.$$
Thus by Lemma~(\ref{ell}), $Q\subseteq R\subseteq I(\Lambda')$ and our observation preceding the Lemma, we are done.
\end{proof}


\section{Set-up and organization of the proof}

For the rest of this paper we let $G$ and $H$ be a balanced $U,V$-bigraphs on $2n$ vertices. Assume $\delta_{U}+\delta_{V}\geq n+2$ and suppose without loss of generality that $\delta_U\leq \delta_V$. Note that this implies $\delta_U\ge 3$. %%Change
Define $\g_1$ by $\delta_U=\g_1 n+1$ and $\g_2$ by $\g_1+\g_2=1$.  Assume  $\g_1<\frac{1}{2}< \g_2$, since the case where $\g_1=\g_2$ was handled in \cite{CK}. Also assume $\Delta(H)\le2 $ and $k=\mathrm{comp}(H)$. Our goal is to show that $G$ contains $H$.   

The rest of the proof is organized as follows. Our main task is to
prove Theorem \ref{main}. This proof divides into three main cases.
In Section 4 we handle the case that $\gamma_{1}<\frac{1}{200k}$.
In this case, we will show that $G$ contains $H$ for any value of
$n$, but will not prove the existence of a spanning ladder. Otherwise,
we consider two cases, the \emph{extremal} case and the \emph{random}
case. The case is determined by whether $G$ is $\alpha$-\emph{splittable}
for a sufficiently small $\alpha$. In Section 5 we define
$G$ to be $\alpha$-splittable if a certain configuration exists
in $G$. The definition is designed to be most useful in the non-extremal
case where $G$ fails to be $\alpha$-splittable. In the remainder
of Section 5 we show that if $G$ is $\alpha$-splittable and $\beta\ge 2\sqrt{\alpha}$ then $G$ has
a much nicer configuration called a $\beta$-\emph{partition}. In Section 6,  we handle the extremal case by showing that for sufficiently small $\beta$, we can obtain a spanning ladder from any $\beta$-partition. In Section 7 we introduce the Regularity
and Blow-up Lemmas. In Section 8 we use these lemmas to prove that in
the non-extremal case, if $n$ is sufficiently large in terms of $\alpha$, then $G$ contains
a spanning ladder. In Section 9 we use our previous results to complete
the proofs of Theorem \ref{mainconstant} and Theorem \ref{amarlarge}.


\section{Pre-extremal Case}

In this section, we will show that Theorem \ref{main} is true in
the case that one of the minimum degrees is very small.

\begin{lemma}\label{preext} If $\g_{1}<\frac{1}{200k}$ then $G$
contains $H$. \end{lemma}

\begin{proof} Let $S=\{u\in U:\deg(u)<\frac{9}{10}n\}$ and $s=\left\vert S\right\vert $.
Then $\gamma_{2}>1-\frac{1}{200k}$ and 
\begin{align}
\left(1-\frac{1}{200k}\right)n^{2}&\leq\sum_{v\in V}\deg(v)=\sum_{u\in U}\deg(u)<\frac{9}{10}ns +n(n-s)\notag\\
s &<\frac{1}{20k}n.\label{e-s-e-3}
\end{align}

 Since $\delta_{U}+\delta_{V}\geq n+2$, $G$ contains a Hamilton
cycle $D$. Suppose $D$ orders $S$ as $x_{1},\dots,x_{s}$,  where  $x_{1}$ is chosen so that $\operatorname*{dist}_{D}(x_{1},x_{s})>2$.
For each $i\in\lbrack s]$, let  $w_{i}x_{i}y_{i}\subseteq D$.
Since\[
|(N(w_{i})\cap N(y_{i}))\smallsetminus S|\geq\left(1-\frac{1}{100k}-\frac{1}{20k}\right)n>s,\]
we can choose distinct $z_{i}\in U$ such that $z_{i}$ is adjacent
to both $y_{i}$ and $w_{i\oplus1}$, if $y_{i}=w_{i\oplus1}$ then $z_{i}=x_{i\oplus1}$,
and otherwise $z_{i}\notin S$. Note that by the choice
of $x_{1}$ we have $y_{s}\neq w_{1}$ and thus $z_{s}\neq x_{1}$.
Set 
$C=w_1x_1y_1z_1\dots w_sx_sy_sz_sw_1$. 
Then $C$ is a cycle with length at most $4s<\frac{2n}{k}$.
Let $G'=G-(C-\{w_{1},z_{s}\})$.  Then $G'$ is
a balanced bipartite graph and  $G'\subseteq G-S$. Thus $$\delta(G')\geq\frac{9}{10}n-2s\stackrel{(\ref{e-s-e-3})}{\geq}\frac{3}{4}n+1\geq\frac{3}{4}\frac{|G'|}{2}+1.$$
So by Lemma \ref{ell}(1), $G'$ contains a spanning ladder $L$ with first rung
$w_{1}z_{s}$. Since $\mathrm{comp}(H)=k$, some component of $H$ must have size at least $\frac{2n}{k}$ and thus $H\subseteq C\cup L\subseteq G$.
\end{proof}

\section{Splitting}

In this section we define the notions of $\alpha$-splitting and $\beta$-partition. We prove that if $G$ has an $\alpha$-splitting then it has a $\beta$-partition.%%Change

\begin{definition} \label{Sp}
$G$ is $\alpha$-\emph{splittable} with $\alpha $-\emph{splitting} $(X,Y)$ if $X\subseteq U$ and $Y\subseteq V$ satisfy
\begin{enumerate}
\item $(\g_1-\alpha)n\leq |X|\leq(\g_{1}+\alpha)n$ and $(\g_2-\alpha)n\leq |Y|\leq(\gamma_{2}+\alpha)n$ and
\item $e(X,Y)\leq\alpha|X||Y|$
\end{enumerate}
\end{definition}
Informally, the following lemma asserts that if $G$ is
$\alpha$-splittable then $G$ can \emph{almost} be split into two
balanced complete bipartite graphs so that one has order approximately
$2\gamma_{1}n$ and the other has order approximately $2\gamma_{2}n$.
Let $(X,Y)$ be an $\alpha$-splitting of $G$ and set $\overline{X}=U\smallsetminus X$
and $\overline{Y}=V\smallsetminus Y$.

\begin{lemma} \label{split} If $G$ is $\alpha$-splittable for $\alpha\leq \left(\frac{\g_1}{4}\right)^2$, then there exist partitions
$U=X_{0}\cup X_{1}\cup X_{2}$ and $V=Y_{0}\cup Y_{1}\cup Y_{2}$
so that

\begin{enumerate}
\item $X_{1}\subseteq X,Y_{1}\subseteq\overline{Y},\left\vert X_{1}\right\vert =\left\vert Y_{1}\right\vert \geq(\gamma_{1}-2\sqrt{\alpha})n ~~\text{and}~~ \delta(G[X_{1}\cup Y_{1}])\geq(\gamma_{1}-4\sqrt{\alpha})n~~$and

\item $X_{2}\subseteq\overline{X},Y_{2}\subseteq Y,\left\vert X_{2}\right\vert =\left\vert Y_{2}\right\vert \geq(\gamma_{2}-2\sqrt{\alpha})n ~~\text{and}~~ \delta(G[X_{2}\cup Y_{2}])\geq(\gamma_{2}-4\sqrt{\alpha})n.$
\end{enumerate}
\end{lemma}

\begin{proof} We will show that there exist $X_{1}\subseteq X$ and
$Y_{1}\subseteq\overline{Y}$ satisfying (i) without
using $\gamma_{1}<\gamma_{2}$. Then by the symmetry of $\gamma_{1},X$
and $\gamma_{2},Y$ it will follow that there exists $Y_{2}\subseteq Y$
and $X_{2}\subseteq\overline{X}$ satisfying (ii). 

%We first observe,
%using splittability, that \[
%|X|\leq(\g_{1}+\alpha)n\text{ and }|Y|\leq(\g_{2}+\alpha)n.\]
%Indeed, if $|X|>(\g_{1}+\alpha)n$ then we have the following %contradiction of Definition \ref{Sp}(2):
%  \[
%e(X,Y)=\sum_{y\in %Y}e(y,X)>(\gamma_{2}n-(n-|X|))|Y|>(\gamma_{2}-1+\gamma_{1}+\alpha)%n|Y|>\alpha|X||Y|.\]
%Using symmetry, we also have $|Y|\leq(\g_{2}+\alpha)n$.
%By Definition \ref{Sp}(1) we conclude that \begin{equation}
%(\gamma_{1}-\alpha)n\leq\left\vert X\right\vert %\leq(\gamma_{1}+\alpha)n\text{ and %}(\gamma_{1}-\alpha)n\leq\left\vert \overline{Y}\right\vert %\leq(\gamma_{1}+\alpha)n.\label{cardX}\end{equation}

Let $S=\{x\in X:e(x,\overline{Y})<(\gamma_{1}-\sqrt{\alpha})n\}.$
Then 
\begin{align}
|S|\sqrt{\alpha}n&<\sum_{x\in X}e(x,Y)=e(X,Y)\leq\alpha|X||Y| \notag \\
|S|&\leq\sqrt{\alpha}|X|\frac{\left\vert Y\right\vert }{n}\leq\sqrt{\alpha}n.\label{S}
\end{align}

Let $\overline{T}=\{y\in\overline{Y}:e(y,X)<(\gamma_{1}-\sqrt{\alpha})n\}$. Then since $\sum_{x\in X} e(x,\overline{Y}) =e(X,\overline{Y})= \sum_{y\in\overline{Y}}e(y,X)$, we have
\[
\gamma_{1}n|X|-\alpha|X||Y| \le e(X,\overline{Y})\le (\gamma_{1}-\sqrt{\alpha})n |\overline{T}|+|X|(|\overline{Y}|-|\overline{T}|).
\]
Thus
\begin{align}
(|X|-(\gamma_1-\sqrt\alpha)n)|\overline T| &\le
(|\overline Y| -\gamma_1n+\alpha|Y|)|X| \notag \\
(\sqrt\alpha-\alpha)n|\overline T| &\le 
((\gamma_1+\alpha-\gamma_1)n+\alpha(\gamma_2+\alpha)n)(\gamma_1+\alpha)n \notag\\
(1-\sqrt\alpha)|\overline T| &\le
(1+\gamma_2+\alpha)(\gamma_1+\alpha)\sqrt\alpha n \notag\\
|\overline T|&\le \frac{3}{2}\sqrt\alpha n. \label{T}
\end{align}

Choose $X_{1}\subseteq X-S$ and $Y_{1}\subseteq\overline{Y}-\overline{T}$
such that $\left\vert X_{1}\right\vert =\left\vert Y_{1}\right\vert \geq(\g_1-2\sqrt{\alpha})$.
This is possible by Definition \ref{Sp}(i) and the upper bounds (\ref{S}) and (\ref{T}) on $\left\vert S\right\vert $
and $\left\vert \overline{T}\right\vert $.
Thus for every $x\in X_{1},y\in Y_{1}$ 
\begin{align*}
e(x,Y_{1})&\geq e(x,\overline{Y})-\left\vert \overline{T}\right\vert \geq((\gamma_{1}-\sqrt{\alpha})-2\sqrt{\alpha})n\geq(\gamma_{1}-4\sqrt{\alpha})n ~~\text{and}\\
%%%
e(y,X_{1})&\geq e(y,X)-\left\vert S\right\vert \geq((\gamma_{1}-\sqrt{\alpha})-2\sqrt{\alpha})n\geq(\gamma_{1}-4\sqrt{\alpha})n.
\end{align*}
\end{proof}

\begin{definition} \label{Part}
A $\beta$-\emph{partition} of $G$ is an ordered partition $(X_{1},S_{1}, S_{2}, X_{2},Y_{1}, T_{1}, T_{2}, Y_{2})$ with $U=U_1\cup U_2, U_1=X_{1}\cup S_{1}, U_2= S_{2}\cup X_{2}, V=V_1\cup V_2, V_1=Y_{1}\cup T_{1},V_2= T_{2}\cup Y_{2}$  such that for $g:=||S_{i}|-|T_{i}||$ and $h\in [2]$ the following conditions are satisfied
\begin{enumerate}
\item $(\g_h-\beta)n\leq|U_h|,|V_h|\leq(\g_h+\beta)n$;
%LD15
\item $|S_{1}|,|S_{2}|,|T_{1}|,|T_{2}|\leq2\beta n$;
%\item $(\gamma_{h}-3\beta)n+g\leq|X_{h}|=|Y_{h}|\leq(\gamma_{h}-\beta)n$;
\item $\delta(X_{h}, Y_{h}),\delta(Y_{h}, X_{h}) \geq(\gamma_{h}-4\beta)n+g$;
\item $\delta(S_{h},Y_{h}),\delta(T_{h},X_{h})\geq22\beta n+g$;
\item if $|S_{i}|>|T_{i}|$ 
then $\Delta(U_i,V_{j}),\Delta(V_{j},U_{i})<24\beta n$ for
$i\in [2]$ and $j=3-i$. 
\end{enumerate}
\end{definition}

\begin{lemma}\label{S2}
If $G$ is $\alpha$-splittable and $2\sqrt{\alpha}\leq\beta\leq\frac{\gamma_1}{268}$ then $G$ has a $\beta$ partition.
\end{lemma}

\begin{proof} (See Fig. \ref{gapfig}.) We start with the partition $U=X_{0}\cup X_{1}\cup X_{2}$
and $V=Y_{0}\cup Y_{1}\cup Y_{2}$ from Lemma \ref{split}. We describe a process for updating the partition so that conditions (i-v) are satisfied. 

Set
\begin{align*}
S_{1}&=\{x\in X_{0}:e(x,Y_{1})\geq 24\beta n\},~ S_{2}=X_{0}\smallsetminus S_{1},\\
T_{1}&=\{y\in Y_{0}:e(y,X_{1})\geq24\beta n\}, ~\text{and}~ T_{2}=Y_{0}\smallsetminus T_{1}.
\end{align*}
Clearly (i,ii) hold. Also (iii) holds with $2\beta n-g$ to spare. Since $50\beta\leq\gamma_{1}\leq\gamma_{2}$, we have $e(x,Y_{2}), e(y,X_{2})\geq24\beta n$ for all $x\in S_{2}$
and $y\in T_{2}$, and thus (iv) also holds with $2\beta n-g$ to spare. If (v) holds, we are done, so suppose not. Choose $i$ such that $|S_{i}|>|T_{i}|$ and set $j=3-i$, then $0<g_{0}:=|S_{i}|-|T_{i}|= |T_{j}|-|S_{j}| \le  2\beta n$.  We will now move vertices so that after each move, the difference $|S_i|-|T_i|$ is reduced while (i-iv) continue to hold.  Once the difference can no longer be reduced by moving vertices we will claim that (v) holds and then we set $g:=|S_i|-|T_i|\geq 0$. On each step we attempt to move vertices $x\in S_{i}$ with $e(x,Y_{j})\geq24\beta n$ from $S_{i}$ to $S_{j}$ and/or vertices $y\in T_{j}$ with $e(y,X_{i})\geq24\beta n$ from $T_{j}$ to $T_{i}$. If no vertices meet this requirement, then we will attempt to move vertices $x\in X_{i}$ with $e(x,Y_{j})\geq24\beta n$ from $X_{i}$ to $S_{j}$. Any time a move of this type is made the size of $X_{i}$ is reduced, so to ensure that $|X_h|=|Y_h|$ we must also move any vertex from $Y_{i}$ to $T_{i}$. Similarly, we may move eligible vertices from $Y_{j}$ to $T_{i}$ and compensate by moving any vertex from $X_{j}$ to $S_{j}$. After each move, any of $|X_{h}|,|Y_{h}|,\delta(X_{i}, Y_{i}),\delta(Y_{i}, X_{i}), \delta(S_{i},Y_{i}),\delta(T_{i},X_{i})$ may decrease, and $|S_{j}|$ and $|T_{i}|$ will increase. Note that these parameters may change by only $1$ per move. Since we will make at most $g_{0}-g$ moves, (iii,iv) will continue to hold. Furthermore, since $|S_i|,|T_j|$ will never be increased, $|U_i|, |V_j|$ may decrease by at most $g_0-g$ and $|U_j|, |V_i|$ may increase by at most $g_0-g$, so (i,ii) will continue to hold. When the the process stops, (v) will hold either because $|S_{i}|=|T_{i}|$ or because there are no more eligible vertices to move, in which case condition (v) is satisfied.
%LD15 (many small changes in proof)

\end{proof}

\section{Extremal case}

In this section we prove Theorem \ref{main} in the case that $G$ is $\alpha$-splittable for sufficiently small $\alpha$.

\begin{lemma}\label{ext}
Let $N_1(k)=408800k+1$. If $n\geq N_1(k)$, $\g_1\geq \frac{1}{200k}$, and $G$ is $\alpha$-splittable for $\alpha=\left(\frac{\g_1}{584}\right)^2$, then $G$ contains a spanning ladder.
\end{lemma}
%LD15 (We could use 1500 instead of 2000.  In fact we could go as low as 1207, but that isn't a very nice number.  Do you think it matters that 2000 is overkill?)

\begin{proof}
Set $\beta=2\sqrt{\alpha}=\frac{\g_1}{292}$, then by Lemma \ref{S2} $G$ has a $\beta$-partition $(X_{1},S_{1}, S_{2}, X_{2},Y_{1}, T_{1}, T_{2}, Y_{2})$.  Since $\g_1\geq \frac{1}{200k}$ we have
\begin{equation} 
\beta n=\frac{\g_1 n}{292} >7. \label{beta}
\end{equation}

Set $G_i=G[U_i \cup V_i]$ for $i \in [2] $. For $L\in\{L_2,L_3\}$ we say that $L$ is a \emph{crossing ladder} if its first rung is in $G_{1}$ and its last rung is in $G_{2}$. Choose $i$ so that $g=|S_i|-|T_i| \ge 0$ and set $j=3-i$.  Roughly, our plan is to find a crossing ladder $L^0$ and then find ladders $L'$, $L''$ spanning $G_1$, $G_2$ such that the last rung of $L'$ is the first rung of $L^0$ and the last rung of $L^0$ is the first rung of $L''$.  However $G_1$, $G_2$ may not be balanced or $G_1$, $G_2$ may have been balanced to begin with, but the crossing ladder created an imbalance.  In both of these situations we will need a way of moving vertices between $G_1$ and $G_2$ so that they may be incorporated into $L'$ and $L''$.

%Roughly, our plan is to find a crossing ladder $L^0$ and then find ladders $L'$, $L''$ spanning $G_1$, $G_2$ such that %the last rung of $L'$ is the first rung of $L^0$ and the last rung of $L^0$ is the first rung of $L''$. However, it is %more complicated than this, since $G_1$,$G_2$ may not be balanced, and even when $g=0$, if $L^0=L_3$ then we would need %to exclude the interior vertices of $L^0$ from $G_1$ and $G_2$, and thus create an imbalance.

Formally, our plan is to construct a set of pairwise disjoint ladders $\Lambda=\{L^0, \dots, L^{s}\}$ with $s\le g+1\le2\beta n+1$ and $I=I(\Lambda)=\bigcup_{L\in\Lambda}\mathring{L}$ such that 
\begin{enumerate}
\item[(a)] $L^0$ is a crossing ladder,

\item[(b)] for all $p\in [s]$, there exists $h\in [2]$ with $ \mathrm{ext}( L^p)\subseteq G_h$ and 

\item[(c)] $G_1 -I$  is balanced (equivalently, $G_2-I$ is balanced). 
\end{enumerate}
%LD15 (I wanted these conditions to be more visible so I enumerated them.  You may have a different idea on how we should write them)

We may also designate one ladder as an initial ladder for each $G_h$. Then we will apply Lemma \ref{elll} to construct a spanning ladder.

We begin with two useful facts. By our degree conditions we have

\begin{equation}
\forall v,v'\in V~~|N(v)\cap N(v')|\ge 2\delta_V-n>2(n/2+1)-n=2 \label{VV}
\end{equation}

Since $\sum_{u\in U}{\deg(u)}= e(U,V)\ge \delta_V|U|$ and $\delta_U<\delta_V$, there exists $u^*\in U$ with $\deg(u^*)>\delta_V$. Thus 
\begin{equation}
\exists u^*\in U~\forall u\in U~~|N(u^*)\cap N(u)|\ge \delta_V+1+\delta_U-n\ge 3. \label{UU}
\end{equation}

\noindent\textbf{Step 1:} (Construct a crossing ladder $L^0$.) We are done unless 
\begin{equation} \text{ \emph{there is no crossing} } L_2. \tag{\aster}\end{equation}
So suppose not, then by (\ref{UU}) there exist vertices $x_1\in U_1$, $x_2\in U_2$ such that $|N(x_1)\cap N(x_2)|\geq 3$ and

\begin{equation} (N(x_1)\cap N(x_2)\subseteq V_1) \lor (N(x_1)\cap N(x_2)\subseteq V_{2}). \label{vv} \tag{\aster1} %%Change subscripts from i,j to 1,2
\end{equation}

Let $y_1,y_2\in N(x_1)\cap N(x_2)$, by (\ref{vv}) there exists $q\in[2]$ such that $\{y_1,y_2\} \subseteq V_q$. Let $q'=3-q$ and $y_3\in N(x_{q'})\cap V_{q'}$. By (\ref{VV}), $y_2$ and $y_3$ have a common neighbor $x_3\ne x_q,x_{q'}$. By (\aster), $x_3\in U_{q'}$. Thus $L^0=x_qy_1x_{q'}y_2x_3y_3$ is a crossing $L_3$. (See Fig. \ref{c1})

%%%%%%%%%%%%%%%%%%

\noindent\textbf{Step 2:} (Construct $L^1,\dots,L^s$ so that (b) and (c) hold.)
For all $u \in U_i$ and $v\in V_{j}$
\[
n+2\le \deg(u)+\deg(v)\le |V_i|+e(u, V_{j})+|U_{j}|+e(v,U_i)\le  n-g+e(u,V_{j})+e(v,U_i).
\]
Therefore 
\begin{equation}
g+2\le \delta(U_i,V_j)+\delta(V_j,U_i). \label{WZ}
\end{equation}

\noindent\textbf{Case 1:} $g=0$. If $G$ has a crossing $L_2$, i.e., (\aster) fails, then there is nothing to do. 
Otherwise, $L^0=L_3$ and $y_2\in \mathring{L^0}\cap V_q$ thus $|U_q\ssm\mathring{L^0}|=|V_q\ssm\mathring{L^0}|+1$. Let $x'\in N(y_2)\cap (U_q-x_q)$ and $y'\in N(x_{q'})\cap (V_{q'}-y_3)$.  Since $g=0$, $i$ and $j$ are interchangeable, so by (\ref{WZ}), either $x'$ has a neighbor in $V_{q'}$ or $y'$ has a neighbor in $U_q$ and by (\aster), neither of these possible neighbors can be in $L^0$. Regardless, there exists an edge $xy\in E(U_q,V_{q'})$ whose ends are not in $L^0$. Let $y^*\in N(x)\cap (V_q\ssm V(L^0))$. By (\ref{VV}), $y$ and $y^*$ have a common neighbor $x^*$  with $x^*\ne x,x_h$. By (\aster), $x^*\in U_q$. Set $L^1=xyx^*y^*$ and specify $L^1$ as the initial ladder for $G_q$. Note that $\mathrm{ext}(L^1)\subseteq G_q$ and $|U_q\ssm (\mathring{L^0}\cup\mathring{L^1})|=|V_q\ssm (\mathring{L^0}\cup\mathring{L^1})|$ so we are done.
%LD15 (made conditions (b) and (c) explicit)

%$x\in U_q$.


%LD15 (I slightly changed the figure so that is more clearly balanced)

\noindent\textbf{Case 2:}  $g\ge1$. 
Using Definition \ref{Part}(i,v) and $g\ge1$ we have  
\begin{align}
\forall v,v^*\in V_{j}~~|(N(v)\cap N(v^*))\cap U_{j}|&\geq 2(\g_2-24\beta)n-|U_j|\ge |U_j|-50\beta n >
\frac{4}{5}|U_{j}|.
\label{com1}
\end{align}

If $U_i=U_1$ we have
\begin{align}
\forall u,u^*\in U_1~~
|(N(u)\cap N(u^*))\cap V_1|&\geq 2(\g_1-24\beta)n-|V_1|\ge |V_1|-50\beta n >
\frac{4}{5}|V_{1}|.
\label{com2}
\end{align}

If $U_i=U_2$ then for all $v\in V_1$, $(\g_1+\beta)n\geq \deg(v,U_1)\geq(\g_2-24\beta)n$ which implies $\g_2>\g_1\geq\g_2-25\beta$.  In which case we have
\begin{align}
\forall u,u^*\in U_2~~
|(N(u)\cap N(u^*))\cap V_2|&\geq 2(\g_1-24\beta)n-|V_2| \geq 2(\g_2-49\beta)n-|V_2|\notag\\
&\geq |V_2|-100\beta n
>\frac{13}{20}|V_2|.
\label{com3}
\end{align}%%Change

Let $m=\max\{\delta(U_i,V_j),\delta(V_j,U_i)\}$ and note that by (\ref{WZ}) and $g\geq 1$, we have $m\geq 2$.  Also note that by (\ref{WZ}), if $g\geq 3$ then $m\geq 3$.  It is the case that if $L^0=L_3$ then $m\geq 3$: if $\delta(V_j,U_i)>0$, then by (\ref{VV},\aster), we have $\delta(V_j,U_i)\geq 3$ otherwise $\delta(V_j,U_i)=0$ and thus $\delta(U_i,V_j)\geq 3$ by (\ref{WZ}).
%LD15

%then every vertex in $V_j$ shares a neighbor in $U_i$ with a %vertex in $V_i$, thus by (\ref{VV},\aster) it shares three, and so %$\delta(V_j,U_i)\ge 3$

%If $g\ge 3$ then $m=\max{\{\delta(U_i,V_j),\delta(V_j,U_i)\}}\ge %3$, by (\ref{WZ}).
%Suppose (\aster). If $\delta(V_j,U_i)>0$ then every vertex in %$V_j$ shares a neighbor in $U_i$ with a vertex in $V_i$. Thus by %(\ref{VV},\aster) it shares three, and so $\delta(V_j,U_i)\ge 3$. %Otherwise by (\ref{WZ}), $\delta(U_i,V_j)\ge g+2$, and again $m\ge %3$.

\noindent\textbf{Case 2a:} $m=2$. Then $L^0=L_2$, $1\le g\le2$ and $1\le \delta(A,B)\le\delta(B,A)=2$ for some choice of $\{A,B\}=\{U_i,V_j\}$. Let $A\cup A',B\cup B'\in\{U,V\}$. By Definition~\ref{Part}(v) and $g>0$ there exists $b_1\in B\ssm V(L^0)$ with no neighbor in $V(L^0)\cap A$ and two neighbors $a_1,a_2\in A$. By (\ref{com1},\ref{com2},\ref{com3}), $a_1$ and $a_2$ have a common neighbor $b_2\in B'\ssm V(L^0)$. Let $L^1=a_1b_1a_2b_2$ be the initial ladder for $G_{h}$, where $b_2\in G_h$ and $\mathrm{ext}(L^1)\subseteq G_h$. If $g=1$ then $|U_i\ssm (\mathring{L^0}\cup \mathring{L^1})|=|V_i\ssm (\mathring{L^0}\cup \mathring{L^1})|$ and we are done. If $g=2$ then also $\delta(A,B)=2$ by (\ref{WZ}), and a similar argument yields an initial ladder $L^{2}=a_3b_3a_4b_4$ for $G_{h-3}$ such that $a_3\in A,b_3,b_4\in B,a_4\in A'$ and $L^0,L^1,L^2$ are disjoint.  We have $\mathrm{ext}(L^2)\subseteq G_{h-3}$ and $|U_i\ssm (\mathring{L^0}\cup \mathring{L^1}\cup \mathring{L^2})|=|V_i\ssm (\mathring{L^0}\cup \mathring{L^1}\cup \mathring{L^2})|$ so we are done.
%LD15 (made conditions (b) and (c) explicit)

\noindent\textbf{Case 2b:} $m\ge 3$. By (\ref{WZ}) there exists $A\in\{U_i,V_j\}=\{A,B\}$ such that $e(a,B)\ge m\ge 3$ for all $a\in A$.  Let $M=\{a_rb_rc_rd_r:r\in [s]\}$ be a maximal set of disjoint claws with root $a_r\in A$ and leaves $b_r,c_r,d_r\in B$. Then every vertex in $\overline A=A\ssm \{a_r:r\in[s]\}$ has at least $m-2$ neighbors in $N=\{b_r,c_r,d_r:r\in [s]\}$. Suppose $s\le g$. Then using Definition \ref{Part}(i,v), $g\le2\beta n$ and $g\leq 2m-2$ (from (\ref{WZ})), we note
\[
(m-2)((\g_1-\beta)n-s)\le |E(\overline{A},N)|\le 3s\cdot 24\beta n.
\]
Thus 
\begin{equation}
\g_1\le 72\beta\frac{g}{m-2}+\beta+\frac{s}{n}\leq 72\beta\frac{2m-2}{m-2}+3\beta\le291\beta<\gamma_1, \label{*}
\end{equation}
a contradiction.  So we conclude that $s\ge g+1$. Choose $B'$ so that $\{B,B'\}=\{U_l,V_l\}$ for some $l\in[2]$.  Let $g':=|B\ssm \mathring{L^0}| -|B'\ssm \mathring{L^0}|$ and note that $g-1\le g'\le g+1$.  
%LD15
In order to balance $G_l-\mathring{L^0}$ we build a set of disjoint $3$-ladders $$\Lambda(M)=\{x_rb_ra_rc_ry_rd_r:r\in[g'], a_rb_rc_rd_r\in M \text{ and } x_r,y_r\in B'\}.$$  
%LD15
This is possible by $s\geq g+1$, (\ref{com1},\ref{com2},\ref{com3}) and 
$$|(N(b_r)\cap N(c_r))\cap (B'\ssm \mathring{L^0})|, |(N(c_r)\cap N(d_r))\cap (B'\ssm \mathring{L^0})|\ge \frac{13}{20}|B'|-2\ge 2g'.$$ 
Thus $|U_l\ssm (\mathring{L^0}\cup I(\Lambda(M)))|=|V_l\ssm (\mathring{L^0}\cup I(\Lambda(M)))|$ and $\mathrm{ext}(L)\subseteq G_l$ for all $L\in \Lambda(M)$ so we are done.
%LD15 (made conditions (b) and (c) explicit)

\noindent\textbf{Step 3:} (Construct the spanning ladder.) Let $\Lambda$ be the set of ladders constructed in Steps 1 and 2 and set $I:=I(\Lambda)$. Let $\Lambda_{h}=\{L\in \Lambda: \mathrm{ext}(L)\subseteq G_{h}\}$ and $G'_{h}=(G_{h}-I)\cup\bigcup \Lambda_{h}$ for $h\in[2]$. Note that $G'_1$, $G'_2$ are balanced and $G'_1\cup G'_2=G-\mathring{L^0}$.
%LD15
For each ladder $L\in \Lambda_{h}$ there is a unique vertex $v'\in \mathring{L}\cap V(G_{3-h})$. Since $v'\in\mathring{L}$, we are unconcerned about its degree in $G'_{h}$ so we add this vertex to the appropriate exceptional set ($S_h$ or $T_h$) in $G'_h$.

Let $e_1$ and $e_2$ be the first and last rungs of $L^{0}$, which we will specify as the terminal ladders in $G'_1$ and $G'_2$ respectively. It will suffice to show using Lemma~\ref{elll} that each $G'_{h}$ has a spanning ladder, starting at its initial ladder, if it is specified in Case 1 or Case 2a, and ending at its terminal ladder. Let $s':=|\Lambda_{h}|\le g+1$ and $t':=\half|\bigcup \Lambda_{h}|\le 3(g+1)$.  Recall that $g=|S_i|-|T_i|$. Since we only add vertices to $S_{j}$ and $T_{i}$ and $\mathring{L^0}\cap V(G'_h)=\emptyset$, we have $n':=\frac{1}{2}|G'_{h}|\le(\gamma_{h}+\beta)n$. Let $Q:=\{v\in V(G'_{h}):\deg(v)<D\}$, where $D:=(\g_{h}-4\beta)n-1$. By Definition \ref{Part}(iii), $Q\subseteq S_{h}\cup T_{h}$. Thus, by Definition \ref{Part}(ii), $q':=|Q|\le4\beta n-g$. By Definition \ref{Part}(iii,iv), if $v\in V(G'_{h})\smallsetminus I$ then $ d:=22\beta n-1\le 22\beta n+g-s'\le \deg(v,G'_{h})$. Thus $G'_{h}$ has the desired spanning ladder by Lemma \ref{elll}, since
%LD15 (many small changes in Step 3)
\begin{align*}
\frac{3n'+3s' +t' +4q'}{4}+1 \le \frac{3\g_{h}n+23\beta n+10}{4}\le D
\end{align*}
and 
\begin{align*}
t'+3q'+2s'+n'-D\le 21\beta n+6\stackrel{(\ref{beta})}{<} d.
\end{align*}
%LD15

\end{proof}
%%%%%%%%%%%%%%%%%%%%%%%%% End of the extremal case %%%%%%%%%%%%%%%%%%%%%%%
% 
% \section{The Regularity and Blow-up Lemmas}
% In this section we review the Regularity and Blow-up Lemmas. Let $\varGamma$ be a simple graph on $n$ vertices. For two disjoint, nonempty subsets $U$
% and $V$ of $V(\varGamma)$, define the density of the pair $(U,V)$ as
% \[
% d(U,V)=\frac{e(U,V)}{|U||V|}.
% \]
% 
% \begin{definition}
% A pair $(U,V)$ is called $\ep $-\emph{regular} if for every $%
% U^{\prime }\subseteq U$ with $|U^{\prime }|\geq \ep |U|$ and every $%
% V^{\prime }\subseteq V$ with $|V^{\prime }|\geq \ep |V|$, $
% |d(U^{\prime },V^{\prime })-d(U,V)|\leq \ep $. The pair $\left(
% U,V\right) $ is $\left( \ep ,\delta \right) $-\emph{super-regular}
% if it is $\ep $-regular and for all $u\in U$, 
% $\deg\left(
% u,V\right) \geq \delta \left| V\right| $ and for all $v\in V$, $\deg\left(
% v,U\right) \geq \delta \left| U\right| $.
% \end{definition}
% 
% First we note the following facts that we will need.
% 
% \begin{lemma}
% \label{deg} If $(U,V)$ is an $\ep $-regular pair with density $\delta$,
% then for any $Y\subseteq V$ with $\left| Y\right| \geq \ep \left|
% V\right| $ there are less than
% $\ep |U|$ vertices $u\in U$ such that $\deg(u,Y)<(\delta-\ep )|Y|$.
% \end{lemma}
% 
% \begin{proposition}
% \label{C4} If $(U,V)$ is a balanced $\ep$-regular pair with density $\delta\ge 2\sqrt\ep>0$ and subsets $A,C\subseteq U$, $B,D\subseteq V$ of size at least $\frac{1}{2}\delta|U|$ then there exist $a\in A,b\in B,c\in C,d\in D$ with  $abcda=C_4$.
% \end{proposition}
% %LD15 
% 
% \begin{lemma}[Slicing Lemma]
% \label{slice} Let $(U,V)$ be an $\ep $-regular pair with density $\delta$, and for some $\lambda >\ep $ let $U^{\prime }\subseteq U$, $V^{\prime }\subseteq V$, with $|U^{\prime }|\geq \lambda |U|$, $|V^{\prime }|
% \geq \lambda |V|$. Then $(U^{\prime },V^{\prime })$ is an $\ep
% ^{\prime }$-regular pair of density $\delta^{\prime }$ where $\ep
% ^{\prime }=\max \{\frac{\ep }{\lambda },2\ep \}$ and $
% \delta^{\prime}\ge \delta-\ep $.
% \end{lemma}
% 
% \begin{lemma}[Augmenting Lemma]
% \label{add}Let $\left( U,V\right) $ be an $\ep $-regular pair.
% Suppose that $U^{\prime }=U\cup S$ and $V^{\prime }=V\cup T$, where $\left|
% S\right| \leq \mu \left| U\right| $, $\left| T\right| \leq \mu \left|
% V\right| $, $S\cap V^{\prime }=\emptyset =T\cap U^{\prime }$, and $0<\mu
% <\ep $. Then $\left( U^{\prime },V^{\prime }\right) $ is an $%
% \ep ^{\prime }$-regular pair, where $\ep ^{\prime }=\max
% \left\{ \frac{\mu }{\ep },6\ep \right\} $.
% \end{lemma}
% 
% \begin{definition}
% A partition $\{V_{0},V_{1},\dots, V_{t}\}$ of $V(\varGamma)$ is called $\ep$-regular if the following conditions are satisfied:
% 
% \begin{enumerate}
% \item  $|V_{0}|\leq \ep |V|$.
% 
% \item  For all $i,j\in[t]$, $|V_{i}|=|V_{j}|$.
% 
% \item  All but at most $\ep t^{2}$ of pairs $(V_{i},V_{j})$, $1\leq i,j\leq t$, are $\ep$-regular.
% \end{enumerate}
% \end{definition}
% 
% The parts of the partition are called \emph{clusters}. Note that the cluster
% $V_{0}$ plays a distinguished role in the above definitions and is usually
% called the exceptional cluster (or class). Our main tool in the proof will
% be the Regularity Lemma of Szemer\'{e}di \cite{Sz} which asserts that for
% every $\ep >0$ every graph which is large enough admits an $\ep$-regular partition into a bounded number of clusters.
% 
% \begin{lemma}[Regularity Lemma]  %%Should we change?Change
% \label{regularity} For every $\ep >0$ 
% %and every positive integer $m$ 
% there exists $N:=N(\ep,m)$ and $M:=M(\ep,m)$ such that every graph on at least $N$ vertices admits an $\ep $-regular partition $\{V_{0},V_{1},\dots, V_{t}\}$ with $m\leq t\leq M$.
% \end{lemma}
% %LD15 (remove m?)
% 
% In the next section we will want a regular partition of a bipartite graph so we will use the following formulation (see for example \cite{CK}).
% 
% %Let $\{U_{0},U_{1},\dots,U_{t_{1}}\}\cup\{V_{0},V_{1},\dots,V_{t_{2}}\}$
% %be an $e_{0}$-regular partition of $G$, with $m\le t_{1},t_{2}\le M(\ep,m)$,
% %starting from the initial partition $\{U,V\}$ of $G$. As in \cite{CK}
% %we {}``clean up'' this partition to obtain an $\ep_{1}$-regular
% %partition $\{U_{0},U_{1},\dots,U_{t}\}\cup\{V_{0},V_{1},\dots,V_{t}\}$
% 
% \begin{corollary}[Regularity Lemma - Bipartite Case]
% \label{bireg} For every $\ep>0$ there exists $N:=N(\ep)$ and $M:=M(\ep)$ such that every balanced $U,V$-bigraph on at least $2N$ vertices admits an $\ep$-regular partition $\{U_0,U_1,\dots,U_t\}\cup\{V_0,V_1,\dots,V_t\}$ with $t\leq M$ satisfying
% 
% \begin{enumerate}
% \item $|U_{0}|=|V_{0}|\leq\ep n$,
% \item for all $i,j\in[t]$, $(1-\ep)\frac{n}{t}\leq|U_{i}|=|V_{j}|\leq\frac{n}{t}$ and
% \item for all $U_i\in\{U_1,\dots,U_t\}$ there are at most $\ep t$ sets $V_j\in\{V_1,\dots,V_t\}$ such that $(U_i,V_j)$ is not $\ep$-regular and for all $V_i\in\{V_1,\dots,V_t\}$ there are at most $\ep t$ sets $U_j\in\{U_1,\dots,U_t\}$ such that $(V_i,U_j)$ is not $\ep$-regular.
% %\item for any $U_i\in\{U_1,\dots,U_t\}$ $(V_i\in\{V_1,\dots,V_t\})$ there are at most $\ep t$ sets $V_j\in\{V_1,\dots,V_t\}$ $(U_j\in \{U_1,\dots,U_t\})$ such that $(U_i,V_j)$ $((V_i,U_j))$ is not $\ep$-regular.
% \end{enumerate}
% \end{corollary}
% 
% %LD15
% 
% In addition, we shall use the following version of the Blow-up Lemma \cite{KSSz}.
% 
% \begin{lemma}[Blow-up Lemma]
% \label{blowup}
% 
% Given $\delta >0$, $\Delta >0$ and $\varrho>0$ there exist $\ep >0$ and $\eta>0$ such that the following holds. Let $S=(W_1,W_2)$ be an $(\ep ,\delta )$-super-regular pair with $|W_1|=n_{1}$ and $|W_2|=n_{2}$. If $T$ is a $A_1,A_2$-bigraph with maximum degree $\Delta (T)<\Delta $ and $T$ is embeddable into the complete bipartite graph $K_{n_{1},n_{2}}$ then it is also embeddable into $S$.  Moreover, for all $\eta n_{i}$-subsets $A_{i}'\subseteq A_{i}$ and functions $f_{i}:A_{i}'\rightarrow \binom{W_{i}}{\varrho n_{i}}$, $i=1,2$, $T$ can be embedded into $S$ so that the image of each $a_{i}\in A_{i}'$ is in the set $f_{i}\left( a_{i}\right)$.
% \end{lemma}


\section{Non-extremal case}

In this section, we will show that if the graph is not $\alpha$-splittable
for sufficiently small $\alpha$ then it contains a spanning ladder.
The proof uses the Regularity-Blow-up method (see Chapter \ref{regularitychapter}).

\begin{lemma}\label{nonext} Let $k$ be a positive integer and suppose
$\g_{1}\geq\frac{1}{200k}$. There exists $N_{2}(k)\in \mathbb{N}$ so that if $G$
is not $\alpha$-splittable for $\alpha=\left(\frac{\g_{1}}{584}\right)^{2}$,
and $n\geq N_{2}(k)$ then $G$ contains a spanning ladder. 
\end{lemma}

\begin{proof} Let $0<d_{0}\leq\frac{\alpha\g_{1}\g_{2}}{8}$,
$\delta_{1}\leq\frac{1}{3072}d_{0}^{2}$, $\delta_{2}\leq\frac{1}{2}\delta_{1}$,
$\delta_{3}\leq\frac{1}{2}\delta_{2}$, $\delta_{4}\leq\half\delta_{3}$,
$\delta\leq\frac{1}{4}\delta_{4}$, $\Delta=4$ and $\varrho=\half\delta$.
For these choices of $\delta$, $\Delta$ and $\varrho$ choose $\ep<\delta^{3}$
and $\eta$ to satisfy the conclusion of Lemma \ref{blowup}. Now let
$\ep_{5}\leq\left(\frac{\ep}{6}\right)^{4}$, $\ep_{4}\leq\frac{1}{4}\ep_{5}$,
$\ep_{3}\leq\frac{1}{2}\ep_{4}$, $\ep_{2}\leq\half\ep_{3}$, and $\ep_{1}\leq\half\ep_{2}$. So \begin{equation}
0<\ep_{1}<\ep_{2}<\ep_{3}<\ep_{4}<\ep_{5}\ll\ep\ll\delta<\delta_{4}<\delta_{3}<\delta_{2}<\delta_{1}\ll d_{0}\ll\alpha.\notag\end{equation}
%and $e_{0}\leq\frac{1}{9}\ep_{1}^{2}$. 
%Let $??m\geq\frac{2}{e_{0}}??$ and 
%LD15 (if we remove this m we should change the statement of the regularity lemma)
%%%Changed \ep < \frac{1}{8}\delta^{3}
Let 
$N_2=\frac{4M(\ep_1)}{\eta}$
%$N_{2}(k)=\max\{N(\ep_1),\frac{4M(\ep_1)}{\eta}\}$, 
where
$M(\ep_1)$ 
%and $N(\ep_1)$ 
is the value obtained from Lemma \ref{bireg}. Apply Lemma \ref{bireg} to $G$ with $\ep_1$ and $\delta_1$ to obtain a partition
$\{U_0,U_{1},\dots,U_{t}\}\cup\{V_0,V_{1},\dots,V_{t}\}$ and a subgraph $G'$ satisfying (i-v). For all $i,j\in [t]$, let $\ell:=|U_i|=|V_j|$ and note that \begin{equation}(1-\ep_1)\frac{n}{t}\leq\ell\leq\frac{n}{t}.\notag\end{equation}

Consider the \emph{cluster graph} $\mathcal{G}$ with $V(\mathcal{G})=\{U_{1},\dots,U_{t}\}\cup\{V_{1},\dots,V_{t}\}$
and two clusters $W,W'$ joined by an edge when the pair $(W,W')$
is $\ep_{1}$-regular and $d(W,W')\geq\delta_{1}$. Then $\mathcal{G}$
is a bipartite graph with bipartition $\{\mathcal{U},\mathcal{V}\}$,
where $\mathcal{U}=\{U_{1},\dots,U_{t}\}$ and $\mathcal{V}=\{V_{1},\dots,V_{t}\}$.  

\begin{claim} \label{cldeg}
$\delta_{\mathcal{U}}\geq(\g_{1}-\delta_{1}-2\ep_{1})t$ and $\delta_{\mathcal{V}}\geq(\g_{2}-\delta_{1}-2\ep_{1})t$.
\end{claim}

\begin{proof}
Suppose there exists $Z\in V(\mathcal{G})$ with
$\deg_{\mathcal{G}}(Z)<(\g_{i}-\delta_{1}-2\ep_{1})t$, where $i=1$
if $Z\in\mathcal{U}$ and $i=2$ if $Z\in\mathcal{V}$. Then 
$$\gamma_in\ell \leq e_G(Z) < (\gamma_i-\delta_1-2\ep_1)t\ell^2 +\ep_1n\ell\leq (\gamma_i-\delta_1-\ep_1)n\ell$$
and thus some vertex $z\in Z$ has $$\deg_{G'}(z)< \gamma_in-(\delta_1+\ep_1)n\leq  \deg_G(z)-(\delta_1+\ep_1)n,$$ contradicting property (iv) of Lemma \ref{bireg}. 
\end{proof}


% 
% For a cluster $Z$, let  $I(Z)=\{W:(Z,W) \mathrm{~is~irregular}\}$ and $\overline{N}(Z)=\{W:d(Z,W)<\delta_1\}$.
% %LD15

\begin{claim}\label{clpath}
$\mathcal{G}$ contains a path $\mathcal{P}$ on $2q$ vertices with $q \geq (1-2\delta_1-4\ep_1)t$. 
\end{claim}

\begin{proof}
If $\mathcal{G}$ is connected, then the claim follows immediately from Claim \ref{cldeg} and Lemma \ref{AC-lem-1}.  So suppose that $\mathcal{G}$ is disconnected, we will obtain a contradiction by showing that this implies that $\mathcal{G}$ is $\alpha$-splittable. Let $\mathcal{A}$ and $\mathcal{B}$ be distinct components of $\mathcal{G}$ and let $X=U\cap\bigcup\mathcal{A}$ and $Y=V\cap\bigcup\mathcal{B}$. Using $e_{\mathcal{G}}(X,Y)=0$, we have 
\[
e_{G}(X,Y)\leq\delta_{1}|X||Y|+\ep_{1}t\ell|X|
\le \delta_{1}|X||Y|+\ep_{1}3|Y||X|
%\leq(\delta_{1}+3\ep_{1})|X||Y|
\le \alpha (\g_1-\alpha)(\g_2-\alpha).
\]
Thus Definition \ref{Sp}(ii) holds. By Claim \ref{cldeg} we have
\begin{align*}
|X|&\geq(\g_{2} -\delta_{1}-2\ep_{1})t\ell\geq(\g_{2} -\delta_{1}-2\ep_{1})(1-\ep_{1})n\geq(\g_{2} -\delta_{1}-3\ep_{1})n\ge(\g_{2}-\alpha)n\\
&\text{and}\\
|Y|&\geq(\g_{1} -\delta_{1}-2\ep_{1})t\ell\geq(\g_{1} -\delta_{1}-2\ep_{1})(1-\ep_{1})n\geq(\g_{1} -\delta_{1}-3\ep_{1})n\ge(\g_{1}-\alpha)n.
\end{align*}
Thus Definition \ref{Sp}(i) holds for some $X'\subseteq X$, $Y'\subseteq Y$ and $(X',Y')$ is an $\alpha$-splitting of $\mathcal{G}$.
\end{proof}

Choose the notation so that $\mathcal{P}$ $=U_{1}V_{1} \dots,U_{q}V_{q}$. 
Add all clusters which are not in $\mathcal{P}$ to the exceptional class $U_0 \cup V_0$. 
%In addition, we add $2\ep |U_i|$ vertices from each $U_i$ and each $V_i$ in $\mathcal{P}$ (vertices with ``small'' %degree as determined by Lemma \ref{deg} to clusters which they are adjacent in $\mathcal{P}$) so that the clusters %$\bar{U}_i, \bar{V}_i$ after this operation
%have the same size $m$ which is at least $(1-3\ep)\frac{n}{k}$ and for 
%$i=1, \dots, k$, the pairs $(\bar{U}_i, \bar{V}_i)$ are 
%$(\delta_1,\ep_1)$-super-regular with $\delta_1 = \delta - 3\ep$ and $\ep_1=2\ep$ (by the Slicing Lemma). 
%In addition, for $i=2, \dots, k$, all pairs $(\bar{U}_i, \bar{V}_{i-1})$ are 
%$(\delta_1,\ep_1)$-super-regular.
As $\delta_1 \gg \ep_1$, the exceptional class may now be much larger: 
\begin{equation}
|U_0|=|V_0| \le 3\delta_1 n.
\notag
\end{equation}

Our next task is to reassign the vertices from the exceptional class
%LD15
%Change the procedure
to $\mathcal P$. Since we will need to do this twice, we state the procedure in general terms. Let $\{X_0,X_1,\dots,X_q\}\cup \{Y_0,Y_1,\dots, Y_q\}$ be the current partition, where $\bigcup_{i=0}^q X_i=U$ and $\bigcup_{i=0}^q Y_i=V$.  Suppose that $(X_i,Y_i)$ and $(X_{i+1},Y_i)$ are $\ep'$-regular pairs of density at least $\delta'$.  Recall that $(1-\ep_1)\frac{n}{t}\leq \ell\leq \frac{n}{t}$ was the common size of the non-exceptional clusters in the initial $\ep_1$-regular partition.  The procedure takes two parameters $\sigma$ and $\tau$ where $\sigma^2n$ is an upper bound on the size of the exceptional sets and $2\tau\ell$ is a minimum degree condition which a vertex must meet in order to be reassigned to a cluster.  We arbitrarily group the vertices from $X_0 \cup Y_0$ into pairs $(u,v)$ and distribute them one pair at a time. In addition to reassigning vertices from $X_0\cup Y_0$ we may move a vertex from one cluster to another.  This process will be completed after $s:=|X_0|=|Y_0|\leq \sigma^2 n$ steps.

We use the following notation. For  a cluster $Z$ let  $Z^r$ denote $Z$ after the $r$-th step of the reassignment. So $Z=Z^0$. Let $O(Z^r):=Z^0\cap Z^r$ denote the original vertices of $Z^0$ that remain after the $r$-th step, $T(Z^r):=Z^r\ssm Z^0$ denote the vertices that have been moved to $Z$ during the first $r$ steps, and $F(Z^r):=Z^0\ssm Z^r$ denote the vertices that have been moved from $Z$ during the first $r$ steps. We say that a cluster $Z^r$ is \emph{full} when $|T(Z^r)|=\sigma \ell$.

\vspace{.1in}

\noindent
\textsc{Procedure: Reassign}

For $r=1,\dots,s$ reassign the $r$-th pair $(u,v)$ as follows:
\begin{enumerate}
\item[(i)] Choose $i,j\in [q]$ so that each of the following holds:
\begin{enumerate}
\item[(a)] None of $V_i^{r-1}, U_i^{r-1}$, and  $U_j^{r-1}$ is full.
\item[(b)] $\deg(v,U_i^0) \geq 2\tau \ell$ and $\deg(u, V_j^0) \geq 2\tau \ell$.
\item[(c)] If $i \neq j$ then $e(U_j^0, V_i^0) \geq 3\tau \ell^2.$
\end{enumerate}
\item[(ii)] Reassign $u$ to $U_j^{r-1}$, $v$ to $V_i^{r-1}$, and if $i \neq j$ then pick $u'\in O(U_j^{r-1})$ with $\deg(u',V_i^0) \geq 2\tau \ell$ and reassign $u'$ to $U_i^{r-1}$.
\end{enumerate}


\begin{lemma}[Reassigning Lemma]\label{reassignlem}
Suppose $\{X_0, X_1,\dots, X_q\}\cup \{Y_0, Y_1,\dots, Y_q\}$ is a partition of $V(G)$ in which the pairs $(X_i,Y_i)$ and $(X_{j+1},Y_j)$ 
%LD15 (do we need to explicitly give the range of the indices?)
for $i\in[q]$ and $j\in[q-1]$, 
are $\ep'$-regular with density at least $\delta'$, where $2\ep'\leq\delta'$, 
$(1-d_0)\ell \leq |X_i|,|Y_i|\leq \ell$ and $s=|X_0|=|Y_0|\leq \sigma^2 n$.  
%LD15 (changed 2d_0 to d_0 based on a change you made later on which removes the need for the 2d_0)
If $\ep_1 \le \ep' \leq \sigma \leq \frac{1}{4}\tau\leq \frac{1}{4}d_0$,
then \textsc{Reassign} distributes all vertices from $X_0 \cup Y_0$ so that the following conditions are satisfied:

\begin{enumerate}
\item If $u\in T(X_i^s)$ then $\deg(u, O(Y_i^s)) \geq \tau \ell$ and if $v\in T(Y_i^s)$ then
$\deg(v, O(X_i^s)) \geq \tau \ell$;
\item $|X_i^s|-|Y_i^s|=|X_i^0|-|Y_i^0|$;
\item $|T(X_i^s)|,|T(Y_i^s)|\leq \sigma \ell$ and $|F(X_i^s)|,|F(Y_i^s)|\leq \sigma \ell$;
\item the pairs $(O(X_i^s),O(Y_i^s))$ and $(O(X_{j+1}^s), O(Y_j^s))$ are $2\ep'$-regular with density at least $\half \delta'$.
\end{enumerate}

\end{lemma}



\begin{proof} 

Suppose that $r$ pairs have been distributed and consider the $(r+1)$-th pair $(u,v)$. Let
$$N'(u) = \{i: \deg(u,Y_i^0) \geq 2 \tau \ell\}~~\text{and}~~ N'(v)= \{i: \deg(v,X_i^0) \geq 2 \tau \ell\}.$$
Since
$$\g_2 n \leq \deg(v) \leq |N'(v)|\ell + 2\tau \ell t +\sigma^2 n\le |N'(v)|\frac{n}{t} + 2\tau n +\sigma^2 n,$$
we have
$$|N'(v)| \geq (\g_2-2\tau-\sigma^2)t \geq (\g_2 - 3\tau)t.$$
In the same way we obtain $$|N'(u)|\geq (\g_1 -3\tau)t.$$ Now let 
$$X=\bigcup_{i \in N'(u)} X_i^0 \subseteq U \mbox{ and } Y = \bigcup_{i \in N'(v)} Y_i^0 \subseteq V.$$
Then we have
$$|Y|\geq |N'(v)|(1-d_0)(1 -\ep_1)\frac{n}{t} \geq (\g_2 -3\tau)(1-d_0)(1- \ep_1) n\geq (\g_2-5d_0)n \geq(\g_2-\alpha)n.$$
Similarly
$$|X| \geq (\g_1 - \alpha)n.$$
Consequently, as the graph is not $\alpha$-splittable, we have 
\begin{equation}
e(X,Y) > \alpha |X||Y| \geq \alpha(\g_1-\alpha)(\g_2-\alpha)n^2 \ge \alpha \g_1\g_2 n^2/2.
\label{rsplit}
\end{equation}

Suppose that we are unable to distribute the pair $(u,v)$. We will derive a contradiction by counting edges incident with full clusters and edges in pairs $(U_i^r,V_j^r)$ with $e(U_i^r,V_j^r)< 3\tau \ell^2$.  At most $s-1\leq \sigma^2 n$ pairs of exceptional vertices have been distributed, and each time a pair is distributed there are at most two indices $i$ such that $|T(X_i^r)|$ or $|T(Y_i^r)|$ increases.  Upon distribution, $|T(X_i^r)|$ or $|T(Y_i^r)|$ can increase by at most one. Thus there are at most
$$\frac{2\sigma^2 n}{\sigma \ell} = 2\sigma \frac{n}{\ell}$$
pairs $(U_i, V_i)$ such that either $U_i$ or $V_i$ is full. The total number of edges of $G$ which are incident with vertices in these clusters is at most $$4 \sigma\frac{n}{\ell}\ell n= 4\sigma n^2.$$  There are 
at most $3\tau n^2$ edges of $G$ in pairs $(X_i^0, Y_j^0)$ with 
$e(X_i^0,Y_j^0)< 3 \tau \ell^2.$
Then, since
$$(3\tau + 4\sigma)n^2 \leq 4\tau n^2 \leq \alpha \g_1\g_2 n^2/2 < e(X,Y)$$
contradicts (\ref{rsplit}), there must exist $i \in N'(v)$ and $j \in N'(u)$ such that none of $X_i^r,Y_i^r, X_j^r, Y_j^r$ is full and $e(X_j^0, Y_i^0) \geq 3 \tau\ell^2.$  Then since $e(O(X_j^r),Y_i^0)\ge (3\tau-\sigma)\ell^2$ there is $u'\in O(X_j^r)$ with $\deg(u',Y_i^0) \geq 2\tau\ell$.  Thus the procedure distributes $(u,v)$.

Conditions (ii) and (iii) hold by design: for (iii) note that a vertex is only reassigned from a cluster if another vertex is reassigned to that cluster. Condition (iv) follows immediately from Lemma \ref{slice}. Finally, condition (i) is satisfied since for every $u\in T(U_{i}^s)$ and $v\in T(V_{i}^s)$ we have
$$\deg(u,O(V_{i}^s)) \geq (2\tau-\sigma)\ell\ge \tau \ell ~~\text{and}~~ \deg\left(v,O(U_{i}^s)\right) \ge (2\tau-\sigma)\ell \ge \tau \ell.$$
\end{proof}

Now we apply Lemma \ref{reassignlem} to the partition $\{U_0,U_1,\dots,U_q\}\cup\{V_0,V_1,\dots,V_q\}$ with $\sigma=\sqrt{3\delta_1}$ and $\tau=d_0$, recalling that $\mathcal{P}=U_{1}V_{1} \dots,U_{q}V_{q}$ and $|U_0|=|V_0|\leq 3\delta_1 n$.
After the exceptional vertices have been distributed we set $U_i^1:=X_i^s$ and $V_i^1:= Y_i^s$. Then $O(U_i^1)=O(X_i^s)$
%LD15
, etc.  By Lemma~\ref{reassignlem}, each  $(O(U_{i}^1),O(V_i^1))$ is $\ep_2$-regular with density at least $\delta_2$ and $\ell\geq|O(U_i^1)|=|O(V_i^1)|\geq(1-\sqrt{3\delta_1})\ell$. While $(U_{i}^1,V_{i}^1)$ may not be $\ep_2$-regular, the exceptional parts  $T(U_i^1)$ and $T(V_i^1)$ satisfy:
\begin{align*}
&\forall u\in T(U_{i}^1), 
\forall v\in T(V_{i}^1),\\ 
&~~~~\deg(u,O(V_{i}^1)), \deg( v,O(U_{i}^1)) \ge d_0 \ell> \sqrt{3\delta_1}\ell\ge |T(V_i^1)|,|T(U_i^1)|.
\end{align*}
%LD15 (d_0 and \sqrt{\delta_1} have the same order.  We can make delta_1 smaller if we need to...)


Our next goal is to find a small ladder in each pair $(U_i,V_i)$ which will contain all of the exceptional vertices $T(U_i^1)$ and $T(V_i^1)$.  Precisely, we will prove the following.
\begin{claim}\label{smalllader} 
For each $i\in [r]$ there exists a ladder $L^i\subseteq U_i^1\cup V_i^1$ 
such that:
\begin{enumerate}
\item  $T(U_i^1)\cup T(V_i^1)\subseteq V(L^i) $.
\item  $|V(L^i)| \leq 16\sqrt{3\delta_{1}}\ell$.
\item  Each $w\in \ext(L^{i})$ satisfies %$w\in O(U_i^1)\cup O(V_i^1)$
%mistake (One vertex in \ext(L^i) will be from T().  It seems like we shouldn't care where the vertices are from just that they have large enough degree)
$\deg(w, (O(V_{i}^1)\cup O(U_{i})^1) \smallsetminus L^{i})\ge  \frac{1}{2}\delta_2\ell$.
%neighbors in $\left( O(V_{i}^1)\cup O(U_{i})^1\right) \smallsetminus L^{i}$.
\end{enumerate}
\end{claim}
%%%%%%%%%%%%%%%%%%%%%%Finish%%%%%%

%LD15


\begin{proof}
Let $w_{1},w_{2},\dots ,w_{s}$ be an ordering of $T(U_i^1) \cup T(V_i^1)$.
%LD15
Then $s\leq 2\sqrt{3\delta _{1}}\ell\le \frac{1}{16}d_0\ell$. Suppose that we have constructed a ladder $L\subseteq U_{i}^1\cup V_{i}^1$
%LD15
on $8r$ vertices ($1\leq r<s$) that contains exactly the first $r$
vertices of  $T(U_i^1) \cup T(V_i^1)$
%LD15
, satisfies (iii), and has
first rung $u^{\prime }v^{\prime }$ and last rung $u^{\prime \prime
}v^{\prime \prime }$. Without loss of generality, assume that $w_{r+1}\in
T(U_{i}^1)$. 

We will first show how to extend $L$ to $L^{\prime }$ by attaching a $3$-ladder $aba^{\prime }b^{\prime }w_{r+1}v$, with $a,a^{\prime }\in O(U_{i}^1)\smallsetminus L$ and $b,b^{\prime },v\in
O(V_{i}^1)\smallsetminus L$, to the end of $L$ so that $w_{r+1}$ and $v$
%LD15
satisfy (iii). By Lemma~\ref{deg}, all but at most $\ep_2\ell$, vertices $v\in O(V_i^1)$ satisfy $\deg(v,O(V_i^1)\ssm V(L))\ge \frac{1}{2}\delta_2\ell +4$.
%LD15
Choose such a vertex $v\in N(w_{r+1})\ssm V(L)$. Each $x\in \{u'',v'',w_{r+1},v\}$ has at least $\half\delta_2\ell$ neighbors in  
$(O(V_{i}^1)\cup O(U_{i}^1)) \smallsetminus L$.
%LD15 (I'm not exactly sure where (\delta_2-\ep_2)^{-2} comes from)
%Changed: added \ell and tied to prop
So by Proposition \ref{C4} 
%LD15
we can find vertices $a,b,a',b'\in \left( O(V_{i}^1)\cup O(U_{i}^1)\right) \smallsetminus L $ such that $a\sim v'',b\sim u'',a'\sim v,b'\sim w_{r+1}$ and $G[\{a,b,a',b'\}]=C_4$,
%LD15 
which completes the extension.

In extending $L$ to $L^{\prime }$ we may have violated
condition (iii) for the first rung $u^{\prime }v^{\prime }$ by using up some
of its neighbors. So now, in a similar way, we choose $a''\in
O(U_{i}^1)\smallsetminus L'$ and $b''\in
O(V_{i}^1)\smallsetminus L'$ such that $u'\sim b''\sim a''\sim v'$ and $\deg(a'',O(V_{i}^1)\smallsetminus L')$, $\deg(b'',O(U_{i}^1)\smallsetminus L') \geq \frac{1}{2}\delta_2\ell+1$. We
then add $a''b''$ to $L'$ as a first
rung to obtain $L''$ satisfying (iii).  Continuing in this fashion we obtain the desired ladder $L^{i}$ satisfying (i-iii).
\end{proof}

For each $i\in [q] $, set $U_{i}^{2}:=U_{i}^1\smallsetminus L^{i}$ and $V_{i}^{2}:=V_{i}^1\smallsetminus L^{i}$.  Then
\[
\ell\geq |U_i^2|=|V_i^2| \geq \left(1-9\sqrt{3\delta _{1}}\right) \ell \geq (1-d_0)\ell.
\]
Move one vertex from $U_1^2$ to $U_q^2$.  By Lemma \ref{slice} each of the pairs $( U_{i}^{2},V_{i}^{2})$ and $(U_{i+1}^{2}, V_{i}^{2}) $ are $\ep_3$-regular
with density at least $\delta_3$.

Our next goal is to reassign some vertices so that each of
the pairs $\left(U_{i}^{2},V_{i}^{2}\right) $ is $(\ep,\delta)$-super-regular.
Let $Q_i\subseteq U_i^2$ and $R_i\subseteq V_i^2$ be sets of size $\ep_3|V_i^2|$ 
%LD15 (say something about the vertex we moved)
such that every vertex $w\in U_i^2\cup V_i^2$ with $\deg(w,U_i^2\cup V_i^2) \leq (\delta_{3}-\ep _{3}) |V_i^{2}| $ is contained in $Q_i\cup R_i$. This is possible by Lemma \ref{deg}. %Changed fixed (2)
 
Move the vertices in $Q_i\cup R_i$ to new exceptional sets to obtain the partition  $$U_0^3:= \bigcup_{i=1}^q Q_i, ~~V_0^3:= \bigcup_{i=1}^q R_i,~~U_i^3:=U_i^2\ssm Q_i, ~~\text{and}~~V_i^3:=V_i^2\ssm R_i.$$
   Then $|U_{0}^3|=|V_{0}^3| \leq \ep _{3}n$. 
  %where $|U_i^3|\geq(1-\ep_3)\ell'$%%Changed removed 2
By Lemma \ref{slice} the pairs $(U_i^3,V_i^3)$ are $(\ep_4,\delta_4)$-super-regular for $i\in[q]$.  The pairs $(U_{j+1}^3,V_j^3)$ may not be super-regular, but they are $\ep_4$-regular with density at least $\delta_4$.

Applying Lemma \ref{reassignlem} to the partition $\{U_0^3, U_1^3,\dots,U_q^3\}\cup\{V_0^3,V_1^3,\dots,V_q^3\}$ with $\sigma=\sqrt{\ep_3}$ and $\tau=\delta_4$, we get a new partition $\{ U_1^4,\dots,U_q^4\}\cup\{V_1^4,\dots,V_q^4\}$. Note that the pairs $(O(U_i^4),O(V_i^4))$ are
 $(\half \ep_5,2\delta)$-super-regular and thus 
 %and thus they are $(\frac{1}{6}\ep,\delta)$-super-regular and the clusters have order 
$$(1-d_0)\ell\le (1-9\sqrt{3\delta _{1}}-\ep_3-\sqrt{\ep_3})\ell\leq |O(U_i^4)|,|O(V_i^4)|\leq \ell \text{~~~~and}$$  
$$|T(U_i^4)|,|T(V_i^4)|\leq\sqrt{\ep_3}\ell \leq \half\sqrt{\ep_5}\ell\leq\sqrt{\ep_5}|O(U_i^4)|,\sqrt{\ep_5}|O(V_i^4)|. $$
%LD15 Changed remove 2
So by Lemma \ref{aug}, since $\deg(u',O(V_i^4))\geq \delta_4|O(V_i^4)|$ and $\deg(v',O(U_i^4))\geq \delta_4|O(U_i^4)|$, for all $u'\in T(U_i^4)$ and $v'\in T(V_i^4)$,  the pairs $(U_i^4,V_i^4)$ are $(\ep,\delta)$-super-regular (with room to spare).  Similarly, each pair $(U_{j+1}^{4},V_{j}^{4}) $ is $\ep$-regular with density at least $\delta$. Also $|U_i^4|=|V_i^4|$, except that $|V_1^4|=|U_1^4|+1, |U_q^4|=|V_q^4|+1$.

%Recall that $\delta' =\frac{\delta _{1}}{2}$ and $\ep' =6\ep_{1}^{1/4}$. 
%Now the minimum degree of each pair $\left(V_{i}^{2},U_{i}^{2}\right) $ is at least
%\[
%\left( \delta _{1}-\ep _{1}-\sqrt{\ep _{1}}\right) \frac{n}{k%
%}\geq \left( \delta _{1}-3\sqrt{\ep _{1}}\right) \left|
%U_{i}^{2}\right| \geq \delta' \left| U_{i}^{2}\right| .
%\]
%By Lemma \ref{add} each pair $\left( U_{i}^{2},V_{i}^{2}\right) $ is $%
%\ep'$-regular. Thus each pair $\left( U_{i}^{2},V_{i}^{2}\right) $
%is $\left( \ep' ,\delta' \right) $-super-regular. 

Using Lemma \ref{deg}, for $i\in \left[q-1\right]$, choose $v_{i}\in V_{i}^{4}$ such that $|A_{i+1}| \geq \half \delta \ell$, where $A_{i+1}:=U_{i+1}^{4}\cap N(v_{i})$. Similarly, choose $u_{i+1}\in A_{i+1}$ such that $|D_{i}| \geq \half \delta \ell$, where 
$D_{i}:=V_i^4\cap N(u_{i+1})$. Set  $P:=\{v_i,u_{i+1}:i\in [q-1]\}$, $U_i^5:=U_i^4\ssm P$, and $V_i^5:=V_i^4\ssm P$. Then (using the spared room) $(U_i^5,V_i^5)$ is still an $(\ep,\delta)$-super-regular pair.
Now set $B_{i+1}:=V_i^5\cap N(u_{i+1})$ and $C_i:=U_i^5\cap N(v_i)$.
Let $x_{i}y_{i}$ be the first rung of $L^{i}$ and let $w_{i}z_{i}$ be the last rung of $L^{i}$, where $x_i,w_{i}\in U$ and $y_i,z_{i}\in V$. Finally let $X_{i}=U_{i}^{5}\cap N(y_{i})$, $Y_{i}=V_{i}^{5}\cap N(x_{i})$, $W_{i}=U_{i}^{5}\cap N( z_{i})$, and $Z_{i}=V_{i}^{5}\cap N(w_{i}) $. Note that each of $X_{i}$, $Y_{i}$, $W_{i}$, and $Z_{i}$ has size at least $\half \delta \ell = \varrho \ell$.
%LD15



%\begin{large}\texttt{
%Also, define $\beta, \eta$ we need $6\ge \eta n$. The start of the section needs to be redone.  Starting with $\delta=?, %\Delta=4, \varrho=\delta/4?$ we apply the Blow-up Lemma to get $\ep, \eta$. Now we must apply the Regularity Lemma.  We need %$m\varrho\ge6$. We need $\ep_0\lll \ep$.  How much bigger do we need $m$?
%}
%\end{large}
%LD15

We now apply Lemma \ref{blowup} to each pair $(U_{i}^{5},V_{i}^{5})$ to find 
%Changed 4 to 5
a spanning ladder $M^{i}$ whose first
rung is contained in $A_{i}\times B_{i}$, whose second rung is contained in $X_{i}\times Y_{i}$, whose third rung is contained in $W_{i} \times Z_{i}$, and whose last rung is contained in $C_{i}\times D_{i}$. This is possible since $\eta \ell\geq 4$. Clearly we can insert $L^{i}$ between the second and third rungs of $M^{i}$ to obtain a ladder $\mathcal{L}^{i}$ spanning $U_{i}^{4}\cup V_{i}^{4}$. Finally, $\mathcal{L}^{1}v_{1}u_{2}\mathcal{L}^{2}\dots v_{r-1}u_{r}\mathcal{L}^{r}$ is a spanning ladder of $G$.
\end{proof}

\section{Proof of Amar's Conjecture}

Theorem \ref{main} follows immediately from Lemmas \ref{preext}, \ref{ext}, \ref{nonext} with $N_0(k)=\max\{N_1(k),N_2(k)\}$.

Now we prove Theorem \ref{mainconstant}.%%Change

\begin{proof}

Let $N_0(1)$ be the value given when $k=1$ in Theorem \ref{main} and set $C:=N_0(1)$.  Suppose $G$ is a balanced $U,V$-bigraph on $2n$ vertices with $\delta_U+\delta_V\geq n+C$.  We may assume without loss of generality that $\delta_U=\delta(G)=:\delta$. We may assume $\delta<\frac{n}{200}+1$, otherwise we would have a spanning ladder by Theorem \ref{main} since the choice of $C$ implies that $n\geq N_0(1)$.

Let $S=\{x\in U:\deg(x)\leq\frac{9n}{10}\}$ and $S^{\prime}\subseteq S$ be a maximal subset such that $| N(S')|<3|S'|$.  Let $\bar{s}:=|S|-|S'|$, then $G[(S\smallsetminus S')\cup (V\smallsetminus N(S'))]$ contains a set of $\bar{s}$ disjoint claws $M=\{a_rb_rc_rd_r:r\in[\bar{s}]$, $a_r\in S\ssm S'$, $b_r,c_r,d_r\in V\ssm N(S')\}$.
%with roots $a_r\in S\ssm S'$ and leaves $b_r,c_r,d_r\in V\ssm N(S')$. 
%We have $S\smallsetminus S'\subseteq V(M)$ and every vertex $x\in S\smallsetminus S'$ satisfies $\deg_{M}(x)=3$.
We have the following bound on the cardinality of $S$,
\begin{align}
(n-\delta+C)n &\leq|E(G)|\leq\frac{9n}{10}| S| +n(n-|S|)\notag \\
|S|&\leq 10\delta-10C.
\end{align}

Note that for all $v_1,v_2\in V\cap V(M)$ we have
\begin{equation}
|(N(v_1)\cap N(v_2))\cap(U\ssm S)|\geq 2(n-\delta+C)-n-|S|>\frac{47}{50}n\geq 2\bar{s}.\label{clawladder}
\end{equation}
Thus by (\ref{clawladder}) there exists a set of $3$-ladders $$\Lambda(M)=\{x_ra_ry_rb_rc_rd_r:r\in[\bar{s}], a_rb_rc_rd_r\in M, x_r,y_r\in U\ssm S\}.$$ Note that $\mathrm{ext}(L)\subseteq V(G)\smallsetminus S$ for all $L\in \Lambda(M)$.  Let $R=\bigcup_{L\in \Lambda(M)}V(L)$.  For all $v'\in V\ssm N(S')$, we have $\deg(v')\geq n-\delta+C$, thus 
\begin{equation}
|S'|\leq \delta-C. \label{S'}
\end{equation}

Now we show that $G$ contains a ladder that spans $S^{\prime}$.
%By Lemmas \ref{ext} and \ref{nonext}, it will suffice to show that there exists a balanced $U',V'$-bigraph $H\subseteq G$ %on $60\delta$ vertices such that 
%\begin{enumerate}
%\item[(a)] $S^{\prime} \subseteq U^{\prime}\subseteq U\ssm V(M) ~~\text{and}~~ V^{\prime} \subseteq V\ssm V(M)$
%\item[(b)] $\deg_{H}(x) \geq\delta ~~\text{for all}~~ x\in U^{\prime} ~~\text{and}~~ \deg_{H}(y)\geq 29\delta+C %~~\text{for all}~~ y\in V^{\prime}.$
%\end{enumerate}
Let $T=\{x\in U:\deg(x)< n-29\delta\}$. Then 
\begin{align}
(n-\delta+C)n &\leq|E(G)| <(n-29\delta)|T|+n(n-|T| )\notag \\
|T|&<\frac{n}{29}\notag.
\end{align}
Let $X^{\prime}$ be any $(30\delta-|S^{\prime}|)$-subset
of $U\ssm (R\cup S\cup T)$ and $U^{\prime}=S^{\prime}\cup X^{\prime}$. Similarly,
let $Y^{\prime}$ be any $(30\delta-\left\vert N(S^{\prime})\right\vert )$-subset
of $V\ssm (N\left(S^{\prime}\right)\cup V(M))$ and $V^{\prime}=N(S^{\prime})\cup Y^{\prime}$.
Let $H:=G[U'\cup V']$.  Then every vertex in $X^{\prime}$ is non adjacent to at most $29\delta$
vertices of $V$ and so $\delta_{U'}:=\delta_{U'}(H)\geq\delta$. Similarly, $\delta_{V'}:=\delta_{V'}(H)\geq 29\delta+C$. Let $m=30\delta$ and note that $\delta_{U'}+\delta_{V'}\geq m+C$, $\delta(H)\geq\frac{m}{30}$ and by the choice of $C$, $m\geq N_0(1)$.  Thus $H$ contains a spanning ladder $L=u_1v_1\dots u_{30\delta}v_{30\delta}$ by Lemmas \ref{ext} and \ref{nonext}. Since $|N(S')|<3|S'|$ we have $|S'\cup N(S')|<4\delta$ by (\ref{S'}). Thus there exists rungs $u_iv_i,u_{i+1}v_{i+1}\in E(L)$ with $2\leq i\leq 30\delta-2$ such that $u_i,v_i,u_{i+1},v_{i+1}\in V(H)\ssm (S'\cup N(S'))$.  Let $L^1=u_1v_1\dots u_iv_i$ and $L^2=u_{i+1}v_{i+1}\dots u_{30\delta}v_{30\delta}$.  We will specify $L^1$ as the initial ladder and $L^2$ as the terminal ladder.  Let $\Lambda:=\Lambda(M)\cup\{L^1,L^2\}$ and let $I=I(\Lambda)=\bigcup_{L\in \Lambda}\mathring{L}$. Set $q':=0$, $s':=\bar{s}+2=|\Lambda|$ and $t':= 30\delta+3\bar{s}$. Note that for all $z\in V(G)\ssm I$ we have,
\[
\deg(z)\geq \frac{9n}{10}\geq \frac{3n+100\delta}{4}+1 \geq\frac{3n+3s'+t'+4q'}{4}+1.
\]
So we may apply Lemma \ref{elll} to $G$ to obtain a spanning ladder which starts with the first rung of $L_1$ and ends with the last rung of $L_2$.

%We will now find a spanning circular ladder of $G'$.  Let $u_1v_1$, $u_sv_s$ be the ends of $L'$.  Let $e'=u'v'$ be %an edge induced by the vertex sets $N(u_1)\cap (B\smallsetminus V(L))$ and $N(v_1)\cap (A\smallsetminus V(L))$ %(\texttt{details...}), likewise let $e''=u''v''$ be an edge induced by $N(u_s)\cap (B\smallsetminus V(L))$ and %$N(v_s)\cap (A\smallsetminus V(L))$.  Let $G''=G-(H\cup (L\smallsetminus \{u',v',u''v''\}))$ and note that $\delta(G')\geq %\frac{9}{10}-(30\delta-30C)\geq \frac{3}{4}n+30(C-1)\geq \frac{3}{4}|V(G'')|+1$ and thus we can use Lemma \ref{ell} %to find a spanning ladder $L''$ of $G''$ which starts with $e'$ and ends with $e''$.  Thus $L''\cup L'$ is a %circular ladder and by equation \ref{enddeg}, we can obtain a spanning ladder $L^*=L\cup L'\cup L''$ which spans $G$.

\end{proof}

Finally, we prove Theorem \ref{amarlarge}.

\begin{proof}
Let $C$ be the constant from Theorem \ref{mainconstant}, let $N_0(1)<N_0(2)<\dots<N_0(C-1)$ be the values given by Theorem \ref{main}, and let $N_0=N_0(C-1)$. Let $G$ be a balanced $U,V$-bigraph on $2n$ vertices with $n\geq N_0$ which satisfies $\delta_U+\delta_V\geq n+\mathrm{comp}(H)$.  By Theorem \ref{main} and Theorem \ref{mainconstant}, we have $H\subseteq G$.
\end{proof}





\chapter{TILING IN BIPARTITE GRAPHS: MINIMUM DEGREE}\label{mindegtilingchapter}

\DoubleSpacing
\setlength{\parindent}{.5in}

This chapter is joint work with Andrzej Czygrinow. 

\section{Introduction}

If $G$ is a graph on $n=sm$ vertices, $H$ is a graph on $s$ vertices and $G$ contains $m$ vertex disjoint copies of $H$, then we say $G$ can be \emph{tiled} with $H$.  In this language, we state the seminal result of Hajnal and Szemer\'edi.

\begin{theorem}[Hajnal-Szemer\'edi \cite{HSz}]
Let $G$ be a graph on $n=sm$ vertices.  If $\delta(G)\geq (s-1)m$, then $G$ can be tiled with $K_s$.
\end{theorem}

For tiling with general $H$, results of Alon and Yuster \cite{Alon} and Koml\'{o}s, S\'{a}rk\"{o}zy, and  Szemer\'edi \cite{KSS} gave sufficient conditions on the minimum degree of a graph $G$ such that $G$ can be tiled with $H$.  Specifically, in \cite{KSS}, it is shown that if $G$ is a graph on $n$ vertices with minimum degree at least $\left (1 - 1/\chi(H)\right)n +K$ for a constant $K$ that only depends on $H$, then $G$ can be tiled with $H$.  A more delicate minimum degree condition that involves the so-called critical chromatic number of $H$ was conjectured by Koml\'os and solved by Shokoufandeh and Zhao \cite{Zh}.  Finally, K\"{u}hn and Osthus \cite{KO} determined exactly when the critical chromatic number or chromatic number is the appropriate parameter and thus settled the problem (for large graphs).

In this paper we study the tiling problem in bipartite graphs.  Denote a bipartite graph $G$ with partition sets $U$ and $V$ by $G[U, V]$.  We say $G[U,V]$ is \emph{balanced} if $|U|=|V|$.  Zhao proved the following Hajnal-Szemer\'edi type result for bipartite graphs.

\begin{theorem}[Zhao \cite{Z}]
For each $s\geq 2$, there exists $m_0$ such that the following holds for all $m\geq m_0$.  If $G$ is a balanced bipartite graph on $2n=2ms$ vertices with 
$$\delta(G)\geq \left\lbrace \begin{array}{ll} \frac{n}{2}+s-1   & \text{ if } m \text{ is even } \\
              \frac {n+3s}{2}-2 & \text{ if } m \text{ is odd, } \end{array} \right. $$
then $G$ can be tiled with $K_{s,s}$.
\end{theorem}
Zhao proved that this minimum degree condition was tight.

\begin{proposition}[Zhao \cite{Z}]
Let $s\geq 2$, and $n=ms\geq 64s^2$.  There exists a balanced bipartite graph, $G$, on $2n$ vertices with
$$\delta(G)= \left\lbrace \begin{array}{ll} \frac{n}{2}+s-2   & \text{ if } m \text{ is even } \\
              \frac {n+3s}{2}-3 & \text{ if } m \text{ is odd } \end{array} \right. $$
such that $G$ cannot be tiled with $K_{s,s}$.
\end{proposition}

Hladk\'{y} and Schacht extended Zhao's result as follows.

\begin{theorem}[Hladk\'{y}-Schacht \cite{HS}]\label{HS ub}
Let $1\leq s<t$ be fixed integers.  There exists $m_0$ such that the following holds for all $m\geq m_0$. If $G$ is a balanced bipartite graph on $2n=2m(s+t)$ vertices with $$\delta(G)\geq \left\lbrace \begin{array}{ll} \frac{n}{2}+s-1   & \text{ if } m \text{ is even } \\
               \frac {n+t+s}{2}-1 & \text{ if } m \text{ is odd,} \end{array} \right. $$
then $G$ can be tiled with $K_{s,t}$.
\end{theorem}
They proved that this minimum degree condition was tight in all cases except when $m$ is odd and $t> 2s+1$.  Note that since we are dealing with balanced bipartite graphs, in any tiling of $G[U,V]$ with $K_{s,t}$ there must be an equal number of copies of $K_{s,t}$ with $s$ vertices in $U$ as copies of $K_{s,t}$ with $t$ vertices in $U$.  This explains why the authors \cite{HS} suppose $2n=2m(s+t)$ instead of $2n=m(s+t)$.  

\begin{proposition}[Hladk\'{y}-Schacht \cite{HS}]
Let $1\leq s<t$ be fixed integers.  There exists $m_0$ such that the following holds for all $m\geq m_0$. There exists a balanced bipartite graph, $G$, on $2n=2m(s+t)$ vertices with
$$\delta(G)= \left\lbrace \begin{array}{ll} \frac{n}{2}+s-2   & \text{ if } m \text{ is even } \\
              \frac {n+t+s}{2}-2 & \text{ if } m \text{ is odd and } t\leq 2s+1 \end{array} \right. $$
such that $G$ cannot be tiled with $K_{s,t}$.
\end{proposition}

Our objective is to give the tight minimum degree condition in the final remaining case, when $m$ is odd and $t> 2s+1$. We will do this in two parts.  First in Section \ref{Extremal} we prove that when $m$ is odd and $t\geq 2s+1$, the following minimum degree condition is sufficient.

\begin{theorem}\label{tilemain}
Let $1\leq s<t$ be fixed integers with $2s+1\leq t$.  There exists $m_0$ such that the following holds for all odd $m$ with $m\geq m_0$. If $G$ is a balanced bipartite graph on $2n=2m(s+t)$ vertices with $$\delta(G)\geq \frac {n+3s}{2}-1,$$ then $G$ can be tiled with $K_{s,t}$.
\end{theorem}

Then in Section \ref{lower bound} we prove that the minimum degree condition in Theorem \ref{tilemain} is tight.

\begin{proposition}\label{counterexample}
Let $1\leq s<t$ be fixed integers with $2s+1\leq t$.  There exists $m_0$ such that the following holds for all odd $m$ with $m\geq m_0$. There exists a balanced bipartite graph, $G$, on $2n=2m(s+t)$ vertices with 
$$\delta(G)= \left\lbrace \begin{array}{ll}\frac{n+3s}{2}-\frac{3}{2}   & \text{ if } t \text{ is odd } \\
              \frac{n+3s}{2}-2 & \text{ if } t \text{ is even } \end{array} \right. $$
such that $G$ cannot be tiled with $K_{s,t}$.
\end{proposition}

Let $m=2k+1$ for some $k\in\mathbb{N}$ and let $n=m(s+t)$.  We note that when $t=2s+1$, $\frac{n+3s}{2}-1=(k+1)(s+t)-\frac{3}{2}$ and $\frac{n+t+s}{2}-1=(k+1)(s+t)-1$.  So the value for the lower bound in Theorem \ref{tilemain} is smaller than the value for the lower bound in Theorem \ref{HS ub} when $t=2s+1$, but since $\delta(G)$ only takes integer values the minimum degree condition in Theorem \ref{tilemain} is not an improvement until $t>2s+1$.


\section{Proof of Theorem \ref{tilemain}} 

For disjoint sets $A,B\subseteq V(G)$, we define $e(A,B)$ to be the number of edges with one end in $A$ and the other end in $B$ and for $v\in V(G)\setminus A$ we write $\deg(v,A)$ instead of $e(\{v\},A)$.  Also, $d(A,B)=\frac{e(A,B)}{|A||B|}$, $\delta(A,B)=\min\{\deg(v,B):v\in A\}$ and $\Delta(A,B)=\max\{\deg(v,B):v\in A\}$.  An \emph{$h$-star from $A$ to $B$}, is a copy of $K_{1,h}$ with the vertex of degree $h$, \emph{the center}, in $A$ and the vertices of degree $1$, \emph{the leaves}, in $B$.

The following theorem appears in \cite{Z}.

\begin{theorem}[Zhao \cite{Z}]\label{stability}
For every $\alpha>0$ and every positive integer $r$, there exist $\beta>0$ and positive integer $m_1$ such that the following holds for all $n=mr$ with $m\geq m_1$.  Given a bipartite graph $G[U,V]$ with $|U|=|V|=n$, if $\delta(G)\geq (\half-\beta)n$, then either $G$ can be tiled with $K_{r,r}$, or there exist \begin{equation}\label{extremalcondition} U_1'\subseteq U,~ V_2'\subseteq V,~\text{ such that } |U_1'|=|V_2'|=\floor{n/2},~ d(U_1',V_2')\leq \alpha.\end{equation}
\end{theorem}

If a balanced bipartite graph $G[U,V]$ on $2n$ vertices with $n$ divisible by $r$ satisfies (\ref{extremalcondition}), we say $G$ is \emph{extremal} with parameter $\alpha$.  In this case we set $U_2':=U\setminus U_1'$ and $V_1':=V\setminus V_2'$.

If we replace $r$ with $s+t$ in Theorem \ref{stability}, we see that either $G$ can be tiled with $K_{s+t,s+t}$ or else we are in the extremal case.  If it is the case that $G$ can be tiled with $K_{s+t,s+t}$, we split each copy of $K_{s+t,s+t}$ into two copies of $K_{s,t}$ to give the desired tiling.  So we must only deal with the extremal case.

\subsection{Pre-processing}\label{preprocess1}

\begin{claim}\label{diagonals}
Let $0<\alpha\ll 1$, $r\in \mathbb{N}$ and let $m_1\in \mathbb{N}$ be given by Theorem \ref{stability}.  Let $m\geq m_1$ and suppose that $G[U,V]$ is a balanced bipartite graph on $2n=2mr$ vertices such that $\delta(G)= \frac{n}{2}+C$, where $0\leq C\leq 3r/2$.  Suppose further that the deletion of any edge of $G$ will cause the resulting graph to have minimum degree less than $\frac{n}{2} +C$.  If $G$ is extremal with parameter $\alpha$, then $d(U_2', V_1')\leq 5\sqrt{\alpha}$.
\end{claim}

\begin{proof}
Let $\gamma:=5\sqrt{\alpha}$ and suppose $d(U_2',V_1')>\gamma$.  Let $X'=\{u\in U_2':\deg(u,V_2')<(1-\sqrt{\alpha})\frac{n}{2}\}$, $Y'=\{v\in V_1':\deg(v,U_1')<(1-\sqrt{\alpha})\frac{n}{2}\}$. Since $e(U_1',V_2')\leq \alpha \frac{n^2}{4}$ and $e(U_1',V)\geq |U_1'|\frac{n}{2}$, we have $e(U_1', V_1')\geq |U_1'|\frac{n}{2}-\alpha\frac{n^2}{4}$.  Thus we can bound the non-edges between $U_1'$ and $V_1'$, $$\sqrt{\alpha}\frac{n}{2}|Y'|\leq \bar{e}(U_1',V_1')\leq \alpha\frac{n^2}{4},$$ which gives $|Y'|\leq\sqrt{\alpha}\frac{n}{2}$.  Similarly we have $|X'|\leq \sqrt{\alpha}\frac{n}{2}$. Let $U_2''=U_2'\setminus X'$ and $V_1''=V_1'\setminus Y'$.  Since $d(U_2', V_1')>\gamma$, we have
\begin{equation}\label{doubleprime}
e(U_2'',V_1'')\geq \gamma\frac{n^2}{4}-2\sqrt{\alpha}\frac{n^2}{4}=3\sqrt{\alpha}\frac{n^2}{4}.
\end{equation}

Let $X''=\{u\in U_2'':\deg(u,V_1'')\geq \sqrt{\alpha}\frac{n}{2}+C+1\}$ and $Y''=\{v\in V_1'':\deg(v,U_2'')\geq \sqrt{\alpha}\frac{n}{2}+C+1\}$.  If there is an edge $uv\in E(X'', Y'')$, then $\deg(u),\deg(y)\geq\frac{n}{2}+C+1$ which contradicts the edge minimality of $G$, so suppose $e(X'', Y'')=0$. Finally, by \eqref{doubleprime} we have 
$$3\sqrt{\alpha}\frac{n^2}{4}\leq e(U_2'',V_1'')\leq e(X'',Y'')+e(U_2''\setminus X'',V_1'')+e(V_1''\setminus Y'', U_2'')\leq 2(\sqrt{\alpha}\frac{n}{2}+C)\frac{n}{2},$$
which is a contradiction, since $n$ is sufficiently large.

\end{proof}

Let $1\leq s<t$ be integers so that $2s+1\leq t$, and let $0<\alpha\ll 1$ (setting $\alpha:=\left(\frac{1}{32t(s+t)}\right)^{3}$ is small enough).  Let $G[U,V]$ be a balanced bipartite graph on $2n=2m(s+t)$ vertices, where $m=2k+1$ and $k$ is a sufficiently large integer with respect to $(\frac{\alpha}{5})^2$.  Suppose that $G$ is extremal with parameter $(\frac{\alpha}{5})^2$ and edge-minimal with respect to the condition $\delta(G)\geq \frac{n+3s}{2}-1$.  By Claim \ref{diagonals} we have $d(U_i',V_{3-i}')\leq \alpha$ for $i=1,2$.  Then for $i=1,2$, we define
\begin{align*}
&U_i=\{u\in U:\deg(u,V_{3-i}')<\alpha^{\frac{1}{3}}\frac{n}{2}\},~ V_i=\{v\in V:\deg(v,U_{3-i}')<\alpha^{\frac{1}{3}}\frac{n}{2}\},\\
&U_0=U-U_1-U_2, \text{ and } V_0=V-V_1-V_2.
\end{align*}
As a consequence of these definitions, we have the following.
\begin{claim}\label{bounds} For $i=1,2$
\begin{align*}
\emph{(i)}&~ (1-\alpha^{2/3})\frac{n}{2}\leq |U_i|,|V_i|\leq (1+\alpha^{2/3})\frac{n}{2},~~~  \emph{(ii)}~|U_0|,|V_0|\leq \alpha^{2/3}n,\\
\emph{(iii)}&~(1-2\alpha^{1/3})\frac{n}{2}<\delta(U_i,V_i),\delta(V_i,U_i),~~~ \emph{(iv)}~(\alpha^{1/3}-\alpha^{2/3})\frac{n}{2}\leq \delta(U_0,V_i),\delta(V_0,U_i),\\ \emph{(v)}&~\Delta(U_i,V_{3-i}), \Delta(V_{3-i},U_i)\leq \alpha^{1/3}n
\end{align*}
\end{claim}

\begin{proof}
A proof of (i)-(iv) can be found in \cite{Z} and was also used in \cite{HS}.  So we prove (v) here.  

Let $i\in \{1,2\}$ and note that 
\begin{equation}\label{eq1}
|U_i'\setminus U_i|\alpha^{1/3}\frac{n}{2}\leq e(U_i'\setminus U_i, V_{3-i}')\leq e(U_i', V_{3-i}')\leq \alpha\frac{n^2}{4}
\end{equation}
and 
\begin{equation}\label{eq2}
|V_i'\setminus V_i|\alpha^{1/3}\frac{n}{2}\leq e(V_i'\setminus V_i, U_{3-i}')\leq e(V_i', U_{3-i}')\leq \alpha\frac{n^2}{4}.
\end{equation}
Then (\ref{eq1}) and (\ref{eq2}) imply 
\begin{equation}\label{eq3}
|U_i'\setminus U_i|, |V_i'\setminus V_i|\leq \alpha^{2/3}\frac{n}{2},
\end{equation}
which gives $\Delta(U_i, V_{3-i})\leq \Delta(U_i, V_{3-i}')+|V_{3-i}\setminus V_{3-i}'|\leq \Delta(U_i, V_{3-i}')+|V_{i}'\setminus V_{i}|\leq \alpha^{1/3}n$ and $\Delta(V_i, U_{3-i})\leq \Delta(V_i, U_{3-i}')+|U_{3-i}\setminus U_{3-i}'|\leq \Delta(V_i, U_{3-i}')+|U_{i}'\setminus U_{i}|\leq \alpha^{1/3}n$.

\end{proof}


We need to define some new sets which were not specified in \cite{Z}.  
\begin{definition}\label{tildehat}
For $i= 1,2$, let 
\begin{align*}
&\tilde{U_i}=\{u\in U_i:\deg(u,V_{3-i})\geq s\},~ \tilde{V_i}=\{v\in V_i:\deg(v,U_{3-i})\geq s\},\\
&\hat{U_i}=U_i\setminus \tilde{U_i}, \text{ and } \hat{V_i}=V_i\setminus \tilde{V_i}. 
\end{align*}
\end{definition}

Note that the following inequalities are satisfied:
\begin{align}
\delta(\hat{U_1},V_0)+\delta(\hat{U_2},V_0)&\geq n+3s-2-(|V_1|+s-1)-(|V_2|+s-1)=|V_0|+s ~~\text{and} \label{V_0}\\
\delta(\hat{V_1},U_0)+\delta(\hat{V_2},U_0)&\geq n+3s-2-(|U_1|+s-1)-(|U_2|+s-1)=|U_0|+s. \label{U_0}
\end{align}

\subsection{Preliminary Claims}

The following useful lemma appears in \cite{Z}.

\begin{lemma}[Zhao \cite{Z}, Fact 5.3]\label{lem:Zhao}
Let $F[A, B]$ be a bipartite graph with $\delta:=\delta(A,B)$ and $\Delta:=\Delta(B,A)$ 
Then $F$ contains $f_h$ vertex disjoint $h$-stars from $A$ to $B$, and $g_h$ vertex disjoint $h$-stars from $B$ to $A$ (the stars from $A$ to $B$ and those from $B$ to $A$ need not be disjoint), where
\begin{align*}
f_h\geq\frac{(\delta-h+1)|A|}{h\Delta+\delta-h+1}, ~~~ g_h\geq\frac{\delta|A|-(h-1)|B|}{\Delta+h\delta-h+1}.
\end{align*}

\end{lemma}

We now prove three claims that we will need in the main proof.

\begin{claim}\label{TildeStars}
Let $i\in\{1,2\}$ and $\{A,B\}=\{U_i,V_{3-i}\}$.  Let $0\leq c\leq \alpha^{1/3}n$, $B_0\subseteq B$ and $A_0=\{v\in A:\deg(v,B_0)\geq s+c\}$.
If $|A_0|\geq\frac{n}{4}$ then there is a set $\mathcal{S}_A$ of at least $\frac{c+1}{8s\alpha^{1/3}}$ vertex disjoint $s$-stars from $A_0$ to $B_0$.
\end{claim}

\begin{proof}
Let $\mathcal{S}_A$ be a maximum set of vertex disjoint $s$-stars from $A_0$ to $B_0$ and let $f_s=|\mathcal{S}_A|$.  We apply Lemma \ref{lem:Zhao} to the graph $G[A_0,B_0]$.   Recall, by Claim \ref{bounds}, that $\Delta(B,A)\leq \alpha^{1/3} n$.  Then
\begin{align*}
f_s\geq \frac{(c+1)|A_0|}{s\alpha^{1/3}n+c+1}\geq\frac{(c+1)\frac{n}{4}}{2s\alpha^{1/3}n}=\frac{c+1}{8s\alpha^{1/3}}.
\end{align*}

\end{proof}

Note that since $n=(2k+1)(s+t)$, we can write $\delta(G)\geq \frac{n+3s}{2}-1=k(s+t)+2s+\frac{t}{2}-1$.  


\begin{claim} \label{Stars}Let $i\in\{1,2\}$ and $\{A,B\}=\{U_i,V_{3-i}\}$.  Let $|A|=k(s+t)+z$ and $|B|=k(s+t)+y$.  Suppose $y\geq z$ and $y\geq \frac{t+1}{2}$.  Then there is a set $\mathcal{S}_B$ of $y$ vertex disjoint $s$-stars with centers $C_B\subseteq B$ and leaves $L_A\subseteq A$.  Furthermore if $z\geq 1$, then there is a set $\mathcal{S}_A$ of $z$ vertex disjoint $s$-stars from $A\setminus L_A$ to $B\setminus C_B$.
\end{claim}

\begin{proof}
Let $\beta:=32s\alpha^{1/3}$ and recall that by the choice of $\alpha$ we have $\frac{1}{t}\gg \beta\gg 2\alpha^{1/3}$.  
We show that the desired set $\mathcal{S}_B$ exists by applying Lemma \ref{lem:Zhao} to the graph $G[A,B]$.  We have $\delta(A,B)\geq k(s+t)+2s+\frac{t}{2}-1-(n-|B|)=y+s-\frac{t}{2}-1$ and $\Delta(B,A)\leq \alpha^{1/3}n$ by Claim \ref{bounds}.  Let $g_s=|\mathcal{S}_B|$, then
\begin{align*}
g_s&\geq \frac{(y-\frac{t}{2}+s-1)(k(s+t)+z)-(s-1)(k(s+t)+z+y-z)}{\alpha^{1/3}n+s(y-\frac{t}{2}+s-1)-s+1}\\
&=\frac{(y-\frac{t}{2})(k(s+t)+z)-(s-1)(y-z)}{\alpha^{1/3}n+s(y-\frac{t}{2})+s^2-2s+1}\\
&\geq \frac{(y-\frac{t}{2})\frac{n}{3}}{2\alpha^{1/3}n}~~~~~~(\text{since } y\leq \alpha^{2/3}\frac{n}{2} \text{ and } -\alpha^{2/3}\frac{n}{2}\leq z, \text{ by Claim \ref{bounds}} )\\
&\geq y ~~~~~~(\text{since } y\geq \frac{t+1}{2} \text{ and } \alpha\ll 1).
\end{align*}
Thus the desired set $\mathcal{S}_B$ exists.

Suppose $z\geq 1$.  Let $c:=\frac{1}{2}y$ if $y\geq 1/\beta$, and let $c:=0$ if $y<1/\beta$.  Let $B_0=B\setminus C_B$ and $A_0=\{v\in A\setminus L_A|\deg(v,B_0)\geq s+c\}$ and $\bar{A}=(A\setminus L_A)\setminus A_0$.  Suppose that $|\bar{A}|\geq\frac{n}{16}$.  Then there exists $u\in C_B$ such that if $y<1/\beta$,
\begin{align*}
\deg(u,A)\geq\frac{e(\bar{A},C_B)}{|C_B|}\geq\frac{\left(y-\frac{t}{2}+s-1-(s-1)\right)\frac{n}{16}}{y}
=\frac{\left(y-\frac{t}{2}\right)\frac{n}{16}}{y}
> \frac{\beta n}{32}
\geq \alpha^{1/3}n
\end{align*}
and if $y\geq 1/\beta$,
\begin{align*}
\deg(u,A)&\geq\frac{e(\bar{A},C_B)}{|C_B|}\\
&>\frac{\left(y-\frac{t}{2}+s-1-(s+\half y)\right)\frac{n}{16}}{y}
=\frac{\left(\frac{y}{2}-\frac{t}{2}-1\right)\frac{n}{16}}{y}
>\frac{n}{64}
\geq\alpha^{1/3}n,
\end{align*}
each contradicting Claim \ref{bounds}.  So $|\bar{A}|<\frac{n}{16}$ and thus $|A_0|\geq |A|-|L_A|-\frac{n}{16}\geq k(s+t)-s\alpha^{2/3}\frac{n}{2}-\frac{n}{16}\geq\frac{n}{4}$. Now let $\mathcal{S}_A$ be a maximum set of disjoint $s$-stars from $A_0$ to $B_0$ and let $f_s=|\mathcal{S}_A|$. By Lemma \ref{TildeStars} we have $f_s\geq \frac{c+1}{8s\alpha^{1/3}}$.  Recall that $1\leq z\leq y$.  If $y\geq 1/\beta$, then $f_s\geq \frac{y}{16s\alpha^{1/3}}\geq z$ and if $y<1/\beta$, then $f_s\geq \frac{1}{8s\alpha^{1/3}}\geq \frac{1}{\beta}\geq z$.  So the desired set $\mathcal{S}_A$ exists.


\end{proof}


\begin{claim}\label{K_1}
Suppose $|U_0|,|V_0|\geq s$.  If $|\hat{U_1}|\geq\frac{n}{8}$ and $|\hat{U_2}|\geq\frac{n}{8}$ (see Definition \ref{tildehat}), then there is a $K_{s,t}=:K^1$ with $s$ vertices in $V_0$, $\ceiling{t/2}$ vertices in $U_1$ and $\floor{t/2}$ vertices in $U_2$.  Likewise, if $|\hat{V_1}|\geq\frac{n}{8}$ and $|\hat{V_2}|\geq\frac{n}{8}$ then there is a $K_{s,t}=:K^2$ with $s$ vertices in $U_0$, $\ceiling{t/2}$ vertices in $V_1$ and $\floor{t/2}$ vertices in $V_2$.
\end{claim}


\begin{proof}

Without loss of generality we will only prove the first statement.  Let $$\ell:=s\binom{|U_2|}{\floor{t/2}}/\binom{\ceiling{(\alpha^{1/3}-\alpha^{2/3})n/2}}{\floor{t/2}}$$
and recall that $|U_1|, |U_2|\leq (1+\alpha^{2/3})\frac{n}{2}$ by Claim \ref{bounds}. Thus we have
\begin{align*}
\ell\leq s\left(\frac{|U_2|}{(\alpha^{1/3}-\alpha^{2/3})\frac{n}{2}-\floor{t/2}}\right)^{\floor{t/2}}&\leq  s\left(\frac{(1+\alpha^{2/3})\frac{n}{2}}{(\alpha^{1/3}-\alpha^{2/3})\frac{n}{3}}\right)^{\floor{t/2}}\\
&\leq s\left(\frac{3(1+\alpha^{2/3})}{2(\alpha^{1/3}-\alpha^{2/3})}\right)^{\floor{t/2}}  .\end{align*}

\noindent
\textbf{Case 1. } $|V_0|\geq  \ell \binom{|U_1|}{\ceiling{t/2}}/\binom{\ceiling{(\alpha^{1/3}-\alpha^{2/3})n/2}}{\ceiling{t/2}}$.
Recall that $\delta(V_0, U_i) \geq (\alpha^{1/3} - \alpha^{2/3})n/2$ for $i=1,2$ by Claim \ref{bounds} and suppose that there is no $K_{\ceiling{t/2},\ell}$ with $\ceiling{t/2}$ vertices in $U_1$ and $\ell$ vertices in $V_0$.  We count the $\ceiling{t/2}$-stars from $V_0$ to $U_1$ in two ways which gives $$|V_0| \binom{\ceiling{(\alpha^{1/3} - \alpha^{2/3})n/2}}{\ceiling{t/2}} < \ell \binom{|U_1|}{\ceiling{t/2}}$$
contradicting the lower bound for $|V_0|$. Consequently there is a complete bipartite graph $K'=K_{\ceiling{t/2},\ell}$ with $\ceiling{t/2}$ vertices in $U_1$ and $\ell$ vertices in $V_0$. If there is no $K_{\floor{t/2},s}$ with $s$ vertices in $V(K') \cap V_0$ and $\floor{t/2}$ vertices in $U_2$, then a similar counting argument gives $$\ell \binom{\ceiling{(\alpha^{1/3} - \alpha^{2/3})n/2}}{\floor{t/2}} < s\binom{|U_2|}{\floor{t/2}}$$
contradicting the definition of $\ell$.

\noindent
\textbf{Case 2.} $|V_0|<  \ell \binom{|U_1|}{\ceiling{t/2}}/\binom{\ceiling{(\alpha^{1/3}-\alpha^{2/3})n/2}}{\ceiling{t/2}}$.  
By \eqref{ell}, we have
$$|V_0|< 
\ell\left(\frac{3(1+\alpha^{2/3})}{2(\alpha^{1/3}-\alpha^{2/3})}\right)^{\ceiling{t/2}}\leq s\left(\frac{3(1+\alpha^{2/3})}{2(\alpha^{1/3}-\alpha^{2/3})}\right)^{t}.$$  Let $p := \delta(\hat{U_1}, V_0)$, and note that $p\geq s$ by (\ref{V_0}).  We claim that there is a complete bipartite graph $K':= K_{\ceiling{t/2},p}$ with $\ceiling{t/2}$ vertices in $\hat{U_1}$ and $p$ vertices in $V_0$. Let $c$ be the number of $p$-stars with centers in $\hat{U_1}$ and leaves in $V_0$.  We have $c\geq |\hat{U_1}|\geq \frac{n}{8}$ and if no $p$-subset of $V_0$ is in $\ceiling{t/2}$ of such stars, i.e. $K'$ does not exist, we have $c\leq(\ceiling{t/2}-1)\binom{|V_0|}{p}$ which contradicts the fact that $|V_0|$ is $O(1)$ and $n$ is sufficiently large (with respect to $\alpha$, $t$, and consequently $|V_0|$). From (\ref{V_0}) we have $\delta(\hat{U_2},V_0)\geq|V_0|-p+s$, so every vertex $u \in  \hat{U_2}$ has at least $s$ neighbors in $V(K')\cap V_0$. Repeating the argument above by counting $s$-stars with centers in $\hat{U_2}$ and leaves in $V(K')\cap V_0$ gives $K'':=K_{s,\floor{t/2}}$.  Now choose $K^1\subseteq K'\cup K''$ having the property that $|V_0 \cap V(K^1)|=s$, $|U_1\cap V(K^1)|=\ceiling{t/2}$, and $|U_2\cap V(K^1)|=\floor{t/2}$ as desired.
\end{proof}

\subsection{Extremal Case}
\label{Extremal}

Recall that $t\geq 2s+1$, $n=(2k+1)(s+t)$ for some sufficiently large $k\in \mathbb{N}$, and $\delta(G)\geq \frac{n+3s}{2}-1=k(s+t)+2s+\frac{t}{2}-1$.  We start with the partition given in Section \ref{preprocess1} and we call $U_0$ and $V_0$ the \emph{exceptional} sets. Let $i\in \{1,2\}$. We will attempt to update the partition by moving a constant number (depending only on $t$) of \emph{special} vertices between $U_1$ and $U_2$, denote them by $X$, and \emph{special} vertices between $V_1$ and $V_2$, denote them by $Y$, as well as partitioning the exceptional sets as $U_0=U_0^1\cup U_0^2$ and $V_0=V_0^1\cup V_0^2$.  Let $U_1^*$, $U_2^*$, $V_1^*$ and $V_2^*$ be the resulting sets after moving the special vertices. %in $X$ and $Y$.  
Our goal is to obtain two graphs, $G_1:=G[U_1^*\cup U_0^1, V_1^*\cup V_0^1]$ and $G_2:=[U_2^*\cup U_0^2, V_2^*\cup V_0^2]$ so that $G_1$ satisfies $$|U_1^*\cup U_0^1|=\ell_1(s+t)+as+bt, |V_1^*\cup V_0^1|=\ell_1(s+t)+bs+at$$ and $G_2$ satisfies $$|U_2^*\cup U_0^2|=\ell_2(s+t)+bs+at, |V_2^*\cup V_0^2|=\ell_2(s+t)+as+bt,$$ for some nonnegative integers $a,b,\ell_1,\ell_2$.  We tile $G_1$ as follows.  We find $a$ copies of $K_{s,t}$, each with $t$ vertices in $U_1^*$, so that each special vertex in $X\cap U_1^*$ is in a unique copy (some copies may not contain any special vertex). Also, we find $b$ copies of $K_{s,t}$, each with $t$ vertices in $V_1^*$ so that each special vertex in $Y\cap V_1^*$ is in a unique copy (some copies may not contain any special vertex).  Note that we only move vertices which will make this step possible. Deleting these $a+b$ copies of $K_{s,t}$ from $G_1$ gives us a balanced bipartite graph on $2\ell_1(s+t)$ vertices.  As noted in \cite{Z} and \cite{HS}, this graph can easily be tiled:  By Claim \ref{bounds} there are at most $\alpha^{2/3}\frac{n}{2}$ exceptional vertices in $U_0^1$ (resp.~$V_0^1$), each with degree at least $(\alpha^{1/3}-\alpha^{2/3})\frac{n}{2}$ to $V_1$ (resp.~$U_1$), so they may greedily be incorporated into unique copies of $K_{s+t,s+t}$.  The remaining graph is still balanced, divisible by $s+t$, and almost complete, thus can be tiled.

So if we are able to split $G$ into graphs $G_1$ and $G_2$ as detailed above, we will conclude that $G$ can be tiled.  However, if it is not possible to carry out this goal, then we will use an alternate method which is explained in Case 2.

\begin{proof}[Proof of Theorem 1.6]
There are two main cases.

\noindent
\textbf{Case 1.} $\max\{|U_1|,|U_2|,|V_1|,|V_2|\}\geq k(s+t)+\frac{t+1}{2}$.  Without loss of generality, suppose $|U_1|=\max\{|U_1|,|U_2|,|V_1|,|V_2|\}$. 

\textbf{Case 1.1.} $|V_2\cup V_0|\geq k(s+t)+s$.  We apply Claim \ref{Stars} to $G[U_1, V_2]$ with $A=V_2$ and $B=U_1$ to obtain $|U_1|-(k(s+t)+s)$ vertex disjoint $s$-stars with centers $C_U\subseteq U_1$ and leaves in $V_2$ and a set of $\max\{0, |V_2|-(k(s+t)+s)\}$ vertex disjoint $s$-stars with centers $C_V\subseteq V_2$ and leaves in $U_1$.  We move the vertices in $C_U$ to $U_2$ and the vertices in $C_V$ to $V_1$.  If $|V_2|<k(s+t)+s$, we choose $V_0'\subseteq V_0$ so that $|(V_2\cup V_0)\setminus V_0')|=k(s+t)+s$ otherwise we set $V_0'=\emptyset$. Then $G_1:=G[U_1\setminus C_U, V_1\cup C_V\cup V_0']$ satisfies $$|U_1|-|C_U|=k(s+t)+s, |V_1|+|V_0'|+|C_V|=k(s+t)+t,$$ and $G_2:=G-G_1$ satisfies $$|U_2\cup U_0|+|C_U|=k(s+t)+t, |V_2|+|V_0\setminus V_0'|-|C_V|=k(s+t)+s.$$  Thus $G_1$ and $G_2$ can be tiled, which completes the tiling of $G$.

\textbf{Case 1.2.} $|V_2\cup V_0|<k(s+t)+s$.  

This implies $|V_1|> k(s+t)+t$.  So we apply Claim \ref{Stars} to $G[V_1,U_2]$ with $A=U_2$ and $B=V_1$ to obtain a set of $|V_1|-k(s+t)$ vertex disjoint $s$-stars with centers $C_V\subseteq V_1$ and leaves in $U_2$. Likewise we apply Claim \ref{Stars} to $G[U_1,V_2]$ with $A=V_2$ and $B=U_1$ to obtain a set of $|U_1|-k(s+t)$ vertex $s$-stars with centers $C_U\subseteq U_1$ and leaves in $V_2$.  We move the vertices in $C_U$ to $U_2$ and the vertices in $C_V$ to $V_2$.  Then $G_1:=G[U_1\setminus C_U, V_1\setminus C_V]$ satisfies $$|U_1|-|C_U|=k(s+t), |V_1|-|C_V|=k(s+t)$$ and $G_2:=G-G_1$ satisfies $$|U_2\cup U_0|+|C_U|=(k+1)(s+t), |V_2\cup V_0|+|C_V|=(k+1)(s+t).$$  Thus $G_1$ and $G_2$ can be tiled, which completes the tiling of $G$.


\noindent
\textbf{Case 2.} $\max\{|U_1|,|U_2|,|V_1|,|V_2|\}\leq k(s+t)+\frac{t}{2}$. Note that this implies $|U_0|, |V_0|\geq s$. 

\textbf{Case 2.1.} $\max\{|\tilde{U}_1|, |\tilde{U}_2|, |\tilde{V}_1|, |\tilde{V}_2|\}\geq \frac{n}{4}$ (see Definition \ref{tildehat}).  Without loss of generality we can assume $|\tilde{U}_1|=\max\{|\tilde{U}_1|, |\tilde{U}_2|, |\tilde{V}_1|, |\tilde{V}_2|\}$.    
Set $h:=\ceiling{t/(2s)}$. Since $|\tilde{U}_1|>\frac{n}{4}$ and $\frac{1}{8s\alpha^{1/3}}\geq (h-1)(s+t)$, we can apply Claim \ref{TildeStars} to $G[\tilde{U}_1, V_2]$ with $c=0$ to obtain a set of $(h-1)(s+t)$ vertex disjoint $s$-stars with centers $C_U\subseteq \tilde{U}_1$ and leaves in $V_2$. We first move the vertices in $C_U$ from $\tilde{U}_1$ to $U_2$.  Then since $$\frac{t}{2}=s\frac{t}{2s}\leq sh\leq s\frac{t+2s-1}{2s}=\frac{t}{2}+s-\frac{1}{2},$$ we can choose sets $U_0'\subseteq U_0$ with $|U_0'|=k(s+t)+\floor{t/2}-|U_1|+sh-\floor{t/2}$ and $V_0'\subseteq V_0$ with $|V_0'|=k(s+t)+\floor{t/2}-|V_1|+s+\ceiling{t/2}-sh$ so that $G_1:=G[(U_1\cup U_0')\setminus C_U, V_1\cup V_0']$ satisfies $$|U_1|+|U_0'|-|C_U|=(k-h+1)(s+t)+hs, |V_1|+|V_0'|=(k-h+1)(s+t)+ht,$$
and $G_2:=G-G_1$ satisfies $$|U_2|+|U_0\setminus U_0'|+|C_U|=k(s+t)+ht, |V_2|+|V_0\setminus V_0'|= k(s+t)+hs.$$  Thus $G_1$ and $G_2$ can be tiled, which completes the tiling of $G$.

\textbf{Case 2.2.} $\max\{|\tilde{U}_1|, |\tilde{U}_2|, |\tilde{V}_1|, |\tilde{V}_2|\}< \frac{n}{4}$.  Thus for $i=1,2$, we have $$|\hat{U_i}|, |\hat{V_i}|\geq (1-\alpha^{2/3})\frac{n}{2}-\frac{n}{4}\geq \frac{n}{8}.$$  So we may apply Claim \ref{K_1} to obtain the two special copies of $K_{s,t}$, $K^1$ and $K^2$.  Note that $|U_i\setminus V(K^1)|$, $|V_i\setminus V(K^2)|\leq k(s+t)$ for $i=1,2$.  Let $U_0'=U_0\setminus V(K^2)$ and $V_0'=V_0\setminus V(K^1)$.  We remove the graphs $K^1$ and $K^2$, then we partition the vertices $U_0'=U_0^1\cup U_0^2$ and $V_0'=V_0^1\cup V_0^2$ so that $G_1:=G[(U_1\cup U_0^1)\setminus V(K^1), (V_1\cup V_0^1)\setminus V(K^2)]$ satisfies $$|U_1|-\ceiling{t/2}+|U_0^1|=k(s+t), |V_1|-\ceiling{t/2}+|V_0^1|=k(s+t)$$ and $G_2=G-G_1-K^1-K^2$ satisfies $$|U_2|-\floor{t/2}+|U_0^2|=k(s+t), |V_2|-\floor{t/2}+|V_0^2|=k(s+t).$$  Thus $G_1$ and $G_2$ can be tiled, so along with $K^1$ and $K^2$, this completes the tiling of $G$.

\end{proof}

\section{Tightness}
\label{lower bound}

In this section we will prove Proposition \ref{counterexample}.  We will need to use the graphs $P(m,p)$, where $m,p\in\mathbb{N}$, introduced by Zhao in \cite{Z}.

\begin{lemma}\label{No K22}

For all $p\in\mathbb{N}$ there exists $m_0$ such that for all $m\in \mathbb{N}$, $m>m_0$, there exists a balanced bipartite graph, $P(m,p)$, on $2m$ vertices, so that the following hold:
\begin{enumerate}
\item $P(m,p)$ is $p$-regular

\item $P(m,p)$ does not contain a copy of $K_{2,2}$.
\end{enumerate}

\end{lemma}

\begin{proof}[Proof of Proposition \ref{counterexample}]

Let $G[U, V]$ be a balanced bipartite graph on $2n$ vertices satisfying the following conditions. Let $n=(2k+1)(s+t)$ for some sufficiently large $k$ (as determined by Lemma \ref{No K22} with $p=s-1$).  Partition $U$ into $U=U_0\cup U_1\cup U_2$ and partition $V$ into $V=V_0\cup V_1\cup V_2$ where, $|U_1|=|V_2|=k(s+t)+\floor{\frac{t+1}{2}}$, $|V_1|=|U_2|=k(s+t)+\ceiling{\frac{t+1}{2}}$ and $|U_0|=|V_0|=s-1$.  Let $G[U_i,V_i]$ be complete for $i\in\{1,2\}$, $G[U_1,V_2]\cong P\left(k(s+t)+\floor{\frac{t+1}{2}},s-1\right)$ and $G[U_2,V_1]\cong P\left(k(s+t)+\ceiling{\frac{t+1}{2}},s-1\right)$.  Let $G[U_0,V_1\cup V_2]$ be complete, $G[V_0,U_1\cup U_2]$ be complete and $G[U_0,V_0]$ be empty.  Note that 
$$\delta(G)= \left\lbrace \begin{array}{ll}\frac{n+3s}{2}-\frac{3}{2}   & \text{ if } t \text{ is odd } \\
              \frac{n+3s}{2}-2 & \text{ if } t \text{ is even. } \end{array} \right. $$
Finally we reiterate the following properties of $G[U_1, V_2]$ and $G[U_2, V_1]$.  For $i=1,2$,
\begin{equation}\label{s-1}
\Delta(U_i,V_{3-i})=\Delta(V_i, U_{3-i})=s-1
\end{equation}
and
\begin{equation}\label{K_{2,2}}
G[U_i, V_{3-i}] \text{ is } K_{2,2}\text{-free}.
\end{equation}

For $i\in\{1,2\}$ and $A\in \{U_i,V_i\}$, let $A^D:=V_{3-i}$ if $A=U_i$ and let $A^D:=U_{3-i}$ if $A=V_i$.  We call $A^D$ the \emph{diagonal set of $A$}.  Let $A^N:=V_i$ if $A=U_i$ and $A^N:=U_i$ if $A=V_i$.  We call $A^N$ the \emph{non-diagonal set of $A$}. Finally, we let $A^M:=V_0$ if $A=U_i$ and $A^M:=U_0$ if $A=V_i$.  We call $A^M$ the \emph{opposite middle set of $A$}.

Suppose $K\cong K_{s,t}$ is a subgraph of $G$. We say $K$ is a \emph{crossing $K_{s,t}$} if $V(K)\cap(U_1\cup V_1)\neq \emptyset$ and $V(K)\cap(U_2\cup V_2)\neq \emptyset$.  Let $\mathcal{W}=\{U_1, U_2, V_1, V_2\}$.  

\begin{claim}\label{claim:properties}
If $K$ is a crossing $K_{s,t}$, then 
\begin{enumerate}
\item $V(K)$ must intersect some member of $\mathcal{W}$ in exactly one vertex, and

\item there is a unique $A_0\in \{U_0, V_0\}$ such that $V(K)\cap A_0\neq \emptyset$.
\end{enumerate}
Furthermore, if $|V(K)\cap A|=1$ for some $A\in \mathcal{W}$, then 
\begin{enumerate}[resume]
\item $V(K)\cap A^D\neq \emptyset$, and

\item either $|V(K)\cap A^N|\geq 2$ and $V(K)\cap (A^N)^D=\emptyset$, or $V(K)\cap A^N=\emptyset$ and $|V(K)\cap (A^N)^D|\geq 2$.

\end{enumerate}

\end{claim}

\begin{proof}
\begin{enumerate}
\item Suppose not. Then without loss of generality, suppose that $|V(K)\cap V_1|\geq 2$. By (\ref{K_{2,2}}) we have, $|V(K)\cap U_2|\leq 1$ and thus $V(K)\cap U_2=\emptyset$.  Since $K$ is crossing, we have $V(K)\cap V_2\neq \emptyset$ and thus $|V(K)\cap V_2|\geq 2$. By (\ref{K_{2,2}}) we have, $|V(K)\cap U_1|\leq 1$ and thus $V(K)\cap U_1=\emptyset$.  This is a contradiction, since $K\cong K_{s,t}$ and $|V(K)\cap U|\leq |U_0|=s-1$.

\item Suppose first that $V(K)\cap U_0=\emptyset=V(K)\cap V_0$.  By Claim \ref{claim:properties} (i), we can assume without loss of generality that $|V(K)\cap U_1|=1$.  Then either $|V(K)\cap U_2|=t-1$ or $|V(K)\cap U_2|=s-1$.  If $|V(K)\cap U_2|=t-1$, then by (\ref{s-1}) we must have $V(K)\cap V_1=\emptyset$ which implies $|V(K)\cap V_2|=s$, contradicting (\ref{s-1}).  If $|V(K)\cap U_2|=s-1$, then since $t\geq 2s+1$ we have $|V(K)\cap V_1|\geq s+1$ or $|V(K)\cap V_2|\geq s+1$, both of which contradict (\ref{s-1}).  Thus there exists $A_0\in \{U_0, V_0\}$ such that $V(K)\cap A_0\neq \emptyset$.  Finally since $G[U_0, V_0]$ is empty, $A_0$ must be unique.

\item Suppose that $V(K)\cap A^D=\emptyset$. Since $|V_0|=s-1$, we have $V(K)\cap A^N\neq \emptyset$ and since $K$ is crossing, we have $V(K)\cap (A^N)^D\neq \emptyset$.  Then by (\ref{s-1}), we have $|V(K)\cap A^N|, |V(K)\cap (A^N)^D|\leq s-1$.  Thus $|V(K)\cap U|\leq 2s-1$ and $|V(K)\cap V|\leq 2s-2$, contradicting the fact that $K\cong K_{s,t}$ and $t\geq 2s+1$.

\item We first show that it is not possible for either $|V(K)\cap A^N|=1$ or $|V(K)\cap (A^N)^D|=1$. If $|V(K)\cap A^N|=1$, then by (\ref{s-1}) and $|U_0|=|V_0|=s-1$, we have $|V(K)\cap U|, |V(K)\cap V|\leq 2s-1$, contradicting the fact that $K\cong K_{s,t}$ and $t\geq 2s+1$.  So suppose $|V(K)\cap (A^N)^D|=1$.  If $V(K)\cap U_0=\emptyset$, then $|V(K)\cap U|=2$ and since $t\geq 3$ we must have $s=2$. Then by (\ref{s-1}) we have $|V(K)\cap V|\leq 3$ contradicting the fact that $K\cong K_{s,t}$ and $t\geq 2s+1$.  If $V(K)\cap U_0\neq \emptyset$, then $V(K)\cap V_0=\emptyset$.  So $|V(K)\cap U|\leq s+1$ and by (\ref{s-1}), $|V(K)\cap V|\leq 2s-2$ contradicting the fact that $K\cong K_{s,t}$ and $t\geq 2s+1$.

Now suppose $V(K)\cap A^N\neq\emptyset$ and $V(K)\cap (A^N)^D\neq\emptyset$. Thus, by the previous paragraph we have $|V(K)\cap A^N|, |V(K)\cap (A^N)^D|\geq 2$, contradicting (\ref{K_{2,2}}).

So suppose that $V(K)\cap A^N=\emptyset=V(K)\cap (A^N)^D$.  Then it must be the case that $|V(K)\cap (A^N)^M|=s-1$ and consequently $|V(K)\cap A^D|=t$, contradicting (\ref{s-1}).

\end{enumerate}

\end{proof}

Let $A\in\mathcal{W}$.  We say $K$ is \emph{crossing from $A$} if either $|V(K)\cap A|=1$ and $|V(K)\cap A^D|\geq 2$, or $|V(K)\cap A|=1$, $|V(K)\cap A^D|=1$ and $V(K)\cap A^M\neq \emptyset$.  We say that a crossing $K_{s,t}$ from $A$ is \emph{Type 1} if $|V(K)\cap (A^N)^M|=s-1$, $|V(K)\cap A^N|=t-p$ and $|V(K)\cap A^D|=p$ for some $2\leq p\leq s-1$.  We say that a crossing $K_{s,t}$ from $A$ is \emph{Type 2} if $|V(K)\cap (A^N)^D|=t-1$, $|V(K)\cap A^M|=s-p$, and $|V(K)\cap A^D|=p$ for some $1\leq p\leq s-1$.



\begin{claim}\label{types}
Every crossing $K_{s,t}$ is either Type 1 or Type 2.
\end{claim}

\begin{proof} (See Figure 1) Let $K$ be a crossing $K_{s,t}$ and without loss of generality suppose $K$ is crossing from $U_1$.  Let $p:=|V(K)\cap V_2|$. By Claim \ref{claim:properties} (iii) and (\ref{s-1}) we have $1\leq p\leq s-1$.  Suppose $K$ is not Type 1.  If $V(K)\cap U_2=\emptyset$, then $|V(K)\cap U_0|=s-1$ which implies $V(K)\cap V_0=\emptyset$ by Claim \ref{claim:properties} (ii).  Since $K$ is not Type 1, it must be the case that $|V(K)\cap V_2|=1$ and $|V(K)\cap V_1|=t-1$ in which case $K$ is not crossing from $U_1$, contradicting our assumption.  So we suppose that $V(K)\cap U_2\neq \emptyset$.  By Claim \ref{claim:properties} (iv) we have $|V(K)\cap U_2|\geq 2$ and $V(K)\cap V_1=\emptyset$, which implies that $|V(K)\cap V_0|=s-p$.  So by Claim \ref{claim:properties} (ii), we have $V(K)\cap U_0=\emptyset$ and thus $|V(K)\cap U_2|=t-1$, so $K$ is Type 2.

\end{proof}

Suppose for a contradiction that $G$ can be tiled with $K_{s,t}$.  Let $\mathcal{F}$ be a tiling of $G$ which minimizes the number of crossing $K_{s,t}$'s.


\begin{claim}\label{forbidden}
For $i=1,2$, if there is a crossing $K_{s,t}$ of Type $2$ from $U_i$ or $V_i$, then there is no crossing $K_{s,t}$ of Type $2$ from $U_{3-i}$ or $V_{3-i}$.
\end{claim}

\begin{proof}
Without loss of generality suppose $K^1$ is a crossing $K_{s,t}$ of Type $2$ from $U_1$.  Suppose that $K^2$ is a crossing $K_{s,t}$ of Type $2$ from $U_2$ (See Figure 2). For $i\in\{1,2\}$, let 
$$K^i_*:=G[U_i\cap (V(K^1)\cup V(K^2)), V(K^{3-i})\cap (V_0\cup V_i)].$$  
We have $K^1_*\cong K_{s,t} \cong K^2_*$, neither of $K^1_*,K^2_*$ are crossing, and $V(K^1)\cup V(K^2)=V(K^1_*)\cup V(K^2_*)$. Thus we obtain a tiling with fewer crossing $K_{s,t}$'s, contradicting the minimality of $\mathcal{F}$.

Now, suppose $K^1$ is a crossing $K_{s,t}$ of Type $2$ from $U_1$ and $K^2$ is a crossing $K_{s,t}$ of Type $2$ from $V_2$ (See Figure 2). Specify an element $L^1\in\mathcal{F}$, such that $V(L^1)\subseteq U_1\cup V_1$ and $|V(L^1)\cap V_1|=t$ and specify an element $L^2\in\mathcal{F}$, such that $V(L^2)\subseteq U_2\cup V_2$ and $|V(L^2)\cap U_2|=t$.  Choose arbitrary vertices $v'\in V(K^1)\cap V_0$ and $u'\in V(K^2)\cap U_0$.  We now define four subgraphs of $G$.  Let
\begin{align*}
K^1_*:&=G[V(L^1)\cap V_1, (V(K^1)\cup V(K^2))\cap ((U_1\cup U_0)\setminus\{u'\})],\\
L^1_*:&=G[V(L^1)\cap U_1, (V(K^2)\cap V_1)\cup\{v'\}],\\
K^2_*:&=G[V(L^2)\cap U_2, (V(K^1)\cup V(K^2))\cap ((V_2\cup V_0)\setminus\{v'\})], \text{ and}\\
L^2_*:&=G[V(L^2)\cap V_1, (V(K^1)\cap U_2)\cup\{u'\}]. 
\end{align*}

All of $K^1_*,K^2_*,L^1_*,L^2_*$ are isomorphic to $K_{s,t}$, none of $K^1_*,K^2_*,L^1_*,L^2_*$ are crossing, and $V(K^1_*)\cup V(K^2_*)\cup V(L^1_*)\cup V(L^2_*)=V(K^1)\cup V(K^2)\cup V(L^1)\cup V(L^2)$. Thus we obtain a tiling with fewer crossing $K_{s,t}$'s, contradicting the minimality of $\mathcal{F}$.
\end{proof}

For $i\in\{1,2\}$, let $\mathcal{F}_i$ be the set of all copies of $K_{s,t}$ in $\mathcal{F}$ which touch $U_i\cup V_i$. And let $U_i^*$ (resp.~$V_i^*$) be all the vertices in $U$ (resp.~$V$) which touch elements of $\mathcal{F}_i$.  Precisely, let $\mathcal{F}_i=\{K\in \mathcal{F}: V(K)\cap(U_i\cup V_i)\neq\emptyset\}$ for $i=1,2$, and let 
\begin{align*}
U_i^*=\left(\cup_{K\in\mathcal{F}_i}V(K)\right)\cap U ~~\text{ and }~~ V_i^*=\left(\cup_{K\in\mathcal{F}_i}V(K)\right)\cap V.
\end{align*}
Note that $U_i\subseteq U_i^*$ and $V_i\subseteq V_i^*$.  We will use the following claim to show that all of the remaining possible configurations of crossing $K_{s,t}$'s lead to contradictions.

\begin{claim}\label{claim:main}
For all $i\in\{1,2\}$, either $$\max\{|U_i^*|, |V_i^*|\}\geq k(s+t)+2t ~\text{ or }~ \min\{|U_i^*|, |V_i^*|\}\geq (k+1)(s+t).$$
\end{claim}

\begin{proof}
Suppose that $\max\{|U_i^*|, |V_i^*|\}< k(s+t)+2t$.  Then since $U_i\subseteq U_i^*$ and $V_i\subseteq V_i^*$, we have \begin{equation}\label{bounded} k(s+t)+s<|U_i^*|, |V_i^*|<k(s+t)+2t,\end{equation} and thus \begin{equation}\label{2t-s}||U_i^*|-|V_i^*||<2t-s.\end{equation}  By definition $G[U_i^*, V_i^*]$ can be tiled, thus there exists nonnegative integers $\ell, a, b$ such that $|U_i^*|=\ell(s+t)+as+bt$ and $|V_i^*|=\ell(s+t)+at+bs$.  By choosing $\ell$ to be maximal, we have $a=0$ or $b=0$.  
If $\ell\leq k-1$, then in order to satisfy the lower bound in \eqref{bounded} we must have $a\geq 3$ or $b\geq 3$.  Since $a=0$ or $b=0$, we have $||U_i^*|-|V_i^*||\geq 3t-3s\geq 2t-s$, which contradicts \eqref{2t-s}.  If $\ell=k$, then in order to satisfy the lower bound in \eqref{bounded}, we must have $a\geq 2$ or $b\geq 2$, but then we violate the upper bound.  So $\ell\geq k+1$ and we have $\min\{|U_i^*|, |V_i^*|\}\geq (k+1)(s+t)$.
\end{proof}

We will also use the following facts.  For $i=1,2$, we have
\begin{equation}\label{eq:upper}
|V_i\cup V_0|+s, |U_i\cup U_0|+s\leq k(s+t)+\frac{t+2}{2}+2s-1<(k+1)(s+t).
\end{equation}
which in particular implies
\begin{equation}\label{eq:upper2t}
|V_i\cup V_0|+t, |U_i\cup U_0|+t<k(s+t)+2t.
\end{equation}


Let $i\in\{1,2\}$ and let $X_i=\{K\in\mathcal{F}:K \text{ is crossing from } U_i \text{ and } K \text{ is Type } 2\}$ and $Y_i=\{K\in\mathcal{F}:K \text{ is crossing from } V_i \text{ and } K \text{ is Type } 2\}$.  Since $|U_0|=|V_0|=s-1$, Claim \ref{claim:properties} (ii) implies, \begin{equation}\label{X_iY_i}0\leq|X_i|,|Y_i|\leq s-1. \end{equation}


\noindent
\textbf{Case 0.} There are no crossing $K_{s,t}$'s.  So $|U_1^*|\leq |U_1\cup U_0|$ and $|V_1^*|\leq |V_1\cup V_0|$.  Then by (\ref{eq:upper}) we have $|U_1^*|, |V_1^*|<(k+1)(s+t)$, contradicting Claim \ref{claim:main}.

\noindent
\textbf{Case 1.} There is a crossing $K_{s,t}$ of Type $1$.  Without loss of generality, suppose $K^1$ is a crossing $K_{s,t}$ of Type $1$ from $U_1$ and let $p:=|V(K^1)\cap V_2|$.  Since $U_0\setminus V(K^1)=\emptyset$, there can be no other crossing $K_{s,t}$'s of Type 1 from $U_1$ or $U_2$ and no crossing $K_{s,t}$'s of Type 2 from $V_1$ or $V_2$. By Claim \ref{types}, we must only consider five subcases:

\textbf{Case 1.0.} $K^1$ is the only crossing $K_{s,t}$. So $|U_1^*|\leq |U_1\cup U_0|$ and $|V_1^*|\leq |V_1\cup V_0|+p< |V_1\cup V_0|+s$.  Then by (\ref{eq:upper}) we have $|U_1^*|, |V_1^*|<(k+1)(s+t)$, contradicting Claim \ref{claim:main}.

\textbf{Case 1.1.i.} There is a crossing $K_{s,t}$ of Type $1$ from $V_1$. Let $K^2$ be a crossing $K_{s,t}$ from $V_1$ and let $q:=|V(K^2)\cap U_2|$.  Since $V_0\setminus V(K^2)=\emptyset$, $K^1$ and $K^2$ are the only crossing $K_{s,t}$'s.  So $|U_1^*|\leq |U_1\cup U_0|+q< |U_1\cup U_0|+s$ and $|V_1^*|\leq |V_1\cup V_0|+p< |V_1\cup V_0|+s$.  Then by (\ref{eq:upper}) we have, $|U_1^*|, |V_1^*|<(k+1)(s+t)$, contradicting Claim \ref{claim:main}.

\textbf{Case 1.1.ii.} There is a crossing $K_{s,t}$ of Type $1$ from $V_2$. Let $K^2$ be a crossing $K_{s,t}$ from $V_2$ and let $q:=|V(K^2)\cap U_1|$.  Since $V_0\setminus V(K^2)=\emptyset$, $K^1$ and $K^2$ are the only crossing $K_{s,t}$'s.  So $|V_1^*|\leq |V_1\cup V_0|+p+1\leq |V_1\cup V_0|+s$ and $|U_1^*|\leq |U_1\cup U_0|+t-q<|U_1\cup U_0|+t$.  Then by (\ref{eq:upper}) and (\ref{eq:upper2t}) we have $|V_1^*|<(k+1)(s+t)$ and $|U_1^*|<k(s+t)+2t$, contradicting Claim \ref{claim:main}.

\textbf{Case 1.2.i.} $1\leq |X_1|$. By Claim \ref{forbidden}, since there exists a crossing $K_{s,t}$ of Type $2$ from $U_1$, there can be no crossing $K_{s,t}$'s of Type $2$ from $U_2$.  So $|U_2^*|\leq |U_2\cup U_0|+|X_1|+1\leq |U_2\cup U_0|+s$ and $|V_2^*|\leq |V_2\cup V_0|+t-p<|V_2\cup V_0|+t$.  Then by (\ref{eq:upper}) and (\ref{eq:upper2t}) we have $|U_2^*|<(k+1)(s+t)$ and $|V_2^*|<k(s+t)+2t$, contradicting Claim \ref{claim:main}.

\textbf{Case 1.2.ii.} $1\leq |X_2|$. By Claim \ref{forbidden}, since there exists a crossing $K_{s,t}$ of Type $2$ from $U_2$, then there can be no crossing $K_{s,t}$'s of Type $2$ from $U_1$.  So $|U_1^*|\leq |U_1\cup U_0|+|X_2|<|U_1\cup U_0|+s$ and $|V_1^*|\leq |V_1\cup V_0|+p<|V_1\cup V_0|+s$.  Then by (\ref{eq:upper}) we have $|U_1^*|, |V_1^*|<(k+1)(s+t)$, contradicting Claim \ref{claim:main}.

\noindent
\textbf{Case 2.} There are no crossing $K_{s,t}$'s of Type $1$.  By Claim  \ref{types}, there can only be crossing $K_{s,t}$'s of Type $2$.  Without loss of generality suppose that $1\leq |X_1|$.  Then there can be no crossing $K_{s,t}$ of Type $2$ from $U_2$ or $V_2$. So $|U_2^*|\leq |U_2\cup U_0|+|X_1|<|U_2\cup U_0|+s$ and $|V_2^*|\leq |V_2\cup V_0|+|Y_1|<|V_2\cup V_0|+s$.  Then by (\ref{eq:upper}) we have $|U_2^*|, |V_2^*|<(k+1)(s+t)$, contradicting Claim \ref{claim:main}.

\end{proof}


\chapter{TILING IN BIPARTITE GRAPHS: ASYMMETRIC MINIMUM DEGREES}\label{sumdegtilingchapter}

\DoubleSpacing
\setlength{\parindent}{.5in}

This chapter is joint work with Andrzej Czygrinow.


\section{Introduction}

If $G$ is a graph on $n=sm$ vertices, $H$ is a graph on $s$ vertices and $G$ contains $m$ vertex disjoint copies of $H$, then we say $G$ can be \emph{tiled} with $H$.  We now state two important tiling results which motivate the current research.

\begin{theorem}[Hajnal-Szemer\'edi \cite{HSz}]
Let $G$ be a graph on $n=sm$ vertices.  If $\delta(G)\geq (s-1)m$, then $G$ can be tiled with $K_s$.
\end{theorem}

Kierstead and Kostochka generalized, and in doing so slightly improved, the result of Hajnal and Szemeredi.

\begin{theorem}[Kierstead-Kostochka \cite{KK}]
Let $G$ be a graph on $n=sm$ vertices.  If $\deg(x)+\deg(y)\geq 2(s-1)m-1$, for all non-adjacent $x,y\in V(G)$ then $G$ can be tiled with $K_s$.
\end{theorem}

Both of these results can be shown to be best possible relative to the respective degree condition, i.e. no smaller lower bound on the degree will suffice.

For the rest of the paper we will consider tiling in bipartite graphs.  Given a bipartite graph $G[U,V]$ we say $G$ is balanced if $|U|=|V|$.  The following theorem is a consequence of Hall's matching theorem, and is an early result on bipartite graph tiling.

\begin{theorem}\label{Hall1}
Let $G$ be a balanced bipartite graph on $2n$ vertices.  If $\delta(G)\geq \frac{n}{2}$, then $G$ can be tiled with $K_{1,1}$.
\end{theorem}

Zhao determined the best possible minimum degree condition for a bipartite graph to be tiled with $K_{s,s}$ when $s\geq 2$.

\begin{theorem}[Zhao \cite{Z}]\label{Zhao theorem}
For each $s\geq 2$, there exists $m_0$ such that the following holds for all $m\geq m_0$.  If $G$ is a balanced bipartite graph on $2n=2ms$ vertices with 
$$\delta(G)\geq \left\lbrace \begin{array}{ll} \frac{n}{2}+s-1   & \text{ if } m \text{ is even } \\
              \frac {n+3s}{2}-2 & \text{ if } m \text{ is odd, } \end{array} \right. $$
then $G$ can be tiled with $K_{s,s}$.
\end{theorem}



Hladk\'y and Schacht, and Czygrinow and DeBiasio determined the best possible minimum degree condition for a balanced bipartite graph to be tiled with $K_{s,t}$.

\begin{theorem}[Hladk\'y, Schacht \cite{HS}; Czygrinow, DeBiasio \cite{CD}]
For each $t>s\geq 1$, there exists $m_0$ such that the following holds for all $m\geq m_0$.  If $G$ is a balanced bipartite graph on $2n=2m(s+t)$ vertices with 
$$\delta(G)\geq \left\lbrace \begin{array}{ll} \frac{n}{2}+s-1   & \text{ if } m \text{ is even } \\
               \frac {n+t+s}{2}-1 & \text{ if } m \text{ is odd and } t\leq 2s \\
               \frac {n+3s}{2}-1 & \text{ if } m \text{ is odd and } t\geq 2s+1  \end{array} \right. $$
then $G$ can be tiled with $K_{s,s}$.
\end{theorem}


Now we consider a more general degree condition than $\delta(G)$.  Given a bipartite graph $G[U,V]$, let $\delta_U(G):=\min\{\deg(u): u\in U\}$ and $\delta_V(G):=\min\{\deg(v): v\in V\}$.  We will write $\delta_U$ and $\delta_V$ instead of $\delta_U(G)$ and $\delta_V(G)$ when it is clear which graph we are referring to.  The following theorem is again a consequence of Hall's matching theorem and is more general than Theorem \ref{Hall1}.

\begin{theorem}\label{Hall2}
Let $G[U,V]$ be a balanced bipartite graph on $2n$ vertices.  If $\delta_U+\delta_V\geq n$, then $G$ can be tiled with $K_{1,1}$.
\end{theorem}

Notice that when $s=2$, Theorem \ref{Zhao theorem} says that if $G[U,V]$ is a balanced bipartite graph on $2n$ vertices with $\delta(G)\geq \frac{n}{2}+1$, then $G$ can be tiled with $K_{2,2}$.  Based on this, one might guess that the optimal value of $\delta_U+\delta_V$ which implies that $G$ can be tiled with $K_{2,2}$ is $\delta_U+\delta_V\geq n+2$.   In fact, Wang made the following conjecture about $2$-factors in bipartite graphs.

\begin{conjecture}[Wang \cite{W2}]
\label{con:W2}
Let $G[U,V]$ and $H$ be balanced bipartite graphs on $2n$ vertices.  If $\delta_U+\delta_V\geq n+2$ and $\Delta(H)\leq 2$, then $H\subseteq G$.
\end{conjecture}

Czygrinow, DeBiasio, and Kierstead \cite{CDK} proved Wang's conjecture when $\delta_V\geq \delta_U=\Omega(n)$.  However, setting $s=2$ in Theorems \ref{main 1} and \ref{main 2}, which are stated below, we obtain the result that if $G[U,V]$ is a balanced bipartite graph on $2n$ vertices with $\delta_U+\delta_V\geq n+1$ and $\delta_V\geq \delta_U=\Omega(n)$, then $G$ can be tiled with $K_{2,2}$.

We prove the following theorems which will generalize the results in \cite{Z} for all $s\geq 2$.%, \cite{HS} and \cite{CD}.

\begin{theorem}\label{main 1}
For each $s\geq 2$ and $\lambda\in (0,\frac{1}{2})$, there exists $m_0$ such that the following holds for all $m\geq m_0$.  If $G[U,V]$ is a balanced bipartite graph on $2n=2ms$ vertices with $\delta_V\geq\delta_U\geq \lambda n$ and $\delta_U+\delta_V\geq n+3s-5$
then $G$ can be tiled with $K_{s,s}$.
\end{theorem}

As mentioned earlier, Zhao gave examples which shows that Theorem \ref{Zhao theorem} is best possible.

\begin{proposition}[Zhao \cite{Z}]\label{Zhao prop}
Let $s\geq 2$, and $n=ms\geq 64s^2$.  There exists a balanced bipartite graph, $G$, on $2n$ vertices with
$$\delta(G)= \left\lbrace \begin{array}{ll} \frac{n}{2}+s-2   & \text{ if } m \text{ is even } \\
             \frac {n+3s}{2}-3 & \text{ if } m \text{ is odd } \end{array} \right. $$
such that $G$ cannot be tiled with $K_{s,s}$.
\end{proposition}

Since there are examples with $\delta(G)=\frac {n+3s}{2}-3$ such that $G$ cannot be tiled with $K_{s,s}$, this implies that there are examples with $\delta_U+\delta_V=2\delta(G)=n+3s-6$ which cannot be tiled with $K_{s,s}$.  This shows that the degree condition in Theorem \ref{main 1} is best possible.  Notice that Theorem \ref{Zhao theorem} gives a better bound on $\delta(G)$ when $m$ is even, which might lead you to guess that $\delta_U+\delta_V\geq n+2s-3$ suffices when $m$ is even (based on Theorem \ref{main 1}).  However, we show that when $m$ is even (or odd) there are graphs with $\delta_U(G)+\delta_V(G)=n+3s-7$ that cannot be tiled with $K_{s,s}$.

\begin{proposition}\label{meven}
Let $s\geq 2$.  For every $j\in \mathbb{N}$, there exists an integer $m$ and a balanced bipartite graph $G[U,V]$ on $2n=2ms$ vertices such that $\delta_U+\delta_V=n+3s-7$ and $2sj-s-1\leq |\delta_V-\delta_U|\leq 2sj-1$, but $G$ cannot be tiled with $K_{s,s}$.
\end{proposition}

Surprisingly, we show that when $\delta_U$ is significantly smaller than $\delta_V$, a smaller sum of degrees will suffice to tile $G$ with $K_{s,s}$, provided $\delta_V\geq \delta_U =\Omega(n)$.  First we must give a definition which allows us to precisely state our result.

We make use of the following fact to split the positive integers into two classes.
\begin{fact}
Let $s$ be a positive integer.  There exists unique $p,q\in \mathbb{N}$ such that $s=p^2+q$ and $0\leq q\leq 2p$.
\end{fact}
Using this fact, we define a function which classifies positive integers depending on their value of $q$.
\begin{definition}
Let $c$ be a function from $\mathbb{Z}^+$ to $\{0,1\}$ such that 
$$c(s) = \left\lbrace \begin{array}{ll} 0   & \text{ if } q=0 ~\text{ or }~ p+1\leq q\leq 2p\\
               1 & \text{ if } 1\leq q\leq p  \end{array} \right. $$
\end{definition} 

\begin{theorem}\label{main 2}
Let $s\geq 2$ and $\lambda\in (0, \frac{1}{2})$.
%and let $p\in \mathbb{N}$ be maximal such that $s=p^2+q$ where $0\leq q\leq 2p$.  Set $c(s)=0$ if $q=0$ or $p+1\leq q\leq 2p$ and set $c(s)=1$ if $1\leq q\leq p$. 
There exists $m_0$ such that the following holds for all $m\geq m_0$.  Let $G$ be a balanced $U,V$-bigraph on $2n=2ms$ vertices with $\delta_V\geq\delta_U\geq \lambda n$, $\delta_U=k_1s+s+r$ for some $0\leq r\leq s-1$, $k_2=m-k_1$, $k_1\leq (1-\frac{1}{2s})k_2$, and $0\leq d\leq s-2\croot{s}+c(s)+1$. If
%\begin{equation}
\begin{enumerate}
\item $\delta_U+\delta_V\geq n+3s-5$ or
\item $k_2\geq (s-d)k_1$ and $\delta_U+\delta_V\geq n+2s-2\croot{s}+d+c(s)$,
\end{enumerate}
%\end{equation}
then $G$ can be tiled with $K_{s,s}$.
\end{theorem}

We also give examples to show that the degree is tight when $d=0$ in the preceding theorem.

\begin{proposition}\label{k1smallexample}
For every $s\geq 2$, there exists a balanced bipartite graph $G$ with $k_2\geq sk_1$ and
$$\delta_U+\delta_V=n+2s-2\croot{s}+c(s)-1$$ such that $G$ cannot be tiled with $K_{s,s}$.  
%Furthermore...
\end{proposition}


% 
% Finally we show that after a certain point the degree condition cannot be relaxed any further.
% 
% \begin{theorem}\label{pre-extremal}
% For each $s\geq 2$, there exists $m_0$ such that the following holds for all $m\geq m_0$.  If $G$ is a balanced $U,V$-bigraph on $2n=2ms$ vertices with $\delta_U<\frac{n}{10s^3}$ and $\delta_U+\delta_V\geq n+2s-2\ceiling{\sqrt{s}}$
% then $G$ can be tiled with $K_{s,s}$.
% \end{theorem}
% 
% \begin{proposition}\label{pre-extremal_example}
% For each $s\geq 2$, there exists $m_0$ such that the following holds for all $m\geq m_0$.  There exists a balanced bipartite graph $G[U,V]$ on $2n=2ms$ vertices with $\delta_U<\frac{n}{10s^3}$ and $\delta_U+\delta_V= n+2s-2\ceiling{\sqrt{s}}-1$
% such that $G$ cannot be tiled with $K_{s,s}$.
% \end{proposition}

Finally, when $\delta_U$ is constant, we first construct two graphs with $\delta_U+\delta_V\geq n+2s-2\croot{s}+c(s)$ which cannot be tiled with $K_{s,s}$.  Then we show that there exists graphs (without constructing them) with $\delta_U+\delta_V$ much larger than $n+3s$ which cannot be tiled with $K_{s,s}$.

\begin{theorem}\label{probexample}
There exists $s_0, n_0\in \mathbb{N}$ such that for all $s\geq s_0$, there exists a graph $G[U, V]$ on $n\geq n_0$ vertices with $\delta_U+\delta_V\geq n+s^{s^{1/3}}$ such that $G$ cannot be tiled with $K_{s,s}$.
\end{theorem}


\section{Extremal Examples}


\subsection{Tightness when $k_2\approx k_1$}

As mentioned in the introduction, Zhao determined the optimal minimum degree condition so that $G$ can be tiled with $K_{s,s}$.  If $n$ is an odd multiple of $s$, then $\delta(G)\geq \frac{n}{2}+\frac{3s}{2}-2$ is best possible; however, if $n$ is an even multiple of $s$, then $\delta(G)\geq \frac{n}{2}+s-1$ is best possible.  In Theorem \ref{main 1} and Theorem \ref{main 2} we show that if $\delta_V\geq \delta_U=\Omega(n)$, then $\delta_U+\delta_V\geq n+3s-5$ suffices to give a tiling of $G$ with $K_{s,s}$.  We now give an example which shows that even when $n$ is an even multiple of $s$, we cannot improve the coefficient of the $s$ term in the degree condition.

We will need to use the graphs $P(m,p)$, where $m,p\in\mathbb{N}$, introduced by Zhao in \cite{Z}.

\begin{lemma}\label{Zhao:No K22}

For all $p\in\mathbb{N}$ there exists $m_0$ such that for all $m\in \mathbb{N}$, $m>m_0$, there exists a balanced bipartite graph, $P(m,p)$, on $2m$ vertices, so that the following hold:
\begin{enumerate}
\item $P(m,p)$ is $p$-regular

\item $P(m,p)$ does not contain a copy of $K_{2,2}$.
\end{enumerate}

\end{lemma}

First we recall Zhao's example which shows that there exist graphs with $\delta_U+\delta_V=n+3s-6$ such that $G$ cannot be tiled with $K_{s,s}$.  Let $G[U, V]$ be a balanced bipartite graph on $2n$ vertices with $n=(2k+1)s$.  Partition $U$ as $U_1\cup U_2$ with $|U_1|=ks+1$, $|U_2|=ks+s-1$ and partition $V$ as $V_1\cup V_2$ with $|V_1|=ks+s-1$, $|V_2|=ks+1$.  Let $G[U_1, V_1]$ and $G[U_2, V_2]$ be complete, let $G[U_1, V_2]\simeq P(ks+1, s-2)$ and let $G[U_2, V_1]\simeq P(ks+s-1, 2s-4)$.


We now recall the argument which shows that $G$ cannot be tiled with $K_{s,s}$.  Suppose $G$ can be tiled with $K_{s,s}$ and let $\mathcal{K}$ be such a tiling.  For $F\in \mathcal{K}$ and $i=1,2$, let $X_i(F):=V(F)\cap U_i$, $Y_i(F):=V(F)\cap V_i$ and $\vec{v}(F)=(|X_1(F)|, |X_2(F)|, |Y_1(F)|, |Y_2(F)|)$.  We say $F\in \mathcal{K}$ is \emph{crossing} if $V(F)\cap (U_1\cup V_1)\neq \emptyset$ and $V(F)\cap (U_2\cup V_2)\neq \emptyset$.  We now claim that if $F$ is crossing then $\vec{v}(F)=(s-1,1,s,0)$ or $\vec{v}(F)=(0,s,1,s-1)$.  It is not possible for $X_1(F)\neq \emptyset$ and $Y_2(F)\neq \emptyset$ since $G[U_1, V_2]\simeq P(ks+1, s-2)$ and $G[V_1, U_2]$ is $K_{2,2}$-free.  Thus if $X_1(F)\neq \emptyset$, then $|Y_1(F)|=s$, $|X_2(F)|\leq 1$, and $|X_1(F)|\geq s-1$.  If $Y_2(F)\neq \emptyset$, then $|X_2(F)|=s$, $|Y_1(F)|\leq 1$, and $|Y_2(F)|\geq s-1$.  This shows that if $F$ is crossing then $\vec{v}(F)=(s-1,1,s,0)$ or $\vec{v}(F)=(0,s,1,s-1)$.  Finally, since we are supposing that $G$ can be tiled, there exists some $\ell\in \mathbb{N}$ and some subset $\mathcal{K}'\subseteq \mathcal{K}$ such that every $F\in \mathcal{K}'$ is crossing and $\sum_{F\in \mathcal{K}'}|X_1(F)|=\ell s+1$ and $\sum_{F\in \mathcal{K}'}|Y_1(F)|=\ell s+s-1$.  Let $i_1$ be the number of $F\in \mathcal{K}'$ with $\vec{v}(F)=(s-1,1,s,0)$ and let $i_2$ be the number of $F\in \mathcal{K}'$ with $\vec{v}(F)=(0,s,1,s-1)$.  Then we have
\begin{align*}
\text{(i)} ~~(s-1)i_1=\ell s+1 ~~~\text{ and }~~~ \text{(ii)} ~~si_1+i_2=\ell s+s-1
\end{align*}
Which implies $i_1+i_2=s-2$.  However, (ii) implies that $i_2\geq s-1$, a contradiction.

Now we prove Theorem \ref{meven}.

\begin{proof}
 
% Let $m=2k$, where $k$ is a sufficiently large integer.  Let $U$ and $V$ be sets of vertices such that $|U|=|V|=2ks$.  Let $U$ be partitioned as $U=U_1\cup U_2$ and $V$ be partitioned as $V=V_1\cup V_2$ with $|U_1|=ks-s+1$, $|U_2|=ks+s-1$, $|V_1|=ks-1$ and $|V_2|=ks+1$.  Let $G[U_i, V_i]$ be complete for $i=1,2$.  Let $G[U_1, V_2]$ be the graph obtained from an $(s-2)$-regular graph $G[U_1', V_2]$ on $2(ks+1)$ vertices with no $K_{2,2}$ in which $s$ vertices have been deleted from $U_1'$ while maintaining $\delta(V_2, U_1)\geq s-3$.  Let $G[U_2, V_1]$ be the graph obtained from a $(3s-5)$-regular graph $G[U_2, V_1']$ on $2(ks+s-1)$ vertices with no $K_{2,2}$ in which $s$ vertices have been deleted from $V_1'$ while maintaining $\delta(U_2, V_1)\geq 3s-6$.
% 
% \begin{figure}[ht]
% \centering
% \input{3s-7.pstex_t}
% \label{3s-7}
% \end{figure}
% 
% Note that $$\delta_U=ks-1+s-2=ks+s-3,$$ $$\delta_V=ks+s-1+s-3=ks-s+1+3s-5=ks+2s-4,$$ and thus $$\delta_U+\delta_V= 2ks+3s-7= n+3s-7.$$

We give two examples of graphs which cannot be tiled with $K_{s,s}$; one when $m$ is even, one $m$ is odd, and both with $\delta_U+\delta_V=n+3s-7$.  

Let $j$ be a non-negative integer and let $m=2k$, where $k$ is sufficiently large.  Let $U$ and $V$ be sets of vertices such that $|U|=|V|=2ks$.  Let $U$ be partitioned as $U=U_1\cup U_2$ and $V$ be partitioned as $V=V_1\cup V_2$ with $|U_1|=(k-j)s+1$, $|U_2|=(k+j)s-1$, $|V_1|=(k-j+1)s-1$ and $|V_2|=(k+j-1)s+1$.  Let $G[U_i, V_i]$ be complete for $i=1,2$.  Let $G[U_1, V_2]$ be the graph obtained from $G[U_1', V_2]\simeq P((k+j)s-s+1, s-2)$ by deleting $(2j-1)s$ vertices from $U_1'$ while maintaining $\delta(V_2, U_1)\geq s-3$ (note that when $s=2$, $\delta(V_2, U_1)=0$).  Let $G[U_2, V_1]$ be the graph obtained from $G[U_2, V_1']\simeq P((k+j)s-1, (2j+1)s-5)$ by deleting $(2j-1)s$ vertices from $V_1'$ while maintaining $\delta(U_2, V_1)\geq (2j+1)s-6$.
We have 
\begin{align*}
\delta_U&=(k-j)s+s-1+s-2=(k-j+2)s-3,\\
\delta_V&=(k+j)s-1+s-3=(k-j)s+1+(2j+1)s-5=(k+j+1)s-4,
\end{align*}
and thus $\delta_U+\delta_V= 2ks+3s-7= n+3s-7.$



Let $j$ be a non-negative integer and let $m=2k+1$, where $k$ is sufficiently large.  Let $U$ and $V$ be sets of vertices such that $|U|=|V|=(2k+1)s$.  Let $U$ be partitioned as $U=U_1\cup U_2$ and $V$ be partitioned as $V=V_1\cup V_2$ with $|U_1|=(k-j)s+1$, $|U_2|=(k+j)s+s-1$, $|V_1|=(k-j)s+s-1$ and $|V_2|=(k+j)s+1$.  Let $G[U_i, V_i]$ be complete for $i=1,2$.  Let $G[U_1, V_2]$ be the graph obtained from $G[U_1', V_2]\simeq P((k+j)s+1, s-2)$ by deleting $2js$ vertices from $U_1'$ while maintaining $\delta(V_2, U_1)\geq s-3$ (note that when $s=2$, $\delta(V_2, U_1)=0$).  Let $G[U_2, V_1]$ be the graph obtained from $G[U_2, V_1']\simeq P((k+j)s+s-1, (2j+2)s-5)$ by deleting $2js$ vertices from $V_1'$ while maintaining $\delta(U_2, V_1)\geq (2j+2)s-6$.
We have 
\begin{align*}
\delta_U&=(k-j)s+s-1+s-2=(k-j+2)s-3,\\
\delta_V&=(k+j)s+s-1+s-3=(k-j)s+1+(2j+2)s-5=(k+j+2)s-4,
\end{align*}
and thus $\delta_U+\delta_V= (2k+1)s+3s-7= n+3s-7.$

The same analysis given before the start of this proof shows that each of these graphs cannot be tiled with $K_{s,s}$.


\end{proof}





\subsection{Tightness when $k_2\gg k_1$}

% We now generalize the graphs $P(m,p)$ given in Lemma \ref{Zhao:No K22}.
% 
% \begin{lemma}\label{No K22:gen}
% 
% For all $p,j\in\mathbb{N}$ there exists $m_0$ such that for all $m\in \mathbb{N}$, $m>m_0$, there exists a bipartite graph, $P_j(m,p)$, on $(j+1)m$ vertices, so that the following hold:
% \begin{enumerate}
% \item $|A|=m$, $|B|=jm$
% 
% \item $\delta(A, B)=jp$ and $\delta(B, A)=p$
% 
% \item $P_j(m,p)$ does not contain a copy of $K_{2,2}$.
% \end{enumerate}
% 
% \end{lemma}
% 
% 
% \begin{proof}
% 
% 
% 
% 
% \end{proof}

Now we prove Theorem \ref{k1smallexample}.

\begin{proof}

%Note that every positive integer $s$ can be written in the form $s=p^2+q$ where $p$ and $q$ are integers and $0\leq q\leq 2p$.


Let $G=(U_1\cup U_2, V_1\cup V_2; E)$ be a bipartite graph with $|U_1|=k_1s+y$, $|U_2|=k_2s-y$, $|V_1|=k_1s+s-1$, $|V_2|=k_2s-s+1$ such that $G[U_1,V_1]$, $G[U_2,V_2]$, and $G[V_1,U_2]$ are complete.  Furthermore suppose $|V_2|\geq (s-x)|U_1|$, every vertex in $U_1$ has $s-x$ neighbors in $V_2$, and for all $u, u'\in U_1$, $(N(u)\cap N(u'))\cap V_2=\emptyset$.  Thus we have $0\leq \delta(V_2,U_1)\leq\Delta(V_2,U_1)\leq 1$ with $\delta(V_2, U_1)=\Delta(V_2, U_1)=1$ only when $|V_2|=(s-x)|U_1|$ and thus
\begin{equation}
\delta_U+\delta_V\geq k_1s+s-1+s-x+k_2s-y=n+2s-(x+y)-1 \label{x+y}
\end{equation}

Every copy of $K_{s,s}$ which touches both $U_1$ and $U_2\cup V_2$ must have one vertex from $U_1$, $s-1$ vertices from $U_2$, at most $s-x$ vertices from $V_2$, and at least $x$ vertices from $V_1$.  So if $xy\geq s$, then $G$ cannot be tiled.   So in order to maximize $\delta_U+\delta_V$ we minimize $x+y$ subject to the condition that $xy\geq s$.  The result is that $x=y=\croot{s}$, unless $1\leq q\leq p$ in which case $x=\croot{s}-1$, $y=\croot{s}$ suffices.  Thus \eqref{x+y} gives $\delta_U+\delta_V= n+2s-2\croot{s}-1$ in general and $\delta_U+\delta_V=n+2s-2\croot{s}$ when $1\leq q\leq p$.

\end{proof}






\section{Non-extremal Case}

In order to prove Theorem \ref{main 1} and Theorem \ref{main 2} we will first prove the following Theorem.  

\begin{theorem}\label{non-extreme}
For every $\alpha>0$ and every positive integer $s$, there exist $\beta>0$ and positive integer $m_1$ such that the following holds for all $n=ms$ with $m\geq m_1$.  Given a bipartite graph $G[U,V]$ with $|U|=|V|=n$, if $\delta_U+\delta_V\geq (1-2\beta)n$, $\delta_V\geq \delta_U\gg \alpha n$ and $\delta_U=k_1s+s+r$ for some $0\leq r\leq s-1$ with $k_1+k_2=m$, then either $G$ can be tiled with $K_{s,s}$, or 
\begin{equation}\label{extremalcondition-gen}
\text{there exist } U_1'\subseteq U,~ V_2'\subseteq V, \text{ such that } |U_1'|=k_1s,~ |V_2'|=k_2s,~ d(U_1',V_2')\leq \alpha.
\end{equation}
\end{theorem}

If $G$ is a graph for which \eqref{extremalcondition-gen} holds, then we say $G$ satisfies the \emph{extremal condition with parameter $\alpha$}.
% 
% \subsection{Regularity and Blow-Up Lemmas}
% 
% In this section we review the Regularity and Blow-up Lemmas. Let $\varGamma$ be a simple graph on $n$ vertices. For two disjoint, nonempty subsets $U$ and $V$ of $V(\varGamma)$, define the density of the pair $(U,V)$ as
% \[
% d(U,V)=\frac{e(U,V)}{|U||V|}.
% \]
% 
% \begin{definition}
% A pair $(U,V)$ is called $\ep$-\emph{regular} if for every $%
% U^{\prime }\subseteq U$ with $|U^{\prime }|\geq \ep |U|$ and every $%
% V^{\prime }\subseteq V$ with $|V^{\prime }|\geq \ep |V|$, $
% |d(U^{\prime },V^{\prime })-d(U,V)|\leq \ep $. The pair $\left(
% U,V\right) $ is $( \ep ,\delta ) $-\emph{super-regular}
% if it is $\ep $-regular and for all $u\in U$, 
% $\deg\left(
% u,V\right) \geq \delta \left| V\right| $ and for all $v\in V$, $\deg\left(
% v,U\right) \geq \delta \left| U\right| $.
% \end{definition}
% 
% First we note the following facts that we will need.
% 
% \begin{fact}[Intersection Property]\label{interprop}
% \label{deg} If $(U,V)$ is an $\ep $-regular pair with density $d$,
% then for any $Y\subseteq V$ with $(d-\ep)^{k-1}|Y|\geq \ep |V|$ there are less than
% $k\ep |U|^k$ $k$-tuples of vertices $(u_1, u_2, \dots, u_k)$, $u_i\in U$, such that $|Y\cap N(u_1, u_2, \dots, u_k)|\leq (d-\ep)^k|Y|$.
% \end{fact}
% 
% 
% \begin{fact}[Slicing Lemma]\label{slicing}
% \label{slice} Let $(U,V)$ be an $\ep $-regular pair with density $d$, and for some $\lambda >\ep $ let $U^{\prime }\subseteq U$, $V^{\prime }\subseteq V$, with $|U^{\prime }|\geq \lambda |U|$, $|V^{\prime }|
% \geq \lambda |V|$. Then $(U^{\prime },V^{\prime })$ is an $\ep
% ^{\prime }$-regular pair of density $d^{\prime }$ where $\ep
% ^{\prime }=\max \{\frac{\ep }{\lambda },2\ep \}$ and $
% d^{\prime}\ge d-\ep $.
% \end{fact}
% 
% Our main tool in the proof will be the Regularity Lemma of Szemer\'{e}di \cite{Sz} which we state in its multipartite form.
% 
% \begin{lemma}[Regularity Lemma - Bipartite Version]\label{bireg} 
% For every $\ep>0$ there exists $M:=M(\ep)$ such that if $G:=G[U,V]$ is a balanced bipartite graph on $2n$ vertices and $d\in[0,1]$, then there is a partition of $U$ into clusters $U_0, U_1,\dots, U_t$, a partition of $V$ into clusters $V_0, V_1,\dots, V_t$, and a subgraph $G':=G'[U,V]$ with the following properties: 
% 
% \begin{enumerate}
% \item $t\leq M$,
% \item $|U_0|\leq \ep n$, $|V_0|\leq \ep n$,
% \item $|U_i|=|V_i|=\ell\leq \ep n$ for all $i\in [t]$,
% \item $\deg_{G'}(x)>\deg_G(x)-(d+\ep)n$ for all $x\in V(G)$,
% \item All pairs $(U_i, V_i)$, $i,j\in [t]$, are $\ep$-regular in $G'$ each with density either $0$ or exceeding $d$.
% 
% \end{enumerate}
% \end{lemma}
% 
% In addition, we will use the Blow-up Lemma of Koml\'{o}s, S\'{a}rk\"{o}zy, and Szemer\'{e}di \cite{KSSbu}.
% 
% \begin{lemma}[Blow-up Lemma]\label{blowup}
% Given $\delta >0$, $\Delta >0$ there exists $\ep >0$ such that the following holds. Let $(U, V)$ be an $(\ep ,\delta )$-super-regular pair. %with $|U|=n_{1}$ and $|W_2|=n_{2}$. 
% If $T$ is a $U',V'$-bigraph with maximum degree $\Delta (T)\leq\Delta $ and $T$ is embeddable into the complete bipartite graph $K_{|U|},{|V|}$ then it is also embeddable into $(U,V)$.
% \end{lemma}


\subsection{Proof of Theorem \ref{non-extreme}}

Here we prove Theorem \ref{non-extreme}.  We show that if $G$ is not in the extremal case, we obtain a tiling with $K_{s,s}$; otherwise $G$ is in the extremal case which we deal with in Section \ref{extremesection}.  The proof is adopted from Zhao \cite{Z}.

\begin{proof}Let $\ep$, $d$, and $\beta$ be positive real numbers such that $$\ep\ll d\ll \beta\ll \alpha$$ and suppose $n$ is large.  Let $G[U,V]$ be a bipartite graph with $|U|=|V|=n$,  $\delta_U+\delta_V\geq (1-\beta)n$, and $\delta_V\geq \delta_U\gg \alpha n$.  We also have $\delta_U=k_1s+s+r$ for some $0\leq r\leq s-1$ and we set $k_2:=m-k_1$.  Let $\gamma_1, \gamma_2$ be positive real numbers such that $\delta_U\geq (\gamma_1-\beta)n$, $\delta_V\geq (\gamma_2-\beta) n$ and $\gamma_1+\gamma_2=1$.  Note that $\gamma_2\geq \gamma_1\gg \alpha$.  We apply Lemma \ref{bireg} to $G$ with parameters $\ep$ and $d$.  We obtain a partition of $U$ into $U_0, U_1,\dots, U_t$ and $V$ into $V_0, V_1,\dots, V_t$ such that $|U_i|=|V_i|=\ell\leq \ep n$ for all $i\in [t]$ and $|U_0|=|V_0|\leq \ep n$.  In the graph $G'$ from Lemma \ref{bireg}, we have $(U_i, V_j)$, is $\ep$-regular with density either $0$ or exceeding $d$ for all $i,j\in [t]$.  We also have $\deg_{G'}(u)>(\gamma_1-\beta)n-(\ep+d)n$ for $u\in U$ and $\deg_{G'}(v)>(\gamma_2-\beta)n-(\ep+d)n$ for $v\in V$.

We now consider the \emph{reduced graph} of $G'$.  Let $G_r$ be a bipartite graph with parts $\mathcal{U}:=\{U_1, \dots, U_t\}$ and $\mathcal{V}:=\{V_1,\dots, V_t\}$ such that $U_i$ is adjacent to $V_j$, denoted $U_i\sim V_j$, if and only if $(U_i, V_j)$ is an $\ep$-regular pair with density exceeding $d$.  A standard calculation gives the following degree condition in the reduced graph, $\delta_\mathcal{U}\geq (\gamma_1-2\beta)t$ and $\delta_\mathcal{V}\geq (\gamma_2-2\beta)t$.

\begin{claim}\label{split-gen}
If $G_r$ contains two subsets $X\subseteq \mathcal{U}$ and $Y\subseteq \mathcal{V}$ such that $|X|\geq (\gamma_1-3\beta)t$, $|Y|\geq (\gamma_2-3\beta)t$ and there are no edges between $X$ and $Y$, then \eqref{extremalcondition-gen} holds in $G$.
\end{claim}

\begin{proof}
Without loss of generality, assume that $|X|=(\gamma_1-3\beta)t$ and $|Y|=(\gamma_2-3\beta)t$.  Let $U'=\cup_{U_i\in X}U_i$ and $V'=\cup_{V_i\in Y}V_i$.  We have 
$$(\gamma_1-4\beta)n<(\gamma_1-3\beta)t\ell=|X|\ell=|U'|\leq (\gamma_1-3\beta)n$$
and
$$(\gamma_2-4\beta)n<(\gamma_2-3\beta)t\ell=|Y|\ell=|V'|\leq (\gamma_2-3\beta)n.$$
Since there is no edge between $X$ and $Y$ we have $e_{G'}(U',V')=0$.  Consequently $e_G(U', V')\leq e_{G'}(U', V')+d|U'||V'|+2\ep n|U'|<dk_1sk_2s$.  By adding at most $4\beta k_1s$ vertices to $U'$ and $4\beta k_2s$ vertices to $V'$, we obtain two subsets of size $k_1s$ and $k_2s$ respectively, with at most $dk_1sk_2s+4\beta k_1sk_2s+4\beta k_1sk_2s<\alpha k_1sk_2s$ edges, and thus \eqref{extremalcondition-gen} holds in $G$.
\end{proof}

For the rest of this proof, we suppose that \eqref{extremalcondition-gen} does not hold in $G$.

\begin{claim}\label{matching}
$G_r$ contains a perfect matching.
\end{claim}

\begin{proof}
Let $M$ be a maximum matching of $G_r$. After relabeling indices if necessary, we may assume that
$M = \{U_iV_i : i\in[k], k\leq t\}$. If $M$ is not perfect, let $x\in \mathcal{U}$ and $y \in \mathcal{V}$ be vertices which are unsaturated by $M$. Then the neighborhood $N(x)$ is a subset of $V(M)$, otherwise we can enlarge $M$ by
adding an edge $xz$ for any $z \in N(x)\setminus V(M)$. We have $N(y) \subseteq V (M)$ for the same reason.
Now let $I = \{i : V_i \in N(x)\}$ and $J = \{j : U_j\in N(y)\}$. If $I \cap J\neq \emptyset$; that is, there exists $i$
such $xV_i$ and $yU_i$ are both edges, then we can obtain a larger matching by replacing $U_iV_i$ in
$M$ by $xV_i$ and $yU_i$. Otherwise, assume that $I \cap J = \emptyset$. Since $|I|\geq (\gamma_1-2\beta)t$ and $|J|\geq (\gamma_2-2\beta)t$
and \eqref{extremalcondition-gen} does not hold in $G$, then by the contrapositive of Claim \ref{split-gen} there exists an edge
between $\{U_i : i \in I\}$ and $\{V_j : j \in J\}$. This implies that there exist $i \neq j$ such that $xV_i$,
$U_iV_j$, and $yU_j$ are edges. Replacing $U_iV_i$, $U_jV_j$ in $M$ by $xV_i$, $U_iV_j$ and $yU_j$, we obtain a larger
matching, contradicting the maximality of $M$.

\end{proof}

By Claim \ref{matching} we assume that $U_i\sim V_i$ for all $i\in [t]$.  If each $\ep$-regular pair $(U_i, V_i)$ is also super-regular and $s$ divides $\ell$, then the Blow-up Lemma (Lemma \ref{blowup}) guarantees that $G'[U_i, V_i]$ can be tiled with $K_{s,s}$ (since $K_{\ell,\ell}$ can be tiled with $K_{s,s}$). If we also know that $U_0=V_0=\emptyset$, then we obtain a $K_{s,s}$-tiling of $G$. Otherwise we do the following steps (details of these steps are given next). \emph{Step 1}: For each $i\geq 1$, we move vertices from $U_i$ to $U_0$ and from $V_i$ to $V_0$ so that each remaining vertex in $(U_i, V_i)$ has at least $(d-2\ep)\ell$ neighbors. \emph{Step 2:} We eliminate $U_0$ and $V_0$ by removing copies of $K_{s,s}$, each of which contains at most one vertex of $U_0\cup V_0$. \emph{Step 3}: We make sure that for each $i\geq 1$, $|U_i|=|V_i| > (1 - d)\ell$ and $|U_i|$ is divisible by
$s$. Finally we apply the Blow-up Lemma to each $(U_i, V_i)$ (which is still super-regular) to finish the proof. Note that we always refer to the clusters as $U_i, V_i, i \geq 0$ even though they may gain or lose vertices during the process.

\emph{Step 1.} For each $i \geq 1$, we remove all $u\in U_i$ such that $\deg(u, V_i) < (d-\ep)\ell$ and all
$v\in V_i$ such that $\deg(v, U_i) < (d-\ep)\ell$. Fact \ref{interprop} (with $k=1$) guarantees that the
number of removed vertices is at most $\ep\ell$. We then remove more vertices from either $U_i$
or $V_i$ to make sure $U_i$ and $V_i$ still have the same number of vertices. All removed vertices
are added to $U_0$ and $V_0$. As a result, we have $|U_0|=|V_0|\leq 2\ep n$.

\emph{Step 2.} This step implies that a vertex in $U_0, V_0$ can be viewed as a vertex in $U_i$ or $V_i$ for
some $i\geq 1$. For a vertex $x \in V(G)$ and a cluster $C$, we say $x$ is adjacent to $C$, denoted $x \sim C$, if
$\deg_G(x,C)\geq d\ell$. We claim that at present, each vertex in $U$ is adjacent to at least
$(\gamma_1-2\beta)t$ clusters.  If this is not true for some $u \in U$, then we obtain a contradiction
$$(\gamma_1-\beta)n\leq \deg_G(u)\leq (\gamma_1-2\beta)t\ell+d\ell t +2\ep n<(\gamma_1-3\beta/2)n.$$
Likewise, each vertex in $V$ is adjacent to at least $(\gamma_2-2\beta)t$ clusters.  Assign an arbitrary order to the vertices in $U_0$. For each $u \in U_0$, we pick some $V_i$ adjacent
to $u$. The selection of $V_i$ is arbitrary, but no $V_i$ is selected more than $\frac{d\ell}{6s}$ times. Such $V_i$
exists even for the last vertex of $U_0$ because $|U_0|\leq 2\ep n<(\gamma_1-2\beta)t\frac{d\ell}{6s}$. For each $u\in U_0$
and its corresponding $V_i$, we remove a copy of $K_{s,s}$ containing $u$, $s$ vertices in $V_i$, and $s-1$ vertices
in $U_i$. Such a copy of $K_{s,s}$ can always be found even if $u$ is the last vertex in $U_0$
because $(U_i, V_i)$ is $\ep$-regular and $\deg_G(u, V_i)\geq d\ell>\ep\ell+\frac{d\ell}{6s}s$ thus Fact \ref{interprop} (with $k=s-1$) allows us to choose $s-1$ vertices from $U_i$ and $s$ vertices from $N(u)\cap V_i$ to complete the copy of $K_{s,s}$. As a result, $U_i$ now has
one more vertex than $V_i$, so one may view this process as moving $u$ to $U_i$. We repeat this
process for all $v \in V_0$ as well. By the end of this step, we have $U_0=V_0=\emptyset$, and each
$U_i$, $V_i$, $i \geq 1$ contains at least $\ell -\ep\ell-d\ell/3$ vertices (for example, $U_i$ may have lost $\frac{d\ell(s-1)}{6s}$
vertices because of $U_0$ and $d\ell/6$ vertices because of $V_0$).  As a result, we have $\delta(G[U_i, V_i])\geq (\frac{2d}{3}-2\ep)\ell$ for all $i\geq 1$. Note that the sizes of $U_i$ and $V_i$ may currently be different.

\emph{Step 3}. We want to show that for any $i \neq j$, there is a path $U_iV_{i_1}U_{i_1}\dots V_{i_a}U_{i_a}V_jU_j$ (resp. $V_iU_{i_1}V_{i_1}\dots U_{i_a}V_{i_a}U_jV_j$) for some $0\leq a\leq 2$. If such a path exists, then for each $i_b$, $1\leq b\leq a+1$ (assume that $i = i_0$ and $j = i_{a+1}$), we may remove a copy of $K_{s,s}$ containing one vertex from $U_{i_{b-1}}$, $s$ vertices
from $V_{i_b}$, and $s-1$ vertices from $U_{i_b}$. This removal reduces the size of $U_i$ by one, increases
the size of $U_j$ by one but does not change the sizes of other clusters (all modulo $s$). We may
therefore adjust the sizes of $U_i$ and $V_i$ (for $i \geq 1$) such that $|U_i|=|V_i|$ and $|U_i|$ is divisible
by $s$.  To do this we will need at most $2t$ paths: (i) Let $r:=\floor{\frac{n}{t}}\hspace{-.1in}\mod s$. (ii) Pair up the current biggest set $U_i$ and current smallest set $U_j$ and move vertices from $U_i$ to $U_j$ until one of the sets has exactly $\floor{\frac{n}{t}}-r$ elements. (iii) Repeat this process until all but one set in $\mathcal{U}$ has exactly $\floor{\frac{n}{t}}-r$ elements (there will be one set, say $U_t$, with as many as $(t-1)^2$ extra vertices) (iv) Do the same for the clusters in $\mathcal{V}$.  

Now we show how to find this path from $U_1$ to $U_2$.  First, if $U_1\sim V_2$, then $U_1V_2U_2$
is a path. Let $I = \{i : U_1 \sim V_i\}$ and $J = \{i : U_i\sim V_2\}$. If there exists $i \in I \cap J$, then
we find a path $U_1V_iU_iV_2U_2$. Otherwise $I \cap J = \emptyset$. Since both $|I|\geq (\gamma_1-2\beta)t$ and $|J|\geq (\gamma_2-2\beta)t$, Claim \ref{split-gen} guarantees that there exists $i\in I$ and $j\in J$ such that $U_i\sim V_j$. We thus have a path $U_1V_iU_iV_jU_jV_2U_2$. Note that in this step we require that a cluster is contained in at most $\frac{d\ell}{3s}$ paths. This restriction has little impact
on the arguments above: we have $|I|>(\gamma_1-3\beta)t$ and $|J| > (\gamma_2-3\beta)t$ instead, still satisfying the conditions
of Claim \ref{split-gen}.  

Now $U_0=V_0=\emptyset$, and for all $i \geq 1$, $|U_i|=|V_i|$ is divisible by $s$.  Let $\mathcal{K}$ be the union of all vertices in existing copies of $K_{s,s}$ and note that, 
\begin{equation*}%\label{unused} 
|U_i\setminus \mathcal{K}|=|V_i\setminus \mathcal{K}|\geq \ell-\ep \ell-2d\ell/3,
\end{equation*} 
which implies $\delta(G[U_i, V_i])\geq (\frac{d}{3}-2\ep)\ell\geq \frac{d}{4}\ell$ for $i\geq 1$.  Thus Fact \ref{slicing} implies that each pair $(U_i, V_i)$ is $(2\ep, \frac{d}{4})$-super-regular.  Applying the Blow-up Lemma to each $(U_i, V_i)$, we find the desired $K_{s,s}$-tiling.

\end{proof}




\section{Extremal Case}\label{extremesection}

Given $s\geq 2$ and $\lambda\in (0,\frac{1}{2})$, let $\alpha>0$ be sufficiently small.  Let $G[U,V]$ be a balanced bipartite graph on $2n=2ms$ vertices for sufficiently large $n$.  Without loss of generality suppose $\delta_V\geq \delta_U$ and note that $\delta_U\geq \lambda n$.  Suppose $G$ is edge minimal with respect to the condition $\delta_U+\delta_V\geq n+c$, and that $G$ satisfies the extremal condition with parameter $\alpha$.  Let $k_1$ be defined by $\delta_U=k_1s+s+r$, where $0\leq r\leq s-1$ and let $k_2s=n-k_1s$.


%Let $\beta$ be a small constant satisfying $\alpha\ll \beta\leq \frac{1}{2s}$.  
The proof will split into cases depending on whether $k_1\leq (1-\frac{1}{2s})k_2$ or $k_1> (1-\frac{1}{2s})k_2$.  When $k_1> (1-\frac{1}{2s})k_2$, we have $\delta_U+\delta_V\geq n+3s-5$.  Since $\delta_U=k_1s+s+r$, we have $\delta_V\geq k_2s+2s-5-r$. Since $G$ is edge minimal we have $\delta_V=k_2s+2s-5-r$, and since $\delta_V\geq \delta_U$, we have $k_2\geq k_1$.  If $\delta_V=\delta_U$, then we have $$\delta(G)\geq \frac{n+3s-5}{2}> \left\lbrace \begin{array}{ll} \frac{n}{2}+s-2   & \text{ if } m \text{ is even } \\ \frac {n+3s}{2}-3 & \text{ if } m \text{ is odd, } \end{array} \right. $$ which is solved in \cite{Z}.  So we may suppose that $\delta_V>\delta_U$.  

\begin{claim}\label{k2approxk1}
If $k_2=k_1$, then $r\leq \frac{s-6}{2}$ and consequently $\delta_V=k_2s+2s-5-r\geq k_2s+s$.  If $k_2=k_1+1$, then $r\leq s-3$ and consequently $\delta_V=k_2s+2s-5-r\geq k_2s+s-2$.
\end{claim}

\begin{proof}
Both statements are implied the following inequality: $k_2s+2s-5-r=\delta_V>\delta_U=k_1s+s+r$.
\end{proof}

When $k_1\leq (1-\frac{1}{2s})k_2$, we will show in Theorem \ref{main 2} that a smaller degree suffices to tile $G$ with $K_{s,s}$.  So Theorem \ref{main 2} provides the second half of the proof of Theorem \ref{main 1}.


\subsection{Pre-processing}\label{preprocesssection}

Let $U_2'=U\setminus U_1'$ and $V_1'=V\setminus V_2'$.  Let
\begin{align*}
U_1&=\{x\in U: %\deg(x, V_1')>(1-\alpha^{1/3})k_1s \wedge 
\deg(x,V_2')<\alpha^{1/3}k_1s\},~~~
V_2=\{x\in V: %\deg(x, U_2')>(1-\alpha^{1/3})k_2s \wedge 
\deg(x,U_1')<\alpha^{1/3}k_2s\},\\
U_2&=\{x\in U: \deg(x, V_1')<\alpha^{1/3}k_1s \vee \deg(x,V_2')>(1-\alpha^{1/3})k_2s\},\\
V_1&=\{x\in V: \deg(x,U_2')<\alpha^{1/3}k_2s \vee \deg(x, U_1')>(1-\alpha^{1/3})k_1s\},\\
U_0&=U\setminus (U_1\cup U_2), \text{ and } V_0=V\setminus (V_1\cup V_2).\\
\end{align*}

\begin{claim}\label{preprocess}
\begin{enumerate}
\item $k_1s-\alpha^{2/3}k_2s\leq|U_1|,|V_1|\leq k_1s+\alpha^{2/3}k_1s$

\item $k_2s-\alpha^{2/3}k_1s\leq|U_2|,|V_2|\leq k_2s+\alpha^{2/3}k_2s$

\item $|U_0|,|V_0|\leq \alpha^{2/3}n$

\item $\delta(U_0,V_1)\geq \alpha^{1/3}k_1s-\alpha^{2/3}k_2s$, $\delta(U_0,V_2)\geq \alpha^{1/3}k_1s-\alpha^{2/3}k_1s$

\item $\delta(V_0,U_1)\geq \alpha^{1/3}k_2s-\alpha^{2/3}k_2s$, $\delta(V_0,U_2)\geq \alpha^{1/3}k_2s-\alpha^{2/3}k_1s$

\item $\delta(G[U_i,V_i])\geq k_is-\alpha^{1/3}k_is-\alpha^{2/3}k_{3-i}s\geq (1-2\alpha^{1/3})k_is$

\item $\Delta(U_1,V_2)\leq 2\alpha^{1/3}k_1s$, $\Delta(V_2,U_1)\leq 2\alpha^{1/3}k_2s$
\end{enumerate}

\end{claim}

\begin{proof}
We have 
\begin{align*}
\alpha^{1/3}k_1s|U_1'\setminus U_1|\leq e(U_1'\setminus U_1,V_2')\leq e(U_1',V_2')\leq \alpha k_1sk_2s 
\end{align*}
which gives $|U_1'\setminus U_1|\leq \alpha^{2/3}k_2s$ and thus $|U_1|\geq k_1s-\alpha^{2/3}k_2s$.

Also 
\begin{align*}
\alpha^{1/3}k_2s|V_2'\setminus V_2|\leq e(V_2'\setminus V_2,U_1')\leq e(V_2',U_1')\leq \alpha k_1sk_2s
\end{align*}
which gives $|V_2'\setminus V_2|\leq \alpha^{2/3}k_1s$ and thus $|V_2|\geq k_2s-\alpha^{2/3}k_1s$.

Since $e(U_1',V_2')\leq \alpha k_1sk_2s$, we have $e(U_2',V_2')\geq k_2sk_2s-\alpha k_1sk_2s$ and $e(U_1',V_1')\geq k_1sk_1s-\alpha k_1sk_2s$.  Thus 
\begin{align*}
\alpha^{1/3}k_2s|U_2'\setminus U_2|\leq \bar{e}(U_2',V_2')\leq \alpha k_1sk_2s
\end{align*}
which gives $|U_2'\setminus U_2|\leq \alpha^{2/3}k_1s$ and thus $|U_2|\geq k_2s-\alpha^{2/3}k_1s$.

Also
\begin{align*}
\alpha^{1/3}k_1s|V_1'\setminus V_1|\leq \bar{e}(U_1',V_1')\leq \alpha k_1sk_2s
\end{align*}
which gives $|V_1'\setminus V_1|\leq \alpha^{2/3}k_2s$ and thus $|V_1|\geq k_1s-\alpha^{2/3}k_2s$.

Putting these results together we have $|U_0|,|V_0|\leq \alpha^{2/3} n$, $|U_1|,|V_1|\leq k_1s+\alpha^{2/3}k_1s$, and $|U_2|,|V_2|\leq k_2s+\alpha^{2/3}k_2s$.

By the definition of $U_1,U_2,V_1,V_2$ and the lower bounds on their sizes, we have $\delta(U_0,V_1)\geq \alpha^{1/3}k_1s-\alpha^{2/3}k_2s$, $\delta(U_0,V_2)\geq \alpha^{1/3}k_1s-\alpha^{2/3}k_1s$, $\delta(V_0,U_1)\geq \alpha^{1/3}k_2s-\alpha^{2/3}k_2s$, and $\delta(V_0,U_2)\geq \alpha^{1/3}k_2s-\alpha^{2/3}k_1s$.  By the definition of $U_1,V_2$ and the upper bounds on their sizes we have $\Delta(U_1,V_2)\leq 2\alpha^{1/3}k_1s$ and $\Delta(V_2,U_1)\leq 2\alpha^{1/3}k_2s$.


\end{proof}


\subsection{Idea of the Proof}

We start with the partition given in Section \ref{preprocesssection} and we call $U_0$ and $V_0$ the \emph{exceptional} sets. Let $i\in \{1,2\}$. We will attempt to update the partition by moving a constant number (depending only on $s$) of \emph{special} vertices between $U_1$ and $U_2$, denote them by $X$, and \emph{special} vertices between $V_1$ and $V_2$, denote them by $Y$, as well as partitioning the exceptional sets as $U_0=U_0^1\cup U_0^2$ and $V_0=V_0^1\cup V_0^2$. Let $U_1^*$, $U_2^*$, $V_1^*$ and $V_2^*$ be the resulting sets after moving the special vertices. %in $X$ and $Y$. 
Suppose $u$ is a special vertex in the set $U_1^*$.  The degree of $u$ in $V_1^*$ may be small, but $u$ will have a set of at least $s$ neighbors in $V_1^*$ which are disjoint from the neighbors of any other special vertex in $U_1^*$.  Furthermore, these neighbors of $u$ in $V_1^*$ will have huge degree in $U_1^*$, so it will be easy to incorporate each special vertex into a unique copy of $K_{s,s}$.

Our goal is to obtain two graphs, $G_1:=G[U_1^*\cup U_0^1, V_1^*\cup V_0^1]$ and $G_2:=[U_2^*\cup U_0^2, V_2^*\cup V_0^2]$ so that $G_1$ satisfies $$|U_1^*\cup U_0^1|=\ell_1s,~ |V_1^*\cup V_0^1|=\ell_1s$$ and $G_2$ satisfies $$|U_2^*\cup U_0^2|=\ell_2s,~ |V_2^*\cup V_0^2|=\ell_2s,$$ for some positive integers $\ell_1,\ell_2$.  We tile $G_1$ as follows.  We incorporate all of the special vertices into copies of $K_{s,s}$.  We now deal with the exceptional vertices: Claim \ref{preprocess} gives $|U_0|, |V_0|\leq \alpha^{2/3}n$ and $\delta(U_0, V_i), \delta(V_0, U_i)\gg s\alpha^{2/3}n$, so they may greedily be incorporated into unique copies of $K_{s,s}$. Then we are left with two balanced ``almost complete'' graphs, which can be easily tiled.

So throughout the proof, if we can make, say $|U_1^*\cup U_0^1|$ and $|V_1^*\cup V_0^1|$ equal and divisible by $s$, we simply state that ``we are done.''


\subsection{Preliminary Lemma's}

In this section we give some lemmas which will be used in the proof of Theorems \ref{main 1} and \ref{main 2}.  Recall that in each of those theorems we suppose $k_2s\geq k_1s\geq \lambda n$.  
%Also recall that $\alpha\ll \beta\leq \frac{1}{2s}$.


\begin{lemma}[Zhao \cite{Z}, Fact 5.3]\label{Zhao lemma}
Let $F$ be an $A,B$-bigraph with $\delta:=\delta(A,B)$ and $\Delta:=\Delta(B,A)$ 
Then $F$ contains $f_h$ vertex disjoint $h$-stars from $A$ to $B$, and $g_h$ vertex disjoint $h$-stars from $B$ to $A$ (the stars from $A$ to $B$ and those from $B$ to $A$ need not be disjoint), where
\begin{align*}
f_h\geq\frac{(\delta-h+1)|A|}{h\Delta+\delta-h+1}, ~~~ g_h\geq\frac{\delta|A|-(h-1)|B|}{\Delta+h\delta-h+1}.
\end{align*}

\end{lemma}


\begin{lemma}\label{STARSlemma}
Let $G[A,B]$ be a bipartite graph with 
%$|A|, |B|\geq \lambda n$.  Suppose 
$|B|=\ell s+b$ for some positive integers $\ell$ and $b$.  Let $0\leq x\leq s-1$ and let $\gamma$ be a small constant such that $\alpha^{1/3}\ll \gamma\ll \frac{1}{2s}$.  If $b<\frac{1}{\gamma}$ and
\begin{enumerate}
\item $\delta(B,A)\geq s-x$, $\Delta(A,B)\leq 2\alpha^{1/3}k_2s$, and $|B|\geq \alpha^{1/6}|A|$
\end{enumerate}
then there are at least $b$ vertex disjoint $(s-x)$-stars from $B$ to $A$.

Suppose $k_2s+\alpha^{2/3}k_2s\geq |A|, |B|\geq k_1s-\alpha^{2/3}k_2s$.  If
\begin{enumerate}[resume]
\item $\delta(A,B)\geq s-1+b$ and $k_1>(1-\frac{1}{2s})k_2$,
\end{enumerate}
then there are at least $b$ vertex disjoint $s$-stars from $B$ to $A$.  If $b<\frac{1}{\gamma}$ and
\begin{enumerate}[resume]

\item $\delta(A,B)\geq s$, $k_1>(1-\frac{1}{2s})k_2$, and $\Delta(B,A)\leq 2\alpha^{1/3}k_2s$ or

\item $\delta(A,B)\geq d$, $|A|\geq \frac{s-1/2}{d}|B|$, and $\Delta(B,A)\leq 2\alpha^{1/3}k_2s$,
\end{enumerate}
then there are at least $b$ vertex disjoint $s$-stars from $B$ to $A$.  Furthermore, if $b\geq \frac{1}{\gamma}$ and 
\begin{enumerate}[resume]
\item $\delta(A,B)\geq b/4$ and $\Delta(B,A)<2\alpha^{1/3}k_2s$ or

\item $\delta(B,A)\geq b/4$ and $\Delta(A,B)<2\alpha^{1/3}k_2s$,

\end{enumerate}
then there are at least $b$ vertex disjoint $s$-stars from $B$ to $A$.

\end{lemma}


\begin{proof}

\begin{enumerate}

\item Suppose $b<\frac{1}{\gamma}$, $\delta(B,A)\geq s-x$, $\Delta(A,B)\leq 2\alpha^{1/3}k_2s$, and $|B|\geq \alpha^{1/6}|A|$.  Let $\mathcal{S}_B$ be the maximum set of vertex disjoint $(s-x)$-stars from $B$ to $A$ and let $f_{s-x}=|\mathcal{S}_B|$.  By Lemma \ref{Zhao lemma}, we have
\begin{align*}
f_{s-x}\geq \frac{|B|}{2(s-x)\alpha^{1/3}k_2s+1}\geq \frac{\alpha^{1/6}}{3s\alpha^{1/3}}
\geq \frac{1}{\gamma}
\geq b
\end{align*}

\item Suppose $\delta(A,B)\geq s-1+b$ and $k_1>(1-\frac{1}{2s})k_2$.  Let $\mathcal{S}_A$ be a maximum set of vertex disjoint $s$-stars with centers $C\subseteq B$ and leaves $L\subseteq A$.  Suppose $|C|\leq b-1$.  Then 
\begin{align*}
s(|A|-|L|) \leq(s-1+b-|C|)(|A|-|L|) &\leq e(B\setminus C, A\setminus L)\\
&\leq (s-1)(|B|-|C|),
\end{align*}
which implies
\begin{align*}
s(k_1s-\alpha^{2/3}k_2s)\leq (s-1)(k_2s+\alpha^{2/3}k_2s)+s|L|-(s-1)|C|.
\end{align*}
Thus $sk_1\leq (s-\frac{1}{2})k_2$, contradicting the fact that $k_1>(1-\frac{1}{2s})k_2$.

\item Suppose $b<\frac{1}{\gamma}$, $\delta(A,B)\geq s$, $k_1>(1-\frac{1}{2s})k_2$, and $\Delta(B,A)\leq 2\alpha^{1/3}k_2s$.  Let $\mathcal{S}_A$ be the maximum set of vertex disjoint $s$-stars from $A$ to $B$ and let $g_s=|\mathcal{S}_A|$.  By Lemma \ref{Zhao lemma}, we have
\begin{align*}
g_s\geq \frac{s|A|-(s-1)|B|}{2\alpha^{1/3}k_2s+s^2-s+1}&\geq \frac{s(k_1s-\alpha^{2/3}k_2s)-(s-1)(k_2s+\alpha^{2/3}k_2s)}{3\alpha^{1/3}k_2s}
\\
&\geq \frac{1}{12\alpha^{1/3}}\geq \frac{1}{\gamma}\geq b
%\frac{s(k_1s-\alpha^{2/3}k_2s)-(s-1)(k_1s+\frac{1}{2s} k_2s+\alpha^{2/3}k_2s)}{3\alpha^{1/3}k_2s}\\
%&\geq \frac{k_1s-s\alpha^{2/3}k_2s-(s-1)(\frac{1}{2s}+\alpha^{2/3})k_2s}{3\alpha^{1/3}k_2s}\\
%&\geq \frac{k_1}{6\alpha^{1/3}k_2}\\
%&\geq 1/\gamma\\
%&\geq b
\end{align*}
Where the third inequality holds since $sk_1s> (s-\frac{1}{2})k_2s$.

\item Suppose $b<\frac{1}{\gamma}$, $\delta(A,B)\geq d$, $|A|\geq \frac{s-1/2}{d}|B|$, and $\Delta(B,A)\leq 2\alpha^{1/3}k_2s$.  Let $\mathcal{S}_B$ be the maximum set of vertex disjoint $s$-stars from $B$ to $A$ and let $g_s=|\mathcal{S}_B|$.  By Lemma \ref{Zhao lemma}, we have
\begin{align*}
g_s\geq \frac{d|A|-(s-1)|B|}{2\alpha^{1/3}k_2s+sd-s+1}\geq \frac{|B|/2}{3\alpha^{1/3}k_2s}
\geq \frac{\lambda}{6\alpha^{1/3}}
%\geq \frac{1}{12(10s^3-1)\alpha^{1/3}}
\geq \frac{1}{\gamma}
\geq b
\end{align*}

\item Suppose $b\geq \frac{1}{\gamma}$, $\delta(A,B)\geq b/4$ and $\Delta(B,A)<2\alpha^{1/3}k_2s$.  Let $\mathcal{S}_B$ be the maximum set of vertex disjoint $s$-stars from $B$ to $A$ and let $g_s=|\mathcal{S}_B|$.  By Lemma \ref{Zhao lemma}, we have
\begin{align*}
g_s\geq \frac{\frac{b}{4}|A|-(s-1)|B|}{2\alpha^{1/3}k_2s+s\frac{b}{4}-s+1}\geq \frac{b\lambda/4-(s-1)}{3\alpha^{1/3}}\geq b
\end{align*}


\item Suppose $b\geq \frac{1}{\gamma}$, $\delta(B,A)\geq b/4$ and $\Delta(A,B)<2\alpha^{1/3}k_2s$.  Let $\mathcal{S}_B$ be the maximum set of vertex disjoint $s$-stars from $B$ to $A$ and let $f_s=|\mathcal{S}_B|$.  By Lemma \ref{Zhao lemma}, we have
\begin{align*}
f_s\geq \frac{(\frac{b}{4}-s+1)|B|}{2s\alpha^{1/3}k_2s+\frac{b}{4}-s+1}\geq \frac{(\frac{b}{4}-s+1)\lambda}{3\alpha^{1/3}}\geq b
\end{align*}

\end{enumerate}

\end{proof}

% \begin{lemma}\label{stars or complete}
% Let $G[A,B]$ be a bipartite graph.  Suppose $k_2s+\alpha^{2/3}k_2s\geq |A|, |B|\geq k_1s-\alpha^{2/3}k_2s$ and suppose $|B|=\ell s+b$ for some positive integers $\ell$ and $1\leq b\leq s-1$.  If $\delta(B,A)\geq 2s-1$, then there is a set of $b$ vertex disjoint $s$-stars from $B$ to $A$, or else there is $K_{b,s}$ with $b$ vertices in $B$ and $s$ vertices in $A$.
% \end{lemma}



%\begin{lemma}
%Let $G[A,B]$ be a balanced bipartite graph on $2n$ vertices with $\delta_A(G)+\delta_B(G)\geq n+3s-4$.  Suppose that $G$ has an $\alpha$-partition.  Suppose $\Delta(G[A_1,B_2])\leq 2\alpha^{1/3}k_2s$ and $|A_1|=\ell_1s+a$ with $\ell_1\geq k_1$ and $a\geq 1$.  If $\ell_1>k_1$ or $a\geq 3$, then there is a set of $a$ vertex disjoint $s$-stars from $A_1$ to $B_2$ and a set of (at most) $s$ vertex disjoint $s$ stars from $B_2$ to $A_1$.  If $a\leq 2$, $\ell_1=k_1$ and $k_2-k_1\leq 1$, then there is a set of $a$ vertex disjoint $s$-stars from $A_1$ to $B_2$ and a set of (at most) $s$ vertex disjoint $s$ stars from $B_2$ to $A_1$.  If $a\leq 2$, $\ell_1=k_1$, $k_2\geq k_1+2$ and $|B_0\cup B_1|<(\ell_1+1)s$, then there is a set of $a$ vertex disjoint $s$-stars from $A_1$ to $B_2$.
%\end{lemma}


\begin{lemma}\label{lemma:diagonalsum}
Let $G[A,B]$ be a bipartite graph with $|A|=\ell_1s+a$ and $|B|=\ell_2s+b$ such that 
%$\ell_1s+\ell_2s=n$ and 
$1\leq b\leq s-1$.  Suppose further that $k_2s+\alpha^{2/3}k_2s\geq |A|,|B|\geq k_1s-\alpha^{2/3}k_2s$ and $\Delta(A, B), \Delta(B, A)\leq 2\alpha^{1/3} k_2s$.  If 
\begin{enumerate}
\item $a\geq 1$ and $\delta(A, B)+\delta(B, A)\geq 2s-3+a+b$ or

\item $a=0$ and $\delta(A, B)+\delta(B, A)\geq 2s-2+b$,
\end{enumerate}
then there is a set $\mathcal{S}_A$ of $a$ vertex disjoint $s$-stars from $A$ to $B$ and a set $\mathcal{S}_B$ of $b$ vertex disjoint $s$-stars from $B$ to $A$ such that the stars in $\mathcal{S}_A$ are disjoint from the stars in $\mathcal{S}_B$.
\end{lemma}

\begin{proof}
Let $\gamma$ be a real number such that $\alpha^{1/3}\ll \gamma\ll \frac{1}{2s}$.  

\noindent
\textbf{Case 1} $a>\frac{1}{\gamma}$.  Suppose first $\delta(B, A)\geq \frac{1}{2}(2s-3+a+b)$.  In this case we apply Lemma \ref{STARSlemma}(vi) to get a set of $b$ vertex disjoint $s$-stars with centers $C\subseteq B$ and leaves $L\subseteq A$.  Then since $\delta(B, A\setminus L)\geq \frac{1}{2}(2s-3+a+b)-bs>\frac{a}{4}$ we apply Lemma \ref{STARSlemma}(v) to get a set of $a$ vertex disjoint $s$-stars from $A\setminus L$ to $B\setminus C$.  Now suppose $\delta(A, B)> \frac{1}{2}(2s-3+a+b)$.  As before, we apply Lemma \ref{STARSlemma}(v) to get a set of $b$ vertex disjoint $s$-stars with centers $C\subseteq B$ and leaves $L\subseteq A$.  Then since $\delta(A, B\setminus C)> \frac{1}{2}(2s-3+a+b)-b>\frac{a}{4}$ we apply Lemma \ref{STARSlemma}(vi) to get a set of $a$ vertex disjoint $s$-stars from $A\setminus L$ to $B\setminus C$.

\noindent
\textbf{Case 2} $1\leq a\leq \frac{1}{\gamma}$.  Suppose first that $\delta(B, A)\geq s-1+a$.  We apply Lemma \ref{STARSlemma}(ii) to get a set of $a$ vertex disjoint $s$-stars with centers $C\subseteq A$ and leaves $L\subseteq B$.  We still have $\delta(B\setminus N(C), A\setminus C)\geq s-1+a$ and $|B\setminus N(C)|\geq |B|-\frac{2\alpha^{1/3}}{\gamma}k_2s\geq \alpha^{1/6}|A|$, thus we can apply Lemma \ref{STARSlemma}(i) to get a set of $b$ vertex disjoint $s$-stars from $B\setminus N(C)$ to $A\setminus C$.  Now suppose $\delta(A, B)\geq s+b$.  We apply Lemma \ref{STARSlemma}(ii) to get a set of $b$ vertex disjoint $s$-stars with centers $C\subseteq B$ and leaves $L\subseteq A$.  We still have $\delta(A\setminus L, B\setminus C)\geq s+b-b=s$ so we apply Lemma \ref{STARSlemma}(i) to get $a$ vertex disjoint $s$-stars from $A\setminus L$ to $B\setminus C$.

\noindent
\textbf{Case 3} $a=0$.  We have $\delta(A, B)+\delta(B, A)\geq 2s-2+b\geq 2s-1$ and thus $\delta(A, B)\geq s$ or $\delta(B, A)\geq s$.  In either case we can apply Lemma \ref{STARSlemma}(i) or (iii) to get a set of $b$ vertex disjoint $s$-stars from $B$ to $A$.

\end{proof}



\begin{lemma}\label{K_1sum}
Suppose $|U_0|\geq s$.  Let $V_1'\subseteq V_1$ and $V_2'\subseteq V_2$ such that $\delta(V_1', U_0)+\delta(V_2', U_0)\geq |U_0|+s$.  If $|V_1'|\geq\frac{n}{8}$ and $|V_2'|\geq\frac{n}{8}$, then for any $0\leq b\leq s$, there is a $K_{s,s}=:K$ with $s$ vertices in $U_0$, $b$ vertices in $V_1$ and $s-b$ vertices in $V_2$. 
\end{lemma}

For a proof see Chapter \ref{mindegtilingchapter} Claim \ref{K_1}.


% 
% \begin{proof}
% Let $$\ell:=s\binom{|U_2|}{\floor{t/2}}/\binom{\ceiling{(\alpha^{1/3}-\alpha^{2/3})n/2}}{\floor{t/2}}$$
% and recall that $|U_1|, |U_2|\leq (1+\alpha^{2/3})\frac{n}{2}$ by Claim \ref{bounds}. Thus we have
% \begin{equation}\label{ell}
% \ell\leq s\left(\frac{|U_2|}{(\alpha^{1/3}-\alpha^{2/3})\frac{n}{2}-\floor{t/2}}\right)^{\floor{t/2}}\leq  s\left(\frac{(1+\alpha^{2/3})\frac{n}{2}}{(\alpha^{1/3}-\alpha^{2/3})\frac{n}{3}}\right)^{\floor{t/2}}\leq s\left(\frac{3(1+\alpha^{2/3})}{2(\alpha^{1/3}-\alpha^{2/3})}\right)^{\floor{t/2}}  .\end{equation}
% 
% \noindent
% \textbf{Case 1. } $|V_0|\geq  \ell \binom{|U_1|}{\ceiling{t/2}}/\binom{\ceiling{(\alpha^{1/3}-\alpha^{2/3})n/2}}{\ceiling{t/2}}$.
% Recall that $\delta(V_0, U_i) \geq (\alpha^{1/3} - \alpha^{2/3})n/2$ for $i=1,2$ by Claim \ref{bounds} and suppose that there is no $K_{\ceiling{t/2},\ell}$ with $\ceiling{t/2}$ vertices in $U_1$ and $\ell$ vertices in $V_0$.  We count the $\ceiling{t/2}$-stars from $V_0$ to $U_1$ in two ways which gives $$|V_0| \binom{\ceiling{(\alpha^{1/3} - \alpha^{2/3})n/2}}{\ceiling{t/2}} < \ell \binom{|U_1|}{\ceiling{t/2}}$$
% contradicting the lower bound for $|V_0|$. Consequently there is a complete bipartite graph $K'=K_{\ceiling{t/2},\ell}$ with $\ceiling{t/2}$ vertices in $U_1$ and $\ell$ vertices in $V_0$. If there is no $K_{\floor{t/2},s}$ with $s$ vertices in $V(K') \cap V_0$ and $\floor{t/2}$ vertices in $U_2$, then a similar counting argument gives $$\ell \binom{\ceiling{(\alpha^{1/3} - \alpha^{2/3})n/2}}{\floor{t/2}} < s\binom{|U_2|}{\floor{t/2}}$$
% contradicting the definition of $\ell$.
% 
% \noindent
% \textbf{Case 2.} $|V_0|<  \ell \binom{|U_1|}{\ceiling{t/2}}/\binom{\ceiling{(\alpha^{1/3}-\alpha^{2/3})n/2}}{\ceiling{t/2}}$.  
% By \eqref{ell}, we have
% $$|V_0|< 
% \ell\left(\frac{3(1+\alpha^{2/3})}{2(\alpha^{1/3}-\alpha^{2/3})}\right)^{\ceiling{t/2}}\leq s\left(\frac{3(1+\alpha^{2/3})}{2(\alpha^{1/3}-\alpha^{2/3})}\right)^{t}.$$  Let $p := \delta(\hat{U_1}, V_0)$, and note that $p\geq s$ by (\ref{V_0}).  We claim that there is a complete bipartite graph $K':= K_{\ceiling{t/2},p}$ with $\ceiling{t/2}$ vertices in $\hat{U_1}$ and $p$ vertices in $V_0$. Let $c$ be the number of $p$-stars with centers in $\hat{U_1}$ and leaves in $V_0$.  We have $c\geq |\hat{U_1}|\geq \frac{n}{8}$ and if no $p$-subset of $V_0$ is in $\ceiling{t/2}$ of such stars, i.e. $K'$ does not exist, we have $c\leq(\ceiling{t/2}-1)\binom{|V_0|}{p}$ which contradicts the fact that $|V_0|$ is $O(1)$ and $n$ is sufficiently large (with respect to $\alpha$, $t$, and consequently $|V_0|$). From (\ref{V_0}) we have $\delta(\hat{U_2},V_0)\geq|V_0|-p+s$, so every vertex $u \in  \hat{U_2}$ has at least $s$ neighbors in $V(K')\cap V_0$. Repeating the argument above by counting $s$-stars with centers in $\hat{U_2}$ and leaves in $V(K')\cap V_0$ gives $K'':=K_{s,\floor{t/2}}$.  Now choose $K^1\subseteq K'\cup K''$ having the property that $|V_0 \cap V(K^1)|=s$, $|U_1\cap V(K^1)|=\ceiling{t/2}$, and $|U_2\cap V(K^1)|=\floor{t/2}$ as desired.
% \end{proof}



\subsection{$k_2\gg k_1$: Proof of Theorem \ref{main 2}}

In this section we prove Theorem \ref{main 2}, which at the same time proves Theorem \ref{main 1} when $k_1\leq (1-\frac{1}{2s})k_2$.  Let $G$ be a graph which satisfies the extremal condition and for which $k_1\leq (1-\frac{1}{2s})k_2$.  Recall the bounds from Claim \ref{preprocess}, specifically $k_1s-\alpha^{2/3}k_2s\leq|U_1|,|V_1|\leq k_1s+\alpha^{2/3}k_1s$, $k_2s-\alpha^{2/3}k_1s\leq|U_2|,|V_2|\leq k_2s+\alpha^{2/3}k_2s$, and $|U_0|,|V_0|\leq \alpha^{2/3}n$.  The fact that $\delta_U+\delta_V\geq n$ implies 
\begin{equation}\label{V1toU2:big}
\delta(V_1,U_2)\geq \delta_V-|U_0\cup U_1|\geq (k_2-k_1-2\alpha^{2/3}k_1)s
\geq(\frac{1}{2s} k_2-2\alpha^{2/3}k_1)s>\frac{1}{4s}k_2s.
\end{equation}

First we prove Theorem \ref{main 2}.

% \begin{proposition}
% Let $s\geq 2$ and let $p\in \mathbb{N}$ be maximal such that $s=p^2+q$ where $0\leq q\leq 2p$.  Set $c(s)=0$ if $q=0$ or $p+1\leq q\leq 2p$ and set $c(s)=1$ if $1\leq q\leq p$.  Suppose \begin{equation}|V_0\cup V_1|<\ell s \vee |U_1|>\ell s,\label{maincondition}\end{equation}
% for all $\ell\in \mathbb{N}$.  Furthermore suppose $\delta_V\geq \delta_U=\Omega(n)$.  If
% \begin{enumerate}
% \item $\delta_U+\delta_V\geq n+3s-5$ or
% 
% \item $k_2\geq (s-d)k_1$ and $\delta_U+\delta_V\geq n+2s-2\croot{s}+d+c(s)$ for some $0\leq d\leq s-2\croot{s}+c(s)+1$,
% \end{enumerate}
% then $G$ can be tiled with $K_{s,s}$.
% \end{proposition}

\begin{proof}
Note that $s-2\croot{s}+c(s)+1\geq 0$ with equality if and only if $s=2$, so $d$ is defined for all $s\geq 2$.  Let $\alpha^{1/3}\ll \gamma\ll \frac{1}{2s}$.  Let $\ell_1$ be maximal so that $|U_1|\geq \ell_1 s$ and $|V_0\cup V_1|\geq \ell_1 s$.  Let $y:=|U_1|-\ell_1 s$ and $z:=|V_0\cup V_1|-\ell_1 s$.  We note that $n+3s-5\geq n+2s-2\croot{s}+d+c(s)$ with equality if and only if $s=2$.  So for this proof we will assume $\delta_U+\delta_V\geq n+2s-2\croot{s}+d+c(s)$ with one exception that we point out.

\begin{claim}\label{V1>U1}
If there exists $\ell$ such that $|V_0\cup V_1|\geq \ell s$ and $|U_1|\leq \ell s$, then $G$ can be tiled with $K_{s,s}$.  
\end{claim}

\begin{proof}
Suppose there exists such an $\ell$.  By the choice of $\ell_1$, we can assume $|U_1|\leq (\ell_1+1)s$ and $|V_0\cup V_1|\geq (\ell_1+1)s$. By \eqref{V1toU2:big} we have $\delta(V_1,U_2)>\frac{1}{4s}k_2s\geq 2s\alpha^{2/3}n$ and thus we can greedily choose a set of $z-s$ vertex disjoint $s$-stars from $V_1$ to $U_2$ with centers $C_V$ and leaves $L_U$.  Let $V_1':=V_1\setminus C_V$ and $U_2':=U_2\setminus L_U$, since $\delta(V_1',U_2')\geq \frac{1}{8s} k_2s$ we may apply Lemma \ref{Zhao lemma} to the graph induced by $U_2'$ and $V_1'$ to get a set of $s-y$ vertex disjoint $s$-stars from $U_2'$ to $V_1'$.  We move the centers of the stars giving $|U_1|+(s-y)=(\ell_1+1)s=|V_0\cup V_1|-(z-s)$ and we are done.
\end{proof}

If $z\geq s$, then by the maximality of $\ell_1$ we have $y<s$ and thus we can apply Claim \ref{V1>U1} to finish.  If $y=0$, then we can also apply Claim \ref{V1>U1} to finish.  So for the rest of the proof, suppose that $0\leq z\leq s-1$ and $1\leq y$.  Our goal is to show that there exists a set $\mathcal{S}_U$ of vertex disjoint $(s-x)$-stars from $U_1$ to $V_2$ such that $|V_0\cup V_1|-x|\mathcal{S}_U|\geq |U_1|-|\mathcal{S}_U|=\ell_1 s$ and a set $\mathcal{T}_V$ of vertex disjoint $s$-stars from $V_1$ to $U_2$ so that $|V_0\cup V_1|-x|\mathcal{S}_U|-|\mathcal{T}_V|=\ell_1 s$ for some $0\leq x\leq s-1$.
%Note that (\ref{maincondition}) implies $|U_1|+s-2\geq|V_0\cup V_1|$.  
Since $\delta_U+\delta_V\geq n+2s-2\croot{s}+d+c(s)$, we have 
\begin{align}
\delta(U_1,V_2)+\delta(V_2,U_1)&\geq n+2s-2\croot{s}+d+c(s)-|V_0\cup V_1|-|U_0\cup U_2|\notag\\
&\geq 2s-2\croot{s}+d+c(s)+y-z\label{diagonal}
%&\geq s-2\croot{s}+2+c(s)\notag
\end{align}

\noindent
\textbf{Case 1} $|U_1|-|V_0\cup V_1|>0$.

\textbf{Case 1.1} $y\geq \frac{1}{\gamma}$. We have 
\begin{align*}
\delta(U_1,V_2)+\delta(V_2,U_1)&\geq2s-2\croot{s}+d+c(s)+y-z\\
&\geq y+s-2\croot{s}+d+c(s)+1
\end{align*}
and thus there are two cases. Either $\delta(U_1,V_2)\geq \frac{1}{2}(y+s-2\croot{s}+d+c(s)+1)$ and we apply Lemma \ref{STARSlemma}(vi) to get $y$ vertex disjoint $s$-stars from $U_1$ to $V_2$ or $\delta(V_2,U_1)> \frac{1}{2}(y+s-2\croot{s}+d+c(s)+1)$ and we apply Lemma \ref{STARSlemma}(v) to get $y$ vertex disjoint $s$-stars from $U_1$ to $V_2$.  We move the centers from $U_1$ to $U_2$ to make $|U_1|=\ell_1 s$.  Then we move vertices from $V_0\cup V_1$ to $V_2$ to make $|V_0\cup V_1|=\ell_1 s$.

\textbf{Case 1.2} $y<\frac{1}{\gamma}$.

%\textbf{Case 2.1.} $y>z$.  %Recall that $\delta(U_1,V_2)+\delta(V_2,U_1)\geq 2s-2\croot{s}+c(s)+y-z$.

\textbf{Case 1.2.1.} $\delta(U_1,V_2)\geq s$. Apply Lemma \ref{STARSlemma}(i) with $x=0$ to get $y$ vertex disjoint $s$-stars from $U_1$ to $V_2$.

\textbf{Case 1.2.2.} $\delta(U_1,V_2)\leq s-1$.  By \eqref{diagonal} we have $\delta(V_2,U_1)\geq 2s-2\croot{s}+d+c(s)+y-z-(s-1)=s-2\croot{s}+d+c(s)+1+y-z\geq d+1$.  Since $k_2\geq (s-d)k_1$ and thus $|V_2|\geq (s-\frac{1}{2}-d)|U_1|\geq \frac{s-\frac{1}{2}}{d+1}|U_1|$, we can apply Lemma \ref{STARSlemma}(iv) to get $y$ vertex disjoint $s$-stars from $U_1$ to $V_2$.


\textbf{Case 2.} $|U_1|-|V_0\cup V_1|\leq 0$.  In this case we have $y\leq z$.  Rearranging \eqref{diagonal} gives 
\begin{equation}
\delta(U_1,V_2)+\delta(V_2,U_1)\geq 2s-2\croot{s}+d+c(s)-(z-y). \label{z-y}
\end{equation} 
Also since $k_1\leq \frac{k_2}{s-d}$, we have
\begin{align}
\delta(V_1, U_2)\geq \delta_V-|U_0\cup U_1|\geq (k_2-k_1-2\alpha^{2/3}k_1)s
&\geq(1-\frac{1+2\alpha^{2/3}}{s-d})k_2s \notag \\
&\geq  \frac{s-d-1-2\alpha^{2/3}}{(s-d)(1+\alpha^{2/3})}|U_2| \notag \\
&\geq \frac{s-d-1-\alpha^{1/3}}{s-d}|U_2|\label{V_1U_2}
\end{align}  

If $\delta_U+\delta_V\geq n+3s-5$, then \eqref{z-y} gives $\delta(U_1,V_2)+\delta(V_2,U_1)\geq 2s-3$ since $z-y\leq s-2$. Thus we have  $\delta(V_2, U_1)\geq s-1$ or $\delta(U_1, V_2)\geq s-1$.  In either case we can get $y$ vertex disjoint $(s-1)$-stars from $U_1$ to $V_2$ by Lemma \ref{STARSlemma}(iii) or Lemma \ref{STARSlemma}(i) with $x=1$.  For each $(s-1)$-star we choose a vertex from $V_1$ and $(s-1)$-vertices in $U_2$, which is possible by \eqref{V_1U_2} and $z\geq y$.  So for the rest of the proof we assume $\delta_U+\delta_V\geq n+2s-2\croot{s}+d+c(s)$.

%If $s=2$, then we must have $y=z=1$.  In this case, we only need a single edge between $U_1$ and $V_2$, which exists since $\delta(U_1,V_2)+\delta(V_2,U_1)\geq 1$.  So we may assume that $s\geq 3$.

\textbf{Case 2.1.} $z-y\leq s-2\croot{s}+c(s)+1$.  %So \eqref{z-y} gives $\delta(U_1,V_2)+\delta(V_2,U_1)\geq s-1$.

\textbf{Case 2.1.1.} $\delta(U_1,V_2)\geq s-1$. We can get $y$ vertex disjoint $(s-1)$-stars from $U_1$ to $V_2$ by Lemma \ref{STARSlemma}(i) with $x=1$.  For each $(s-1)$-star we choose a vertex from $V_1$ and $(s-1)$-vertices in $U_2$, which is possible by \eqref{V_1U_2} and $z\geq y$.

\textbf{Case 2.1.2.} $\delta(U_1,V_2)\leq s-2$. So \eqref{z-y} and the condition of Case 2.2.1. gives $$\delta(V_2,U_1)\geq 2s-2\croot{s}+d+c(s)-(s-2\croot{s}+c(s)+1)-(s-2)=d+1.$$  We can get $y$ vertex disjoint $s$-stars from $U_1$ to $V_2$ by Lemma \ref{STARSlemma}(iv) as in Case 1.2.2.

\textbf{Case 2.2.} $z-y\geq s-2\croot{s}+c(s)+2$.  If $\delta(U_1,V_2)\geq s-1$ or $\delta(V_2,U_1)\geq d+1$, then we would be done as in the previous two cases.  So suppose $\delta(U_1,V_2)\leq s-2$ and $\delta(V_2,U_1)\leq d$ .  By \eqref{z-y}, we have
\begin{align}
s-2\geq s-x=\delta(U_1,V_2)&\geq 2s-2\croot{s}+d+c(s)-(z-y)-\delta(V_2, U_1)\label{s-x} \\
&\geq s-2\croot{s}+c(s)+2\geq d+1 \notag
\end{align}
for some $2\leq x\leq s-d-1$.

Let $\mathcal{S}_U$ be a set of $y$ vertex disjoint $(s-x)$-stars from $U_1$ to $V_2$, which exists by Lemma \ref{STARSlemma}(i).  For each $(s-x)$-star in $\mathcal{S}_U$ we will choose $s-1$ vertices from $U_2$ and $x$ vertices from $V_1$ to complete a copy of $K_{s,s}$. Let $u_1$ be the center of a star in $\mathcal{S}_U$ and let $v_1^1,v_1^2,\dots, v_1^x$ be a set of $x$ vertices in $N(u_1)\cap V_1$. By \eqref{V_1U_2}, we have $|N(v_1^1, v_1^2, \dots, v_1^x)\cap U_2|\geq \left(1-\frac{x(1+\alpha^{1/3})}{s-d}\right)|U_2|$.  Let $v_2^1,v_2^2,\dots, v_2^{s-x}$ be a set of $s-x$ vertices in $V_2$. By Claim \ref{preprocess}, we have $|N(v_2^1, v_2^2,\dots,v_2^{s-x})\cap U_2|\geq (1-(s-x)\alpha^{1/3})|U_2|$.  Thus 
\begin{align*}
|N(v_1^1, v_1^2, \dots, v_1^x,v_2^1,v_2^2,\dots, v_2^{s-x})\cap U_2|&\geq \left(1-\frac{x(1+\alpha^{1/3})}{s-d}-(s-x)\alpha^{1/3}\right)|U_2|\\ &\geq \alpha|U_2|
\end{align*}
and we can choose $x$ vertices from $V_1$ and $s-1$ vertices from $U_2$ to turn each $s-x$ star into a copy of $K_{s,s}$.

Finally we must be sure that $|V_0\cup V_1|-xy\geq \ell s$, i.e. $z\geq xy$.  There are two cases.

\textbf{Case 2.2.1.} $1\leq q\leq p$ and consequently $c(s)=1$.  By \eqref{s-x} and $\delta(V_2, U_1)\leq d$, we get 
\begin{equation}
x+y\leq z-(s-2\croot{s}+1)\label{x+y first}
\end{equation}  
and thus
\begin{align*}
xy\leq\left(\frac{z-(s-2\croot{s}+1)}{2}\right)^2\leq z.
\end{align*}
The first inequality is by \eqref{x+y first} and the arithmetic mean-geometric mean inequality.  To verify the second inequality, let $F(z)=z-\left(\frac{z-(s-2\croot{s}+1)}{2}\right)^2$ and note $s-2\croot{s}+3\leq z\leq s-1$.  Using calculus, we see that $F$ achieves a maximum at $s-2\croot{s}+3$, $F$ is decreasing on the interval $[s-2\croot{s}+3, s-1]$ and $F(s-1)=s-1-(\croot{s}-1)^2=p^2+q-1-p^2\geq 0$.

\textbf{Case 2.2.2.} $q=0$ or $p+1\leq q\leq 2p$ and consequently $c(s)=0$.  By \eqref{s-x} and $\delta(V_2, U_1)\leq d$, we get
\begin{equation}
x+y\leq z-(s-2\croot{s}).\label{x+y second}
\end{equation}  
If $z=s-1$, then \eqref{x+y second} gives $x+y\leq 2\croot{s}-1$.  Since $2\croot{s}-1$ is odd, we have
\begin{align*}
xy\leq\left(\frac{2\croot{s}}{2}\right)\left(\frac{2\croot{s}-2}{2}\right)=\croot{s}(\croot{s}-1)\leq s-1=z
\end{align*}
where the last inequality holds by the assumption of this case.  So we may assume $z\leq s-2$.  So we have
\begin{align*}
xy\leq\left(\frac{z-(s-2\croot{s})}{2}\right)^2\leq z.
\end{align*}
The first inequality holds by (\ref{x+y second}) and the arithmetic mean-geometric mean inequality.  To verify the second inequality, let $F(z)=z-\left(\frac{z-(s-2\croot{s})}{2}\right)^2$ and note $s-2\croot{s}+2\leq z\leq s-2$.  Using calculus, we see that $F$ achieves a maximum at $s-2\croot{s}+2$, $F$ is decreasing on the interval $[s-2\croot{s}+2, s-2]$ and $F(s-2)=s-2-(\croot{s}-1)^2$.  When $q=0$ we have $p\geq 2$, and thus $F(s-2)=s-2-(\croot{s}-1)^2=p^2-2-(p^2-2p+1)=2p-3\geq 1$.  When $q\geq p+1$, we have $F(s-2)=s-2-(\croot{s}-1)^2=p^2+q-2-p^2=q-2\geq 0$.

\end{proof}



\subsection{$k_2\approx k_1$: Proof of Theorem \ref{main 1}}

In this section we prove Theorem \ref{main 1} when $k_1> (1-\frac{1}{2s})k_2$.  Recall that $k_1\leq k_2$.  We first give a proof when $s=2$ since this is often a special case in the general argument.  Also, the case $s=2$ may be of independent interest considering Conjecture \ref{con:W2}.

We start with a graph which satisfies the extremal condition after pre-processing.  For $i=1,2$, let $U_i^M=\{u\in U_i:\deg(u, V_{3-i})>\alpha^{1/3}n\}$ and $V_i^M=\{v\in V_i: \deg(v, U_{3-i})>\alpha^{1/3}n \}$.  We call these vertices \emph{movable}.  Note that $U_1^M=\emptyset=V_2^M$ by Claim \ref{preprocess}.

\subsubsection*{s=2}
Let $\gamma$ be a real number such that $\alpha^{1/3}\ll \gamma\ll \frac{1}{2s}$.  We assume that $n=2m$ and $\delta_V>\delta_U$, thus $\delta_V \geq \frac{n}{2}+1$. As a result
\begin{equation}\label{twocommon}
\forall v, v'\in V, |N(v)\cap N(v')|\geq 2
\end{equation}
Furthermore, since $\delta_V\geq \frac{n}{2}+1$, and since there is some vertex $u\in U$ with $\deg(u, V)\leq \frac{n}{2}$, 
\begin{equation}\label{u*}
\exists u^* \in U \mbox{ such that } \deg(u^*, V)\geq \frac{n}{2}+2.  
\end{equation}

\noindent
\textbf{Case 1.} $U_0\cup U_2^M\neq \emptyset$ or $|U_2|$ is even.  There are two cases: (i) $|V_0\cup V_1|>|U_1|$ 
%or $|V_0\cup V_1|=|U_1|=2p$ for some $p\in \mathbb{N}$, 
or (ii) $|V_2|\geq |U_0\cup U_2|$.  If (i) is the case 
%and $|V_0\cup V_1|=|U_1|=2p$ for some $p\in \mathbb{N}$, then we are done.  Otherwise 
there exists some $\ell_1\in \mathbb{N}$, $X\subseteq U_0\cup U_2^M$, and $Y\subseteq V_0\cup V_1^M$ such that $|U_1\cup X|=\ell_1s$, $|(V_0\cup V_1)\setminus Y|\geq \ell_1s$ and $|(V_0\cup V_1)\setminus Y|-|U_1\cup X|$ is as small as possible.  If $|(V_0\cup V_1)\setminus Y|-|U_1\cup X|=0$, then we are done.  Otherwise there are no movable vertices left in $(V_0\cup V_1)\setminus Y$.   If (ii) is the case, then there exists some $\ell_2\in \mathbb{N}$ and $X\subseteq U_0\cup U_2^M$ with $|X|\leq 1$ such that $|(U_0\cup U_2)\setminus X|=\ell_2s$, $|V_2|\geq \ell_2s$ and $|V_2|-|(U_0\cup U_2)\setminus X|$ is as small as possible.

Notice that in either case, we are either done or there are no movable vertices left in $(V_0\cup V_1)\setminus Y$ or $V_2$.  Because of this symmetry we can suppose without loss of generality that that (i) is the case.  We reset $U_1:=U_1\cup X$ , $U_0:=(U_0\cup U_2^M)\setminus X$, $U_2:=U_2\setminus U_2^M$, $V_1:=V_1\setminus Y$, and $V_0:=V_0\cup Y$.  Let $\ell_2=m-\ell_1$.  Let $a:=|V_1|-\ell_1s$.  If $a=0$, then we are done, so suppose $a\geq 1$.  Note that there are no movable vertices in $V_1$ or $U_2$.  We have \begin{equation}\label{bothdirections}\delta(V_1, U_0\cup U_2)+\delta(U_0\cup U_2, V_1)\geq a+1.\end{equation} 

\textbf{Case 1.1.} $a>\frac{1}{\gamma}$.  We know that $|U_0|\leq 1$, otherwise we could make $a$ smaller by moving $2$ vertices from $U_0$ to $U_1$ while maintaining the fact that $|U_1|$ is even.  Either $\delta(V_1, U_2)\geq \delta(V_1, U_0\cup U_2)-1\geq \frac{a+1}{2}-1$ and we apply Lemma \ref{STARSlemma}(vi) to get $a$ vertex disjoint $2$-stars from $V_1$ to $U_2$ or else $\delta(U_0\cup U_2, V_1)> \frac{a+1}{2}$ and we apply Lemma \ref{STARSlemma}(v) to get $a$ vertex disjoint $2$-stars from $V_1$ to $U_2$.  We move the centers from $V_1$ to $V_2$ to make $|V_1|=\ell_1 s$.

\textbf{Case 1.2.} $a\leq \frac{1}{\gamma}$. If $\delta(U_0\cup U_2, V_1)\geq 2$, then we apply Lemma \ref{STARSlemma}(iii) to get a set of $a$ vertex disjoint $2$-stars from $V_1$ to $U_2$. So suppose $\delta(U_0\cup U_2, V_1)\leq 1$ and thus \begin{equation}\label{V1toU2}\delta(V_1, U_0\cup U_2)\geq a.\end{equation}

\textbf{Case 1.2.1.} $a\geq 3$.  We know that $|U_0|\leq 1$, otherwise we could make $a$ smaller by moving $2$ vertices from $U_0$ to $U_1$ while maintaining the fact that $|U_1|$ is even.  Since $a\geq 3$, we have $\delta(V_1, U_2)\geq \delta(V_1, U_0\cup U_1)-1\geq 2$ by \eqref{V1toU2}, and thus we can apply Lemma \ref{STARSlemma}(i) to get a set of $a$ vertex disjoint $2$-stars from $V_1$ to $U_2$.
So we only need to deal with the case $a\leq 2$.  

\textbf{Case 1.2.2.} $a=2$.  If $U_0=\emptyset$, then we can use \eqref{V1toU2} and apply Lemma \ref{STARSlemma}(i) to get a set of $a$ vertex disjoint $2$-stars from $V_1$ to $U_2$.  So suppose $U_0=\{u_0\}$.  If there is a vertex $u\in U_2$ with $\deg(u, V_1)=0$, then by (\ref{bothdirections}) we have $\delta(V_1, U_0\cup U_2)\geq 3$ and we are done since $\delta(V_1, U_2)\geq \delta(V_1, U_0\cup U_1)-1\geq 2$. So suppose $\delta(U_0\cup U_2)\geq 1$.  If there is a vertex $u\in U_2$ with $\deg(u, V_1)\geq 2$, then we can move $u_0$ and $u$ to $U_1$, thus for all $u\in U_2$, $\deg(u, V_1)=1$.  Now suppose there is a vertex $v_1\in V_1$ with $\deg(v_1, U_2)\geq 2$ and let $u_2, u_2'\in N(v)\cap U_2$.  Let $v_1'\in N(u_0)\cap (V_1\setminus \{v_1\})$.  Since $\Delta(U_2, V_1)\leq 1$, there exists some $u'\in (U_2\setminus \{u_2, u_2'\})\cap N(v_1')$.  Thus we can move $v_1$ and $v_1'$.  So for all $v\in V_1$, $\deg(v, U_2)=1$.  This implies that $\ell_2s-1=|U_2|=|V_1|=\ell_1s+2$, a contradiction.

\textbf{Case 1.2.3.} $a=1$.  If $U_0\neq \emptyset$, then let $u_0\in U_0$.  Let $u_2v_1\in E(V_1, (U_0\cup U_2)\setminus \{u_0\})$, which exists be \eqref{bothdirections}.  Let $v_2\in N(u_2)\cap V_2$.  By \eqref{twocommon}, $v_1$ and $v_2$ have a common neighbor $u'$ different than $u_2$.  If $u'\in U_0\cup U_2$, then we are done by simply moving $v_1$, so we have $u'\in U_1$ which completes a $K_{2,2}$.  Now we move $u_0$ to $U_1$ to finish.

Finally, suppose $U_0=\emptyset$.  If there exists a vertex $v\in V_1$ such that $\deg(v, U_2)\geq 2$, then we can move $v$ and be done.  So suppose $\Delta(V_1, U_2)\leq 1$.  Furthermore if there was a vertex $v\in V_1$ such that $\deg(v, U_2)=0$, then \eqref{bothdirections} would imply $\delta(U_2, V_1)\geq 2$ contradicting the fact that $\Delta(V_1, U_2)\leq 1$.  So every vertex in $V_1$ has exactly one neighbor in $U_2$ and \eqref{bothdirections} implies $\delta(U_2, V_1)\geq 1$.  Since $|U_2|$ is even and $|V_1|$ is odd, we must have $|V_1|\neq |U_2|$.  If $|U_2|>|V_1|$, then $\delta(U_2, V_1)\geq 1$ would imply that there was a vertex in $V_1$ with two neighbors in $U_2$, so suppose $|V_1|>|U_2|$. This implies that there exists some $u_0\in U_2$ such that $\deg(u_0, V_1)\geq 2$. Let $u_2v_1\in E(V_1, U_2\setminus \{u_0\})$, which exists be \eqref{bothdirections}.  Let $v_2\in N(u_2)\cap V_2$.  By \eqref{twocommon}, $v_1$ and $v_2$ have a common neighbor $u'$ different than $u_2$.  If $u'\in U_2$, then we are done by simply moving $v_1$, so we have $u'\in U_1$ which completes a $K_{2,2}$.  Now we move $u_0$ to $U_1$ to finish.


\noindent
\textbf{Case 2.} $U_0\cup U_2^M= \emptyset$ and $|U_2|$ is odd.  Now there are no movable vertices in $U_1$ or $U_2$.  So choose $\ell_1, \ell_2$ such that $|U_1|=\ell_1s+1$, $|U_2|=\ell_2s-1$.  If it is not the case that $|V_0\cup V_1|\geq \ell_1s+2$ or $|V_0\cup V_2|\geq \ell_2s$, then $V_0=\emptyset$, $|V_1|=\ell_1s+1$, $|V_2|=\ell_2s-1$, and $V_1^M=\emptyset$.  Without loss of generality, suppose $|V_0\cup V_1|\geq \ell_1s+1$.  
Let $b:=|V_1\cup V_0|-|U_1|$.

%We move vertices from $V_0$ to $V_2$ so that the gap  is as small as possible with respect to the condition $|V_0\cup V_1|\geq \ell_1s+1$.  Note that if $a=0$, then there are no movable vertices in any set and if $a\geq 2$, then there are no movable vertices in $V_1$.  

\textbf{Case 2.1.} $b=0$.  Note that since $b=0$, $U_0=V_0=U_2^M=V_1^M=\emptyset$ for $i=1,2$.  We first show that if there is a vertex $u_i\in U_i$ such that $\deg(u_i, V_{3-i})\geq 2$, then we would be done.  Without loss of generality, suppose there exists $u_1\in U_1$ such that $\deg(u_1, V_2)\geq 2$.  Let $v,v'\in N(u_1)\cap V_2$.  Since $\delta(V_1, U_2)+\delta(U_2, V_1)\geq 1$, there is an edge $v_1u_2\in E(V_1, U_2)$.  Let $v_2\in V_2\cap N(u_2)\setminus\{v,v'\}$.  By \eqref{twocommon} we know that $v_1$ and $v_2$ have a common neighbor $u_0$ which is different than $u_2$.  If $u_0\in U_1$, then we have a copy of $K_{2,2}$ with one vertex in each of $U_1, U_2, V_1, V_2$ and we are done, so suppose $u_0\in U_2$.  Then we choose $u'\in (N(v)\cap N(v'))\cap (U_2\setminus\{u_0\})$.  Thus we can move $u$ and $v_1$ to finish.  So we may suppose that 
\begin{equation}\label{atmost1}
\Delta(U_1, V_2), \Delta(U_2, V_1)\leq 1.
\end{equation}
By (\ref{u*}), there is a vertex $u^*\in U$ such that $\deg(u^*, V)\geq \frac{n}{2}+2$.  Without loss of generality, suppose $u^*\in U_1$.  Then by (\ref{atmost1}) we have $|U_1|=|V_1|\geq \frac{n}{2}+1$, which in turn implies that $|U_2|=|V_2|\leq \frac{n}{2}-1$.  However, now we have $\delta(V_2, U_1)\geq 2$, and thus there exists $u\in U_1$ such that $\deg(u, V_2)\geq 2$, contradicting (\ref{atmost1}). 


\textbf{Case 2.2.} $b\geq 1$.  Suppose first that $|V_1\setminus V_1^M|\geq \ell_1s+3$.  Let $b_1':=|V_1\setminus V_1^M|-(\ell_1s+2)$.   We have 
$$
\delta(V_1\setminus V_1^M, U_2)+\delta(U_2, V_1\setminus V_1^M)\geq n+1-(\ell_1s+1+\ell_2s-2-b_1')=b_1'+2.
$$
So we apply Lemma \ref{lemma:diagonalsum}(i) with $A=V_1\setminus V_1^M$ and $B=U_2$ to get a set of $b_1'$ vertex disjoint $s$ stars from $V_1\setminus V_1^M$ to $U_2$ and one $s$-star from $U_2$ to $V_1\setminus V_1^M$.

So we may suppose $|V_1\setminus V_1^M|\leq \ell_1s+2$.  Reset $V_1:=V_1\setminus V_1^M$ and $V_0:=V_0\cup V_1^M$, then partition $V_0=V_0^1\cup V_0^2$ so that $|V_1\cup V_0^1|=l_1s+2$ and $|V_2\cup V_0^2|=l_2s-2$.  We have
\begin{equation}\label{b=1}
\delta(V_1\cup V_0^1, U_2)+\delta(U_2, V_1\cup V_0^1)\geq n+1-(\ell_1s+1+\ell_2s-2)=2.
\end{equation}
We first observe that if $\delta(V_1\cup V_0^1, U_2)\geq 2$, then there will be a vertex $u_2\in U_2$ such that $\deg(u_2, V_1)\geq 2$ in which case we would be done, so suppose not.  This implies that $|U_1|\geq \frac{n}{2}$.

First assume that $|V_0^1|\leq 1$.  By (\ref{b=1}), one of $\delta(U_2, V_1\cup V_0^1)\geq 2$ or $\delta(V_1\cup V_0^1, U_2)\geq 1$ must hold.  Since $|V_1\cup V_0^1|>|U_2|$, in either case there is a vertex $u\in U_2$ such that $\deg(u, V_1\cup V_0^1)\geq 2$, in which case we are done since $|V_0^1|\leq 1$.

So suppose $|V_0^1|\geq 2$.  Now if $\delta(V_2\cup V_0^2, U_1)\geq 2$, then there will be a vertex $u_1\in U_1$ such that $\deg(u_1, V_2)\geq 2$ in which case we would be done, since we can also move two vertices from $V_0^2$, so suppose not.  This implies that $|U_2|\geq \frac{n}{2}$ and since $|U_1|\geq \frac{n}{2}$, we have $|U_1|=|U_2|=\frac{n}{2}$.  So let $v_2\in V_2$ with $\deg(v_2, U_1)=1$ and let $v_1\in N(u_1)\cap V_1$.  By \eqref{twocommon}, $v_1$ and $v_2$ have a common neighbor in $U_2$ (since $\deg(v_2, U_1)=1$) which completes a $K_{2,2}$.  We finish by moving one additional vertex from $V_0^1$ to $V_2$.





\subsubsection*{$s\geq 3$}

The following proof has many cases, so we provide an outline for reference.

\begin{easylist} 

\ListProperties(Style*=\bfseries)

# $|V_1|\leq k_1s$ and $|V_0\cup V_1|\leq k_1s+r$
 
# $\exists \ell_1\geq k_1$, $\exists Y\subseteq V_1^M$ and $\exists V_0'\subseteq V_0$ such that $|(V_1\setminus Y)\cup V_0'|=\ell_1s$.  
%Let $\ell_1\geq k_1$ be minimal. 
%In this case we reset $V_1:=V_1\setminus Y$ and $V_0:=V_0\cup (V_1^M\setminus Y)$.

## $|V_1|\leq k_1s$

### $|V_0\cup V_1|\geq k_1s+s$

%#### $k_2=k_1$

%#### $k_2=k_1+1$

%#### $k_2\geq k_1+2$

### $|V_0\cup V_1|<k_1s+s$

%#### $k_2=k_1$

%#### $k_2=k_1+1$

%#### $k_2\geq k_1+2$

%##### $|U_0\cup U_1|\geq k_1s+s$

%##### $|U_0\cup U_1|<k_1s+s$

## $|V_1|>k_1s$

### $|V_1\setminus V_1^M|\leq k_1s$

#### $|U_0\cup U_2|\geq k_2s$

#### $|U_0\cup U_2|<k_2s$ 
%(i.e. $|U_1|>k_1s$)

##### $|V_0\cup V_1|\geq k_1s+s$

###### $|U_0\cup U_1|\geq k_1s+s$

###### $|U_0\cup U_1|<k_1s+s$

##### $|V_0\cup V_1|<k_1s+s$

### $|V_1\setminus V_1^M|>k_1s$

#### $\exists \ell_1$, $\exists Y\subseteq V_1^M$ such that $|V_1\setminus Y|=\ell_1s$ 
%(choose $\ell_1$ minimal, note $\ell_1>k_1$)

##### $|U_0\cup U_2|< \ell_2s$ (i.e. $|U_1|>\ell_1s$)

##### $|U_0\cup U_2|\geq \ell_2s$

#### $\exists \ell_1$, $\exists V_0'\subseteq V_0$ such that $|V_1\cup V_0'|=\ell_1s$ 
%(choose $\ell_1$ minimal, note $\ell_1>k_1$)

##### $|U_0\cup U_2|<\ell_2 s$ 
%(i.e. $|U_1|>\ell_1s$)

##### $|U_0\cup U_2|\geq \ell_2 s$

%###### $|U_1|\leq \ell_1s-s$
% 
% ####### $a_2=2$, $b_1=1$
% 
% ####### $a_2=1$, $b_1=2$
% 
% ####### $a_2=1$, $b_1=1$

%###### $\ell_1s-s<|U_1|$
% 
% ####### $a_2=1$, $b_1=2$
% 
% ####### $a_2=2$, $b_1=1$
% 
% ####### $a_2=1$, $b_1=1$

# For some $\ell_1\geq k_1$ we have $\ell_1s<|V_1\setminus V_1^M|\leq |V_1\cup V_0|<\ell_1s+s$

## $|U_2\setminus U_2^M|\geq \ell_2s$

## $|U_2\setminus U_2^M|<\ell_2s$

### $|U_0\cup U_1|\geq \ell_1s+s$

#### $|U_1|\leq \ell_1s$

#### $|U_1|>\ell_1s$

##### $\ell_1>k_1$

##### $\ell_1=k_1$

%###### $k_2=k_1$

%###### $k_2=k_1+1$

% ####### $b_1=2$, $b_2=2$
% 
% ####### $b_1=1$, $b_2=2$
% 
% ####### $b_1=2$, $b_2=1$
% 
% ####### $b_1=1$, $b_2=1$

%###### $k_2\geq k_1+2$
% 
% ####### $a_1=2$, $b_2=1$
% 
% ####### $a_1=1$, $b_2=2$
% 
% ####### $a_1=1$, $b_2=1$

### $\ell_1s<|U_0\cup U_1|<\ell_1s+s$

#### $|U_1|\leq \ell_1s$
% 
% ##### $a_2=2$, $b_1=1$
% 
% ##### $a_2=1$, $b_1=2$
% 
% ##### $a_2=1$, $b_1=1$
% 
% ###### $|V_0|\leq s-3$
% 
% ###### $|V_0|=s-2$

%###### $|V_0|\geq s-1$ ?????? (This case may overlap a previous case)

#### $|U_1|>\ell_1s$

%##### $|V_0\cup V_1|\geq \ell_1s+s$ ?????? (This case may overlap a previous case)

%###### $a_2=1$, $b_1=2$

%###### $a_2=2$, $b_1=1$

%###### $a_2=1$, $b_1=1$

%#### $\ell_1s<|V_1\setminus V_1^M|\leq |V_1\cup V_0|<\ell_1s+s$

##### For some $i\in\{1,2\}$ we have $\delta(V_i, U_{3-i})\geq s$ or $\delta(U_{3-i}, V_i)\geq s$

##### For all $i\in\{1,2\}$ we have $\delta(V_i, U_{3-i})<s$ and $\delta(U_{3-i}, V_i)<s$

\end{easylist}

Recall the following definitions. For $i=1,2$, $U_i^M=\{u\in U_i:\deg(u, V_{3-i})>\alpha^{1/3}n\}$ and $V_i^M=\{v\in V_i: \deg(v, U_{3-i})>\alpha^{1/3}n \}$. Also recall $U_1^M=\emptyset=V_2^M$ by Claim \ref{preprocess}.

\noindent
\textbf{Case 1} $|V_1|\leq k_1s$ and $|V_0\cup V_1|\leq k_1s+r$.  Let $b_2:=|V_2|-k_2s$ and note that $b_2\geq -r$.  We have 
\begin{equation} 
\label{eq1.1} \delta(U_1, V_2)\geq k_1s+s+r-(k_1s-b_2)\geq s+r+b_2\geq s. 
\end{equation}

\begin{claim}\label{Claim1}
If $|V_0\cup V_1|\geq k_1s$, then there exists $V_0'\subseteq V_0$ such that $|V_1\cup (V_0\setminus V_0')|=k_1s$.  If $|V_0\cup V_1|<k_1s$, then there exists a set of vertex disjoint $s$-stars with centers $C\subseteq V_2$  and leaves in $U_1$ such that $|V_0\cup V_1|+|C|=k_1s$.
\end{claim}

\begin{proof}
If $|V_0\cup V_1|\geq k_1s$, we just choose $V_0'\subseteq V_0$ such that $|V_1\cup (V_0\setminus V_0')|=k_1s$.  Otherwise $b_2\geq 0$ and thus by \eqref{eq1.1} and $\Delta(V_2, U_1)<2\alpha^{1/3}k_2s$, we can apply Lemma \ref{STARSlemma}(ii) to get a set of $b_2$ vertex disjoint $s$-stars from $V_2$ to $U_1$ with centers $C$.  So we have $|V_0\cup V_1\cup C|=k_1s$.
\end{proof}

Let $a_2:=|U_2|-k_2s$. We have two cases.

Suppose $a_2\geq 0$. Claim \ref{k2approxk1} gives $\delta(V_1, U_2)\geq k_2s+2s-5-r-(k_1s-a_2)\geq s+a_2$.  So by Lemma \ref{STARSlemma}(ii) there are $a_2$ vertex disjoint $s$-stars from $U_2$ to $V_1$ with centers $C_U$.  So we can make $|U_0\cup U_1\cup C_U|=k_1s$ and apply Claim \ref{Claim1} to finish.

Suppose $a_2<0$. Then $|U_0\cup U_1|>k_1s$.  If $|U_1|\leq k_1s$, then there exists $U_0'\subseteq U_0$ such that $|U_1\cup (U_0\setminus U_0')|=k_1s$ and we apply Claim \ref{Claim1} to finish.  Otherwise $|U_1|>k_1s$ and let $a_1:=|U_1|-k_1s>0$.  If $b_2>0$, then we have $$\delta(U_1, V_2)+\delta(V_2, U_1)\geq 3s-5+a_1+b_2,$$ and we use Lemma \ref{lemma:diagonalsum}(i) to get a set of $a_1$ vertex disjoint $s$-stars from $U_1$ to $V_2$ with centers $C_U$ and a set of $b_2$ vertex disjoint $s$-stars from $V_2$ to $U_1$ with centers $C_V$.  Thus $|U_1\setminus C_U|=k_1s$ and $|V_0\cup V_1\cup C_V|=k_1s$.  Finally suppose $b_2\leq 0$, i.e. $|V_0\cup V_1|\geq k_1s$.  If there exists a set of $a_1$ vertex disjoint $s$-stars from $U_1$ to $V_2$, then we can apply Claim \ref{Claim1} to finish.  We show that such a set exists.  We have 
\begin{equation} 
\label{eq1.2} \delta(V_2, U_1)\geq k_2s+2s-5-r-(k_2s-a_1)=2s-5-r+a_1\geq s-4+a_1. 
\end{equation} 
If $a_1\leq 3$, we use \eqref{eq1.1} and Lemma \ref{STARSlemma}(i) with $x=0$ to get a set of $a_1$ vertex disjoint $s$-stars from $U_1$ to $V_2$ with centers $C_U$.  Otherwise $a_1\geq 4$ and we use \eqref{eq1.2} and Lemma \ref{STARSlemma}(iii) or (v) to get a set of $a_1$ vertex disjoint $s$-stars from $U_1$ to $V_2$ with centers $C_U$.

\noindent
\textbf{Case 2.} There exists $\ell_1\geq k_1$, $Y\subseteq V_1^M$ and $V_0'\subseteq V_0$ such that $|(V_1\setminus Y)\cup V_0'|=\ell_1s$.  Let $\ell_1\geq k_1$ be minimal.

\textbf{Case 2.1.}
%(*****Remember to deal with $U_2^M$*****) 
$|V_1|\leq k_1s$.  By Case 1 we have $|V_0\cup V_1|>k_1s+r$.  This implies that there exists $V_0'\subseteq V_0$ such that $|V_1\cup V_0'|=k_1s$ and $|(V_0\cup V_2)\setminus V_0'|=k_2s$.  We now try to make $|U_1|=k_1s$ or $|U_2|=k_2s$.  Reset $U_2:=U_2\setminus U_2^M$ and $U_0:=U_0\cup U_2^M$.  Let $a_1:=|U_1|-k_1s$ and $a_2:=|U_2|-(k_2s-s)$. We have 
\begin{equation}\label{eq2.1.a}
\delta(V_2, U_1)\geq k_2s+2s-5-r-(k_2s-a_1)=2s-5-r+a_1
\end{equation}
and
\begin{equation}\label{eq2.1.b}
\delta(V_1, U_2)\geq k_2s+2s-5-r-(k_1s+s-a_2)=(k_2-k_1)s+s-5-r+a_2.
\end{equation}
If $|U_2|\geq k_2s$ i.e. $a_2\geq s$, then by \eqref{eq2.1.b} and Claim \ref{k2approxk1} we have $\delta(V_1, U_2)\geq s-1+(a_2-s)$ and thus Lemma \ref{STARSlemma}(ii) gives $a_2-s$ vertex disjoint $s$-stars from $U_2$ to $V_1$ with centers $C_U$ such that $|U_2\setminus C_U|=k_2s$.  Otherwise we have $|U_0\cup U_1|>k_1s$.  If $|U_1|\leq k_1s$, then we choose $U_0'\subseteq U_0$ such that $|U_1\cup (U_0\setminus U_0')|=k_1s$.  So suppose $|U_1|>k_1s$, i.e. $a_1>0$.  

\textbf{Case 2.1.1.} $|V_0\cup V_1|\geq k_1s+s$.  If $|U_0\cup U_1|\geq k_1s+s$, then we are done: either $a_1\leq s$ and we just choose $U_0'\subseteq U_0$ and $V_0'\subseteq V_0$ such that $|V_1\cup (V_0\setminus V_0')|=k_1s+s$ and $|U_1\cup (U_0\setminus U_0')|=k_1s+s$ or else $a_1> s$ and thus \eqref{eq2.1.a} gives $\delta(V_2, U_1)\geq 2s-4+(a_1-s)\geq s-1+(a_1-s)$ and thus Lemma \ref{STARSlemma}(ii) allows us to find $a_1-s$ vertex disjoint $s$-stars from $U_1$ to $V_2$.  So suppose $|U_0\cup U_1|<k_1s+s$ and thus $a_2>0$.  

%\textbf{Case 2.1.1.1.} 
$k_2=k_1$.  By Claim \ref{k2approxk1}, $r\leq \frac{s-6}{2}$ which implies $\delta(V_2, U_1)\geq s-1+a_1$ by \eqref{eq2.1.a}.  So there are $a_1$ vertex disjoint $s$-stars from $U_1$ to $V_2$ by Lemma \ref{STARSlemma}(ii).

%\textbf{Case 2.1.1.2.} 
$k_2=k_1+1$.  By Claim \ref{k2approxk1}, $r\leq s-3$ which implies $\delta(V_2, U_1)\geq s-2+a_1$ by (\ref{eq2.1.a}).  If $a_1\geq 2$ or $r\leq s-4$, then there are $a_1$ vertex disjoint $s$-stars from $U_1$ to $V_2$ by Lemma \ref{STARSlemma}(iii), so suppose $a_1=1$ and $r=s-3$.  Furthermore we have $\delta(V_1, U_2)\geq s-2+a_2$ by (\ref{eq2.1.b}).  If $a_2\geq 2$, then there are $a_2$ vertex disjoint $s$-stars from $U_2$ to $V_1$ by Lemma \ref{STARSlemma}(iii), so suppose $a_2=1$.  Note that we would be done unless $\Delta(U_1, V_2)\leq s-1$ and $\Delta(U_2, V_1)\leq s-1$.  Let $d_1:=k_1s-|V_1|$ and let $d_2:=k_2s-|V_2|$.  Note that $|V_0|=d_1+d_2\geq s$.  
%We have
%\begin{equation*}
%\delta(U_1, V_0)\geq k_1s+s+r-(k_1s-d_1+s-1)=d_1+s-2\geq s-2
%\end{equation*}
%and
%\begin{equation*}
%\delta(U_2, V_0)\geq k_1s+s+r-(k_2s-d_2+s-1)=d_2-2\geq s-2.
%\end{equation*}
Let $\hat{U_1}=\{u\in U_1:\deg(u, V_1)\leq k_1s-d_1-4\}$ and suppose that $\hat{U_1}\neq \emptyset$.  
So we have $$\delta(\hat{U_1}, V_0)+\delta(U_2, V_0)\geq 2(k_1s+s+r)-(k_1s-d_1-4+s-1)-(k_2s-d_2+s-1)\geq |V_0|+s.$$
This implies that we can find a $K_{s,s}$ with one vertex in $U_1$, $s-1$ vertices in $U_2$ and $s$ vertices in $V_0$.  So we may suppose that $\hat{U_1}=\emptyset$.  Note that $\delta(U_1, V_1)\geq k_1s-d_1-3=|V_1|-3$.  Since $\delta(V_1, U_2)\geq s-1$, there exists a set of $3s-2$ vertex disjoint $(s-1)$-stars from $U_2$ to $V_1$ with centers $C_U$.  Let $v_2\in N(C_U)\cap V_2$. Since $\delta(V_2, U_1)\geq s-1$, we can let $L_U\subseteq N(v_2)\cap U_1$ such that $|L_U|=s-1$.  Since $\delta(U_1, V_1)\geq |V_1|-3$, the leaves of at least one of the $(s-1)$-stars from $U_2$ to $V_1$ forms a $K_{s-1,s-1}$ with $L_U$.  This allows us to move a vertex $u_2\in U_2$ to $U_1$ and $v_2$ to $V_1$. This makes $|U_2\setminus \{u_2\}|=k_2s-s$, and we choose $V_0'\subseteq V_0$ such that $|V_0'\cup V_2\setminus\{v_2\}|=k_2s-s$.

%\textbf{Case 2.1.1.3} 
$k_2\geq k_1+2$.  In this case, we see from \eqref{eq2.1.b} that $\delta(V_1, U_2)\geq 2s-4+a_2\geq s-1+a_2$.  So there are $a_2$ vertex disjoint $s$-stars from $U_2$ to $V_1$ by Lemma \ref{STARSlemma}(ii).  Then we choose $V_0'\subseteq V_0$ such that $|V_1\cup (V_0\setminus V_0')|=k_1s+s$.


\textbf{Case 2.1.2.} $|V_0\cup V_1|<k_1s+s$. Let $b_2:=|V_2|-(k_2s-s)$ and note that $b_2>0$.

%\textbf{Case 2.1.2.1.} 
$k_2=k_1$.  Then $r\leq \frac{s-6}{2}$ which implies $\delta(V_2, U_1)\geq s-1+a_1$ by (\ref{eq2.1.a}).  So by Lemma \ref{STARSlemma}(ii) there are $a_1$ vertex disjoint $s$-stars from $U_1$ to $V_2$. 

%\textbf{Case 2.1.2.2.} 
$k_2=k_1+1$.  Then $r\leq s-3$ which implies $\delta(V_2, U_1)\geq s-2+a_1$ by (\ref{eq2.1.a}).  If $a_1\geq 2$, then there are $a_1$ vertex disjoint $s$-stars from $U_1$ to $V_2$, so suppose $a_1=1$.  We have $|V_2|=k_2s-s+b_2=k_1s+b_2$.  If $b_2\geq 2$, then $|V_2|>|U_1|$ which together with $\delta(V_2, U_1)\geq s-1$ implies that there is a vertex in $U_1$ with at least $s$ neighbors in $V_2$, in which case we are done.  So suppose $b_1=1$ and thus $|V_2|=|U_1|$.  So if there is a vertex in $V_2$ with $s$ neighbors in $U_1$, then there is a vertex in $U_1$ with $s$ neighbors in $V_2$, so suppose not.  Together with $\delta(V_2, U_1)\geq s-1$, this implies that $G[U_1, V_2]$ is $(s-1)$-regular.  So we have $\delta(V_2, U_0\cup U_2)\geq k_2s+2s-5-r-(s-1)\geq k_2s-1=|U_0\cup U_2|$ which implies that $G[V_2, U_0\cup U_2]$ is complete, and thus we can choose a vertex $u_1\in U_1$ and a vertex $v_1\in N(u_1)\cap V_1$.  Since $\deg(u_1, V_2)=s-1$ and $\deg(v_1, U_0\cup U_2)\geq k_2s+2s-5-r-(k_1s+1)\geq 2s-3\geq s$ we can move $u_1$ and $v_1$.  Then we replace $v_1$ with a vertex from $V_0$ as $V_0\neq \emptyset$.

%\textbf{Case 2.1.2.3.} 
$k_2\geq k_1+2$. 

\begin{claim}\label{Claim 2123}
If $|U_0\cup U_1|\geq k_1s+s$ and $|U_1|\leq k_1s+s$, then there exists $U_0'\subseteq U_0$ such that $|(U_0\cup U_1)\setminus U_0'|=k_1s+s$. If $|U_0\cup U_1|< k_1s+s$, then there exists a set of vertex disjoint $s$-stars with centers $C\subseteq U_2$ and leaves in $V_1$ such that $|U_0\cup U_1|+|C|=k_1s+s$.
\end{claim}

\begin{proof}
Suppose first that $|U_0\cup U_1|\geq k_1s+s$ and $|U_1|\leq k_1s+s$.  Let $U_0'\subseteq U_0$ so that $|(U_0\cup U_1)\setminus U_0'|=k_1s+s$.  
%If $a_1>s$, then we can find $a_1-s$ vertex disjoint $s$-stars from $U_1$ to $V_2$ by \eqref{eq2.1.a} and Lemma \ref{STARSlemma} so that $|U_1|-(a_1-s)=k_1s+s$.  
Now suppose $|U_0\cup U_1|< k_1s+s$ and let $a_2:=|U_2|-(k_2s-s)$.  Since $k_2\geq k_1+2$, \eqref{eq2.1.b} gives $\delta(V_1, U_2)\geq 2s-4+a_2\geq s-1+a_2$ and thus by Lemma \ref{STARSlemma}(ii) there is a set of $a_2$ vertex disjoint $s$-stars with centers $C\subseteq U_2$ and leaves in $V_1$ such that $|U_0\cup U_2|+|C|=k_1s+s$.
\end{proof}

We have 
\begin{equation}
\delta(U_1, V_2)\geq k_1s+s+r-(k_1s+s-b_2)=r+b_2.
\end{equation}
If $r\geq s-b_2$, then $\delta(U_1, V_2)\geq s$ and we apply Lemma \ref{STARSlemma}(iii) to get a set of $b_2$ vertex disjoint $s$-stars from $V_2$ to $U_1$.  So suppose $r\leq s-1-b_2$.  By \eqref{eq2.1.a} we have 
\begin{equation}
\delta(V_2, U_1)\geq s-4+a_1+b_2.\label{2123}  
\end{equation}
We would be done unless $2\leq a_1+b_2\leq 3$. Note also that we have 
\begin{equation}
\delta(V_1, U_0\cup U_2)\geq k_2s+2s-5-r-(k_1s+a_1)\geq (k_2-k_1)s+s-4+b_2-a_1\geq 3s-4+b_2-a_1.\label{2123i}
\end{equation}

First suppose $b_2=2$ and $a_1=1$.  By \eqref{2123} we have $\delta(V_2, U_1)\geq s-1$, and since $|V_2|>|U_1|$ there exists some $u\in U_1$ such that $\deg(u, V_2)\geq s$.  Thus we can move one vertex from $U_1$.

Now suppose $b_2=1$.  If there is a vertex $v_2\in V_2$ such that $\deg(v_2, U_1)\geq s$, then $|(V_0\cup V_1)\cup\{v_2\}|=k_1s+s$ and we apply Claim \ref{Claim 2123} to finish.  So suppose $\Delta(V_2, U_1)\leq s-1$.  

If $a_1=2$, we have $\delta(V_2, U_0\cup U_2)\geq k_2s+2s-5-r-(s-1)\geq k_2s-2=|U_0\cup U_2|$ which implies that $G[V_2, U_0\cup U_2]$ is complete.  Since $\delta(V_2, U_1)\geq s-1$ and $|V_2|>|U_1|$, there is a vertex $u_1\in U_1$ such that $\deg(u_1, V_2)\geq s$ and since $\delta(V_2, U_1)\geq s-1$ and $\Delta(U_1, V_2)<2\alpha^{1/3}k_1s$, there is another vertex $u_1'\in U_1$ such that $\deg(u_1', V_2)\geq s-1$ and the neighborhoods of $u_1$ and $u_1'$ in $V_2$ are disjoint.  Let $v_1'\in N(u_1')\cap V_1$; by \eqref{2123i} $\deg(v_1', U_0\cup U_2)\geq s-1$ and thus since $G[V_2, U_0\cup U_2]$ is complete we can move $u_1$, $u_1'$ to make $|U_1|=k_1s$.

If $a_1=1$, we have $\delta(V_2, U_0\cup U_2)\geq k_2s+2s-5-r-(s-1)\geq k_2s-2=|U_0\cup U_2|-1$.  Since $\delta(V_2, U_1)\geq s-2$ and $|V_2|>|U_1|$, there is a vertex $u_1\in U_1$ such that $\deg(u_1, V_2)\geq s-1$.  Let $v_1\in V_1\cap N(u_1)$; by \eqref{2123i} we have $\deg(v_1, U_0\cup U_2)\geq 3s-4\geq 2s-1$.  Since $\delta(V_2, U_0\cup U_2)\geq |U_0\cup U_2|-1$, $K_{s-1,s-1}\subseteq G[N(u_1)\cap V_2, N(v_1)\cap (U_0\cup U_2)]$.  Thus we can move $u_1$.


\textbf{Case 2.2} $|V_1|>k_1s$.

\textbf{Case 2.2.1.} $|V_1\setminus V_1^M|\leq k_1s$.  Let $Y\subseteq V_1^M$ such that $|V_1\setminus Y|=k_1s$.

\textbf{Case 2.2.1.1.} $|U_0\cup U_2|\geq k_2s$. If $|U_2|\leq k_2s$, then there exists $U_0'\subseteq U_0$ such that $|U_1\cup U_0'|=k_1s=|V_1\setminus Y|$ and we are done.  If not, then we have $|U_2|>k_2s$. So let $a_2:=|U_2|-k_2s$.  We have $\delta(V_1, U_2)\geq k_2s+2s-5-r-(k_1s-a_2)=(k_2-k_1)s+2s-5-r+a_2\geq s-1+a_2$ by Claim \ref{k2approxk1}, and thus we can apply Lemma \ref{STARSlemma}(ii) to get a set of $a_2$ vertex disjoint $s$-stars from $U_2$ to $V_1$.  Since $|(V_0\cup V_2)\cup Y|=k_2s$, we are done.  


\textbf{Case 2.2.1.2.} $|U_0\cup U_2|<k_2s$.  Set $a_1:=|U_1|-k_1s$ and note that $a_1\geq 1$.  We have \begin{equation}\label{V2U1-2212} \delta(V_2, U_1)\geq k_2s+2s-5-r-(k_2s-a_1)=2s-5-r+a_1.\end{equation}

\textbf{Case 2.2.1.2.1.} $|V_0\cup V_1|\geq k_1s+s$.

\textbf{Case 2.2.1.2.1.1.} $|U_0\cup U_1|\geq k_1s+s$. If $a_1\leq s$, we can choose $U_0'\subseteq U_0$ and $Y'\subseteq V_1^M\cup V_0$ so that $|(U_0\cup U_1)\setminus U_0'|=|(V_0\cup V_1)\setminus Y'|=k_1s+s$.  If $a_1>s$, then \eqref{V2U1-2212} implies $\delta(V_2, U_1)\geq 2s-4+(a_1-s)\geq s-1+(a_1-s)$ and thus we can apply Lemma \ref{STARSlemma}(ii) to get $a_1-s$ vertex disjoint $s$-stars from $U_1$ to $V_2$.  Now let $Y'\subseteq V_1^M\cup V_0$ so that $|U_1|-(a_1-s)=|(V_0\cup V_1)\setminus Y'|=k_1s+s$.

\textbf{Case 2.2.1.2.1.2.} $|U_0\cup U_1|<k_1s+s$.  Let $a_2=|U_2|-(k_2s-s)$.  We have 
\begin{equation}\label{V1U2-22122}
\delta(V_1, U_2)\geq k_2s+2s-5-r-(k_1s+s-a_2)=(k_2-k_1)s+s-5-r+a_2.
\end{equation}  

If $k_2=k_1$, then $r\leq \frac{s-6}{2}$.  By \eqref{V2U1-2212} we have $\delta(V_2, U_1)\geq \frac{3s-4}{2}+a_1\geq s-1+a_1$.  So by Lemma \ref{STARSlemma}(ii), we can move $a_1$ vertices from $U_1$ so that $|U_1|-a_1=k_1s=|V_1\setminus Y|$.

If $k_2=k_1+1$, then $r\leq s-3$. By \eqref{V1U2-22122} and \eqref{V2U1-2212} we have $\delta(V_1, U_2)\geq s-2+a_2$ and $\delta(V_2, U_1)\geq s-2+a_1$.  We would be done if either $\delta(V_1, U_2)\geq s$ or $\delta(V_2, U_1)\geq s$, because $|V_0\cup V_1|\geq k_1s+s$ and $|V_1\setminus V_1^M|\leq k_1s$.  So we may suppose $a_1=a_2=1$ and $r=s-3$.  We have $|V_1|\geq |U_2|$, $\delta(V_1, U_2)\geq s-1$, and at least one vertex $v_1\in V_1^M$ such that $\deg(v_1, U_2)\geq \alpha^{1/3}n$. Thus there is a vertex $u_2\in U_2$ such that $\deg(u_2, V_1)\geq s$.  So we have $|(U_0\cup U_2)\cup \{u_2\}|=k_1s+s$ and $|V_0\cup V_1|\geq k_1s+s$ with $|V_1\setminus V_1^M|\leq k_1s$ so we are done.

Finally, suppose that $k_2\geq k_1+2$.  We have $\delta(V_1, U_2)\geq (k_2-k_1)s+s-5-r+a_2\geq 2s-4+a_2\geq s-1+a_2$ since $s\geq 3$.  Thus we can find $a_2$ vertex disjoint $s$-stars from $U_2$ to $V_1$ by Lemma \ref{STARSlemma}(ii) and we have $|(U_0\cup U_1)|+a_2=k_1s+s$.  Since $|V_0\cup V_1|\geq k_1s+s$ and $|V_1\setminus V_1^M|\leq k_1s$ we are done.


\textbf{Case 2.2.1.2.2.} $|V_0\cup V_1|<k_1s+s$.  Set $b_2:=|V_2|-(k_2s-s)$ and $b_1:=|V_1|-k_1s$.  Note that  
$1\leq b_1, b_2\leq s-1$.

If $k_2=k_1$, then $r\leq \frac{s-6}{2}$ by Claim \ref{k2approxk1}. So by \eqref{V2U1-2212} we have $\delta(V_2, U_1)\geq \frac{3s-4}{2}+a_1\geq s-1+a_1$.  By Lemma \ref{STARSlemma}(ii), we can move $a_1$ vertices from $U_1$ so that $|U_1|-a_1=k_1s=|V_1\setminus Y|$.

If $k_2=k_1+1$, then $r\leq s-3$ and by \eqref{V2U1-2212} we have \begin{equation}\label{V2U1-221212}\delta(V_2, U_1)\geq s-2+a_1.\end{equation}  If $a_1\geq 2$ or $r\leq s-4$, then \eqref{V2U1-221212} gives $\delta(V_2, U_1)\geq s-2+a_1\geq s$ in which case we can apply Lemma \ref{STARSlemma}(iii) to get a set of $a_1$ vertex disjoint $s$-stars from $U_1$ to $V_2$. So suppose $a_1=1$ and $r=s-3$.  We have $\delta(U_1, V_2)\geq k_1s+s+r-(k_1s+s-b_2)=r+b_2\geq s-3+b_2$.  If $b_2\geq 3$, then we have $\delta(U_1, V_2)\geq s$ and thus we can move a single vertex from $U_1$ to make $|U_1|-a_1=k_1s=|V_1\setminus Y|$.  So suppose $1\leq b_2\leq 2$.  By \eqref{V2U1-221212}, we have $\delta(V_2, U_1)\geq s-1$. If $b_2=2$, then $|V_2|=k_1s+2>k_1s+1=|U_1|$ and since $\delta(V_2, U_1)\geq s-1$ there exists $u\in U_1$ such that $\deg(u, V_2)\geq s$. So we move $u$ to $U_2$ and $|U_1\setminus\{u\}|=k_1s=|V_1\setminus Y|$.  So we may suppose that $b_2=1$.  Since $\delta(V_2, U_1)\geq s-1$, if there was a vertex $v\in V_2$ such that $\deg(v, U_1)\geq s$, then there exists $u\in U_1$ such that $\deg(u, V_2)\geq s$ in which case we would be done.  So we can suppose $\Delta(U_1, V_2), \Delta(V_2, U_1)\leq s-1$.  Then since $\delta(V_2, U_1)\geq s-1$ by \eqref{V2U1-221212}, we have that $G[U_1, V_2]$ is $(s-1)$-regular.  So we have $\delta(V_2, U_0\cup U_2)\geq k_2s+2s-5-r-(s-1)\geq k_2s-1=|U_0\cup U_2|$ and thus $G[V_2, U_0\cup U_2]$ is complete.  Since $|V_1|=k_1s+1$ and $|V_1\setminus V_1^M|\leq k_1s$, there exists some $v_1\in V_1$ with $\deg(v_1, U_2)> \alpha^{1/3}n$.  Let $u_1\in U_1\cap N(v_1)$.  Since $\deg(u_1, V_2)=s-1$ and $G[V_2, U_0\cup U_2]$ is complete there is a copy of $K_{s,s}$ which contains $u_1$ and $v_1$.  Thus $|U_1\setminus\{u_1\}|=k_1s=|V_1\setminus Y|$.

Finally, suppose $k_2\geq k_1+2$.  We first prove the following claim.

\begin{claim}\label{Claim 22122}
If $|U_0\cup U_1|\geq k_1s+s$ and $|U_1|\leq k_1s+s$, then there exists $U_0'\subseteq U_0$ such that $|(U_0\cup U_1)\setminus U_0'|=k_1s+s$. If $|U_0\cup U_1|< k_1s+s$, then there exists a set of vertex disjoint $s$-stars with centers $C\subseteq U_2$ and leaves in $V_1$ such that $|U_0\cup U_1|+|C|=k_1s+s$.
\end{claim}

\begin{proof}
Suppose first that $|U_0\cup U_1|\geq k_1s+s$ and $|U_1|\leq k_1s+s$.  Let $U_0'\subseteq U_0$ so that $|(U_0\cup U_1)\setminus U_0'|=k_1s+s$.  %If $a_1>s$, then we can find $a_1-s$ vertex disjoint $s$-stars from $U_1$ to $V_2$ by \eqref{V2U1-2212} and Lemma \ref{STARSlemma} so that $|U_1|-(a_1-s)=k_1s+s$.  
Now suppose $|U_0\cup U_1|< k_1s+s$ and let $a_2:=|U_2|-(k_2s-s)$.  Equation \eqref{V1U2-22122} holds in this case.  Since $k_2\geq k_1+2$, \eqref{V1U2-22122} gives $\delta(V_1, U_2)\geq 2s-4+a_2\geq s-1+a_2$ and thus by Lemma \ref{STARSlemma}(ii) there is a set of $a_2$ vertex disjoint $s$-stars with centers $C\subseteq U_2$ and leaves in $V_1$ such that $|U_0\cup U_2|+a_2=k_1s+s$.
\end{proof}

We have 
\begin{equation}
\delta(U_1, V_2)\geq k_1s+s+r-(k_1s+s-b_2)=r+b_2.\label{221:rb2}
\end{equation}
If $r\geq s-b_2$, then $\delta(U_1, V_2)\geq s$ and we can apply Lemma \ref{STARSlemma}(iii) to get a set of $a_1$ vertex disjoint $s$-stars from $U_1$ to $V_2$ giving $|U_1|-a_1=k_1s=|V_1\setminus Y|$.  So suppose $r\leq s-1-b_2$.  By \eqref{V2U1-2212} we have 
\begin{equation}
\delta(V_2, U_1)\geq s-4+a_1+b_2.\label{221:a1b2}
\end{equation}  
If $\delta(V_2, U_1)\geq s$, we would be done by moving $a_1$ vertices from $U_1$, so suppose $2\leq a_1+b_2\leq 3$.

If $b_2=2$ and $a_1=1$, then $\delta(V_2, U_1)\geq s-1$ and since $|V_2|>|U_1|$, there is a vertex $u\in U_1$ with $\deg(u, V_2)\geq s$, which we can move $|U_1|-a_1=k_1s=|V_1\setminus Y|$.

If $a_1=2$ and $b_2=1$, then $\delta(V_2, U_1)\geq s-1$ by \eqref{221:a1b2}.  If $r\leq s-3$, then \eqref{V2U1-2212} would give $\delta(V_2, U_1)\geq s$ in which case we would be done by moving two vertices from $U_1$, so suppose $r=s-2$. If there is a vertex $v_2\in V_2$ with $\deg(v_2, U_1)\geq s$, we can move $v_2$ so that $|(V_0\cup V_2) \cup \{v_2\}|=k_1s+s$ and apply Claim \ref{Claim 22122} to finish.  So suppose $\Delta(V_2, U_1)\leq s-1$. So for all $v\in V_2$, $\deg(v, U_0\cup U_2)\geq k_2s+2s-5-r-(s-1)=k_2s-2=|U_0\cup U_2|$, which implies $G[V_2, U_0\cup U_2]$ is complete.  
%\begin{equation}\label{V1U0U2}
%\delta(V_1, U_0\cup U_2)\geq k_2s+2s-5-r-(k_1s+2)\geq 3s-5\geq s.
%\end{equation}
Since $|V_2|>|U_1|$ and $\delta(V_2, U_1)\geq s-1$, there is a vertex $u_1\in U_1$ with $\deg(u_1, V_2)\geq s$.  Let $L$ be a subset of $N(u_1)\cap V_2$ of size $s$.  Let $v_1\in V_1^M$ and note that $\delta(U_1, V_2)\geq s-1$ by \eqref{221:rb2} and the fact that $r=s-2$.  Since $\Delta(V_2, U_1)\leq s-1$ there must be a vertex $u_1'\in U_1\cap N(v_1)$ such that $\deg(u_1', V_2\setminus L)\geq s-1$.   Then since $G[V_2, U_0\cup U_2]$ is complete, $u_1$ and $v_1$ are contained in a copy of $K_{s,s}$.  Thus $|U_1\setminus\{u_1, u_1'\}|=k_1s=|V_1\setminus Y|$.

Now in the final case we have $a_1=1=b_2$.  If there were a vertex $v_2\in V_2$ such that $\deg(v_2, U_1)\geq s$, then $|(V_0\cup V_1)\cup \{v_2\}|=k_1s+s$ and we apply Claim \ref{Claim 22122} to finish.  So suppose $\Delta(V_2, U_1)\leq s-1$.  Since $r\leq s-2$, we have $\delta(V_2, U_0\cup U_2)\geq k_2s+2s-5-r-(s-1)\geq k_2s-2=|U_0\cup U_2|-1$.  Also $\delta(V_1, U_0\cup U_2)\geq k_2s+2s-5-r-(k_1s+1)\geq(k_2-k_1)s+s-4\geq 3s-4\geq 2s-2$.  Since $\delta(V_2, U_1)\geq s-2$ and $|V_2|>|U_1|$, there exists $u_1\in U_1$ with $\deg(u_1, V_2)\geq s-1$.  Let $v_1\in N(u_1)\cap V_1$.  Since $v_1$ has $2s-2$ neighbors in $U_0\cup U_2$ and $\delta(V_2, U_0\cup U_2)\geq |U_0\cup U_2|-1$ there is a copy of $K_{s,s}$ which contains $u_1$ and $v_1$ with $s-1$ vertices in $U_0\cup U_2$ and $s-1$ vertices in $V_2$.  If $v_1\in V_1^M$, then $|U_1\setminus\{u_1\}|=k_1s=|V_1\setminus Y|$. If $v_1\notin V_1^M$, then let $Y'\subseteq Y$ with $|Y'|=|Y|-1$ and thus  $|U_1\setminus\{u_1\}|=k_1s=|(V_1\setminus\{v_1\})\setminus Y'|$.


% 
% \textbf{Case 2.2.1.2.2.} $|U_0\cup U_1|<k_1s+s$.  Let $a_2=|U_2|-(k_2s-s)$.  We have 
% \begin{equation}\label{V1U2-22122}
% \delta(V_1, U_2)\geq k_2s+2s-5-r-(k_1s+s-a_2)=(k_2-k_1)s+s-5-r+a_2.
% \end{equation}  
% 
% \textbf{Case 2.2.1.2.2.1.} $|V_0\cup V_1|\geq k_1s+s$.  If $k_2=k_1$, then $r\leq \frac{s-6}{2}$.  By \eqref{V2U1-2212} we have $\delta(V_2, U_1)\geq \frac{3s-4}{2}+a_1\geq s-1+a_1$.  So by Lemma \ref{STARSlemma}, we can move $a_1$ vertices from $U_1$ so that $|U_1|-a_1=k_1s=|V_1\setminus Y|$.
% 
% If $k_2=k_1+1$, then $r\leq s-3$. By \eqref{V1U2-22122} and \eqref{V2U1-2212} we have $\delta(V_1, U_2)\geq s-2+a_2$ and $\delta(V_2, U_1)\geq s-2+a_1$.  We would be done if either $\delta(V_1, U_2)\geq s$ or $\delta(V_2, U_1)\geq s$, because $|V_0\cup V_1|\geq k_1s+s$ and $|V_1\setminus V_1^M|\leq k_1s$.  So we may suppose $a_1=a_2=1$ and $r=s-3$.  We have $|V_1|\geq |U_2|$, $\delta(V_1, U_2)\geq s-1$, and at least one vertex $v_1\in V_1^M$ such that $\deg(v_1, U_2)\geq \alpha^{1/3}n$, thus there is a vertex $u_2\in U_2$ such that $\deg(u_2, V_1)\geq s$.  So we have $|(U_0\cup U_2)\cup \{u_2\}|=k_1s+s$ and $|V_0\cup V_1|\geq k_1s+s$ with $|V_1\setminus V_1^M|\leq k_1s$ so we are done.
% 
% Finally, suppose that $k_2\geq k_1+2$.  We have $\delta(V_1, U_2)\geq (k_2-k_1)s+s-5-r+a_2\geq 2s-4+a_2\geq s-1+a_2$ since $s\geq 3$.  Thus we can find $a_2$ vertex disjoint $s$-stars from $U_2$ to $V_1$ by Lemma \ref{STARSlemma} and we have $|(U_0\cup U_1)|+a_2=k_1s+s$.  Since $|V_0\cup V_1|\geq k_1s+s$ and $|V_1\setminus V_1^M|\leq k_1s$ we are done.
% 
% 
% \textbf{Case 2.2.1.2.2.2.} $|V_0\cup V_1|<k_1s+s$. Let $b_2=|V_2|-(k_2s-s)$.  
% 
% Here we repeat the argument of Case 2.2.1.2.1.2 with the following difference.  When $k_2\geq k_1+2$, \eqref{V1U2-22122} gives us $\delta(V_1, U_2)\geq (k_2-k_1)s+s-5-r+a_2\geq 2s-4+a_2\geq s-1+a_2$.  Thus if we need to add vertices to $U_1$, we can do so by finding $a_2$ vertex disjoint $s$-stars from $U_2$ to $V_1$.  This takes the place of the fact that $|U_0\cup U_1|\geq k_1s+s$ in Case 2.2.1.2.1.2.


%DONT DELETE YET
%%%%%%%%%%%%%%%%%%%%%%%%%%%%%%%%%%%%%%%%%%%%%%%%%%%%%%%%%%%%%%%%%%%%%%%%%%%%%%%%%%%%%%%%%
% If $k_2=k_1$, then $r\leq \frac{s-6}{2}$.  By \eqref{V2U1-2212} we have $\delta(V_2, U_1)\geq \frac{3s-4}{2}+a_1\geq s-1+a_1$.  So by Lemma \ref{STARSlemma}, we can move $a_1$ vertices from $U_1$ so that $|U_1|-a_1=k_1s=|V_1\setminus Y|$.
% 
% If $k_2=k_1+1$, then $r\leq s-3$.  By \eqref{V2U1-2212} we have $\delta(V_2, U_1)\geq s-2+a_1$, so we would be done unless $a_1=1$ and $r=s-3$.  Since $|V_2|=k_2s-s+b_2=k_1s+b_2$, we would be done if $b_2>1$, since in this case we would have $|V_2|>|U_1|$ which would give a vertex $u_1\in U_1$ with $\deg(u_1, V_2)\geq s$.  So suppose $b_2=1$ which implies $|U_1|=|V_2|$.  Since $\delta(V_2, U_1)\geq s-1$, if there was a vertex $v\in V_2$ such that $\deg(v, U_1)\geq s$, then there exists $u\in U_1$ such that $\deg(u, V_2)\geq s$ in which case we would be done.  So we may suppose that $G[U_1, V_2]$ is $(s-1)$-regular.  For all $v_2\in V_2$, $\deg(v_2, U_0\cup U_2)\geq k_2s+2s-5-r-(s-1)=k_2s-1=|U_0\cup U_2|$ and thus $G[V_2, U_0\cup U_2]$ is complete.  Finally $\delta(V_1, U_0\cup U_2)\geq k_2s+2s-5-r-(k_1s+1)=2s-3\geq s$.  So we may choose any vertex $u_1\in U_1$ and a vertex $v_1\in N(u_1)\cap V_1$, which forms a $K_{s,s}$ allowing us to move $u_1$.
% 
% Finally suppose $k_2\geq k_1+2$.  We have $\delta(U_1, V_2)\geq k_1s+s+r-(k_1s+s-b_2)=r+b_2$, so we would be done unless $r\leq s-1-b_2$.  If $b_2\geq 2$, then $r\leq s-3$ and $\delta(V_2, U_1)\geq 2s-5-r+a_1\geq s-2+a_1$.  We are done even if $a_1=1$ since $k_2\geq k_1+2$ implies that $|V_2|>|U_1|$.  So suppose $b_2=1$ and thus $\delta(V_2, U_1)\geq 2s-5-r+a_1\geq s-3+a_1$.  We have 
% \begin{equation}\label{V1U2-221222}
% \delta(V_1, U_2)\geq k_2s+2s-5-r-(k_1s+s-a_2)=(k_2-k_1)s-3+a_2\geq 2s-2.
% \end{equation}
% If there were a vertex $v_2\in V_2$ with $\deg(v_2, U_1)\geq s$, then we would be done since (\ref{V1U2-221222}) allows us to move vertices from $U_2$. So $\Delta(V_2, U_1)\leq s-1$ and thus 
% \begin{equation}\label{V2U0U2}
% \delta(V_2, U_0\cup U_2)\geq k_2s+2s-5-r-(s-1)=k_2s-2.
% \end{equation}
% 
% If $a_1=2$, then (\ref{V2U0U2}) implies that $G[V_2, U_0\cup U_2]$ is complete.  Since $\delta(U_1, V_2)\geq s-1$, we can take two vertex disjoint $(s-1)$-stars from $U_1$ to $V_2$ with centers $u_1$ and $u_2$.  Let $v_1, v_2\in (N(u_1)\cap N(u_2))\cap V_1$.  We have $\delta(V_1, U_0\cup U_2)\geq k_2s+2s-5-r-(k_1s+2)=(k_2-k_1)s+s-5\geq 3s-5\geq 2s-2$, where the last inequality holds since $s\geq 3$.  Thus we obtain two vertex disjoint copies of $K_{s,s}$ which allows us to move $u_1$ and $u_2$.  
% 
% So suppose $a_1=1$.  We have $\delta(V_1, U_0\cup U_2)\geq k_2s+2s-5-r-(k_1s+1)=(k_2-k_1)s+s-4\geq 3s-4\geq 2s-2$.  So we choose any vertex $u_1\in U_1$, $v_1\in N(u_1)\cap V_1$.  Then $u_1$ has $s-1$ neighbors in $V_2$ and $v_1$ has $2s-2$ neighbors in $U_0\cup U_2$, since every vertex in $V_2$ only misses at most one vertex in $U_0\cup U_2$, we have a copy of $K_{s,s}$ which allows us to move $u_1$. 
%%%%%%%%%%%%%%%%%%%%%%%%%%%%%%%%%%%%%%%%%%%%%%%%%%%%%%%%%%%%%%%%%%%%%%%%%%%%%%%%%%%%%%%%%
%DONT DELETE YET



\textbf{Case 2.2.2.} $|V_1\setminus V_1^M|>k_1s$.

\textbf{Case 2.2.2.1.} $\exists \ell_1$, $\exists Y\subseteq V_1^M$ such that $|V_1\setminus Y|=\ell_1s$.  Choose $\ell_1$ minimal and note that $\ell_1>k_1$ by Case 2.2.2.  Let $\ell_2:=m-\ell_1$.


\textbf{Case 2.2.2.1.1.} $|U_0\cup U_2|< \ell_2s$.  Let $a_1:=|U_1|-\ell_1s$. We have $\delta(V_2, U_1)\geq k_2s+2s-5-r-(\ell_2s-a_1)=(k_2-\ell_2)s+2s-5-r+a_1\geq 2s-4+a_1\geq s-1+a_1$, and thus we can find a set of $a_1$ vertex disjoint $s$-stars from $U_1$ to $V_2$.  This gives $|U_1|-a_1=\ell_1s=|V_1\setminus Y|$.  

\textbf{Case 2.2.2.1.2.} $|U_0\cup U_2|\geq \ell_2s$.  If $|U_2|\leq \ell_2s$, then there exists $U_0'\subseteq U_0$ such that $|U_1\cup U_0'|=\ell_1s=|V_1\setminus Y|$.  Otherwise $|U_2|>\ell_2s$.  Set $a_2:=|U_2|-\ell_2s$.  

We have $|V_1\setminus Y|=\ell_1s$ and since $\ell_1>k_1$ and $\ell_1$ is minimal, we have $|V_1^M\setminus Y|<s$.  Set $b_1:=|V_1\setminus V_1^M|-(\ell_1s-s)$.  We have 
\begin{equation}\label{V1-Y}
\delta(V_1\setminus Y, U_2)+\delta(U_2, V_1\setminus Y)\geq n+3s-5-(\ell_1s-a_2+\ell_2s)=3s-5+a_2
\end{equation}
and 
\begin{equation}\label{V1-V1M}
\delta(V_1\setminus V_1^M, U_2)+\delta(U_2, V_1\setminus V_1^M)\geq n+3s-5-(\ell_1s-a_2+\ell_2s+s-b_1)=2s-5+b_1+a_2.  
\end{equation}
If $\delta(V_1\setminus Y, U_2)\geq s$, then there are $a_2$ vertex disjoint $s$-stars from $U_2$ to $V_1$ by Lemma \ref{STARSlemma}(iii) and we are done.  Otherwise by \eqref{V1-Y} we have $\delta(U_2, V_1\setminus Y)\geq 2s-4+a_2\geq s$.  If $\delta(U_2, V_1\setminus V_1^M)\geq s$, then since $\Delta(V_1\setminus V_1^M, U_2)<\alpha^{1/3}n$ we can apply Lemma \ref{STARSlemma}(iii) to get a set of $a_2$ vertex disjoint $s$-stars from $U_2$ to $V_1$.  Likewise if $\delta(V_1\setminus V_1^M, U_2)\geq s$.  These two facts, together with \eqref{V1-V1M} imply $2\leq a_2+b_1\leq 3$.  If $a_2=1$, then since $\delta(U_2, V_1\setminus Y)\geq 2s-3\geq s$ and we only need to move one vertex, we are done.  So we only need to deal with the case when $a_2=2$, $b_1=1$, and $\delta(U_2, V_1\setminus V_1^M)=s-1=\delta(V_1\setminus V_1^M, U_2)$. Since $b_1=1$ we have $|V_1^M\setminus Y|=s-1$. If there exists a vertex $u_2\in U_2$ such that $\deg(u_2, V_1\setminus V_1^M)\geq s$, then since $\delta(U_2, V_1\setminus Y)\geq s$, we either have another vertex disjoint $s$-star and we are done, or every vertex in $U_2$ must have a neighbor in $N(u_2)\cap (V_1\setminus V_1^M)$.  However this implies that some vertex in $v'\in N(u_2)\cap (V_1\setminus V_1^M)$ has $\deg(v', U_2)>\alpha^{1/3}n$ contradicting the fact that vertices in $V_1\setminus V_1^M$ are not movable.  So we have $\Delta(U_2, V_1\setminus V_1^M)\leq s-1$.  Since $\delta(U_2, V_1\setminus Y)\geq 2s-4+a_2=2s-2$, $\Delta(U_2, V_1\setminus V_1^M)\leq s-1$ and $|V_1^M\setminus Y|=s-1$, every vertex in $U_2$ is adjacent to every vertex in $V_1^M\setminus Y$.  Since $\delta(V_1\setminus V_1^M, U_2)=s-1$, we can choose $v_1\in V_1\setminus V_1^M$ and $u_2, u_2'\in N(v_1)\cap U_2$.  Thus $\{v_1\}\cup (V_1^M\setminus Y)$ and $\{u_2, u_2'\}$ form a $K_{s,2}$ and thus we can move $u_2,u_2'$ from $U_2$, giving $|U_0\cup U_1|+2=\ell_1s=|V_1\setminus Y|$.


\textbf{Case 2.2.2.2.} $\exists \ell_1$, $\exists V_0'\subseteq V_0$ such that $|V_0'\cup V_1|=\ell_1s$. Choose $\ell_1$ to be minimal and note that since we are in Case 2.2.2. but not Case 2.2.2.1. we have $|V_1\setminus V_1^M|> \ell_1s-s$ and thus
\begin{equation}
\ell_1\geq k_1+1.\label{l1k1}
\end{equation}
Set $\ell_2:=m-\ell_1$. Since $|V_1\setminus V_1^M|>\ell_1s-s$, we reset $V_1:=V_1\setminus V_1^M$, $V_0:=V_0\cup V_1^M$ and set $b_1:=|V_1|-(\ell_1s-s)$. 

\textbf{Case 2.2.2.2.1.} $|U_0\cup U_2|<\ell_2 s$.  Set $a_1:=|U_1|-\ell_1s$.  Then we have $\delta(V_2, U_1)\geq k_2s+2s-5-r-(\ell_2s-a_1)=(k_2-\ell_2)s+2s-5-r+a_1\geq 2s-4+a_1\geq s-1+a_1$, and thus we are done by Lemma \ref{STARSlemma}(ii).  

\textbf{Case 2.2.2.2.2.} $|U_0\cup U_2|\geq \ell_2 s$.  If $|U_2|\leq \ell_2s$, then there exists $U_0'\in U_0$ such that $|U_1\cup U_0'|=\ell_1s=|V_1\cup Y|$.  Otherwise $|U_2|>\ell_2s$.  Set $a_2:=|U_2|-\ell_2s$.  Note that if $\ell_2\geq \ell_1$, then $\ell_2s\geq \frac{n}{2}$ and consequently $\delta(V_1, U_2)\geq \frac{n+3s-4}{2}-(\ell_1s-a_2)\geq \frac{3s-4}{2}+a_2\geq s-1+a_2$.  Then by Lemma \ref{STARSlemma}(ii) we can move $a_2$ vertices from $U_2$ and we are done.  So for the rest of this case we may suppose that
\begin{equation}
\ell_2\leq \ell_1-1.\label{l2l1}
\end{equation}

Since $|U_2|=\ell_2s+a_2$, we have 
\begin{equation}\label{22222:V1U2}
\delta(V_1, U_2)+\delta(U_2, V_1)\geq n+3s-5-(\ell_1s-a_2+\ell_2s+s-b_1)=2s-5+a_2+b_1.  
\end{equation}
If $\delta(V_1, U_2)\geq s$ or $\delta(U_2, V_1)\geq s$, then we can apply Lemma \ref{STARSlemma}(i) or (iii) to get a set of $a_2$ vertex disjoint $s$-stars from $U_2$ to $V_1$, giving $|U_1|+a_2=\ell_1s=|V_0'\cup V_1|$. So suppose for the rest of the case that %$\delta(V_1, U_2)+\delta(U_2, V_1)\leq 2s-2$, 
\begin{equation}\label{s-1:V1U2}
\delta(V_1, U_2)\leq s-1 \text{ and } \delta(U_2, V_1)\leq s-1.
\end{equation} 
Thus \eqref{22222:V1U2} and \eqref{s-1:V1U2} imply $2\leq a_2+b_1\leq 3$.  Furthermore, if $\delta(V_1, U_2)+\delta(U_2, V_1)= 2s-2$, then we have $\delta(V_1, U_2)= s-1$ and $\delta(U_2, V_1)= s-1$.  

\begin{claim}\label{Claim 22222}
If $|U_1|\leq \ell_1s-s$, then there exists $U_0'\subseteq U_0$ such that $|U_1\cup U_0'|=\ell_1s-s$.  If $|U_1|\geq \ell_1s-s+1$, then there exists a set of vertex disjoint $s$-stars with centers $C\subseteq U_1$ and leaves in $V_2$ such that $|U_1\setminus C|=\ell_1s-s$ or else $\delta(V_1, U_2)\geq s-2+a_2$.

%$|U_1|=\ell_1s-s+1$, $r=s-3$, $\ell_1=k_1+1$, $\ell_2=k_2-1$, $k_2=k_1+1$.
\end{claim}

\begin{proof}
First suppose $|U_1|\leq \ell_1s-s$.  Since $|U_2|=\ell_2s+a_2\leq \ell_2s+2<\ell_2s+s$, there exists $U_0'\subseteq U_0$ such that $|U_0'\cup U_1|=\ell_1s-s$.  Now suppose $|U_1|\geq \ell_1s-s+1$ and set $a_1:=|U_1|-(\ell_1s-s)$.  If $\ell_1\geq k_1+2$, then 
\begin{align}
\delta(V_2, U_1)\geq k_2s+2s-5-r-(\ell_2s+s-a_1)&=(k_2-\ell_2)+s-5-r+a_1 \notag \\
&\geq 2s-4+a_1 \geq s-1+a_1.  \label{claim:V2U1}
\end{align}
Thus we may apply Lemma \ref{STARSlemma}(ii) to get a set of $a_1$ vertex disjoint $s$-stars from $U_1$ to $V_2$ giving $|U_1|-a_1=\ell_1s-s$.  So suppose $\ell_1\leq k_1+1$, which implies $\ell_1=k_1+1$ by \eqref{l1k1}.  Consequently $\ell_2=k_2-1$.  By \eqref{l2l1}, we have $k_2-1=\ell_2\leq \ell_1-1=k_1$.  By \eqref{claim:V2U1}, we have $\delta(V_2, U_1)\geq 2s-5-r+a_1$.  If $k_2=k_1$, then $r\leq \frac{s-6}{2}$ and thus $\delta(V_2, U_1)\geq s-1+a_1$. So suppose $k_2=k_1+1$, which implies $r\leq s-3$ by Claim \ref{k2approxk1}.  If $r\leq s-4$, then \eqref{claim:V2U1} gives $\delta(V_2, U_1)\geq s-1+a_1$.  So suppose $r=s-3$.  If $a_1\geq 2$, we have $\delta(V_2, U_1)\geq s$.  Otherwise $a_1=1$ and $\delta(V_1, U_2)\geq k_2s+2s-5-r-(\ell_1s-a_2)\geq s-2+a_2$.
\end{proof}


%\textbf{Case 2.2.2.2.2.1.2.} 
$a_2=1$, $b_1=2$.  In this case, $|V_1|>|U_2|$ by \eqref{l2l1} and since $\delta(V_1, U_2)\geq s-1$, there is a vertex $u_2\in U_2$ such that $\deg(u, V_1)\geq s$ and we are done.

%\textbf{Case 2.2.2.2.2.1.1.} 
$a_2=2$, $b_1=1$.  If there is a vertex $v\in V_1$ with $\deg(v, U_2)\geq s$, then we apply Claim \ref{Claim 22222} to either finish or get $\delta(V_1, U_2)\geq s-2+a_2$.  However, if $\delta(V_1, U_2)\geq s-2+a_2$, then the fact that $a_2=2$, contradicts \eqref{s-1:V1U2}. So suppose $\Delta(V_1, U_2)\leq s-1$.  Since $\delta(U_2, V_1)\leq s-1$, there exists $u\in U_2$ such that for all $v\in V_1$ we have $$n+3s-5\leq \deg(v)+\deg(u)\leq \ell_1s+s-1+s-1+\ell_2s-2+s-1=n+3s-5,$$ thus $G[V_1, U_0\cup U_1]$ is complete.  Let $v_0,v_0'\in V_0$.  Let $u_2\in N(v_0)\cap U_2$ and choose a set of $s-1$ vertices $L\subseteq N(u_2)\cap V_1$.  Since $\Delta(V_1, U_2)\leq s-1$, there exists $u_2'\in N(v_0')\cap U_2$ such that $\deg(u_2', V_1\setminus L)\geq s-1$.  Let $L'$ be a set of $s-1$ vertices in $N(u_2')\cap (V_1\setminus L)$.  Since $G[V_1, U_0\cup U_1]$ is complete we can move $u_2$ and $u_2'$.


%\textbf{Case 2.2.2.2.2.1.3.} 
$a_2=1$, $b_1=1$.  If there is a vertex $v_1\in V_1$ with $\deg(v_1, U_2)\geq s$, then we apply Claim \ref{Claim 22222} to either finish or get $\delta(V_1, U_2)\geq s-2+a_2$. Since $a_2=1$, we have $\delta(V_1, U_2)\geq s-1$.  Since $|V_1|\geq |U_2|$, $\delta(V_1, U_2)\geq s-1$, and $\deg(v_1, U_2)\geq s$, there exists a vertex $u_2\in U_2$ such that $\deg(u_2, V_1)\geq s$ and we are done.  So we may suppose $\Delta(V_1, U_2),\Delta(U_2, V_1)\leq s-1$.  This implies that $\delta(U_2, V_0\cup V_2)\geq |V_0\cup V_2|-1$ and $\delta(V_1, U_0\cup U_1)\geq |U_0\cup U_1|-1$.  Since $\delta(V_1, U_2)+\delta(U_2, V_1)\geq 2s-3$, we can choose $u_2\in U_2$ such that $\deg(u_2, V_1)\geq s-1$.  Let $v_0\in V_0\cap N(u_2)$, which exists since $|V_0|\geq s-1$ and $\delta(U_2, V_0\cup V_2)\geq |V_0\cup V_2|-1$.  We have $\deg(v_0, U_1)>2s-2$ and thus $G[N(u_2)\cap V_1, N(v_0)\cap U_1]$ contains a copy of $K_{s-1,s-1}$.  This allows us to move one vertex from $U_2$ as needed.




%\textbf{Case 2.2.2.2.2.1.} $|U_1|\leq \ell_1s-s$. 
%Furthermore, we have 
%\begin{equation}\label{U0U1<->V0V2}
%\delta(U_0\cup U_1, V_0\cup V_2)+\delta(V_0\cup V_2, U_0\cup U_1)\geq n+3s-5-(\ell_1s-s+b_1+\ell_2s+a_2)=4s-5-a_2-b_1
%\end{equation}




% 
% \textbf{Case 2.2.2.2.2.2.} $\ell_1s-s<|U_1|$.  Let $a_1:=|U_1|-(\ell_1s-s)$.  Note first that if $\ell_1\geq k_1+2$, then $\delta(V_2, U_1)\geq s$.  Otherwise $\ell_1=k_1+1$ (by the case) and thus $\delta(V_2, U_1)\geq 2s-5-r+a_1$.  Note that $2s-5-r+a_1\geq s$ unless $k_2=k_1+1$, $r=s-3$ and $a_1=1$, in which case we have $\delta(V_2, U_1)\geq s-1$. 
% 
% \textbf{Case 2.2.2.2.2.2.1.} 
% $a_2=1$, $b_1=2$. In this case $|V_1|>|U_2|$ and since $\delta(V_1, U_2)\geq s-1$, there is a vertex $u_2\in U_2$ such that $\deg(u, V_1)\geq s$ and we are done.
% 
% \textbf{Case 2.2.2.2.2.2.2.} 
% $a_2=2$, $b_1=1$.  If we don't have $\delta(V_2, U_1)\geq s$, then we have $\ell_1=k_1+1$, $k_2=k_1+1$ and $r=s-3$, thus $\delta(V_1, U_2)\geq k_2s+2s-5-r-(k_1s+s-2)=s$ in which case we can move two vertices from $U_2$ and be done.  Thus we may assume that $\delta(V_2, U_1)\geq s$.   If there is a vertex $v\in V_1$ with $\deg(v, U_2)\geq s$, then we are done, so suppose $\Delta(V_1, U_2)\leq s-1$.  Since $\delta(U_2, V_1)\leq s-1$, there exists $u\in U_2$ such that for all $v\in V_1$ we have $n+3s-5\leq \deg(v)+\deg(u)\leq \ell_1s+s-1+s-1+\ell_2s-2+s-1=n+3s-5$, thus $G[V_1, U_0\cup U_1]$ is complete.  If there exists a vertex $u'\in U_2$ such that $\deg(u', V_1)\geq 2s$, then $\Delta(U_2\setminus\{u'\}, V_1)\leq s-1$ or else we could move two vertices from $U_2$ and be done.  Let $U_2'=\{u\in U_2:\deg(u, V_1)\leq 2s-1$ and note that $|U_2|\geq |U_2|-1$.  We have $\delta(U_2', V_0\cup V_2)\geq |V_0\cup V_2|-s$.  Let $v_1$ be a vertex in $V_1$ with $s-1$ neighbors in $U_2'$ and let $u_0\in U_0$.  We have $\deg(u_0, V_2)>s^2$ and thus $G[N(v_1)\cap U_2, N(u_0)\cap V_2]$ contains a copy of $K_{s-1,s-1}$.  This allows us to move one vertex from $V_1$.  Then we use the fact that $\delta(V_2, U_1)\geq s$ to move $a_1$ vertices from $U_1$.
% 
% \textbf{Case 2.2.2.2.2.2.3.} 
% $a_2=1$, $b_1=1$. If we don't have $\delta(V_2, U_1)\geq s$, then we have $\ell_1=k_1+1$, $k_2=k_1+1$ and $r=s-3$, thus $\delta(V_1, U_2)\geq k_2s+2s-5-r-(k_1s+s-1)=s-1$.  So if there is a vertex $v_1\in V_1$ such that $\deg(v_1, U_2)\geq s$, then we are done either because $\delta(V_2, U_1)\geq s$ and we can also move $a_1$ vertices from $U_1$ or because $|V_1|=|U_2|$ and thus $e(V_1, U_2)>(s-1)|U_2|$ which gives us a vertex $u_2\in U_2$ such that $\deg(u_2, V_1)\geq s$.  So we have $\Delta(V_1, U_2),\Delta(U_2, V_1)\leq s-1$ or else we are done.  This implies that $\delta(U_2, V_0\cup V_2)\geq |V_0\cup V_2|-1$ and $\delta(V_1, U_0\cup U_1)\geq |U_0\cup U_1|-1$.  Since $\delta(V_1, U_2)+\delta(U_2, V_1)\geq 2s-3$, we can choose $u_2\in U_2$ such that $\deg(u_2, V_1)\geq s-1$.  Let $v_0\in V_0\cap N(u_2)$, which exists since $|V_0|\geq s-1$.  We have $\deg(v_0, U_1)>2s-2$ and thus $G[N(u_2)\cap V_1, N(v_0)\cap U_1]$ contains a copy of $K_{s-1,s-1}$.  This allows us to move one vertex from $U_2$ as needed.
% 
% $a_2=1$, $b_1=1$.  Note that $\Delta(V_1, U_2),\Delta(U_2, V_1)\leq s-1$ or else we are done.  This implies that $\delta(U_2, V_0\cup V_2)\geq |V_0\cup V_2|-1$ and $\delta(V_1, U_0\cup U_1)\geq |U_0\cup U_1|-1$. Since $\delta(V_1, U_2)+\delta(U_2, V_1)\geq 2s-3$, we can choose $u_2\in U_2$ such that $\deg(u_2, V_1)\geq s-1$.  Let $v_0\in V_0\cap N(u_2)$, which exists since $|V_0|\geq s-1$.  We have $\deg(v_0, U_1)>2s-2$ and thus $G[N(u_2)\cap V_1, N(v_0)\cap U_1]$ contains a copy of $K_{s-1,s-1}$.  This allows us to move one vertex from $U_2$ as needed.

\noindent
\textbf{Case 3}  For some $\ell_1\geq k_1$, we have $\ell_1s<|V_1\setminus V_1^M|\leq |V_0\cup V_1|<\ell_1s+s$.  Set $b_1:=|V_1\setminus V_1^M|-\ell_1s>0$ and $b_2:=|V_2|-(\ell_2s-s)$.  Reset $V_1:=V_1\setminus V_1^M$ and $V_0:=V_0\cup V_1^M$.  Set $\ell_2=m-\ell_1$.
%*****Show that Case 3 is the complement of Case 1 and 2*****

\noindent
\textbf{Case 3.1} $|U_2\setminus U_2^M|\geq \ell_2s$.  Let $a_2:=|U_2\setminus U_2^M|-\ell_2s$.  Reset $U_2:=U_2\setminus U_2^M$ and $U_0:=U_0\cup U_2^M$.  We have 
\begin{equation}
\delta(V_1, U_2)+\delta(U_2, V_1)\geq 3s-5+a_2+b_1\geq 2s-2+a_2+b_1.
\end{equation}
Note that $a_2\geq 0$, $b_1>0$, so we are done by Lemma \ref{lemma:diagonalsum}.

\noindent
\textbf{Case 3.2} $|U_2\setminus U_2^M|<\ell_2s$.  Reset $U_2:=U_2\setminus U_2^M$ and $U_0:=U_0\cup U_2^M$.  We have $|U_1\cup U_0|>\ell_1s$.  

\textbf{Case 3.2.1.} $|U_0\cup U_1|\geq \ell_1s+s$.

\textbf{Case 3.2.1.1.} First suppose that $|U_1|\leq \ell_1s$.  Let $\bar{V}_i=\{v\in V_i:\deg(v, U_{3-i})\geq s\}$.  If $|\bar{V}_1|\geq \frac{n}{8}$ or $|\bar{V}_2|\geq \frac{n}{8}$, then we either get a set of $b_1$ vertex disjoint $s$-stars from $\bar{V}_1$ to $U_2$ or a set of $b_2$ vertex disjoint $s$-stars from $\bar{V}_2$ to $U_1$ by Lemma \ref{STARSlemma}(i).  Since $|U_1|\leq \ell_1s$ and $\ell_1s+s\leq |U_0\cup U_1|$ we can choose a set $U_0'\subseteq U_0$ such that $|(U_0\cup U_1)\setminus U_0'|=\ell_1s$ or we can choose a set $U_0'\subseteq U_0$ such that $|(U_0\cup U_1)\setminus U_0'|=\ell_1s+s$.  For $i=1,2$, let $\tilde{V}_i=\{v\in V_i\setminus \bar{V}_i:\deg(v, U_1\cup U_2)\leq |U_i|+s-2\}$.  We have 
\begin{align}
\delta(\tilde{V}_1, U_0)+\delta(\tilde{V}_2, U_0)\geq n+3s-4-(|U_1|+s-2+|U_2|+s-2)=|U_0|+s\label{gadget1}
\end{align}

%Let $c_1, c_2\in \mathbb{N}$, we have
%\begin{align}
%\delta(V_1\setminus V_1(c_1), U_0)+\delta(V_2\setminus V_2(c_2), U_0)&\geq n+3s-4-(|U_1|+c_1-1+|U_2|+c_2-1)\notag\\
%&=|U_0|+3s-2-c_1-c_2\label{gadget1}
%\end{align}
 
If $|\tilde{V}_1|\geq \frac{n}{8}$ and $|\tilde{V}_2|\geq \frac{n}{8}$, then by (\ref{gadget1}) and Lemma \ref{K_1sum} we can find a $K_{s,s}$ with $b_1$ vertices in $V_1$ and $s-b_1$ vertices in $V_2$.  Then we choose $U_0'\subseteq U_0$ such that $|V_1|-b_1=\ell_1s=|(U_0\cup U_1)\setminus U_0'|$.  Otherwise we have $|\tilde{V}_1|<\frac{n}{8}$ or $|\tilde{V}_2|<\frac{n}{8}$.  Suppose that $|\tilde{V}_1|<\frac{n}{8}$.  First note that for all $v\in V_1\setminus (\bar{V}_1\cup \tilde{V}_1)$, $\deg(v, U_2)=s-1$.  Since $|V_1\setminus (\bar{V}_1\cup \tilde{V}_1)|>\frac{n}{8}$, we can apply Lemma \ref{STARSlemma}(i) to get a set of $b_1$ vertex disjoint $(s-1)$-stars from $V_1\setminus (\bar{V}_1\cup \tilde{V}_1)$ to $U_2$.  Let $v_1, v_2,\dots, v_{b_1}$ be the centers and $L(v_i)$ be the leaf sets for each star.  

If $|\tilde{V}_2|\geq \frac{n}{8}$, then for every star we have $|N(L(v_i))\cap \tilde{V}_2|>\frac{n}{16}$ and for all $\tilde{v}\in N(L(v_i))\cap \tilde{V}_2$ we have $$n+3s-4\leq \deg(v_i)+\deg(\tilde{v})\leq |U_1|+s-1+\deg(v_i, U_0)+|U_2|+s-2+\deg(\tilde{v}, U_0),$$ which implies $\deg(v_i, U_0)+\deg(\tilde{v}, U_0)\geq |U_0|+s-1$.  So for each $v_i$, we can find a $K_{s-1,s-1}$ with $s-1$ vertices in $N(v_i)\cap U_0$ and $s-1$ vertices in $N(L(v_i))\cap \tilde{V}_2$.  Since we only need to move at most $s-1$ vertices from $V_1$, we can always choose a unique vertex from $U_0$ for each center in $V_1$ to complete the copy of $K_{s,s}$.  

If $|\tilde{V}_2|< \frac{n}{8}$, then  $|V_i\setminus (\bar{V}_i\cup \tilde{V}_i)|>\frac{n}{8}$ for $i=1,2$.  Set $V_i':=V_i\setminus (\bar{V}_i\cup \tilde{V}_i)$ for $i=1,2$.  We know that $\min\{b_1, s-b_1\}\leq \frac{s}{2}$ and since $s\geq 3$,  $\min\{b_1, s-b_1\}\leq s-2$.  Without loss of generality, suppose $b_1\leq s-b_1$.  Since $|V_1'|>\frac{n}{8}$, we start by taking a set of $b_1$ vertex disjoint $(s-1)$-stars from $V_1'$ to $U_2$.  Let $v_1, v_2,\dots, v_{b_1}$ be the centers and $L(v_i)$ be the leaf sets for each star.  For every star we have $|N(L(v_i))\cap V_2'|>\frac{n}{16}$ and for all $v'\in N(L(v_i))\cap V_2'$ we have $$n+3s-4\leq \deg(v_i)+\deg(v')\leq |U_1|+s-1+\deg(v_i, U_0)+|U_2|+s-1+\deg(v', U_0),$$ which implies $\deg(v_i, U_0)+\deg(v', U_0)\geq |U_0|+s-2$. So for each $v_i$, we can find a $K_{s-2,s-1}$ with $s-2$ vertices in $U_0\cap N(v_i)$ and $s-1$ vertices in $N(L(v_i))\cap V_2'$.  Since we only need to move at most $s-2$ vertices from $V_1$, we can always choose a unique vertex from $U_0$ for each center in $V_1$ to complete the copy of $K_{s,s}$.  



\textbf{Case 3.2.1.2.} $|U_1|>\ell_1s$.  Let $a_1:=|U_1|-\ell_1s$.  In this case we have 
\begin{equation}\label{3212}
\delta(V_2, U_1)\geq k_2s+2s-5-r-(\ell_2s-a_1)=(k_2-\ell_2)s+2s-5-r+a_1.
\end{equation}

\textbf{Case 3.2.1.2.1.} $\ell_1>k_1$. Then $\ell_2<k_2$ and \eqref{3212} gives $\delta(V_2, U_1)\geq s-1+a_1$ and we are done by moving vertices to $V_1$. 

\textbf{Case 3.2.1.2.2.} $\ell_1=k_1$ and so $\ell_2=k_2$.  

Suppose $k_2=k_1$. Then $r\leq \frac{s-6}{2}$ and we have $\delta(V_2, U_1)\geq s-1+a_1$ so we are done by moving vertices to $V_1$.

Suppose $k_2=k_1+1$. This implies $r\leq s-3$. Now we have 
$\delta(V_2, U_1)\geq s-2+a_1$.  
If $\delta(V_2, U_1)\geq s$, then we would be done by moving vertices to $V_1$. So suppose $a_1=1$ and $r=s-3$. Recall $b_2=|V_2|-(k_2s-s)$.  We have 
$\delta(U_1, V_2)\geq k_1s+s+r-(k_1s+s-b_2)=s-3+b_2$, 
so we would be done by moving vertices to $V_1$ unless $1\leq b_2\leq 2$.  Furthermore, we have 
\begin{equation}\label{s-3+b1}
\delta(U_2, V_1)\geq k_1s+s+r-(k_1s+s-b_1)=s-3+b_1
\end{equation}

%\textbf{Case 3.2.1.2.2.2.1.} 
Suppose $b_2=2$.  Since $a_1=1$ and $k_2=k_1+1$ we have $|V_2|>|U_1|$.  Since $\delta(V_2, U_1)\geq s-1$, there exists a vertex $u_1\in U_1$ such that $\deg(u_1, V_2)\geq s$.  If $b_1\geq 3$, then \eqref{s-3+b1} implies $\delta(U_2, V_1)\geq s$ and thus we can move $b_1$ vertices from $V_1$ by Lemma \ref{STARSlemma}(iii).  Otherwise let $V_2'=\{v\in V_2:\deg(v, U_1)\leq s-1\}$.  If $|V_2\setminus V_2'|>2s\alpha^{1/3}k_2s$, then since $\Delta(U_1, V_2)\leq 2\alpha^{1/3}k_2s$ there would be two vertex disjoint $s$-stars from $V_2\setminus V_2'$ to $U_1$.  So suppose $|V_2'|> \frac{n}{4}$.  Note that for all $v\in V_2'$, $\deg(v, U_0\cup U_2)\geq k_2s+2s-5-r-(s-1)=k_2s-1=|U_0\cup U_2|$, so $G[V_2', U_0\cup U_2]$ is complete. If $b_1=1$, then since $\delta(V_1, U_0\cup U_2)\geq 2s-3\geq s$ we can move a vertex from $V_1$, giving $|U_1\setminus\{u_1\}|=k_1s=|V_1|-1$.  So suppose $b_1=2$.  If there is a vertex $v_1\in V_1$ such that $\deg(v_1, U_0\cup U_2)\geq 2s$, then we would be done since $\delta(V_1, U_0\cup U_2)\geq 2s-3\geq s$ and $G[V_2', U_0\cup U_2]$ is complete so we can move two vertices from $V_1$.  So suppose $\Delta(V_1, U_0\cup U_2)\leq 2s-1$.  Then $\delta(V_1, U_1)\geq k_2s+2s-5-r-(2s-1)=k_2s-s-1=k_1s-1=|U_1|-2$.  Since $b_1=2$, we have $\delta(U_2, V_1)\geq s-1$ by \eqref{s-3+b1}.  Thus there are two vertex disjoint $s$-stars from $U_2$ to $V_1$ with leaf sets $L_1$ and $L_2$.  Let $\tilde{U}_1:=U_1\cap (N(L_1)\cap N(L_2))$ and note that since $\delta(V_1, U_1)\geq |U_1|-2$, we have $|\tilde{U}_1|\geq |U_1|-4s$.  Now since $\delta(V_2', U_1)\geq s-1$ and $\Delta(U_1, V_2)\leq 2\alpha^{1/3}k_2s$, there exist two vertex disjoint $(s-1)$-stars from $V_2'$ to $\tilde{U}_1$.  Since $G[\tilde{U}_1, L_1\cup L_2]$ and $G[V_2', U_0\cup U_2]$ are complete, we can move two vertices from $V_2$ to $V_1$ and $U_2$ to $U_1$.  We finish by moving $s-3$ vertices from $U_0$ to $U_1$ and $s-4$ vertices from $V_0$ to $V_1$, giving $|U_1|+2+s-3=k_1s+s=|V_1|+2+s-4$. 


% 
% We have $\delta(V_2, U_1)\geq s-1$ and $\delta(U_1, V_2),\delta(U_2, V_1)\geq s-1$.  If there is a vertex $v'\in V_1$ such that $\deg(v, U_2)\geq 2s$, then we would be done unless for all $v\in V_1\setminus\{v'\}$, $\deg(v, U_2)\leq s-1$.  Either way $G[U_1, V_1]$ is very close to complete (the same goes for $G[U_2, V_2]$).  Since $\delta(U_2, V_1)\geq s-1$, there exists two vertex disjoint $(s-1)$-stars from $V_1$ to $U_2$ with centers $v_1$ and $v_2$.  Now since $\delta(U_1, V_2)\geq s-1$, we can find two $(s-1)$-stars from $N(v_i)\cap U_1$ to $N(L(v_i))\cap V_2$.  
%\textbf{Case 3.2.1.2.2.2.2.} $b_1=1$, $b_2=2$.  


%\textbf{Case 3.2.1.2.2.2.3.} $b_1=2$, 
Suppose $b_2=1$.  If there exists a vertex $v_2\in V_2$ such that $\deg(v_2, U_1)\geq s$, then we would be done by moving $v_2$ to $V_1$.  So suppose $\Delta(V_2, U_2)\leq s-1$ and thus $\delta(V_2, U_0\cup U_2)\geq k_2s+2s-5-r-(s-1)=k_2s-1=|U_0\cup U_2|$.  Let $v_2\in V_2$ and let $L$ be the set of leaves in $U_1$ of an $(s-1)$-star with center $v_2$.  Let $V_1'=N(L)\cap V_1$ and note that $|V_1'|\geq |V_1|-2s\alpha^{1/3}k_1s$.  Since $\delta(V_1', U_0\cup U_2)\geq k_2s+2s-5-r-(k_1s+1)=2s-3\geq s$, there exists a vertex $u_2\in U_0\cup U_2$ such that $\deg(u, V_1')\geq s-1$.  Since $G[V_2, U_0\cup U_2]$ is complete, we can move $v_2$ and $u_2$.  We finish by moving $s-2$ vertices from $U_0$ to $U_1$ and $s-1-b_1$ vertices from $V_0$ to $V_1$ giving $|U_1|+1+s-2=k_1s+s=|V_1|+1+s-1-b_1$.  

%\textbf{Case 3.2.1.2.2.2.4.} $b_1=1=b_2$.  If there exists a vertex $v\in V_2$ such that $\deg(v, U_1)\geq s$, then we would be done.  So for all $v\in V_2$, $\deg(v, U_0\cup U_2)=|U_0\cup U_2|$ and thus $G[U_0\cup U_2, V_2]$ is complete.  Since $\delta(V_2, U_1)\geq s-1$, there exists a vertex $u_1\in U_1$ such that $\deg(u_1, V_2)\geq s-1$.  Let $v_1\in N(u_1)\cap V_1$.  Note that $\deg(v_1, U_0\cup U_2)\geq 2s-3\geq s$, thus we can move $u_1$ and $v_1$ together since $G[U_0\cup U_2, V_2]$ is complete.

Finally, suppose $k_2\geq k_1+2$.  Here we have $\delta(U_1, V_2)\geq k_1s+s+r-(k_1s+s-b_2)=r+b_2$.  If $r\geq s-b_2$, then $\delta(U_1, V_2)\geq s$ and we would be done by moving vertices from $V_2$ to $V_1$, so suppose $r\leq s-1-b_2$.  Then we have 
\begin{equation}\label{321223}
\delta(V_2, U_1)\geq k_2s+2s-5-r-(k_2s-a_1)\geq s-4+a_1+b_2.  
\end{equation}
We would have $\delta(V_2, U_1)\geq s$ and be done unless $2\leq a_1+b_2\leq 3$.

%\textbf{Case 3.2.1.2.2.3.1.} 
Suppose $a_1=2$, $b_2=1$.  If $r\leq s-3$, then $\delta(V_2, U_1)\geq s$ by \eqref{321223}, so suppose $r= s-2$.  We have $\delta(U_1, V_2), \delta(V_2, U_1)\geq s-1$ and $\delta(V_1, U_0\cup U_2)\geq k_2s+2s-5-r-(k_1s+2)\geq 3s-5$.  If there was a vertex $v_2\in V_2$ such that $\deg(v_2, U_1)\geq s$, then we would be done by moving $v_2$ to $V_1$.  So suppose $\Delta(V_2, U_1)\leq s-1$ and thus $\delta(V_2, U_0\cup U_2)\geq k_2s+2s-5-r-(s-1)=k_2s-2=|U_0\cup U_2|$. Let $v_2\in V_2$ and let $L:=N(v_2)\cap U_1$.  Every vertex in $N(L)\cap V_1=:V_1'$ has at least $3s-5\geq s$ neighbors in $U_0\cup U_2$, so there exists a vertex $u_2\in U_0\cup U_2$ such that $\deg(u_2, V_1')\geq 3s-5\geq s-1$.  Then since $G[V_2, U_0\cup U_2]$ is complete, we have a copy of $K_{s,s}$ which allows us to move $v_2$.  We finish by moving $s-3$ vertices from $U_0$ to $U_1$ and $s-1-b_1$ vertices from $V_0$ to $V_1$ giving $|U_1|+1+s-3=k_1s+s=|V_1|+1+s-1-b_1$.


%\textbf{Case 3.2.1.2.2.3.2.} 
Suppose $a_1=1$, $b_2=2$.  If $r\leq s-4$, then $\delta(V_2, U_1)\geq s$ by \eqref{321223}, so suppose $r= s-3$.  We have $\delta(U_1, V_2), \delta(V_2, U_1)\geq s-1$ and $\delta(V_1, U_0\cup U_2)\geq k_2s+2s-5-r-(k_1s+1)\geq 3s-3$.  Let $V_2'=\{v\in V_2:\deg(v, U_1)\leq s-1\}$.  If $|V_2\setminus V_2'|>2s\alpha^{1/3}k_2s$, then since $\Delta(U_1, V_2)\leq 2\alpha^{1/3}k_2s$ there would be two vertex disjoint $s$-stars from $V_2\setminus V_2'$ to $U_1$, so suppose not.  Then $|V_2'|>\frac{n}{4}$.  Note that $G[V_2', U_0\cup U_2]$ is complete.  Since $|V_2|>|U_1|$ and $\delta(V_2, U_1)\geq s-1$, there exists a vertex $u_1\in U_1$ such that $\deg(u_1, V_2)\geq s$.  Now we must move $b_1$ vertices from $V_1$.  If say $\frac{n}{8}$ vertices in $V_1$ have at least $s$ neighbors in $U_0$, then we can find a $K_{s,s}$ with $s$ vertices in $U_0$, $b_1$ vertices in $V_1$ and $s-b_1$ vertices in $V_2$ by Lemma \ref{gadget1} and the fact that $G[V_2', U_0\cup U_2]$ is complete.  Otherwise we have $\frac{n}{4}$ vertices with at most $s-1$ neighbors in $U_0$ and consequently at least $3s-3-(s-1)\geq s$ neighbors in $U_2$.  Either way there exists $b_1$ vertex disjoint $s$-stars from $V_1$ to $U_2$. 


%\textbf{Case 3.2.1.2.2.3.3.} 
Suppose $a_1=1=b_2$.  If there is a vertex in $V_2$ with $s$ neighbors in $U_1$, then we would be done, so suppose not.  Since $b_2=1$, we have $r\leq s-2$.  If $r=s-2$, then $\delta(U_1, V_2)\geq s-1$.  If $r\leq s-3$, then $\delta(V_2, U_1)\geq s-1$.  So either way there is a vertex $v_2\in V_2$ such that $\deg(v_2, U_1)= s-1$.  Let $L:=N(v_2)\cap U_1$.  We have $\delta(V_2, U_0\cup U_2)\geq k_2s+2s-5-r-(s-1)\geq k_2s-2=|U_0\cup U_1|-1$.  Since $\delta(V_1, U_0\cup U_2)\geq 3s-4$, every vertex in $N(L)\cap V_1=:V_1'$ has at least $3s-5$ neighbors in $N(v_2)\cap (U_0\cup U_2)$. So there exists a vertex $u_2\in N(v_2)\cap (U_0\cup U_2)$ with at least $3s-5\geq s-1$ neighbors in $V_1'$.  This gives us a copy of $K_{s,s}$ which allows us to move $v_2$.

\textbf{Case 3.2.2.} $\ell_1s<|U_0\cup U_1|<\ell_1s+s$.  

\textbf{Case 3.2.2.1.} $|U_1|\leq \ell_1s$.  Thus there exists $U_0'\subseteq U_0$ such that $|(U_0\cup U_1)\setminus U_0'|=\ell_1s$.  So we try to make $|V_1|=\ell_1s$ or $|V_2|=\ell_2s$.  Recall $\ell_2=m-\ell_1$ and $b_1=|V_1|-\ell_1s$.  Let $a_2:=|U_2|-(\ell_2s-s)$.  We have 
\begin{equation}
\delta(V_1, U_2)+\delta(U_2, V_1)\geq n+3s-5-(\ell_1s+s-a_2+\ell_2s-b_1)=2s-5+a_2+b_1.  
\end{equation}
If $\delta(V_1, U_2)\geq s$ or $\delta(U_2, V_1)\geq s$, then we would be able to find $b_1$ vertex disjoint $s$-stars from $V_1$ to $U_2$ by Lemma \ref{STARSlemma}(i) or (iii) and we are done.  So suppose $\delta(V_1, U_2)\leq s-1$ and $\delta(U_2, V_1)\leq s-1$, thus $2\leq a_2+b_1\leq 3$.  If $\delta(V_1, U_2)+\delta(U_2, V_1)= 2s-2$, then we have $\delta(V_1, U_2)= s-1$ and $\delta(U_2, V_1)= s-1$.  
%****If $\ell_2\geq \ell_1+2$, then $|U_2|>\frac{n}{2}$ and thus $\delta(V_1, U_2)\geq s$ and we would be done, so suppose for the rest of this case that $\ell_2\leq \ell_1+1$.  *****Do we need this?****
Furthermore, we have 
\begin{align}
\delta(U_0\cup U_1, V_0\cup V_2)+\delta(V_0\cup V_2, U_0\cup U_1)&\geq n+3s-5-(\ell_1s+b_1+\ell_2s-s+a_2) \notag \\
&=4s-5-a_2-b_1. \label{U0U1<->V0V2-3221}
\end{align}
Let $U_2':=\{u\in U_2: \deg(u, V_1)\leq s-1\}$.

%\textbf{Case 3.2.2.1.1} 
Suppose $a_2=2$, $b_1=1$.  If there is a vertex $v_1\in V_1$ with $\deg(v_1, U_2)\geq s$, then we are done by moving $v_1$ to $V_2$.  If $e(U_2, V_1)>(s-1)|V_1|$, then there exists a vertex $v_1\in V_1$ such that $\deg(v_1, U_2)\geq s$, so suppose not. If $|U_2\setminus U_2'|>3\alpha^{2/3}k_2s$, then since $|V_1|-|U_2|\leq 2\alpha^{2/3}k_2s$ we have $e(U_2, V_1)>(s-1)|V_1|$, so suppose not.  Then $|U_2'|\geq |U_2|-3\alpha^{2/3}k_2s$.  For all $v\in V_1$ and $u\in U_2'$ we have 
\begin{equation}\label{32211}
n+3s-5\leq \deg(v)+\deg(u)\leq \ell_1s+s-1+s-1+\ell_2s-2+s-1=n+3s-5,
\end{equation}
thus $G[V_1, U_0\cup U_1]$ is complete and $G[U_2', V_0\cup V_2]$ is complete.  Since $\delta(U_2', V_1)\geq s-1$, there exists a vertex $v_1\in V_1$, such that $\deg(v_1, U_2')= s-1$.  Let $u_0\in U_0$ and note that $\deg(u_0, V_2)>s$.  Since $G[V_1, U_0\cup U_1]$ is complete we can move $v_1$ from $V_1$ along with $u_0$.

%\textbf{Case 3.2.2.1.2} 
Suppose $a_2=1, b_1=2$.  First suppose that there exists $v_1\in V_1$ with at least $s$ neighbors in $U_2$.  Let $L\subseteq N(v_1)\cap U_2$ with $|L|=s$.  In this case we can apply the argument of the previous paragraph to the sets $V_1\setminus v_1$ and $U_2\setminus L$.  
% Let $U_2'=\{u\in U_2:\deg(u, V_1)\leq s-1$ and let $U_2''=U_2'\setminus N(v_1)$.  If there are $c\alpha^{1/3}\frac{n}{2}$ vertices in $U_2\setminus N(v_1)$ each with $s$ neighbors in $V_1$, then we would have a vertex in $V_1\setminus\{v_1\}$ with $s$ neighbors in $U_2\setminus N(v_1)$ and we would be done, so suppose not.  We have $|U_2''|\geq \ell_2s-s+1-(2\alpha^{1/3}+c\alpha^{1/3})\frac{n}{2}$ and since every vertex in $U_2''$ has $s-1$ neighbors in $V_1$, there exists a vertex $v_1'\in V_1$ such that $\deg(v_1', U_2'')\geq s-1$.  Also by (\ref{U0U1<->V0V2-3221}), we have a vertex $u_1\in N(v_1')\cap (U_0\cup U_1)$ with at least $2s-4\geq s-1$ neighbors in $V_0\cup V_2$.  Then since $G[U_2'', V_0\cup V_2]$ is complete we can move $u_1, v_1$, and $v_1'$.  
So suppose that $\Delta(V_1, U_2)\leq s-1$ and $|U_2'|\geq |U_2|-2\alpha^{2/3}k_2s$.  Equation \eqref{32211} holds which implies that $G[V_1, U_0\cup U_1]$ is complete and $G[U_2', V_0\cup V_2]$ is complete.  Every vertex in $U_2'$ has $s-1$ neighbors in $V_1$, so there are two vertex disjoint $(s-1)$-stars from $V_1$ to $U_2'$ with centers $v_1$ and $v_1'$.  Since $G[V_1, U_0\cup U_1]$ is complete and $|U_0|\geq s-1\geq 2$, there exist $u_0, u_0'\in U_0$.  Since $\deg(u_0, V_2), \deg(u_0', V_2)>2s$, we can move $v_1$ and $v_1'$ by taking $u_0$ and $u_0'$.  Then let $U_0'\subseteq U_0$ so that $|U_1|+|U_0'|=\ell_1s=|V_1|-2$.

% If $\delta(V_0\cup V_2, U_0\cup U_1)\geq 2s-3$, then there is a vertex $u_1\in U_0\cup U_1$ such that $\deg(u, V_0\cup V_2)\geq 2s-3$.  Let $C\subseteq N(u)\cap (V_0\cup V_2)$ where $|C|=2s-3$.  For all $v\in (V_0\cup V_2)\setminus C$, $\deg(v, (U_0\cup U_1)\setminus\{u_1\})\geq 2s-4\geq s-1$, so there exists $u_1'\in (U_0\cup U_1)\setminus\{u_1\}$ such that $\deg(u_1', (V_0\cup V_2)\setminus C)\geq s-1$.  So we can move two vertices from $V_1$ by taking $u_1$ and $u_1'$ as well.
% So suppose $\delta(V_0\cup V_2, U_0\cup U_1)\leq 2s-4$, which implies $\delta(U_0\cup U_1, V_0\cup V_2)\geq 2s-4$.  Either there are two vertex disjoint $(s-1)$-stars from $U_0\cup U_1$ to $V_0\cup V_2$ or else there is a $K_{s-1,s-2}$ with $s-1$ vertices in $U_0\cup U_1$ and $s-2$ vertices in $V_0\cup V_2$.  If there are two vertex disjoint $(s-1)$-stars from $U_0\cup U_1$ to $V_0\cup V_2$, then we find two vertex disjoint $(s-1)$-stars from $V_1$ to $U_2$ and we can move two vertices from $V_1$ each time taking one of the centers from $U_1$.  If there is a $K_{s-1,s-2}$ with $s-1$ vertices in $U_0\cup U_1$ and $s-2$ vertices in $V_0\cup V_2$, then we take a vertex $u_2\in U_2$ which is in the neighborhood of the $s-2$ vertices from the $K_{s-1,s-2}$ in $V_0\cup V_2$.  Since $\delta(U_2, V_1)\geq s-1\geq 2$, we can choose two neighbors of $u_2$ in $V_1$ to complete the $K_{s,s}$.  

%\textbf{Case 3.2.2.1.3} 
Suppose $a_2=1, b_1=1$.  If there is a vertex $v_1\in V_1$ such that $\deg(v_1, U_2)\geq s$, then we can move $v_1$ to $V_2$ and be done, so suppose $\Delta(V_1, U_2)\leq s-1$.  First suppose that $\Delta(U_2, V_1)\leq s-1$.  For all $v\in V_1$ and $u\in U_2$ we have $n+3s-5\leq \deg(u)+\deg(v)\leq \ell_1s+s-1+s-1+\ell_2s-1+s-1=n+3s-4$.  Thus $\delta(V_1, U_0\cup U_1)\geq |U_0\cup U_1|-1$ and $\delta(U_2, V_0\cup V_2)\geq |V_0\cup V_2|-1$.  Let $v_1\in V_1$ such that $\deg(v_1, U_2)= s-1$, which exists since $\delta(V_1, U_2)\geq s-1$ or $\delta(U_2, V_1)\geq s-1$.  Let $L:=N(v_1)\cap U_2$ and $V_2':=N(L)\cap V_2$; note that $|V_2'|\geq |V_2|-s$ since $\delta(U_2, V_0\cup V_2)\geq |V_0\cup V_2|-1$.  Finally let $u_0\in U_0\cap N(v_1)$, which exists since $\delta(V_1, U_0\cup U_1)\geq |U_0\cup U_1|-1$ and $|U_0|\geq s-1$.  Since $\deg(u_0, V_2')>s$, we can move $v_1$ along with $u_0$.  So we may suppose that there exists some $u_2\in U_2$ such that $\deg(u_2, V_1)\geq s$.  Let $V_2':=\{v\in V_2:\deg(v, U_1)\leq s-1\}$.  If say $|V_2\setminus V_2'|>\frac{n}{8}$, then since $\Delta(U_1, V_2)\leq 2\alpha^{1/3}k_2s$ we could move $b_2$ vertices from $V_2$ and we would be done. So we may suppose that $|V_2'|>\frac{n}{4}$.  Note that we have
\begin{equation}\label{32213}
\delta(V_1, U_0)+\deg(V_2', U_0)\geq n+3s-4-(|U_1|+s-1+|U_2|+s-1)=|U_0|+s-2.
\end{equation}
Let $v_1\in V_1$ such that $\deg(v_1, U_2)=s-1$ and let $L:=N(v_1)\cap U_2$.  Let $\tilde{V}_2:=V_2'\cap N(L)$ and note that $|\tilde{V}_2|>\frac{n}{8}$.  For all $\tilde{v}\in \tilde{V}_2$ we have $\deg(\tilde{v}, N(v_1)\cap U_0)\geq s-2$ by \eqref{32213}.  Since $|\tilde{V}_2|>|N(v_1)\cap U_0|$, there exists $u_0\in N(v_1)\cap U_0$ such that $\deg(u_0, \tilde{V}_2)\geq s-1$.  This completes a copy of $K_{s,s}$ which allows us to move $v_1$.


% We have $\delta(U_0\cup U_1, V_0\cup V_2)+\delta(V_0\cup V_2, U_0\cup U_1)\geq n+3s-5-(\ell_1s+1)-(\ell_2s-s+1)=4s-7$.  We may suppose $\Delta(V_1, U_2)\leq s-1$ for the rest of this case because otherwise we would be done.  Also recall that $\delta(U_2, V_1)\leq s-1$.  This implies that $\delta(V_1, U_0\cup U_1)\geq |U_0\cup U_1|-1$.  Let $\hat{V_2}=\{v\in V_2:\deg(v, U_0\cup U_1)\geq (2s-1)(s-1-|V_0|)\}$.  If $|\hat{V_2}|\geq s-1-|V_0|$, then there are $s-1-|V_0|$ vertex disjoint $(2s-1)$-stars from $V_2$ to $U_1$.  For each $(2s-1)$-star, there is a $K_{s,s-1}$ with $s$ vertices in $U_1$ and $s-1$ vertices in $V_1$ since $\delta(V_1, U_0\cup U_1)\geq |U_0\cup U_1|-1$.  So if $|\hat{V_2}|\geq s-1-|V_0|$, then we may assume $\Delta(U_2, V_1)\leq s-1$ or else we are done.  If $\Delta(U_2, V_1)\leq s-1$, then $\delta(U_2, V_0\cup V_2)\geq |V_0\cup V_2|-1$.   Thus we can choose some vertex $u_2\in U_2$ with $s-1$ neighbors in $V_1$ and choose some center, $v_2$, of a $(2s-1)$-star which is adjacent to $u_2$ (if $s-1-|V_0|=1$, we can observe that since $\Delta(U_2, V_1)\leq s-1$ almost all of the vertices in $U_2$ have $s-1$ neighbors in $V_1$).  Then we can move $s-1-|V_0|$ vertices from $V_2$ and $1$ vertex from $U_2$.  So for the rest of this case we may suppose $|\hat{V_2}|<s-1-|V_0|$ and set $\tilde{V_2}:=V_2\setminus \hat{V_2}$.
% 
% \textbf{Case 3.2.2.1.3.1.} $|V_0|\leq s-3$. First suppose $\delta(U_0\cup U_1, V_0\cup V_2)\geq 2s-3$ in which case we have $\delta(U_0\cup U_1, V_2)\geq s$ and thus can move $b_2$ vertices from $V_2$.  So now we are done unless $\Delta(U_2, V_1)\leq s-1$ which implies $\delta(U_2, V_0\cup V_2)\geq |V_0\cup V_2|-1$.  Let $v_1$ be a vertex in $V_1$ with $s-1$ neighbors in $U_2$.  Since $\delta(U_1, V_2)\geq s$, there are $s$ vertex disjoint $(s-1)$-stars from $N(v_1)\cap U_1$ to $V_2$.  One of these stars will form a $K_{s-1,s-1}$ and we can move $v_1$.
% 
% Now suppose $\delta(V_0\cup V_2, U_0\cup U_1)\geq 2s-3$.  Recall that $|\hat{V_2}|<s-1-|V_0|$ and $\tilde{V_2}:=V_2\setminus \hat{V_2}$.  Since $\delta(\tilde{V_2}, U_0\cup U_2)\geq 2s-3$, there exists $u_1, u_1'\in U_1$ such that $\deg(u_1, \tilde{V_2})\geq 2s-3$ and $\deg(u_1', \tilde{V_2}\setminus N(u_1))\geq s-1$.  Let $L_1\subseteq N(u_1)\cap \tilde{V_2}$ such that $|L_1|=s-1$ and let $L_2\subseteq N(u_1')\cap \tilde{V_2}$ such that $|L_2|=s-1$.  Let $U_2':=N(L_1\cup L_2)\cap U_2$ and note that $|U_2'|\geq |U_2|-2(s-1)(2s-1)(s-1-|V_0|)$.  Let $V_1':=\{v\in V_1:\deg(v, U_2')\geq s-1\}$.  If $V_1'\neq \emptyset$, then we are done since $v_1\in V_1'$ is adjacent to $u_1$ or $u_1'$.  Otherwise every vertex in $V_1$ has at least one neighbor in $U_2\setminus U_2'$.  Since $|U_2\setminus U_2'|\leq 2(s-1)(2s-1)(s-1-|V_0|)$, this contradicts the fact that $\Delta(U_2\setminus U_2^M, V_1)\leq 2\alpha^{1/3}n$. 
% 
% \textbf{Case 3.2.2.1.3.2.} $|V_0|=s-2$.  In this case we have $\hat{V_2}=\emptyset$ and thus $V_2=\tilde{V_2}$.  
% 
% First suppose $\delta(U_0\cup U_1, V_0\cup V_2)\geq 2s-3$.  Since $|V_0|=s-2$, we have $\delta(U_1, V_2)\geq s-1$.  There exists a set of $s$ vertex disjoint $(s-1)$ stars from $V_2$ to $U_1$ with centers $C_V=\{v_1,\dots, v_s\}$.  Let $U_2':=N(C_U)\cap U_2$.  If there exists a vertex $u_2\in U_2'$ such that $\deg(u_2, V_1)\geq s-1$, then we would be done since $\delta(V_1, U_0\cup U_1)\geq |U_0\cup U_1|-1$.  Otherwise $\Delta(U_2', V_1)\leq s-2$.  But since $|U_2'|\geq |U_2|-s(2s-1)$ and $\Delta(U_2, V_1)\leq 2\alpha^{1/3}n$ and $e(V_1, U_2)\geq (s-1)|V_1|$ we have a contradiction.
% 
% Now suppose $\delta(V_0\cup V_2, U_0\cup U_1)\geq 2s-3$.  Since $\delta(V_2, U_0\cup U_2)\geq 2s-3$, there exists $u_1, u_1'\in U_1$ such that $\deg(u_1, V_2)\geq 2s-3$ and $\deg(u_1',V_2\setminus N(u_1))\geq s-1$.  Let $L_1\subseteq N(u_1)\cap V_2$ such that $|L_1|=s-1$ and let $L_2\subseteq N(u_1')\cap V_2$ such that $|L_2|=s-1$.  Let $U_2':=N(L_1\cup L_2)\cap U_2$ and note that $|U_2'|\geq |U_2|-2(s-1)(2s-1)$.  Let $V_1':=\{v\in V_1:\deg(v, U_2')\geq s-1\}$.  If $V_1'\neq \emptyset$, then we are done since $v_1\in V_1'$ is adjacent to $u_1$ or $u_1'$.  Otherwise every vertex in $V_1$ has at least one neighbor in $U_2\setminus U_2'$.  Since $|U_2\setminus U_2'|\leq 2(s-1)(2s-1)$, this contradicts the fact that $\Delta(U_2\setminus U_2^M, V_1)\leq 2\alpha^{1/3}n$. 
% 





%\textbf{Case 3.2.2.1.3.3.}??

\textbf{Case 3.2.2.2.} $|U_1|>\ell_1s$.  Let $a_1:=|U_1|-\ell_1s$.  Recall $\ell_2=m-\ell_1$, $b_1=|V_1|-\ell_1s$, $a_2=|U_2|-(\ell_2s-s)$, and $b_2:=|V_2|-(\ell_2s-s)$.  We have
\begin{equation}\label{3222:V1U2}
\delta(V_1, U_2)+\delta(U_2, V_1)\geq n+3s-5-(\ell_1s+s-a_2)-(\ell_2s-b_1)=2s-5+a_2+b_1
\end{equation}
and
\begin{equation}\label{3222:V2U1}
\delta(V_2, U_1)+\delta(U_1, V_2)\geq n+3s-5-(\ell_2s-a_1)-(\ell_1s+s-b_2)=2s-5+a_1+b_2
\end{equation}

\textbf{Case 3.2.2.2.1.} For some $i\in\{1,2\}$ we have $\delta(V_i, U_{3-1})\geq s$ or $\delta(U_{3-i}, V_i)\geq s$.  Without loss of generality (all cases are similar, but not exactly the same), suppose $\delta(V_2, U_1)\geq s$.  This implies by Lemma \ref{STARSlemma}(iii) that there is a set of $a_1$ vertex disjoint $s$-stars from $U_1$ to $V_2$ and a set of $b_2$ vertex disjoint $s$-stars from $V_2$ to $U_1$.  So if we can move $a_2$ vertices from $U_2$ or $b_1$ vertices from $V_1$, then we say that we are done.  If $\delta(V_1, U_2)\geq s$ or $\delta(U_2, V_1)\geq s$, then we can apply Lemma \ref{STARSlemma}(i) or (iii) and we are done, so suppose not.  This implies $2\leq a_2+b_1\leq 3$ by \eqref{3222:V1U2}.  Furthermore, if $a_2+b_1=3$, then $\delta(V_1, U_2)+\delta(U_2, V_1)\geq 2s-2$ and we may suppose $\delta(V_1, U_2)=s-1$ and $\delta(U_2, V_1)=s-1$.  Let $U_2':=\{u\in U_2:\deg(u, V_1)\leq s-1\}$ and $V_1':=\{v\in V_1:\deg(v, U_2)\leq s-1\}$.  

Since $2\leq a_2+b_1\leq 3$, either $a_2=1$ or $b_1=1$.  Without loss of generality suppose $a_2=1$ and thus $1\leq b_1\leq 2$.  If there is a vertex $u_2\in U_2$ such that $\deg(u_2, V_1)\geq s$, then we can move $u_2$ and we are done, so suppose $\Delta(U_2, V_1)\leq s-1$.  For all $u\in U_2$ and $v\in V_1'$ we have $n+3s-5\leq \deg(u)+\deg(v)\leq \ell_1s+s-1+s-1+\ell_2s-b_1+s-1\leq n+3s-4$ and thus $\delta(U_2, V_0\cup V_2)\geq |V_0\cup V_2|-1$ and $\delta(V_1', U_0\cup U_1)\geq |U_0\cup U_1|-1$.  If $b_1=1$, then we may suppose $\Delta(V_1, U_2)\leq s-1$ or else we are done.  In this case $V_1'=V_1$.  If $b_1=2$, then $\delta(V_1, U_2)\geq s-1$. If there are two vertex disjoint $s$-stars from $V_1$ to $U_2$, then we are done since $b_1\leq 2$.  This implies that $|V_1'|\geq |V_1|-2s\alpha^{1/3}k_2s$.  So in either case there exists a vertex $u_2\in U_2$ such that $\deg(u_2, V_1')=s-1$.  Since $\delta(V_2, U_1)\geq s$, there is a set of $s$ vertex disjoint $s$-stars from $N(u_2)\cap V_2$ to $U_1$.  Finally since $\delta(V_2', U_0\cup U_1)\geq |U_0\cup U_1|-1$, the leaf set of one of the $s$-stars from $V_2$ to $U_1$ will form a $K_{s-1,s-1}$ with $s-1$ vertices in $N(u_2)\cap V_1'$ and $s-1$ vertices in $U_1$.  Then we move $b_2-1$ more vertices from $V_2$.


% Suppose $a_2=1$, $b_1=2$.  If there is a vertex $u_2\in U_2$ such that $\deg(u_2, V_1)\geq s$, then we can move $u_2$ and we are done, so suppose $\Delta(U_2, V_1)\leq s-1$.  For all $u\in U_2$ and $v\in V_1'$ we have $n+3s-5\leq \deg(u)+\deg(v)\leq \ell_1s+s-1+s-1+\ell_2s-2+s-1=n+3s-5$ and thus $G[U_2, V_0\cup V_2]$ and $G[V_1', U_0\cup U_1]$ are complete.  If there are two vertex disjoint $s$-stars from $V_2$ to $U_1$, then we are done.  This implies that $|V_1'|\geq |V_1|-2s\alpha^{1/3}k_2s$. So choose $a_1$ vertex disjoint $s$-stars from $U_1$ to $V_2$ and two vertex disjoint $(s-1)$-stars from $V_1'$ to $U_2$ with centers $v_1$ and $v_1'$.  Since $G[U_2, V_0\cup V_2]$ and $G[V_1', U_0\cup U_1]$ are complete, if $a_1-2\geq 0$ we move $v_1$ and $v_1'$ along with two of the centers from $U_1$ and move the remaining $a_1-2$ centers from $U_1$.  If $a_1-2< 0$, then we move $v_1$ and $v_1'$ along with one of the centers from $U_1$ and one vertex from $U_0$.
% 
% Suppose $a_2=2$, $b_1=1$.   If there is a vertex $v_1\in V_1$ such that $\deg(v_1, U_2)\geq s$, then we can move $v_1$ and we are done, so suppose $\Delta(V_1, U_2)\leq s-1$.  For all $u\in U_2'$ and $v\in V_1$ we have $n+3s-5\leq \deg(u)+\deg(v)\leq \ell_1s+s-1+s-1+\ell_2s-2+s-1=n+3s-5$ and thus $G[U_2', V_0\cup V_2]$ and $G[V_1, U_0\cup U_1]$ are complete.   If there are two vertex disjoint $s$-stars from $U_2$ to $V_1$, then we are done.  This implies that $|U_2'|\geq |U_2|-2s\alpha^{1/3}k_2s$.  So choose $b_2$ vertex disjoint $s$-stars from $V_2$ to $U_1$ and two vertex disjoint $(s-1)$-stars from $U_2'$ to $V_1$ with centers $u_2$ and $u_2'$.  Since $G[U_2', V_0\cup V_2]$ and $G[V_1, U_0\cup U_1]$ are complete, if $b_2-2\geq 0$ we move $u_2$ and $u_2'$ along with two of the centers from $V_2$ and move the remaining $a_1-2$ centers from $V_2$.  If $b_2-2< 0$, then we move $u_2$ and $u_2'$ along with one of the centers from $V_2$ and one vertex from $V_0$.
% 
% If $a_2=1$ and $b_1=1$, then we may suppose $\Delta(V_1, U_2), \Delta(U_2, V_1)\leq s-1$ or else we are done.  This implies that $\delta(V_1, U_0\cup U_1)\geq |U_0\cup U_2|-1$ and $\delta(U_2, V_0\cup V_2)\geq |V_0\cup V_2|-1$.  Let $u_2\in U_2$ such that $\deg(u_2, V_1)\geq s-1$ and choose a set of $s$ vertex disjoint $s$-stars from $N(u_2)\cap V_2$ to $U_1$.  Since $\delta(V_1, U_0\cup U_1)\geq |U_0\cup U_2|-1$, one of these $s$-stars will form a $K_{s-1,s-1}$ with $s-1$ vertices in $N(u_2)\cap V_1$ and $s-1$ vertices in $U_1$.  Then we move $b_2-1$ remaining vertices from $V_2$.

\textbf{Case 3.2.2.2.2.} For all $i\in\{1,2\}$ we have $\delta(V_i, U_{3-i})\leq s-1$ and $\delta(U_{3-i}, V_i)\leq s-1$.  So by \eqref{3222:V1U2} and \eqref{3222:V2U1}, we may suppose $2\leq a_1+b_2\leq 3$ and $2\leq a_2+b_1\leq 3$.  We have 
\begin{equation}\label{32222a}
\delta(V_2, U_1)\geq k_2s+2s-5-r-(\ell_2s-a_1)=(k_2-\ell_2)s+2s-5-r+a_1\geq (k_2-\ell_2)s+s-4+a_1.
\end{equation}
If $\ell_1>k_1$, then $k_2>\ell_2$ and $\delta(V_2, U_1)\geq s$ by \eqref{32222a}.  So suppose $\ell_1=k_1$ and thus $\ell_2=k_2$.  We also have
\begin{equation}\label{32222b}
\delta(V_1, U_2)\geq k_2s+2s-5-r-(k_1s+s-a_2)=(k_2-k_1)s+s-5-r+a_2.
\end{equation}
If $k_2\geq k_1+2$, then $\delta(V_1, U_2)\geq s$ by \eqref{32222b}. So suppose $k_2\leq k_1+1$.  If $k_2=k_1$, then $r\leq \frac{s-6}{2}$ by Claim \ref{k2approxk1} and thus \eqref{32222a} gives $\delta(V_2, U_1)\geq 2s-5-\frac{s-6}{2}+a_1\geq s$. So suppose $k_2=k_1+1$ which implies $r\leq s-3$ by Claim \ref{k2approxk1}.  If $r\leq s-4$, then \eqref{32222a} implies $\delta(V_2, U_1)\geq s-1+a_1\geq s$.  So suppose $r=s-3$.  Finally if either $a_1\geq 2$ or $a_2\geq 2$, then \eqref{32222a} or \eqref{32222b} implies $\delta(V_1, U_2)\geq s$ or $\delta(V_2, U_1)\geq s$.  So suppose $a_1=1=a_2$ and thus $\delta(V_1, U_2)=s-1=\delta(V_2, U_1)$.  For $i=1,2$, let $V_i':=\{v\in V_i:\deg(v, U_{3-i})\leq s-1\}$.  For all $v\in V_i$, $\deg(v, U_0\cup U_i)\geq k_2s+2s-5-r-(s-1)=k_2s-1=|U_0\cup U_i|$, thus $G[V_i', U_0\cup U_i]$ is complete.



First suppose $b_1=2=b_2$. Since $|V_1|>|U_2|$ and $|V_2|>|U_1|$, there are vertices $u_1\in U_1$ and $u_2\in U_2$ such that $\deg(u_1, V_2)\geq s$ and $\deg(u_2, V_1)\geq s$.  If $|V_i\setminus V_i'|>2s\alpha^{1/3}k_2s$ for some $i$, then we would be done by moving two vertices from $V_i\setminus V_i'$ and moving $u_i$ from $U_i$ for some $i=1,2$.  So we may assume that $|V_i'|\geq |V_i|-s\alpha^{1/3}n$ for $=1,2$.  Since $\delta(V_1', U_2)\geq s-1$ and $|V_1'|\geq |V_1|-s\alpha^{1/3}n$, there exists $u_2\in U_2$ such that $\deg(u_2, V_1')\geq s-2$ and there exists $u_1\in U_1$ such that $\deg(u_1, V_2')\geq 2$.  Now since $G[V_1', U_0\cup U_1]$ and $G[V_2', U_0\cup U_2]$ are complete, we have a copy of $K_{s,s}$ with $s-2$ vertices in $V_1'$, $2$ vertices in $V_2'$, $s-2$ vertices in $U_0$, $1$ vertex in $U_1$ and $1$ vertex in $U_2$.  Then we move the remaining $s-4$ vertices from $V_0$ to $V_1$

Now suppose $b_i=2$ and $b_{3-i}=1$ for some $i$.  Without loss of generality, suppose $b_1=1$ and $b_2=2$.  Since $|V_2|>|U_1|$, there is a vertex $u_1\in U_1$ such that $\deg(u_1, V_2)\geq s$.  So we would be done unless $\Delta(V_1, U_2)\leq s-1$ and thus $V_1'=V_1$.  Let $u_2, u_2'\in U_2$ be the centers of two vertex disjoint $(s-1)$-stars from $U_2$ to $V_1$.  Then since $\delta(V_2, U_1)\geq s-1$ we can choose two vertex disjoint $(s-1)$-stars from $(N(u_2)\cap N(u_2'))\cap V_2$ to $U_1$.  Then since $G[V_1, U_0\cup U_1]$ is complete we are done.

Finally suppose $b_1=1=b_2$.  If there exists $v_2\in V_2$ (without loss of generality) such that $\deg(v, U_1)\geq s$, then there is a vertex $u_1\in U_1$ such that $\deg(u_1, V_2)\geq s$.  So we would be done unless $\Delta(V_1, U_2)\leq s-1$ and $\Delta(U_2, V_1)\leq s-1$.  Thus $G[V_1, U_0\cup U_1]$ is complete.  Let $u_2, u_2'\in U_2$ be the centers of two vertex disjoint $(s-1)$-stars from $U_2$ to $V_1$.  Then since $\delta(V_2, U_1)\geq s-1$ we can choose two vertex disjoint $(s-1)$-stars from $N(u_2)\cap N(u_2')\cap V_2$ to $U_1$.  Then since $G[V_1, U_0\cup U_1]$ is complete we are done.  Otherwise $\Delta(V_i, U_{3-i})\leq s-1$ for $i=1,2$ in which case $G[V_i, U_0\cup U_i]$ is complete for $i=1,2$.  Let $u_1\in U_1$ such that $\deg(u_1, V_2)\geq s-1$ and let $v_1u_2\in E(V_1, U_2)$.  Since $G[V_1, U_0\cup U_1]$ and $G[V_2, U_0\cup U_2]$ are complete, we have a copy of $K_{s,s}$ with $s-1$ vertices in $V_2$, $1$ vertex in $V_1$, $s-2$ vertices in $U_0$, $1$ vertex in $U_1$, and $1$ vertex in $U_2$.  Then we move the remaining $s-2$ vertices from $V_0$ to $V_2$.



\section{Examples when $\delta_U$ is small}


\subsection{A probabilistic example}

We prove Theorem \ref{probexample}.  We ignore floors and ceilings since they are not vital to our calculations.

\begin{proof}
Given a positive integer $s$, let $c:=s^{1/3}$, $d:=2c$, $a := s^c$, and $b := \frac{s}{d}a=\frac{s^{c+1}}{d}$. Let $s$ be large enough so that $s^{2s^{2/3}}\left(\frac{(3d)^{d}}{s^{(c-1)s}}\right)^{s}<\frac{1}{2}$.
Let $A, B$ be sets such that $|A| = a$ and $|B| = b$. Consider the random bipartite graph by adding the pair from $A \times B$ with
probability $p := \frac{3d}{s}$ (all choices made independently). Then for $u\in A$,
$\mathbb{E}(\deg(u)) = pb = 3s^c$ and for $v\in B$, $\mathbb{E}(\deg(v)) = pa = 3ds^{c-1}$.  The probability that there exists $u\in A$ with $\deg(u) < 2s^c$ or $v \in B$ with $\deg(v) < 2ds^{c-1}$ %is less than $s^{c+1}\exp(-\frac{s^c}{6}) < 0.5$
is less than $1/2$ by a standard application of Chernoff's bound. In addition,
the probability that there exists $K_{d,s}$ with $d$ vertices in $A$ is at most
\begin{align*}
\binom{a}{d}\binom{b}{s}p^{ds}<a^d b^s p^{ds}=s^{cd} \frac{s^{(c+1)s}}{d^s} \frac{(3d)^{ds}}{s^{ds}}&=\frac{s^{cd}}{s^{(d-(c+1))s}}\left(\frac{(3d)^d}{d}\right)^s \\
&\leq s^{2s^{2/3}}\left(\frac{(3d)^d}{s^{(c-1)s}}\right)^s <\frac{1}{2}.
\end{align*}
Consequently there exists a graph $H$ on $A\cup B$ such that
\begin{itemize}
\item $\deg(u) \geq 2s^c$ for every $u \in A$, $\deg(v) \geq 2ds^{c-1}$ for $v \in B$ and
\item  $H$ has no $K_{d,s}$ with $d$ vertices in $A$.
%\item $|W| < s|U|/l$.
\end{itemize}

Let $G$ be obtained from $H$ by adding a set $A'$ of $n - a$ vertices to $A$ and a set $B'$ of $n-b$ vertices to $B$ with $n$ large as usual. We add all edges between $A'$ and $B\cup B'$.  The sum of degrees in $G$ is at least $2s^c+(n-s^c)= n + s^c$.   

Suppose that $G$ can be tiled with $K_{s,s}$.  Since $G[A, B']$ is empty, any copy of $K_{s,s}$ touching $A$ must have $s$ vertices in $B$.  Also, any copy touching $A$ must have at most $d-1$ vertices from $A$, since $H$ has no $K_{d,s}$ with $d$ vertices in $A$.  So the number of copies touching $A$ is at least $\frac{a}{d-1}$.  However, this implies that $s\frac{a}{d-1}\leq |B|= \frac{s}{d}a$, a contradiction.
\end{proof}



\subsection{Concrete examples}

% 
% Supposing $n$ is sufficiently large, we proved (in Proposition 3.11) that if $G$ is balanced bipartite graph on $2n=2ms$ vertices with $\delta_U+\delta_V\geq n+2s-2\croot{s}+1$ and $\delta_V\geq \delta_U\geq \frac{n}{10s^3}$, then $G$ can be tiled with $K_{s,s}$.  Furthermore, we have an example of a graph $G$ with $\delta_U+\delta_V= n+2s-2\croot{s}$ such that $G$ cannot be tiled with $K_{s,s}$. (This example requires that $s=p^2+q$, where $1\leq q\leq p$, for instance when $s=5$ we need $\delta_U+\delta_V\geq n+2s-2\croot{s}+1$)

We do not provide a general class of counterexamples in this section, however we provide two specific cases of graphs with $\delta_U=O(1)$ and $\delta_U+\delta_V\geq n+2s-2\croot{s}+c(s)$ which cannot be tiled with $K_{s,s}$.

Let $s=5$.  First note that $n+2s-2\croot{s}+c(s)=n+5$.  We will show that there exists a graph with $\delta_U+\delta_V=n+5$ which cannot be tiled with $K_{5,5}$.  Let $G[U,V]$ be a balanced bipartite graph with the following properties.  Let $|U|=|V|=5m=:n$.  Partition $U$ as $U=U_1\cup U_2$ where $|U_1|=3$, $|U_2|=n-3$ and $V$ as $V=V_1\cup V_2$ where $|V_1|=4$ and $|V_2|=n-4$.  Let $G[U_i, V_i]$ be complete for $i=1,2$.  Let $G[V_1, U_2]$ be complete.  Finally suppose $U_1=\{a,b,c\}$ and let $N(a)\cap V_2=\{a_1, a_2, a_3, a_4\}$, $N(b)\cap V_2=\{b_1, b_2, b_3, b_4\}$, and $N(c)\cap V_2=\{c_1, c_2, c_3, c_4\}$ where $a_4=b_1$, $b_4=c_1$, $c_4=a_1$, and $a_2, a_3, b_2, b_3, c_2, c_3$ are distinct (see Figure \ref{s=5}).  Note that $\delta_U=8$, $\delta_V=n-3$ and thus $\delta_U+\delta_V=n+5=n+2s-2\croot{s}+c(s)$.  Suppose $G$ can be tiled with $K_{5,5}$.  Since $|N(a,b,c)|=4$, it is not the case that $a,b,c$ all belong to one copy.  So either $a$, $b$, and $c$ are in distinct copies, or say $b$ and $c$ belong to the same copy.  First suppose that $a$, $b$, and $c$ are in distinct copies and let $A$, $B$ and $C$ be copies of $K_{5,5}$ such that $a\in A$, $b\in B$, and $c\in C$.  Let $\alpha:=|V(A)\cap V_1|$, $\beta:=|V(B)\cap V_1|$, and $\gamma:=|V(C)\cap V_1|$.  Since $|V_1|=4$, we have $\alpha+\beta+\gamma\leq 4$.  Also since $|(N(a)\cup N(b)\cup N(c))\cap V_2|=9$, we have $5-\alpha+5-\beta+5-\gamma \leq 9$ which implies $6\leq \alpha+\beta+\gamma $, a contradiction.  So suppose that $b$ and $c$ belong to the same copy.  But since $|N(b, c)\cap V_2|=1$, we have $|N(b,c)\cap V_1|=4$.  But since $|N(a)\cap V_2|=4$, it is not possible for $a$ to belong to a disjoint copy of $K_{5,5}$.


% This shows that when $s=5$, we start with needing $\delta_U+\delta_V\geq n+3s-5=n+10$ when $\delta_U\approx \delta_V$.  The situation gradually improves to $\delta_U+\delta_V\geq n+2s-2\croot{s}+1=n+5$ as $\delta_V$ gets larger than $\delta_U$.  However, this example shows that $\delta_U+\delta_V= n+2s-2\croot{s}+1=n+5$ does not suffice when $\delta_U$ is very small.

Let $s=10$.  First note that $n+2s-2\croot{s}+c(s)=n+13$.  We will show that there exists a graph with $\delta_U+\delta_V=n+15$ which cannot be tiled with $K_{10,10}$.  Let $G[U,V]$ be a balanced bipartite graph with the following properties.  Let $|U|=|V|=10m=:n$.  Partition $U$ as $U=U_1\cup U_2$ where $|U_1|=4$, $|U_2|=n-4$ and $V$ as $V=V_1\cup V_2$ where $|V_1|=9$ and $|V_2|=n-9$.  Let $G[U_i, V_i]$ be complete for $i=1,2$.  Let $G[V_1, U_2]$ be complete.  Finally suppose $U_1=\{a,b,c,d\}$ and let $N(a)\cap V_2=\{a_1, \dots, a_{10}\}$, $N(b)\cap V_2=\{b_1,\dots, b_{10}\}$, $N(c)\cap V_2=\{c_1, \dots, c_{10}\}$, and $N(d)=\{d_1,\dots, d_{10}\}$ where $\{a_7,a_8,a_9,a_{10}\}=\{b_1,b_2, b_3, b_4\}$, $\{b_7,b_8,b_9,b_{10}\}=\{c_1,c_2, c_3, c_4\}$, $\{c_7,c_8,c_9,c_{10}\}=\{d_1,d_2, d_3, d_4\}$, $\{d_7,d_8,d_9,d_{10}\}=\{a_1,a_2, a_3, a_4\}$ and $a_5, a_6, b_5, b_6, c_5, c_6, d_5, d_6$ are distinct (see Figure \ref{s=10}).  Note that $\delta_U=19$, $\delta_V=n-4$ and thus $\delta_U+\delta_V=n+15=n+2s-2\croot{s}+c(s)$.  Suppose $G$ can be tiled with $K_{10,10}$.  Since $|N(x,y,z)|=9$, for any $x,y,z\in\{a,b,c,d\}$ it is not the case that any three of $a,b,c,d$ all belong to one copy.  A similar analysis as given in the $s=5$ case will lead to a contradiction here.

% So either $a$, $b$, and $c$ are in distinct copies, or say $b$ and $c$ belong to the same copy.  First suppose that $a$, $b$, and $c$ are in distinct copies and let $A$, $B$ and $C$ be copies of $K_{5,5}$ such that $a\in A$, $b\in B$, and $c\in C$.  Let $\alpha:=|V(A)\cap V_1|$, $\beta:=|V(B)\cap V_1|$, and $\gamma:=|V(C)\cap V_1|$.  Since $|V_1|=4$, we have $\alpha+\beta+\gamma\leq 4$.  Also since $|(N(a)\cup N(b)\cup N(c))\cap V_2|=9$, we have $5-\alpha+5-\beta+5-\gamma \leq 9$ which implies $6\leq \alpha+\beta+\gamma $, a contradiction.  So suppose that $b$ and $c$ belong to the same copy.  But since $|N(b, c)\cap V_2|=1$, we have $|N(b,c)\cap V_1|=4$.  But since $|N(a)\cap V_2|=4$, it is not possible for $a$ to belong to a disjoint copy of $K_{5,5}$.


%It may be possible that there is just a mistake with Proposition 3.11 and that $n+2s-2\croot{s}+1$ isn't really good enough, however 
%If we think about what this picture looks like in Proposition 3.11, we have a bunch of $(s-1)$-stars from $U_1$ to $V_2$.  When $\delta_U\geq \frac{n}{10s^3}$, we have $|U_1|\approx \delta_U$ and there we will be able to find a matching of $(s-1)$-stars, but here in Figure \ref{n+5} we get stuck.


% 
% 
% \section{Appendix: Technical Inequalities}
% 
% % \begin{fact}\label{fact 1}
% % For all $s\in\mathbb{N}$, if $s\geq 2$ then
% % \begin{align*}
% % s-2\croot{s}\geq \left(\croot{s}-\floor{\frac{\sqrt{4\sqrt{s-1}-3}-1}{2}}-3\right)\left(\croot{s}+\floor{\frac{\sqrt{4\sqrt{s-1}-3}-1}{2}}\right)-1
% % \end{align*}
% % 
% % \end{fact}
% % 
% % \begin{fact}\label{fact 2}
% % For all $s\in\mathbb{N}$, if $s\geq 2$ and $s=p^2+q$ where $p\in \mathbb{N}$ is maximal and $(q=0 \vee p+1\leq q\leq 2p)$, then
% % \begin{align*}
% % s-2\croot{s}+1\geq \left(\croot{s}-\floor{\sqrt[4]{s-1}}-2\right)\left(\croot{s}+\floor{\sqrt[4]{s-1}}\right)-1
% % \end{align*}
% % 
% % \end{fact}
% 
% \begin{fact}\label{fact 3}
% For all $s\in\mathbb{N}$, if $s\geq 2$ then
% \begin{align*}
% s+3-2\croot{s}+2\sqrt{s+2-2\croot{s}}\geq s-1
% \end{align*}
% 
% \end{fact}
% 
% \begin{fact}\label{fact 4}
% For all $s\in\mathbb{N}$, if $s\geq 2$ and $s=p^2+q$ where $p\in \mathbb{N}$ is maximal and $(q=0 \vee p+1\leq q\leq 2p)$, then
% \begin{align*}
% s+2-2\croot{s}+2\sqrt{s+1-2\croot{s}}\geq s-2
% \end{align*}
% 
% \end{fact}

\section{Conclusion}

In Theorem \ref{main 1} and Theorem \ref{main 2} we show that if $\delta(G)$ is $\Omega(n)$, then $\delta_U+\delta_V\geq n+3s-5$ suffices to tile $G$ with $K_{s,s}$.  The only example we have which shows $n+3s-5$ is best possible has the property that $\delta_U=\delta_V$.  When $\delta_V>\delta_U$ we have examples which show that we can't do better than $n+3s-7$.  This raises the question of whether $n+3s-6$ suffices when $\delta_V>\delta_U$.

In Theorem \ref{probexample}, we show that there exist balanced bipartite graphs on $2n$ vertices with $\delta_U+\delta_V\geq n+s^{s^{1/3}}$ which cannot be tiled with $K_{s,s}$.  An interesting problem would be to determine the largest possible value of $\delta_U+\delta_V$ such that $G[U,V]$ cannot be tiled with $K_{s,s}$.  We note that if $G[U,V]$ is a graph with $\delta_U+\delta_V\geq (1+\ep)n$, then $\delta_U\geq \ep n$ and thus we can apply Theorem \ref{main 1} or Theorem \ref{main 2} to obtain a tiling of $G$.

Finally, while we don't address the case of tiling with $K_{s,t}$ here, we point out that it is easy to prove an analog of Theorem \ref{main 2} for $K_{s,t}$.  In fact, even if we only assume $\delta_U+\delta_V\geq n$, we can tile $G$ with $K_{s,t}$:  the proof of Theorem \ref{main 2} is easy when there exists $\ell$ such that $|U_1|\leq \ell s$ and $|V_0\cup V_1|\geq \ell s$ by Claim \ref{V1>U1}, so we just remove copies of $K_{s,t}$ from $G[U_1, V_1]$, each with $t$ vertices in $U_1$, until the desired property holds and then we can finish the tiling as we do here.



\clearpage


\newpage
%\chapter*{REFERENCES\hfill} \addcontentsline{toc}{chapter}{REFERENCES}
%[Enter your text here]
%\clearpage
 


\begin{thebibliography}{99}
\SingleSpacing

% SIAM citation style:
% http://www.siam.org/journals/instruct.php

\bibitem{Abas} S. Abbasi, The solution of the El-Zahar problem, \emph{Ph.D. Thesis, Rutgers University, New Brunswick, NJ, 1998.}

\bibitem{AB}  M. Aigner and S. Brandt, Embedding arbitrary graphs of maximum
degree two, \emph{J. London Math. Soc.} \textbf{48} (1993), 39--51.

\bibitem{Alon} N. Alon, R. Yuster, $H$-factors in dense graphs, \emph{J. Combin. Theory B}, \textbf{66}, (1996), 269--282.

\bibitem{A} D. Amar, Partition of a hamiltonian graph into two cycles, \emph{Discrete Mathematics} \textbf{58} (1986), 1--10.

\bibitem{BE1} B. Bollob\'as, S. E. Eldridge,
Maximal matchings in graphs with given maximal and minimal
degrees,  \emph{Congressus Numerantium}  \textbf{XV} (1976),  165--168.

\bibitem{Cat} P.A. Catlin, Embedding subgraphs and coloring graphs under extremal degree conditions, \emph{Ph. D. Thesis, Ohio State Univ., Columbus, 1976}.

\bibitem{C}P. Ch\^au, An Ore-type theorem on hamiltonian square cycles
(2010), preprint. %consistency 

\bibitem{Ch} H. Chernoff, A measure of asymptotic efficiency for tests of a hypothesis based on the sum of observations, \emph{Ann. Math. Statistics} \textbf{23} (1952). 493--507. 

\bibitem{CH} K. Corradi, A. Hajnal, On the maximal number of independent circuits in a graph, \emph{Acta Math. Acad. Sci. Hungar.} \textbf{14} (1963), 423--439. 


\bibitem {CSS} B. Csaba, A. Shokoufandeh, E. Szemer\'edi,  Proof of a conjecture of Bollob\'as and Eldridge for graphs of maximum degree three, \emph{Combinatorica} \textbf{23} (2003), 35--72.

\bibitem{CD} A. Czygrinow, L. DeBiasio, A note on bipartite graph tiling, \emph{to appear in SIAM J. Discrete Math.}

\bibitem{CDK} A. Czygrinow, L. DeBiasio, H.A. Kierstead, $2$-factors of bipartite graphs with asymmetric minimum degrees, \emph{SIAM J. Discrete Math.} \textbf{24} (2010), no. 2, 486--504. 

\bibitem{CK} A. Czygrinow, H. A. Kierstead, $2$-factors in bipartite graphs, 
\emph{Discrete Mathematics} \textbf{257}, no. 2-3 (2002), 357--369.

\bibitem{Di} R. Diestel, Graph Theory, 4th Edition, Springer
(2010).

\bibitem{D} G. A. Dirac, Some theorems on abstract graphs, \emph{Proceedings
of the London Mathematical Society} \textbf{2} (1952), 68--81.

\bibitem{E} P. Erd\H{o}s, Problem 9, in: M. Fiedler (Ed.), \emph{Theory 
of Graphs and Its Applications}, Czech. Acad. Sci. Publ., Prague, 1964,
p. 159.

\bibitem{EZ} M. H. El-Zahar, On circuits in graphs, \emph{Discrete Mathematics} \textbf{50} (1984), 227--230.

\bibitem{FK1}G. Fan, H. A. Kierstead, The square of paths and cycles,
\emph{Journal of Combinatorial Theory, Series B} \textbf{63} (1995), 55--64.

\bibitem{FK2}G. Fan, H.A. Kierstead, Hamiltonian square-paths, \emph{Journal 
of Combinatorial Theory, Series B} \textbf{67} (1996), 167--182.

%\bibitem{FK}  H. A. Kierstead and G. Fan, Hamiltonian square paths, \emph{J.
%of Combinatorial Theory B} \textbf{63} (1995), 55-64.

\bibitem{FK3}G. Fan, H.A. Kierstead, Partitioning a graph into two
square-cycles, \emph{Journal of Graph Theory} \textbf{23} (1996), 241--256.

\bibitem{HSz} A. Hajnal, E. Szemer\'edi, Proof of a conjecture of P. Erd\H{o}s, \emph{Combinatorial Theory and its Application} ( P. Erd\H{o}s, A. R\'enyi, and V. T. S\'os, Eds.) North-Holland, London (1970), 601--623.

\bibitem{HS} J. Hladk\'y, M. Schacht, Note on bipartite graph tilings, \emph{SIAM J. Discrete Math.}  \textbf{24}, no. 2 (2010), 357--362.

\bibitem{JLR} S. Janson, T. \L{}uczak, A. Ruci\'{n}ski, Random Graphs,
Wiley, New York, 2000.

\bibitem{KKO} P. Keevash, D. K\"uhn, D. Osthus, An exact minimum degree condition for Hamilton cycles in
oriented graphs, \emph{J. London Math. Soc.} \textbf{79} (2009), 144--166.

\bibitem{KK} H.A. Kierstead, A.V. Kostochka, An Ore-type theorem on equitable coloring, \emph{J. Combin. Theory Ser. B} \textbf{98},  no. 1 (2008), 226--234.

\bibitem{KSSp}  J. Koml\'{o}s, G. N. S\'{a}rk\"{o}zy, E. Szemer\'{e}di, On the square of a Hamiltonian cycle in dense graphs, Proceedings of the Seventh
International Conference on Random Structures and Algorithms (Atlanta, GA, 1995). \emph{Random
Structures and Algorithms} \textbf{9} (1996), no. 1-2, 193--211.

\bibitem{KSSbu}  J. Koml\'{o}s, G. N. S\'{a}rk\"{o}zy, E. Szemer\'{e}di, Blow-up
lemma, \emph{Combinatorica} \textbf{17} (1997), no. 1, 109--123.

\bibitem{KSSps}J. Koml\'{o}s, G.N. S\'{a}rk\"{o}zy, E. Szemer\'{e}di, On the P\'osa-Seymour
conjecture, \emph{Journal of Graph Theory} \textbf{29}, no. 3 (1998), 167--176.

\bibitem{KSSs} J. Koml\'{o}s, G. N. S\'{a}rk\"{o}zy, E. Szemer\'{e}di, Proof of the Seymour Conjecture for Large Graphs, \emph{Annals of Combinatorics} \textbf{2} (1998), 43--60.

\bibitem{KSS}  J. Koml\'{o}s, G. N. S\'{a}rk\"{o}zy,  E. Szemer\'edi, Proof of the Alon-Yuster conjecture, \emph{Discrete Math.}, \textbf{235} (2001), 255--269.

\bibitem{KS}  J. Koml\'{o}s, M. Simonovits, Szemer\'{e}di's regularity
lemma and its applications in graph theory, in \emph{Combinatorics, Paul Erd\H{o}s is Eighty}, Vol. 2 (D. Mikl\'{o}s, V. T. S\'{o}s, T. Szonyi, eds.),
J\'{a}nos Bolyai Math. Soc., Budapest (1996), 295--352.

\bibitem{KY1} A.V. Kostochka, G. Yu, Ore-type graph packing problems, \emph{Combin. Probab. Comput.} \textbf{16}, no. 1 (2007), 167--169.

\bibitem{KY2} A.V. Kostochka, G. Yu, Graphs containing every 2-factor, \emph{manuscript} (2009).

\bibitem{KO}D. K\"{u}hn, D. Osthus. The minimum degree threshold for perfect graph packings, \emph{Combinatorica} \textbf{29}, no. 1 (2009), 65--107.

\bibitem{LSS} I. Levitt, G. N. S\'{a}rk\"{o}zy, E. Szemer\'{e}di, How to avoid
using the Regularity Lemma: P\'osa's conjecture revisited, \emph{Discrete
Mathematics} \textbf{310} (2010), 630--641. 

\bibitem{M} W. Mantel, Problem 28, soln. by H. Gouwentak, W. Mantel, J. Teixeira de Mattes, F. Schuh and W.A. Wythoff, \emph{Wiskundige Opgaven} \textbf{10} (1907), pp. 60--61.

\bibitem{MM} J. Moon, L. Moser, On hamiltonian bipartite graphs, \emph{Israel J. Math.} \textbf{1} (1963), 163--165.


\bibitem{RRS1}V. R\"{o}dl, A. Ruci\'{n}ski, E. Szemer\'{e}di, A Dirac-type
theorem for 3-uniform hypergraphs, \emph{Combinatorics, Probability and
Computing} \textbf{15} (2006), 229--251.

\bibitem{RRS2}V. R\"{o}dl, A. Ruci\'{n}ski, E. Szemer\'{e}di, Dirac-type
conditions for hamiltonian paths and cycles in 3-uniform hypergraphs,
preprint.

%\bibitem{S}P. Seymour, Problem section, in: T.P. McDonough, V.C.
%Mavron (Eds.), Combinatorics: Proceedings of the British Combinatorial
%Conference 1973, Cambridge University Press, 1974, pp. 201--202.

\bibitem{Sey} P. Seymour, Problem section, \emph{Combinatorics: Proceedings of the British Combinatorial Conference 1973}, T.P. McDonough and V.C. Mavron, Eds., Cambridge University Press (1974), 201--202.

\bibitem{Zh} A. Shokoufandeh, Y. Zhao, Proof of a conjecture of Koml\'{o}s, \emph{Random Struct. Alg.} \textbf{23}, (2003), 180--205. 

\bibitem{Sz}  E. Szemer\'{e}di, Regular partitions of graphs, \emph{
Colloques Internationaux C.N.R.S., Problemes Combinatories et Theorie des
Graphes} (1978), 399--402.

\bibitem{SzPro} E. Szemer\'edi, On sets of integers containing no $k$ elements in arithmetic progression, \emph{Acta Arith.}  \textbf{27} (1975), 199--245.

\bibitem{T} P. Tur\'an, On an extremal problem in graph theory (in Hungarian), \emph{Matematikai \'es Fizikai Lapok} \textbf{48} (1941), 436--452.

\bibitem{Twel} P. Tur\'an, A note of welcome, \emph{Journal of Graph Theory} \textbf{1} (1977), 7--9.

\bibitem{Tr} A. Treglown, A note on some embedding problems for oriented graphs, \emph{to appear in Journal of Graph Theory.}

\bibitem{W1}  H. Wang, On 2-factors of a bipartite graph, \emph{J. of Graph
Theory} \textbf{31} (1999), 101--106.

\bibitem{W2} H. Wang, Bipartite graphs containing every possible pair of cycles, \emph{Discrete Mathematics} \textbf{207} (1999), 233--242.

\bibitem{Z} Y. Zhao, Bipartite Graph Tiling, \emph{SIAM J. Discrete Math.} \textbf{23}, no. 2 (2009), 888--900.






%\bibitem{KSSp}J. Koml\'{o}s, G.N. S\'{a}rk\"{o}zy, E. Szemer\'{e}di, On the square
%of a hamiltonian cycle in dense graphs, Random Structures and Algorithms
%9 (1996) 193--211.

%\bibitem{KSSbu}J. Koml\'{o}s, G.N. S\'{a}rk\"{o}zy, E. Szemer\'{e}di, Blow-up Lemma,
%Combinatorica 17 (1) (1997) 109--123.

%\bibitem{KSSps}J. Koml\'{o}s, G.N. S\'{a}rk\"{o}zy, E. Szemer\'{e}di, On the P\'osa-Seymour
%conjecture, Journal of Graph Theory 29 (1998) 167--176.

%\bibitem{KSSs}J. Koml\'{o}s, G. N. S\'{a}rk\"{o}zy, E. Szemer\'{e}di, Proof of the
%Seymour conjecture for large graphs, Annals of Combinatorics 2 (1998)
%43--60.



\end{thebibliography}










%% maybe endnotes 
%% maybe bibliography
% if appendices, then

% \appendix
% \addcontentsline{toc}{chapter}{APPENDIX}
% \chapter{\uppercase{Insert Appendix A Title here}}
% \clearpage
% \chapter{\uppercase{Insert Appendix B Title here}}
% \clearpage

% if Biographical sketch then
% \newpage
% \chapter*{BIOGRAPHICAL SKETCH\hfill} \addcontentsline{toc}{chapter}{BIOGRAPHICAL SKETCH}
% [Enter your text here]
% \clearpage
% \newpage	
% This LaTeX document was generated using the Graduate College Format Advising tool. Please turn a copy of this page in when you submit your document to Graduate College format advising. You may discard this page once you have printed your final document. DO NOT TURN THIS PAGE IN WITH YOUR FINAL DOCUMENT! 
% font type: Times New Roman
% font size: 12

\end{document}		
		
